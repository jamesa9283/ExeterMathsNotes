\documentclass{article}


% Packages
\usepackage{fullpage}
\usepackage{amssymb}
\usepackage{multicol}
\usepackage{amsmath}
\usepackage{amsfonts}
\usepackage{bm}
\usepackage{float}
\usepackage{tikz}
\usepackage{xcolor}
\usetikzlibrary{shapes.geometric, positioning, arrows, intersections}
\tikzset{point/.style={circle,draw=black,inner sep=0pt,minimum size=3pt}}
\usepackage{amsthm}
\usepackage{tcolorbox}
\usepackage{hyperref}
\hypersetup{
    colorlinks=true, %set true if you want colored links
    linktoc=all,     %set to all if you want both sections and subsections linked
    linkcolor=black,  %choose some color if you want links to stand out
}
\usepackage{fancyhdr}

\renewcommand{\Re}{\operatorname{Re}}
\renewcommand{\Im}{\operatorname{Im}}


% Macros
\newcommand{\R}{\mathbb{R}}
\newcommand{\N}{\mathbb{N}}
\newcommand{\Q}{\mathbb{Q}}
\newcommand{\Z}{\mathbb{Z}}
\newcommand{\C}{\mathbb{C}}
\newcommand{\sub}{\subset}
\newcommand{\sm}{\setminus}
\renewcommand{\a}{\alpha}
\renewcommand{\b}{\beta}
\newcommand{\g}{\gamma}
\renewcommand{\d}{\delta}
\newcommand{\e}{\varepsilon}
\newcommand{\ex}{\exists\,}
\newcommand{\overbar}[1]{\mkern 1.5mu\overline{\mkern-1.5mu#1\mkern-1.5mu}\mkern 1.5mu}
\newcommand{\eR}{\overbar{\R}}
\newcommand{\xb}{\bar{x}}
\newcommand{\pd}[2]{\frac{\partial #1}{\partial #2}}

%\setlength{\columnsep}{20pt}

%ToC stuff
\newtheorem{problem}{Problem}
\newtheorem{solution}{Solution}

 % Document stuff

\title{Complex Analysis Coursework 2}
\author{James Arthur - 690055793}

\begin{document}
\maketitle

\begin{problem}
  By using the Cauchy-Riemann equations, or otherwise, find a function $f$, holomorphic on $\C$, such that
  $$ \Re(f(x + iy)) = 2x^3 - 6xy^2 + 2xy $$
\end{problem}


\begin{solution}
  We can say that $u(x,\, y) = 2x^3 - 6xy^2 + 2xy$ and hence we can differentiate and solve the PDE produced,
  $$ \pd{u}{x} = 6x^2 - 6y^2 + 2y = \pd{v}{y} $$
  and so,
  \begin{align*}
    v &= \int {6x^2 - 6y^2 + 2y , dy}\\
    &= 6yx^2 - 2y^3 + y^2 + f(x)\\
  \end{align*}
  and we can differentiate with respect to $x$,
  \begin{align*}
    \pd{v}{x} = 12xy + f'(x) &= -\pd{u}{y}\\
    &= - (- 12xy + 2x)
  \end{align*}
  Hence, $f'(x) = -2x$ and so $f(x) = C - x^2$. Now we can write his together as,
  $$ f(x + iy) = 2x^3 - 6xy^2 + 2xy + i(6yx^2 - 2y^3 + y^2 - x^2) + C \quad C \in \C$$
  and hence by partial converse of the Cauchy Riemann equations, this function is holomorphic.
\end{solution}

\newpage
\begin{problem}
  Suppose that $f$ is a function holomorphic at every point of the open disc
  $$ D = \{ z \in \C : |z| < 1 \} $$
  such that
  $$\Re(f(z)) + \Im(f(z)) = 10$$
  for all $z \in D$. Show that $f$ is constant in $D$.
\end{problem}

\begin{solution}
  As $f$ is holomorphic on the open disc $D$, we can use the Cauchy Riemann equations to prove the required result. We can rewrite the constraint as we know that $f(x + iy) = u(x, y) + iv(x, y)$ and so,
  $$ u + v = 10 $$
  by differentiating $u = 10 - v$, we can find that,
  \begin{align}
    \pd{u}{x} &= - \pd{v}{x} \tag{$*$}\\
    \pd{u}{y} &= - \pd{v}{y} \tag{$**$}
  \end{align}
  and we can hence rewrite the Cauchy Riemann equations using $(*)$ and $(**)$,\\
  \begin{minipage}{0.48\textwidth}
    \begin{align*}
      \pd{v}{y} &= \pd{u}{x}\\
      &= -\pd{v}{x}
    \end{align*}
  \end{minipage}
  \begin{minipage}{0.48\textwidth}
    \begin{align*}
      \pd{u}{y} &= -\pd{v}{x}\\
      &= \pd{u}{x}
    \end{align*}
  \end{minipage}\\

  \noindent
  Hence we can say that $v_y = -v_x$ which then leads us to say that $v_x = v_y = 0$ and with $u_y = u_x$ we can say $u_x = u_y = 0$ and hence $f$ is constant.
\end{solution}\qed

\newpage
\begin{problem}
  Show that
  $$ \left | \int_\g \frac{dz}{2 + z^2} \right | \le \pi $$
  where $\g$ is the upper half of the unit circle.
\end{problem}

\begin{solution}
  Firstly we say,
  \begin{align*}
    \left | \int_\g \frac{dz}{2 + z^2} \right | &\le \int_\g\left | \frac{dz}{2 + z^2} \right |\\
  \end{align*}
  and as $\g$ is the upperhalf of the unit circle we can say that,
  $$ \left | \frac{1}{2 + z^2} \right | \le 1 $$
  where the maximum is at $z = i$. Now applying the $ML$-bound we can say,
  $$ \left | \int_\g \frac{dz}{2 + z^2} \right | \le \pi $$
  as $\ell (\g) = \pi$.
\end{solution}\qed

\newpage
\begin{problem}
  If $\displaystyle{\sum_{n = 0}^\infty {a_nz^n}}$ has radius of convergence $R$, show that,
  $$ \sum_{n = 0}^\infty {\Re({a_n})z^n} $$
  has radius of convergence greater than or equal to $R$.
\end{problem}

\begin{solution}
  We define the radius of convergence as,
  $$ \lim_{n\to\infty} {\left | \frac{a_{n+1}}{a_n} \right |}$$
  and so we can look at the expression inside the limit,
  \begin{align*}
    \left | \frac{a_{n+1}}{a_n} \right | &\ge \frac{\Re(a_{n+1})}{|a_n|}\\
    &= \frac{|\overline{a_n}|\Re(a_{n+1})}{|\overline{a_n}a_n|}\\
    &\ge \frac{\Re(\overline{a_n})\Re(a_{n+1})}{|\overline{a_n}a_n|}\\
    &= \frac{\Re(\overline{a_n})\Re(a_{n+1})}{\Re(\overline{a_n}a_n)}\\
    &= \frac{\Re(\overline{a_n})\Re(a_{n+1})}{\Re(\overline{a_n})\Re(a_n)}\\
    &= \frac{\Re(a_{n+1})}{\Re(a_n)}\\
  \end{align*}
  and so
  $$ \left | \frac{a_{n+1}}{a_n} \right | \ge \frac{\Re(a_{n+1})}{\Re(a_n)} $$
  and now taking limits,
  \begin{align*}
    \left | \frac{a_{n+1}}{a_n} \right | &\ge \frac{\Re(a_{n+1})}{\Re(a_n)}\\
    \left | \frac{a_{n}}{a_{n+1}} \right | &\le \frac{\Re(a_{n})}{\Re(a_{n+1})}\\
    \lim_{n \to \infty} \left | \frac{a_{n}}{a_n+1} \right | &\le \lim_{n \to \infty} \frac{\Re(a_{n})}{\Re(a_{n+1})}\\
    R &\le \lim_{n \to \infty} \frac{\Re(a_{n})}{\Re(a_{n+1})}
  \end{align*}
  and so the radius of convergence of the real part of a series is greater than the radius of convergence of the series.
\end{solution}\qed

\end{document}
