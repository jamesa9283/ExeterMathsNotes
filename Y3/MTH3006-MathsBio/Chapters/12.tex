% !TEX root = ../notes.tex

\subsection{Turing Mechanism}
We now consider $n = 2$, we write these as $u$ and $v$.The application is pattern formation. In absence of diffusion, $u(\vec x, t) = u_0$ for all $\vec x, t$. This would produce a uniform pattern. We also assume in the absence of diffusion $ v(\vec x, t) = v_0$. We further assume that steady states are linearly stable. We will focus on deriving conditions such that the introduction of diffusion destabilises this and we get inhomogeneous patterning. \\

\noindent
We consider the reaction-diffusion equations for $u$ and $v$ in their normalised form ($\g > 0$),
\begin{equation}
  \begin{aligned}
    \pd u t &= \g f(u, v) + \nab^2 u \\
    \pd v t &= \gamma g(u, v) + \nab^2 v
  \end{aligned}\label{equ:RDNF}
\end{equation}
We consider some domain $\O$ with boundary $\partial \O$ with outward normal $\vec n$. Further, we take zero-flux boundary conditions,
$$ \vec n \cdot \nab u = \vec n \cdot \nab v = 0. $$
Suppose we have a spatially uniform steady state of \refeq{equ:RDNF}, $(u_0, v_0)$. Then, $(u_0, v_0)$ satisfies,
$$ f(u_0, v_0) = g(u_0, v_0) = 0. $$
Also note that $\vec n \cdot \nab u_0 = \vec n \cdot \nab v_0 = 0$. We first linearise these about the steady states $(u_0, v_0)$ in the absence of diffusion, (i.e. when $\nab^2 u = \nab^2 v = 0$). Write,
\begin{align*}
  u &= u_0 + \hat u \\
  v &= v_0 + \hat v.
\end{align*}
Substituting into \ref{equ:RDNF}, taking a Taylor expansion and considering lowest order terms, we obtain,
\begin{align*}
  \pd {\hat u} t &= \g (f_u\hat u + f_v\hat v)\\
  \pd {\hat v} t &= \g (g_u\hat u + g_v\hat v).
\end{align*}
In vector form, $\vec w = (\hat u \quad \hat v)^T$, the equations become,
$$ \pd {\vec w} t = L\vec w \qquad L = \g \left.\begin{pmatrix}
  f_u & f_v \\ g_u & g_v
\end{pmatrix}\right|_{(u_0, v_0)}. $$
Recall that we want the diffusion-free problem to be stable. We want the eigenvalues of $L$ to have negative real part. This is guarenteed if $\tr L < 0$ and $\det L > 0$. Note,
\begin{align*}
  \Tr(L) &= \g(f_u + g_v)\\
  \det L &= \g (f_u g_v - f_v g_u)
\end{align*}
Thus we obtain stability if,
\begin{align}
  f_u + g_v < 0 \tag{I}\\
  f_ug_v - f_vg_u > 0\tag{II}.
\end{align}
Let us consider the full equations with diffusion,
\begin{align*}
  \pd u t &= \g f(u, v) + \nab^2 u \label{cond:linStab1}\\
  \pd v t &= \gamma g(u, v) + \nab^2 v \label{cond:linStab2}
\end{align*}
We introduce again $u = u_0 + \hat u$ and $v = v_0 + \hat v$ and we then obtain the linearisation,
\begin{align*}
  \pd {\hat u} t = \g(f_u \hat u + f_v \hat v) + \nab^2 \hat u \\
  \pd {\hat u} t = \g(g_u \hat u + g_v \hat v) + d\nab^2 \hat v
\end{align*}
In vector form, $\vec w = (\hat u \quad \hat v)^T$, obtaining,
\begin{equation}
  \pd {\vec w} t = L\vec w + D\nab^2 \vec w \qquad D = \begin{pmatrix}
    1 & 0 \\ 0 & d
  \end{pmatrix}\label{equ:linTM}
\end{equation}
To solve \ref{equ:linTM} we are going to separate space and time. First, let $\vec w_k(\vec x) = (\hat u_k(\vec x) \quad \hat v_k(\vec x))^T$ be the solution to the spatial eigenvalue problem,
$$ \nab^2 \vec w_k = -k^2 \vec w_k. $$
Subject to the following boundary conditions $\vec n \cdot \nab \hat u_k = \vec n \cdot \hat w_k = 0$.

\begin{eg}
  Consider the 1D domain $\O = \{0 \le x \le a\}$, then $\partial \O = \{0\} \cup \{a\}$. The eigenvalue problem is then,
  $$ \pdd {\vec w_k} x = -k^2 \vec w_k$$
  which we can write as the shorthand for $w_k = \hat u_k$ or $\hat v_k$. The boundary conditions are,
  $$ \pd {w_k} x = 0 \text{ at } x = o \text{ and } x= a. $$
  The general solution is,
  $$ w_k(x) = A_k\cos (kx) + B_k\sin (kx) \qquad A_k, B_k \in \R. $$
  Imposing the boundary conditions, the solution are,
  $$ w_k(x) = A_k \cos \left(\frac{n\pi x}{a}\right) $$
  Thus we have shown,
  $$ \vec w_k(x) = \begin{pmatrix}
    A_{k_1} \\ A_{k_2}
  \end{pmatrix} \cos \left(\frac{n\pi x}{a}\right)$$
  where $A_{k_1}, A_{k_2} \in \R$.
\end{eg}

\noindent
In general we see a solution to (\ref{equ:linTM}) in the following form,
\begin{equation}
  \vec w(\vec x, t) = \sum_k c_k e^{\l_k t}\vec w_k(\vec x)\label{equ:wksol}
\end{equation}
The stability of the solutions in time will be determined by the spatial eigenvalues. If $\l_k < 0$ then we get the perturbation dying out, but if $\l_k > 0$ then we have a growing solution with the trig function in space with a periodic solution. Hence for instability we want some $\l_k > 0$. We are going to substitute this (\ref{equ:wksol}) back into (\ref{equ:linTM}),
$$ \sum_k c_k\l_k e^{\l_k t} \vec w_k = \sum_k c_ke^{\l_k t} (\g A - k^2 D)\vec w_k $$
where $A$ is just the Jacobean at $(u_0, v_0)$. But the $\vec w_k$'s are linearly independent, and so
$$ (\g A - k^2D)\vec w_k = \l_k \vec w_k $$
and so,
$$ (\l_kI - \g A + k^2D)\vec w_k = 0 $$
We are interested in the temporal eigenvalues $\l_k$. For non-trivial solutions, we require that,
$$ |\l_k I - \g A + k^2D| = 0 $$
This is equivalent to the following,
$$ \l_k^2 + (k^2(1 + d) - \g(f_u + g_v))\l_k + h(k^2) = 0 $$
where
$$h(k^2) = dk^4 - \g(df_u + g_v)k^2 + \g(f_ug_v - f_vg_u).$$
We want stable eigenvalues and so we will use (I) and (II) to find new conditions on this problem. From condition \ref{cond:linStab1}, $f_u + g_v < 0$. Hence, for $\Re (\l_k) > 0$, we must $h(k^2) < 0$. From condition \ref{cond:linStab2}, we have that $f_ug_v - f_vg_u > 0$, so for $h(k^2) < 0$, require
\begin{equation}
  df_u + g_v > 0 \tag{III}\label{cond:linStab3}
\end{equation}
Further we can write $h(k^2)$ as,
$$ h(k^2) = d\left[ (k^2 - \frac{\g}{2d}(f_u + g_v)) + \frac{\g^2}{d}\left( f_ug_v - f_vg_u \right) - \frac{\g^2}{4d^2} \left( df_u + g_v \right)^2 \right]. $$
The last two terms are independent of $k^2$, hence $h(k^2)$ is minimised when,
\begin{equation}
  k^2 = \frac{\g}{2d}\left( f_u + g_v \right)\label{equ:hkmin}
\end{equation}
and $h(k^2)$ has the following minimum value,
$$ h_{\text{min}} = \g^2 \left( f_ug_v - f_vg_u \right) - \frac{\g^2}{4d} \left( df_u + g_v \right)^2 $$
Then $h_{\text{min}} < 0$ and so $h(k^2) < 0$, so for instability we require,
\begin{equation}
  (df_u + g_v)^2 < 4d(f_ug_v - f_vg_u)\label{linStab4}\tag{IV}
\end{equation}

\noindent
To recap. We have show that conditions required for diffusion driven instability are,
\begin{align}
  f_u + g_v < 0 \tag{I} \\
  f_ug_v - f_vg_u > 0 \tag{II} \\
  df_u + g_v > 0 \tag{III} \\
  (df_u + g_v)^2 > 4d(f_ug_v - f_vg_u) \tag{IV}
\end{align}
The first two will guarantee that without diffusion it is stable and the last two guarantee it is unstable with diffusion. Note, each partial derivatives are evaluated at $(u_0, v_0)$. \\

\noindent
This is obviously a bifurcation. Let us thing about $d$ as a parameter. Then there is a critical value of $d$, $d_c$ at which instability occurs (When $h_{\text{min}}$). The derivation for condition \refeq{linStab4} tells us it is,
\begin{equation}
  (d_cf_u + g_v)^2 = 4d_c (f_ug_v - f_vg_u)\label{equ:kc}
\end{equation}
We can plot this as,

\begin{figure}[!ht]
\centering
\resizebox{0.6\textwidth}{!}{\input{./figures/TC4.pdf_tex}}
\caption{Different values of $d$ with respect to $d_c$.}
\end{figure}

\noindent
We consider the the perturbation from the steady states was,
$$ \vec w(\vec x, t) = \sum_k c_k e^{\l_k t} \vec w_k (\vec x). $$
This tells is that for all $k_1^2 k^2 < k_2^2$ we have $\Re(\l_k) > 0$. For $d < d_c$ we have all the $\Re(\l_k) < 0$ and so $\vec w(\vec x, t) \to 0$, but for $d > d_c$ we will have $\vec w_k$ grow to an inhomogeneous solution.\\

\noindent
The bifurcation we see is sometimes called the Turing bifurcation, we call the $k$'s the wave numbers. There is a critical wave number, $k_c$, the wave number we get at the bifurcation. This is the first node to lose stability. From \ref{equ:hkmin}, we can see this is,
$$ k_c ^2 = \frac{\g}{2d_c} (d_c f_u + g_v). $$
Then \ref{equ:kc} then gives us,
$$ k_c^2 = \g\sqrt{\frac{f_ug_v - f_vg_u}{d_c}} $$

\subsection{Turing Mechanism: Example}
Consider the following system in $\R$ ($u = u(x, t)$ and $v = v(x, t)$). Then the system is,
\begin{align*}
  u_t &= \frac{u^2}{v} - bu + u_{xx} \\
  v_t &= u^2 - v + du_{xx}
\end{align*}
where $b, k > 0$. The reaction terms in this case are the following,
$$ f(u,v) = \frac{u^2}{v} - bu \qquad g(u, v) = u^2 - v $$
This define the system as an activator-inhibitor system because, an increase of $u$ increases $v$, that is $u$ is having a positive increasing affect on $v$. Increasing $v$ decreases the amount of $u$ created. Hence we have an activator coupled with an inhibitor. We can take partial derivatives,
$$ f_u = \frac{2u}{v} - b \qquad f_v = - \frac{u^2}{v^2} $$
$$ g_u = 2u \qquad g_v = -1 $$
We are interested in spatially homogenous steady states $u = u_0$ and $v = v_0$ satisfy,
$$ f(u_0, v_0) = g(u_0, v_0) = 0 $$
That gives,
$$ \frac{u_0^2}{v^2} - bu_0 = 0 \qquad u_0^2 = v_0 $$
This is pretty easy to solve and we get,
$$ u_0 = \frac{1}{b} \qquad v_0 = \frac{1}{b^2}.$$
So we have the steady state of $\left( \frac{1}{b}, \frac{1}{b^2}\right)$. At the steady state,
$$ f_u = b \qquad g_v = -b^2 \qquad g_u = \frac{2}{b} \qquad g_v = -1  $$

\noindent
Stability in the presence of diffusion requires,
\begin{align*}
  f_u + g_v < 0 \implies b < 1 \tag{I}\\
  f_ug_v - f_vg_u > 0 \implies b > 0 \tag{II}
\end{align*}
Therefore, stability in the presence of diffusion happens when $0 < b< 1$. For diffusion-driven instability, requires,
\begin{align*}
  df_u + g_v > 0 \implies db > 1 \tag{III}\\
  (df_u + g_v)^2 > 4d(f_ug_v - f_vg_u) \implies (db - 1)^2 > 4db \tag{IV}
\end{align*}
To simplify the analysis, we let $x = db$. Then (IV) becomes,
$$ (x - 1)^2 > 4x $$
This can be manipulated to find that $(x - 3) > 8$ and further $3 + 2\sqrt 2 < x < 3 - 2\sqrt{2}$. We let $x = db$ and (III) said that $db > 1$, we see that the later half of the above inequality violated that. This $db > 3 + 2\sqrt 2$. So for diffusion-driven instability we require,
$$ 0 < b < 1 \quad \text{ and } d > \frac{3 + 2\sqrt 2}{b}. $$
In the $(b, d)$ parameter plane,

\begin{figure}[!ht]
\centering
\resizebox{0.6\textwidth}{!}{\input{./figures/TMex.pdf_tex}}
\caption{Parameter variation for activator-inhibitor model.}
\end{figure}

The shaded value shows where diffusion-driven instability can occur. If we think of $d$ as a bifurcation parameter. For each fixed $b$, f we move $d$ through the curved dotted line a Turing bifurcation will occur. That is, as $d$ increases through,
\begin{equation}
  d_c = \frac{3 + 2\sqrt{2}}{b},\tag{$*$}\label{equ:dcex}
\end{equation}
$(u_0, v_0)$ loses stability in a Turing bifurcation. Now,
$$ k_c^2 = \g \sqrt{\frac{f_ug_v - f_vg_u}{d_c}} = \sqrt{\frac{b}{d_c}}. $$
So from \refeq{equ:dcex},
$$ k_c = \sqrt{\frac{3 + 2\sqrt 2}{d_c^2}} = \frac{\sqrt{3 + 2\sqrt 2}}{d_c} = \frac{1 + \sqrt 2}{d_c}. $$

\subsection{Biological Pattern Formation (Non-examinable)}
We have been examining,
\begin{align*}
  u_t = \g f(u, v) + \nab^2 u \\
  v_t &= \g g(u, v) + d\nab^2 v
\end{align*}
involving two reactants and two zero-flux conditions,
$$ \pd u n = \vec n \cdot \nab u = 0 \quad \pd v n = \vec n \cdot \nab v = 0 $$
and then the linearisation,
$$ \vec w_t = L\vec w + D\nab^2 \vec w. $$
This has solutions,
$$ \vec w(\vec x, t) = \sum_k c_ke^{\l_k t}\vec w_k(\vec x) $$
and we have the usual $I - IV$ conditions. So let us now consider $\O = \{0 < x< p, 0 < y < q\}$. The spatial eigenvalue problem is,
$$ \nab^2 \vec w = -k^2 \vec w $$
with boundary conditions,
$$ \pd {\vec w} x = 0 \text{ at } x = 0, p \qquad \pd {\vec w} y = 0 \text{ at } y = 0, q $$
Thus,
$$ \pdd w x + \pdd w y = -k^2 w $$
with,
$$ \pd w x = 0 \text{ at } x = 0, p \qquad \pd w y = 0 \text{ at } y = 0, q, $$
where $w = \hat u, \hat v$. We seek separable solutions $w(x, y) = X(x)Y(y)$. These satisfy,
$$ X''Y + XY'' + k^2XY = 0 $$
that is,
$$ -\frac{X''}{X} = k^2 + \frac{Y''}{Y} = \a^2, $$
for some $\a \in \R$. This then yields,
$$ X(x) = A\cos (\a x) + B\sin (\a x). $$
Now, $\pd w x = X'Y$ and $\pd w y = XY'$. The boundary conditions say,
$$ X' = 0 \text{ at } x = 0, p \qquad Y' = 0 \text{ at } y = 0, q. $$
Imposing boundary conditions,
$$ \a = \frac{m\pi }{p} \qquad \b = \frac{n\pi}{q} \qquad m,n \in \Z $$
This,
$$ w = XY = A\cos\left( \frac{m\pi x}{p}\right)\cos\left( \frac{n\pi y}{q}\right) $$
where,
$$ k^2 = a^2 + b^2 = \pi^2 \left( \frac{m^2}{p^2} + \frac{n^2}{q^2} \right). $$
Our solution is then of the form,
$$ w(x, y, t) = \sum_{m, n} c_{mn}e^{\l_{mn} t} \cos\left( \frac{m\pi x}{p}\right)\cos\left( \frac{n\pi y}{q}\right) $$
We now recall the stability discussion. Any $m, n$ that satisfy our stability condition will grow in time, these develop into different states that are dependent on the eigenfunctions.\\

\noindent
Suppose the only unstable wave numbers are $m = 1$ $n = 0$. The only unstable mode is then,
$$ w(x, y, t) = Ae^{\l t} \cos \left( \frac{\pi x}{p} \right) \quad \text{ for } \Re \l > 0$$
Thus,
$$ u(x, y, t) = u_0 + \e e^{\l t}\cos \left( \frac{\pi x}{p} \right) $$
where we have a steady state and perturbation terms. Similarly for $v$. Typically a patter will be laid down where the morphogen $u$ is about it's steady state. So we get some halved pattern,

\begin{figure}[!ht]
\centering
\resizebox{0.4\textwidth}{!}{\input{./figures/Pattern1.pdf_tex}}
\caption{Pattern Formation from $m = 1$, $n = 0$ unstable.}
\end{figure}

\noindent
We can generalise this. If we have some pairing $(m, n)$ corresponds to an unstable solution then,
$$ w(x, y, t) = Ae^{\l t} \cos \left( \frac{m\pi x}{p} \right)\cos \left( \frac{n\pi y}{q} \right) \quad \text{ for } \Re \l > 0$$
Thus,
$$ u(x, y, t) = u_0 + \e e^{\l t}\cos \left( \frac{m\pi x}{p} \right)\cos \left( \frac{n\pi y}{q} \right) $$
and similarly for $v$. We therefore expect $u > u_0$ in an $(m + 1) \ti (n + 1)$ chessboard pattern. We get spots.\\

\noindent
More complex geometries and reaction kinetics lead to more complex patterns. Murray has a detailed discussion. We know that the wave numbers satisfy of our problem (a Helmhotlz Problem) satisfy,
$$ k^2 = \pi \left( \frac{m^2}{p^2} + \frac{n^2}{q^2} \right). $$
Typically, the chemistry, rather than the geometry will force unstable wave numbers $k$ to satisfy the condition $k_1^2 < k^2 < k_2^2$. \\

\noindent
For a tail, length $>>>$ width and so $p >>> q$. Then we might allow only $n = 0$ and $m = M >>> 1$ say, giving a striped tail and if $n = 2$ we might get a spotted tail. 