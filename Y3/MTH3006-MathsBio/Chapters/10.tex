% !TEX root = ../notes.tex

\section{PPMs}
We have $x(t+1) = Ax(t)$, we can look at several things, firstly eigenvalues. So we calculate $\l_1, \dots, \l_n$ of $A$. We could be lucky to have $w_1, \dots, w_n$ linearly independent eigenvectors. We can consider the right and the left eigenvalues. Now if we have some $x(0)$, we can write it as a linear combination of the linearly independent eigenvectors. $x(0) = \a_1w_1 + \dots + \a_nw_n$ and so we can write our solution,
\begin{align*}
  x(t) &= A^tx(0) \\
  &= A^t \sum \a_iw_i \\
  &= \sum \a_i\l_i^t w_i
\end{align*}
Now we have a theorem,
\begin{nthm}[]
  The maximum has a positive eigenvalue that has the greatest modulus.
\end{nthm}
Hence for large time we consider $\l_1 > |\l_i|$. Hence we can say $\l_1^{-t} x(t) \sim \a_1w_1$ if $t > 0$. We can say more than this, $\l^{-t}x(t) \sim \a_1w_1$, but we know $x(0) = \sum \a_iw_i$ and so we multiply by $v_1^\top x(0) = \a_1v_1^\top w_1$ and so,
$$ \a_1 = \frac{v_1^\top x(0)}{v_1^\top w_1} $$
and so,
$$ x(t) \sim \frac{v_1^\top x(0)}{v_1^\top w_1}w_1 $$
This is just for the long term, the question remains of what happens when we have small time. We go back to $x(t) = A^tx(0)$, we firstly focus on the sum of the components $\vec{1} x(t) = \norm{x(t)}_1$. Then $N(t) = \vec{1} A^t x(0)$. Hence we write $N(t) = \begin{pmatrix}
  c_1 & c_2 & \dots & c_n
\end{pmatrix} x(0)$ and now assume that $N(0) = \norm{x(0)}_1 = 1$. We now ask how big this can really be. We can write that,
$$ N(t) = \begin{pmatrix}
  c_1 & \dots & c_n
\end{pmatrix} \begin{pmatrix}
  x_1(0) \\ \vdots \\ x_n(0)
\end{pmatrix}$$
and we know $\sum x(0)^i$ and so $1 \le N(t) \le \bar c$. Where $\bar c$ is the maximum column sum. It may also be written as $a_t \le N(t) \le \rho(t)$. This isn't that good, so back to our equation

\subsection{Bound 2}
Now we multiply on the left by $v^\top$ and get, (wlog when we write $v$ and $\l$ we mean $v_1$ and $\l_1$, the dominant eigenvector. )
\begin{align*}
  v^\top x(t+1) &= v^\top Ax(t) \\
  &= \l v^\top x(t) \\
  &= \sum v_i x_i(t) \\
\end{align*}
and so we can write,
$$ v_{\text{min}}N(t) \le v^\top x(t) \le v_{\text{max}}N(t) $$
and so we can form another bound,
$$ v_{\text{min}}N(t+1) \le \l v^\top_{\max}N(t) $$
and the other way around and so,
$$\frac{v_{\text{min}}}{v_{\text{max}}}\l \le N(t) \le \frac{v_{\text{max}}}{v_{\text{min}}}\l $$

\subsection{Bound 3}
We now look at $x(t) = A^tx(0)$ and we write this as
\begin{align*}
  N(t+1) &= \mathbf{1} Ax(t)\\
  &= \begin{pmatrix}
    c_1(A) & \dots & c_n(A)
\end{pmatrix} x(t) \\
\end{align*}
Now we will get $\underline c(A) N(t) \le N(t+1) \le \bar c(A) N(t)$ and now we iterate and get that $\underline c(A)^t N(0) \le N(t) \le (\bar c(A))^tN(0)$.

Then we can plot these and form a idea for what it looks like. These are just going to be bounds of $a_t$ and $\rho_t$.
