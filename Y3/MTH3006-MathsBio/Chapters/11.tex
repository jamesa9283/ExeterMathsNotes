% !TEX root = ../notes.tex

\section{Spatially distributed reaction-diffusion kinetics}

\subsection{Background}
Consider the concentration $c(\vec x, t)$. This is the concentration of some chemical species at $\vec x \in \R^3$ at time $t$.\\
\begin{figure}[!ht]
\centering
\resizebox{0.3\textwidth}{!}{\input{./figures/spatialRK.pdf_tex}}
\caption{Model setup}
\end{figure}

\noindent
Let $f = f(c, \vec x, t)$ be the source of the chemical in $V$. Introduce the flux vector $\vec J = J_1\vec i + J_2\vec j + J_3\vec k$. We define this as, $\vec J \cdot d\vec S = \vec J \cdot \vec n dS$. This is the rate of movement of chemical across $dS$ per unit time. We can write the following conservation equation,
$$ \pd{}{t} \iiint_V c(\vec x, t) dV = \iiint_V fdV - \iint_S \vec J \cdot d\vec S $$
We recall the divergence theorem,
$$ \iint_S \vec J \cdot d\vec S = \iiint_V \nab \cdot \vec J dV $$
Thus,
$$ \iiint_V \pd{}{t} c(\vec x, t) dV = \iiint_V f - \nab \cdot \vec J dV $$
This holds for an arbitrary volume $V$. We can equate the integrands,
\begin{equation}
  \pd {c(\vec x, t)}{t} = -\nab \cdot \vec J + f \label{equ:CoMSK}
\end{equation}
If transport is via diffusion, then,
$$ \vec J = -D\nab c $$
where $D$ is the diffusion coefficient. In general $D = D(\vec x, t)$. This is referred to as Fick's law. In $1D$, then $J = -D\pd c x$. We now substitute into \ref{equ:CoMSK}. We obtain,
\begin{equation}
  \pd c t = f + \nab \cdot (D \nab c).\label{equ:RD}
\end{equation}
This is the reaction-diffusion equation. We now go further and assume that $D(\vec x, t) = D$ is constant. Then \refeq{equ:RD} simplifies to,
\begin{equation}
  \pd c t = f + D\nab \cdot \nab c = f + D\nab^2 c\label{equ:simpRD}
\end{equation}
where $\nab^2$ is just the Laplacian. If $f = 0$, then we obtain the diffusion equation in three dimensions. We consider $m$ reactants, with concentration,
$$ u_1(\vec x, t), \dots, u_m(\vec x, t) $$
with diffusion constants, $ D_1, \dots, D_m$ and source terms $f_1(u_1, \dots, u_m), \dots, f_m(u_1, \dots, u_m)$. We then write the equations as,
$$ \pd {u_i} t = f_i(u_1, \dots, u_m) + D_i \nab^2 u_i \qquad 1 \le i \le m. $$
We can write this in vector form,
$$ \pd {\vec u} t = \vec f(\vec u) + \vec D\nab^2 \vec u $$
where,
$$ \vec u = (u_1, \dots, u_m)^T \qquad \vec f = (f_1, \dots, f_m)^T $$
and,
$$ D = \begin{pmatrix}
  D_1 & & 0 \\
  & \ddots & \\
  0 & & D_m
\end{pmatrix} $$
First, we consider one reactant, $m = 1$ and $x \in \R$. This gives arise to,
$$ \pd u t = f(u) + D\pdd u x. $$
\subsection{Fisher Equation}

We will consider the Fisher equation. We consider $u(\vec x, t)$ as the concentration of the gene in a population at spatial location $\vec x$ at time $t$,
$$ \pd u t = ku(1 - u) + D\pdd u x. $$
We assume that $k, D > 0$. We can non-dimensionalise this set of equations with $\t = kt$, $\xi = \sqrt{\frac{k}{D}}$ to give,
\begin{equation}
  \pd u \t = u(1 - u) + \pdd u \xi\label{equ:fisher}
\end{equation}
To analyse systems of this kind we consider the spatially homogenous system, we ignore $\pdd u \xi$. It has two steady states, where $\pd u \t = 0$. That is $u = 0$ and $u = 1$,\newpage

\begin{figure}[!ht]
\centering
\resizebox{0.4\textwidth}{!}{\input{./figures/SPFM.pdf_tex}}
\caption{Stability of stable states}
\end{figure}

\noindent
We know that $0$ is unstable and $1$ is stable, that is any profile between $0$ and $1$ it will asymptotically go to $1$. This isn't very interesting. We now consider the system with diffusion. We seek a solution of \refeq{equ:fisher}, as,
$$ u(\xi, \t) = U(\xi - c\t) \qquad c > 0 $$

\begin{figure}[!ht]
\centering
\resizebox{0.6\textwidth}{!}{\input{./figures/FMwaves.pdf_tex}}
\caption{Characteristics of solutions of \refeq{equ:fisher}}
\end{figure}

\noindent
We introduce the travelling wave coordinate $z = \xi - c\tau$, which now means, $u(\xi, \tau) = U(\xi - c\tau) = U(z)$. Now we can transform this into a system in terms of $z$. We see by the chain rule,
\begin{align*}
  \pd u \tau &= -c \di U z\\
  \pdd u \xi &= \dii U z
\end{align*}
Then substituting this into the PDE (\refeq{equ:fisher}) we get,
$$ \dii U z + c\di U z + u(1 - u) = 0 $$
We seek solutions with respect to the boundary conditions,
$$ U(-\infty) = 1 \quad \text{ and } U(\infty) = 0 $$
These equate to a solution of the form,\newpage

\begin{figure}[!ht]
\centering
\resizebox{0.6\textwidth}{!}{\input{./figures/fisherBC.pdf_tex}}
\caption{BC of the solution.}
\end{figure}

\noindent
We now let $v = u'$, which allows us to now present this as a system of two ODEs,
\begin{equation}
  \begin{aligned}
    v &= u' = f(u, v)\\
    v' &= -cv - u(1 - u) = g(u, v)
  \end{aligned}
\end{equation}
\textbf{Steady states and stabilities} \\
The steady states are where $u' = v' = 0$, that then implies for $v = 0$ we have $u(1 - u) = 0$ and so $u = 0$ or $u = 1$. Thus we have two steady states, $(0, 0)$ and $(1, 0)$. To find the stabilities we consider the jacobian at the two steady states,
$$ \mathcal{J}(u, v) = \begin{pmatrix}
  \pd f u & \pd f v \\
  \pd g u & \pd g v
\end{pmatrix} = \begin{pmatrix}
  0 & 1 \\ -1 + 2u & -c
\end{pmatrix} $$
We now consider $\mathcal{J}(0, 0)$. We see,
$$ \mathcal{J}(0, 0) = \begin{pmatrix}
  0 & 1 \\ -1 & c
\end{pmatrix} $$
Then we get the eigenvalues are,
$$ \l_{\pm} = \frac{1}{2}\left( -c \pm \sqrt{c^2 - 4} \right) $$
Now we consider two cases. If $c^2 \ge 4$, then $\l_{-} \le \l_+ < 0$. This is a stable node. If $c^2 < 4$, then $\l_\pm \in \C$ and $\Re(\l_\pm) = -c \le 0$ and so we have a stable focus.

\begin{figure}[!ht]
\centering
\resizebox{0.3\textwidth}{!}{\input{./figures/fisherO.pdf_tex}}
\caption{Stability of the origin}
\end{figure}

\noindent
We then must have $c^2 \ge 4$ and so $c \ge 2$. This is because $u < 0$ is biologically infeasible. We now have a bound on $c$. We now consider the other stable state.
$$ \mathcal{J}(1, 0) =  \begin{pmatrix}
  0 & 1 \\ 1 & -c
\end{pmatrix}$$
where we have eigenvalues of $\l_\pm = \frac{1}{2}\left( -c \pm \sqrt{c^2 - 4} \right)$. As $\sqrt{c^2 + 4} > c$, we have $\l_{-} < 0 < \l_+$ and so we have a saddle node. We can now draw a phase portrait of this system,

\begin{figure}[!ht]
\centering
\resizebox{0.5\textwidth}{!}{\input{./figures/fisherSS.pdf_tex}}
\caption{Phase Portrait for the system}
\end{figure}

\noindent
We see the red trajectory is the only trajectory with the property we are interested in for the boundary conditions. We can now infer the form of $U(z)$.

\begin{figure}[!ht]
\centering
\resizebox{0.6\textwidth}{!}{\input{./figures/fisherSol.pdf_tex}}
\caption{Phase Portrait for the system}
\end{figure}