% !TEX root = ../notes.tex

\section{Week 2 - stuff}

test2
