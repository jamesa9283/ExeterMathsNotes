% !TEX root = ../notes.tex

\section{Linear Input-State-Output Systems}
We will not consider systems of the form,\marginnote{\emph{Lecture 13}}[0mm]
\begin{align*}
  \di x t &= Ax + Bu\\
  y &= Cx + Du
\end{align*}
We can solve the DE by using an integrating factor,
\begin{align*}
  \di x t &= Ae^{-At} + Bue^{-At}\\
  \di x t e^{-At} - Ae^{-At} &= Bue^{-At}\\
  \di{}{t} (xe^{-At}) &= Bue^{-At}\\
  x(t)e^{-At} - x(0) &= \int_0^t Bue^{a\t}\,d\t
\end{align*}

%part ii
\subsection{Input-State Stability}
\begin{ndefi}[Input-State Stability]
  Consider the system,
  $$ \di x t = Ax + Bu \qquad y = Cx + Du $$
  We call this stable if,
  \begin{enumerate}
    \item if $u= 0$ for all $t \ge 0$ then $x(t) \to 0$ as $t \to \infty$
    \item If $\sup_{t \ge 0} \max |u_i (t)| < \infty$ then $sup_{t \ge 0} \max |x_i(t)| < \infty $
  \end{enumerate}
\end{ndefi}

\noindent
If a system is input-state stable, then it is input -output stable. We aim to show that a system is input-state stable if and only if all the eigenvalues of $A$ is in OLHP.\\
% Lect 14
\begin{proof}
  We first prove that condition (1) and forward holds and then we seek to prove (2) and backwards holds.\\

  We recall,
  $$ x(t) = e^{At}x(0) + \int_0^t e^{A(t - \t)}Bu(\t)\,d\t $$
  We then consider each entry in $(sI - A)^{-1}$ and the obtain a partial fraction decomposition. Then we take inverse laplace transforms and the result follows.\\

  For the second part, we take our decomposition and then consider the inverse laplace transform. Then proceed by contradiction, assume we have a eigenvalue in the ORHP, then by our inverse laplace, if follows that all matrices corresponding to our ORHP eigenvalue is zero, then we find that the ORHP eigenvalue has a non-zero eigenvector which can be written using our $(sI - A)^{-1}$ equation and so implies it's zero, but this causes a contradiction.\\

  Now we prove a bounded input leads to a bounded state, we use variation of constants, then we proceed very similarly to input-output systems.
\end{proof}


We now consider choosing $u$ to shape some sort of desired behaviour. We now consider controllability,

\subsection{Controllability}

\begin{ndefi}[Controllability]
  We call a system controllable if for $z_0,\,z_1 \in \R^d$ there exists some $t_1 > 0$ and an input $u$ which steeps the system from the initial state $x(0) = z_0$ to the final state $x(t_1) = z_1$
\end{ndefi}

\begin{nthm}
  We will show the following are equivilent,
  \begin{enumerate}
    \item The system is controllable
    \item For any $T > 0$, $\int_0^T e^{-A\t}BB^T e^{-A^T\t}\,d\t$ is non-singular.
    \item If $z \in \R^d$ satisfies $z^T [B \quad AB \quad \dots \quad A^{d-1}B]$ then $z = 0$
  \end{enumerate}
\end{nthm}

The third is easiest to use, the second is more complicated, however, it does give a neat expression for $u$ from $x(0) = z_0$ to $x(t_1) = z_1$ while minimising the integral. Finally the third leads to a linear control law $u = Kx$ where $K \in M_{m \times n}(\R)$.\\

\noindent
We will show that if $(ii)$ does not hold, then $(iii)$ does not hold, if $(iii)$ does not hold, then $(i)$ does not hold and if $(ii)$ holds then so does $(i)$.\\

\begin{proof}[Proof of Theorem 5.3]
  Case $(i)$, Firstly, suppose there exists a $T > 0$ and $0 \ne z \in \R^d$ such that $z^T\int_0^T e^{-A\t}BB^T e^{-A^T\t}\,d\t = 0$. This implies that $z^T e^{-A\t}BB^T e^{-A^T\t} = 0$ for all $0 \le \t \le T$. If we define $w(t) = z^T e^{-At}B = 0$, hence we see $w(0) = 0$ and all the derivatives are zero, so then it follows that $z^TB = 0$ and $z^TAB = 0$ and so on.
  It hence follows that $z^T [B \quad AB \quad \dots \quad A^{d-1}B] = 0$.\\

  \noindent
  The case (ii) requires Cayley Hamilton Theorem. It follows from the theorem that $z^T A^jB = 0$ for all $j \in \Z$ and moreover and considering the powerseries of matrix exponetial $z^Te^{At}B = 0$ and so the result follows from variation of constants formula. Thus the system is not controllable.\\

  \noindent
  The case (iii), follows from definiting the integral as $W(T)$ and if it's nonsingular, then we can define an input $u(t)$ using $W^{-1}(T)$. Then from variations of constant the result follows. Moreover we can show that the input minimises $u(t)^Tu(t)$.
\end{proof}

\subsection{State Space Isomorphism}
Consider the system,
$$ \di x t = Ax + Bu \qquad y = Cx + Du $$
let $T \in R^{d \ti d}$ be non-singular, and let
$$ \hat A = TAT^{-1},\, \hat B = TB,\, \text{ and } \hat C = CT^{-1} $$
Then, defining $z = Tx$, we find that
$$ \di x t = \hat Az + \hat Bu,\, y = \hat Cz + Du $$
Thus the trajectories are in an one-to-one correspondence with the first. \\

Consider a controllable system, we have our $V$ non-singular and there is a $w^TV = [0 \qquad \dots \qquad 1]$ and construct $w^TA^{k-1}$ and then,
$$ TAT^{-1} = \begin{bmatrix}
  0 & 1 & 0 & \dots & 0\\
  0 & 0 & 1 & \dots & 0\\
  \vdots & \vdots &\vdots && \vdots\\
  -a_0 & -a_1 & -a_2 & \dots & -a_{d-1}
\end{bmatrix} \qquad TB = \begin{bmatrix}
  0 \\ 0 \\ \vdots \\ 0 \\ 1
\end{bmatrix}$$
