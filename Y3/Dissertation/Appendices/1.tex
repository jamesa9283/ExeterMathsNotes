% !TEX root = ../notes.tex

\section{Automorphisms}

This is an appendices for Chapter II of the thesis, where we mention briefly Inner Automorphisms. Here I would like to define what they are and give some general background. We have seen before what an isomorphism is, here is the defintion of an Automorphism,
\begin{ndefi}[Automorphism]
  Let $X$ be some mathematical object or structure, then an Automorphism is a bijection $f : X \to X$
\end{ndefi}
Here are a few examples of what automorphisms are,
\begin{enumerate}
  \item Consider $(\Z,\,+)$ as a group, then negation is an automorphism. Moreover, if we consider $(\Z, +, \ti)$ as a group, then that automorphism is the only automorphism.
  \item Of any abelian group there will always be the automorphism of negation.
\end{enumerate}

As we saw in the main text, the inner automorphism is just the action of conjugation on the group itself. A further interesting fact is that the group of inner automorphisms form a normal subgroup of $\Aut(G)$ that we denote $\Inn(G)$.
\begin{proof}
  by Goursat's Lemma\footnote{This lemma leads to my favourite lemma in the whole of mathematics, Snake Lemma.}
\end{proof}
