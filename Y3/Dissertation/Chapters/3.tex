% !TEX root = ../notes.tex

\section{Integrating Mechanics and Geometry}
In this final pure mathematics chapter we will use the Geometric ideas we met in Chapter 1 and Chapter 2 and integrate them to start to be able to discuss systems of equations. We will first talk about different coordinate systems and rotations that connect them. This section introduces specific Lagrangians that we will use several times in the next chapter. In the next section we present an introduction to calculus of variations. We will focus on the first variation and how that relates to Hamilton's variational principle, which we will use to derive our first equation for the Lagrangian in section one. Finally, we will formally introduce the hat map, we will prove it's a Lie algebra isomorphism and explain some of the theory surrounding it. This will then all be used in the next chapter which is the main topic of the thesis, Euler-Poincar\'e reduction.

\subsection{Rotation}
Consider a spatial coordinate system with its origin at the centre of mass of a given rigid body. We denote it by, $\vec{x}(t) \in \R^3$, where $\vec x = X$.
\begin{figure}[!ht]
\centering
\resizebox{0.2\textwidth}{!}{\input{./figures/RotationCoord.pdf_tex}}
\caption{Spatial and Body Coordinates}
\end{figure}
% draw two coordinate axis (x, y, z) and (X, Y, Z)
We need a way to rotate the rigid body without constraints, so we denote a rotation tensor $R(t)$ and say $\vec x(t) = R(t)\vec X$ where $\vec X$ is in the body coordinate system. The configuration of the body particle at time $t$ is given by a rotation matrix that takes the label $\vec X$ to current position $\vec x(t)$ where $R \in \SO(3)$ is a proper rotation matrix. This means,
$$ R^T = R^{-1} \qquad \det R = 1. $$
The map $\vec X \to R(t)\vec X$ is called the body-to-space map. We can now talk about kinetic energy,
$$ K = \frac{1}{2}\int_\mathcal{B} \rho \norm{\vec{\dot x}}^2\, d^3\vec X$$
which we can change to,
\begin{align*}
  \frac{1}{2}\int_\mathcal{B} \rho \norm{\vec {\dot x}}^2\, d^3\vec X &= \frac{1}{2}\int_\mathcal{B} \rho(\vec X)\norm{\dot R(t)\vec X}^2 \,d^3\vec X\\
  &= \frac{1}{2}\int_\mathcal{B} \rho(\vec X)\dot R(t)\vec X \cdot \dot R(t)\vec X \,d^3\vec X.
\end{align*}
Now we can say if $V = 0$. Hence, $L = K$ and so,
$$ \di{}{t}\pd{K}{\dot R} - \pd{K}{R} = \vec 0. $$
This is difficult to deal with. We shall now derive kinetic energy that is more useful to Euler-Poincar\'e reduction. \\

\noindent
We know that $R^{-1}= R^T$ and so $RR^T = RR^{-1} = I$. If we have $\vec v$, $\vec w \in \R^3$, then $\vec v \cdot\vec w = R \vec v \cdot R\vec w$. Hence, consider $\norm{\dot R \vec X}^2$ and we know,
\begin{align*}
  \norm{\dot R \vec X}^2 &= \dot R \vec X \cdot \dot R \vec X\\
  &= R^{-1}(\dot R \vec X) \cdot R^{-1}(\dot R \vec X)\\
  &= \norm{R^{-1}\dot R \vec X}^2
\end{align*}
and so,
$$ K = \frac{1}{2} \int_\mathcal{B} \rho(\vec X) \norm {R^{-1}\dot R \vec X}^2 d^3\vec X.$$
It can be seen that $K = K(R, \dot R) = K(R^{-1}R, R^{-1}\dot R)$. This is called left symmetry. Hence, we can reduce this to $K(e, R^{-1}\dot R)$ and change notation. Let $K(e, R^{-1}\dot R) := \kappa(R^{-1}\dot R)$, where $R^{-1}\dot R$ is the angular velocity of the body. We can see this from the body and from an observation outside the system. Hence, we call $R^{-1}\dot R = \Oh$.\\
Interestingly, we know $RR^T = RR^{-1} = I$. Therefore,
$$ \di {}{t} I = \di {}{t} (RR^{-1}) = \dot R R^{-1} + R\di{}{t} R^{-1} = \vec 0 $$
and we can also write this as,
\begin{align*}
  I &= RR^T\\
  \vec 0 &= \di {}{t} (RR^T)\\
  \vec 0 &= \dot R^T R + R^T\dot R \\
  R^T\dot R &= - (R^T\dot R)^T
\end{align*}
and so $R^{-1}\dot R = - (R^{-1}\dot R)^T$ that is, $\Oh = - \Oh^T$. This is the antisymmetric property we have noted about this vector. Now we go back to kinetic energy to nicely write it in terms of $\Oh$,
$$ K = \frac{1}{2} \int_\mathcal{B} \rho(\vec X) \norm{\Oh \vec X}^2 d^3\vec X. $$
We can now prove that $\Oh \vec X = \vec \O \times \vec X$
where,
$$ \Oh = \begin{bmatrix}
  0 & -\O_3 & \O_2\\
  \O_3 & 0 & -\O_1\\
  -\O_2 & \O_1 & 0 \\
\end{bmatrix} \qquad \vec\O = \begin{bmatrix}
  \O_1\\ \O_2\\\O_3
\end{bmatrix}. $$
This can be proven by just multiplying out the matrix,
\begin{align*}
  \begin{bmatrix}
    0 & -\O_3 & \O_2\\
    \O_3 & 0 & -\O_1\\
    -\O_2 & \O_1 & 0 \\
  \end{bmatrix}\begin{bmatrix}
    X_1\\ X_2\\ X_3
  \end{bmatrix} = \begin{bmatrix}
    \O_2X_3 - X_2\O_3\\ \O_3X_1 - \O_1X_3\\ X_2\O_1 - X_1\O_2
  \end{bmatrix} = \vec X \ti \Ov
\end{align*}
where $\Ov$ is the axel vector. We now write again,
\begin{align*}
  K &= \frac{1}{2} \int_\mathcal{B} {\rho (\vec X)\norm{\vec\O \times \vec X}^2 \,d^3\vec X}\\
  &= \frac{1}{2} \int_\mathcal{B} {\rho (\vec X)(\vec\O\times\vec X) \cdot (\vec\O\times\vec X) \,d^3\vec X}\\
  &= \frac{1}{2}\int_\mathcal{B} {\rho (\vec X)\left(\norm{\vec \O^2}\norm{\vec X^2} - (\vec\O \cdot \vec X)^2\right) \,d^3\vec X}\\
  &= \frac{1}{2}\int_\mathcal{B} {\rho (\vec X)\left(\vec\O^T\vec\O\norm{\vec X^2} - \vec\O^T\vec X \vec X^T\vec\O\right) \,d^3\vec X}\\
  &= \frac{1}{2}\int_\mathcal{B} {\rho (\vec X)\vec\O^T\left(\norm{\vec X^2}I - \vec X \vec X^T\right)\vec\O\, d^3\vec X}\\
  &= \frac{1}{2}\int_\mathcal{B} {\rho (\vec X)\vec\O^T\vec\O\left(\norm{\vec X^2}I - \vec X \vec X^T\right)\, d^3\vec X}\\
  &= \frac{1}{2}\vec\O \cdot \vec\O \int_\mathcal{B} {\rho (\vec X)\left(\norm{\vec X^2}I - \vec X \vec X^T\right)\, d^3\vec X}\\
  &= \frac{1}{2}\mathbb{I}\vec\O \cdot \vec \O
\end{align*}
where $\I$ is the moment of inertia tensor. This is defined as,
$$ \mathbb{I} = \int_\mathcal{B} \rho(\vec X)\left(\norm{\vec X}^2I - \vec X\vec X^T\right) d^3\vec X $$
where $\vec X\vec X^T = \vec X \otimes \vec X$ and,
$$ (\vec a \otimes \vec b)\vec c = (\vec b \cdot \vec c)\vec a $$
for all $\vec a, \vec b, \vec c \in \R^3$.
