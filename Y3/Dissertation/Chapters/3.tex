% !TEX root = ../notes.tex

Let $g(t) \in G$ such that $g(0) = e$ where $\eta = \dot g (0) \in T_eG$
$$ ad_\eta \xi := \di {}{t} _{t = 0} Ad_{g(t)}\xi \qquad \fa \xi \in \mathfrak{g} $$
qhich we can see to be
$$ \dot g(0)\xi g(0)^{-1} + g(0)\xi \di {}{t}_{t=0} g(t)^{-1}= \eta\xi - \xi\eta $$
Hence we can say that $\ad_{\eta}^\xi = [\eta, \xi] = \eta\xi - \xi\eta$.\\
Hence now we define the coadjoint action on $\mu$,
\begin{ndefi}[Adjoint / Coadjoint action on $\mathfrak{g}/\mathfrak{g^*}$]
  The adjoint action of the matrix lie algebra on itself is given by,
  $$ \ad : \mathfrak{g}\times \mathfrak{g} \to \mathfrak{g} $$
  $$ \ad_\eta \xi = [\eta,\, \xi] $$
  The dual map $\ip{ad^*_\eta \mu}{\xi} = \ip{\mu}{\ad_\eta \xi}$ is the coadjoint action of $\mathfrak{g}$ on $\mathfrak{g}^*$.
\end{ndefi}

\begin{exercise}
  Find $\ad^*_\eta \mu$.
\end{exercise}

\section{Rotation}
\subsection{Inertial Frame}
A spatial coordinate system with origin at the centre of mass of the given rigid body. We denote it by, $\vec{x}(t) \in \R^3$, where $\vec x = X$. Assume we have a spacial coordinate system,
% draw two coordinate axis (x, y, z) and (X, Y, Z)
We need a way to rotate things without constraints, so we denote a tensor $R(t)$ and say $\vec x(t) = R(t)\vec X$ where $\vec X$ is in the body coordinate system. The configuration of the body particle at time $t$ is given by a rotation matrix that takes the label $\vec X$ to current position $\vec x(t)$ where $R \in \SO(3)$ is a proper rotation matrix; this means,
$$ R^T = R^{-1} \qquad \det R = 1 $$
The map $\vec X \to R(t)\vec X$ is called the body-to-space map. \\

We can now talk about kinetic energy,
$$ K = \frac{1}{2}\int_\b \rho \norm{\vec x}^2\, d^3\vec X$$
which we can change to,
\begin{align*}
  \frac{1}{2}\int_\b \rho \norm{\vec x}^2\, d^3\vec X &= \frac{1}{2}\int_\b \rho(\vec X)\norm{\dot R(t)\vec X}^2 \,d^3\vec X\\
  &= \frac{1}{2}\int_\b \rho(\vec X)\dot R(t)\vec X \circ \dot R(t)\vec X \,d^3\vec X\\
\end{align*}
Now we can say if $V = 0$. Hence, $L = K$ and so,
$$ \di{}{t}\pd{K}{\dot R} - \pd{K}{R} = \vec 0 $$
This is difficult to deal with, so let's do something more cool! \\

We know that $R^{-1}= R^T$ and so $RR^T = RR^{-1} = I = e$. If we have $\vec v$, $\vec w \in \R^3$, then $\v \cdot\vec w = R \vec v \cdot R\vec w$. Hence,  consider $\norm{\dot R \vec X}^2$ and we know
\begin{align*}
  \norm{\dot R \vec X}^2 &= \dot R \vec X \cdot \dot R \vec X\\
  &= R^{-1}(\dot R \vec X) \cdot R^{-1}(\dot R \vec X)\\
  &= \norm{R^{-1}\dot R \vec X}
\end{align*}
and so,
$$ K = \frac{1}{2} \int_\b \rho(\vec X) \norm {R^{-1}\dot R \vec X} d^3\vec X$$
Then $K = K(R, \dot R) = K(R^{-1}R, R^{-1}\dot R)$ this is called left symmetry. Hence, we can reduce this to $K(e, R^{-1}\dot R)$ and change notation let $\kappa (R^{-1}\dot R)$ and $R^{-1}\dot R$ is angular velocity of the body. We can see this from the body and from an observation outside the system. Hence, we call this $R^{-1}\dot R = \Oh$.\\
Interestingly, we know $RR^T = RR^{-1} = I$. Hence,
$$ \di {}{t} I = \di {}{t} (RR^{-1}) = \dot R R^{-1} + R\di{}{t} R^{-1} = \vec 0 $$
and we can also write this as,
\begin{align*}
  I &= RR^T\\
  \vec 0 &= \di {}{t} (RR^T)\\
  \vec 0 &= \dot R^T R + R^T\dot R \\
  R^T\dot R &= - (R^T\dot R)^T
\end{align*}
and so $R^{-1}\dot R = - (R^{-1}\dot R)^T$ and so $\Oh = - \Oh^T$. This is the antisymmetric property we have noted about this vector.\\

Now we go back to kinetic energy to nicely write it as $\Oh$
$$ K = \frac{1}{2} \int_\b \rho(\vec X) \norm{\Oh \vec X}^2 d^3\vec X $$
and now we can prove that $\Oh \vec X = \vec \O \times \vec X$
where,
$$ \Oh = \begin{bmatrix}
  0 & -\O_3 & \O_2\\
  \O_3 & 0 & -\O_1\\
  -\O_2 & \O_1 & 0 \\
\end{bmatrix} $$
where $\vec\O = \begin{bmatrix}
  \O_1\\ \O_2\\\O_3
\end{bmatrix}$
where $\O$ is the axel vector.
and so,
\begin{align*}
  K &= \frac{1}{2} \int_\b {\rho (\vec X)\norm{\vec\O \times \vec X}^2 d^3\vec X}\\
  &= \frac{1}{2} \int_\b {\rho (\vec X)(\vec\O\times\vec X) \cdot (\vec\O\times\vec X) d^3\vec X}
\end{align*}
