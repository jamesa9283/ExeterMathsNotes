% !TEX root = ../notes.tex

Euler Poincare equtions,
$$ \di {}{t} \pd{\ell}{\vec\O} - \pd{\ell}{\vec\O} \ti \O = \vec 0 $$
\begin{exercise}
  Derive these
\end{exercise}
{\color{red} \begin{solution}
  We shall start from Hamilton's Principle and move forward to derive our Euler-Poincare Equations.
  \begin{align*}
    \d\int_{t_1}^{t_2} \ell(\Oh)\,dt &= \int_{t_1}^{t_2} \ip{\pd{\ell}{\Oh}}{\d\Oh}\, dt\\
    &= \int_{t_1}^{t_2} \ip{\pd{\ell}{\Ov}}{\d\Ov}
  \end{align*}
  Now, we shall use a fact we proved in the last exercise $\d\Ov = \Ld + (\Ov \ti \Lv)$ to derive the Euler-Poincare equations we wanted,
  \begin{align*}
    \int_{t_1}^{t_2} \ip{\pd{\ell}{\Ov}}{\d\Ov} &= \int_{t_1}^{t_2} \ip{\pd{\ell}{\Ov}}{\Ld + \Ov \ti \Lv} \\
    &= \int_{t_1}^{t_2} \ip{\pd{\ell}{\Ov}}{\dit\Lv} + \ip{\pd{\ell}{\Ov}}{\Ov \ti \Lv} \\
    &= \int_{t_1}^{t_2} \ip{-\dit\pd{\ell}{\Ov}}{\Lv} + \ip{-\Ov \ti \pd{\ell}{\Ov}}{\Lv} \\
    &= \int_{t_1}^{t_2} \ip{-\dit\pd{\ell}{\Ov}}{\Lv} + \ip{ \pd{\ell}{\Ov} \ti \Ov}{\Lv} \\
    &= \int_{t_1}^{t_2} \ip{-\dit\pd{\ell}{\Ov} + \pd{\ell}{\Ov} \ti \Ov}{\Lv} = \vec 0\\
  \end{align*}
  Hence we say that
  \begin{equation*}
    \dit\pd{\ell}{\Ov} - \pd{\ell}{\Ov} \ti \Ov = \vec 0
  \end{equation*}
\end{solution} }
These are the Euler-Poincare equations for rotational dynamics with symmetry under left multiplication.

\newpage
\begin{nthm}[]
  The spatial angular momentum (in the spatial frame) is conserved along solutions of the Euler-Poincare equations.
\end{nthm}
\begin{proof}
  We know $\di {}{t}\pd{\ell}{\vec \O} - \pd{\ell}{\vec \O} \times \O = 0$ and we know that $R\pd{\ell}{\vec\O}$ and the multiplcation by $R$ means spatial frame. Hence we prove,
  $$ \di{}{t} R\pd{\ell}{\vec \O} = \vec 0$$
  \begin{align*}
    \di{}{t} R\pd{\ell}{\vec \O} &= \dot R \pd{\ell}{\vec \O} + R\di { }{t} \pd{\ell}{\vec\O}\\
    &= R\Oh\pd{\ell}{\vec\O} + R(\pd{\ell}{\vec \O} \times \vec\O)\\
    &= R(\O \times \pd{\ell}{\vec\O}) + R(\pd{\ell}{\vec \O} \times \vec\O)\\
    &= \vec 0
\end{align*}
\end{proof}

Now we want to write a general form of the Euler-Poincare Equations for left invariant systems. \\
Let $L$ be a lagrangian on the tangent bundle of a matrix lie group $G$,
$$ L : TG \to \R $$
$$ L = L(g, \dot g) \qquad \fa g \in G $$
Assume that the lagrangian is left-invariant,
$$ L(g,\,\dot g) = L(hg,\, h\dot g) \quad \fa h\in G $$
and now let $h = g^{-1}$, and so $L(g,\,\dot g) = L(g^{-1}g\,g^{-1}\dot g) = \ell(\xi)$. We have gone from a lie group to a lie algebra, $\xi = g^{-1}\dot g \in T_eG = \mathfrak{g}$ which is a matrix lie algebra. We now aim to use the action functional and variational derivative,
\begin{align*}
  \d \int_{t_1}^{t_2} L(g,\,\dot g) &= \vec 0\\
  \d \int_{t_1}^{t_2} \ell(\xi)\,dt &= \vec 0\\
  \int_{t_1}^{t_2} \ip{\pd{\ell}{xi}}{\d \xi}\,dt &= \vec 0\\
\end{align*}
Now we want to consider $\d \xi = \d (g^{-1}\dot g)$,
\begin{align*}
  \d (g^{-1}\dot g) &= \d g^{-1}\dot g + g^{-1}\d \dot g\\
  &= g^{-1}\d gg^{-1}\dot g + g^{-1}\di {}{t}\d g\\
  &= -(g^{-1}\d g)g^{-1}\dot g + g^{-1}\di{}{t}\d g\\
  &= - \eta\xi + \di{}{t}\d (g^{-1}\d g) + (g^{-1}\dot g)(g^{-1}\di{}{t}\d g)\\
  &= -\eta\xi \dot\eta+\xi\eta\\
  &= \dot\eta + [\xi,\,\eta]\\
  &= \dot\eta + \ad_{\xi}{\eta}
\end{align*}
and so back to the derivation,
\begin{align*}
  \int_{t_1}^{t_2} \ip{\pd{\ell}{xi}}{\d \xi}\,dt &= \vec 0\\
  \int_{t_1}^{t_2} \ip{\pd{\ell}{\xi}}{\dot\eta + \ad_{\xi}\eta}\\
  \int_{t_1}^{t_2} \ip{-\di{}{t} \left(\pd{\ell}{\xi}\right)}{\eta} + \ip{\ad^{*}_{\xi}\pd{\ell}{\xi}}{\eta}\,dt
\end{align*}
Since $\eta$ is arbitrary our equation is of this form,
$$ \di{}{t} \pd{\ell}{\xi} - \ad^{*}_{\xi}\pd{\ell}{\xi} = 0 $$
and these are our Euler-Poincare equations for a left invariant system.

\begin{nthm}[Noether Theorem]
  The Euler Poincare equations associated a left-invariant system preserve the generalised momentum along solutions of the Euler-Poincare equations, that is,
  $$ \di{}{t} \left(\Ad^*_{g^{-1}(t)}\pd{\ell}{\xi}(t)\right) = 0 $$
\end{nthm}
\begin{proof}
  Exercises
\end{proof}

\begin{exercise}
  Repeat derivations for the Euler-Poincare Equations for right-invariant systems. What is Noether Theorem?\footnote{What about both left and right invariant?}
\end{exercise}
