% !TEX root = ../notes.tex

\subsection{The Hat Map as a Lie Algebra Isomorphisms}
In this section we will take a quick jaunt back into some purer topics, more specifically a Lie Algebra Isomorphism called the Hat Map. We can define a Lie Algebra Isomorphism as a bijective Lie Algebra homomorphism, which we then define as,
\begin{ndefi}[Lie Algebra Homomorphism]
  A Lie Algebra homomorphism is a linear map, $\phi : \mathfrak{g} \to \mathfrak{g}'$ compatible with the respective lie brackets,
  $$ \phi([x, y]_\mathfrak{g}) = [\phi(x), \phi(y)]_{\mathfrak{g}'} \qquad \forall x, y \in \mathfrak{g} $$
\end{ndefi}

\noindent
We note that the Lie bracket for $\R^3$ is the cross product, and more specifically $(\R^3, \ti)$ is a Lie Algebra. Then we can talk about a Lie Algebra homomorphism, $\phi : \R^3 \to \mathfrak{so}(3)$, which is a homomorphism that maps,
$$ \phi(\vec x \ti \vec y) = [\phi (x), \phi (y)]_{\mathfrak{so}(3)}. $$
Given $\ov \in \R^3$, we let this $\phi$ be the hat map where we define,
$$ \phi(\ov) = \oh = \begin{pmatrix}
  0 & -\o_1 & \o_2 \\ \o_3 & 0 & -\o_1 \\ -\o_2 & \o_1 & 0
\end{pmatrix}. $$

\noindent
The Lie algebra of $\SO(3)$ is the space of skew-symmetric matrices, $\SO(3)$. Then we can conclude that the Euler-Poincare Equations are written as:
$$ \di {}{t} \pd{\ell}{\Oh} - \ad^*_{\Oh}\pd{\ell}{\Oh} = 0. $$
Let $\vec\Pi$ be any element in $\mathfrak{g}^*$, then $\ad^*$ operator is defined by $\ip{\ad_{\Oh}^*\vec\Pi}{\oh} = \ip{\vec\Pi}{\ad_{\Oh}{\oh}}$ where $\oh \in \mathfrak{g}^*$.
\begin{align*}
  \ip{\ad_{\Oh}^*\vec\Pi}{\oh} &= \ip{\vec\Pi}{\ad_{\Oh}{\oh}}\\
  &= \ip{\vec\Pi}{[\Oh,\,\oh]}\\
  &= \Tr(\vec\Pi^\top [\Oh,\,\oh])\\
  &= \Tr(\vec\Pi^\top \Oh\oh - \vec\Pi^\top\oh\Oh)\\
  &= \Tr(\vec\Pi\Oh^\top\oh - \vec\Pi\Oh\oh^\top)\\
  &= \Tr(\vec\Pi\Oh\oh^\top - \Oh\vec\Pi\oh^\top) \\
  &= \Tr((\vec\Pi\Oh - \Oh\vec\Pi)\oh^\top) \\
  &= \Tr([\vec\Pi,\,\Oh]\oh^\top) \\
  &= \ip{[\vec\Pi,\,\Oh]}{\oh}.
\end{align*}

\noindent
Then, $\ad^*_{\Oh}\vec\Pi = [\vec\Pi,\,\Oh]$. From here we can conclude that, the hap map is a Lie algebra isomorphism, i.e. $ [\Oh,\,\oh] = \wh{\vec\O\ti\vec\o} $. We prove this from the definition of the Lie Bracket, that is,

\begin{align*}
  [\Oh, \oh] &= \Oh\oh - \oh\Oh\\
  &= \begin{pmatrix}
    0 & -\O_1 & \O_2 \\ \O_3 & 0 & -\O_1 \\ -\O_2 & \O_1 & 0
  \end{pmatrix}\begin{pmatrix}
    0 & -\o_1 & \o_2 \\ \o_3 & 0 & -\o_1 \\ -\o_2 & \o_1 & 0
  \end{pmatrix} - \begin{pmatrix}
    0 & -\o_1 & \o_2 \\ \o_3 & 0 & -\o_1 \\ -\o_2 & \o_1 & 0
  \end{pmatrix}\begin{pmatrix}
    0 & -\O_1 & \O_2 \\ \O_3 & 0 & -\O_1 \\ -\O_2 & \O_1 & 0
  \end{pmatrix}\\
  &= \begin{pmatrix}
    -\O_3\o_3 - \O_2\o_2 & \O_2\o_1 & \O_3\o_1 \\
    \O_1\o_2 & -\O_3\o_3 - \O_1\o_1 & \O_3\o_2 \\
    \O_1\o_3 & \O_2\o_3 & -\O_2\o_2 - \O_1\o_1
\end{pmatrix} - \begin{pmatrix}
  -\o_3\O_3 - \o_2\O_2 & \o_2\O_1 & \o_3\O_1 \\
  \o_1\O_2 & -\o_3\O_3 - \o_1\O_1 & \o_3\O_2 \\
  \o_1\O_3 & \o_2\O_3 & -\o_2\O_2 - \o_1\O_1
\end{pmatrix} \\
&= \begin{pmatrix}
  0 & \O_2\o_1 - \O_1\o_2 & \O_3\o_1 - \O_1\o_3 \\
  \O_1\o_2 - \O_2\o_1 & 0 & \O_3\o_2 - \O_2\o_3 \\
  \O_1\o_3 - \O_3\o_1 & \O_2\o_3 - \O_3\o_2 & 0
\end{pmatrix} \\
&= \wh{\vec\O \ti \vec\o}.
\end{align*}

\noindent
We have proved that where $\phi$ is the hap map that $[\phi(\vec x), \phi(\vec y)] = \phi(\vec x \ti \vec y)$. We now need to prove that the hap map is a bijective linear map. It is linear as it can be represented as a matrix in $\SO(3)$. Now we prove bijectivity by first proving injectivity, then surectivity. Let $\vec x, \vec y \in \R$ and we know that $\hat{\vec x} = \hat{\vec y}$, that is,
$$ \begin{pmatrix}
  0 & -x_1 & x_2 \\ x_3 & 0 & -x_1 \\ -x_2 & x_1 & 0
\end{pmatrix} = \begin{pmatrix}
  0 & -y_1 & y_2 \\ y_3 & 0 & -y_1 \\ -y_2 & y_1 & 0
\end{pmatrix}. $$
Then we can see that $x_1 = y_1$, $x_2 = y_2$ and $x_3 = y_3$. Therefore, $\vec x = \vec y$. Hence the hat map is injective. Now we seek to prove that the hap map is surjective. Consider some $\hat{\vec z} \in \mathfrak{so}(3)$, then we can write it as,
$$ \begin{pmatrix}
  0 & -z_1 & z_2 \\ z_3 & 0 & -z_1 \\ -z_2 & z_1 & 0
\end{pmatrix}. $$
Then this will uniquely define some $\vec z = (z_1, z_2, z_3)^T \in \R^3$. Hence the hat map is surjective. Therefore the hat map is a Lie Algebra Isomorphism.\\
