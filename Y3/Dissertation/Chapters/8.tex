% !TEX root = ../notes.tex

\subsection{More Noether Theory}

\subsubsection{Noethers Theorem for EL equations}

Consider $L(\vec q,\, \dot{\vec q})$ for $\vec q \in \R^3$ and $\dot{\vec q} \in T_{\vec q}\R^3$. Suppose that $L$ is left invariant with respect to the tangent lift on $\SO(3)$, ie. $R\in\SO(3)$ with $L(R\vec q\, R\dot{\vec q}) = L(\vec q,\,\dot{\vec q})$. Then we can prove that,
$$ \mathcal{E} := \dit \pd{L}{\dot{\vec q}} - \pd{L}{\vec q} = 0 $$

Now we have theorem,
\begin{nthm}
  Corresponding to each one-parameter subgroup of $\SO(3)$ $R(s)$ where $R(0) = e$ and $R'(0) = \hat\xi \in \mathfrak{SO}(3)$. There is a conserved quantity,
  $$ A_\xi := \ip{\vec q \ti \pd{L}{\dot q}}{\xi} $$
  with $\di{A_\xi}{t} = 0$ along solutions of the Euler Legrange equations $\mathcal{E}(\vec q) = 0$
\end{nthm}
\begin{proof}
  Associated with the one parameter subgroup $R(s)$ is the generator $\xi_{\mathcal{M}}(\vec q) := \di{}{s}_{s=0} R(s)\vec q = \hat\xi \vec q = \xi\ti\vec q$ where $\mathcal{M} = \R^3$. Now we consider,
  \begin{align*}
    \int_{t_1}^{t_2} L(R(s)\vec q,\,R(s)q) = \int_{t_1}^{t_2} L(q,\,\dot q) \,dt \\
  \end{align*}
  and now we differentiate this wrt $s$ and then set $s = 0$ to obtain
  $$ \ip{\pd{L}{\dot q}}{\xi_{\mathcal{M}}(\vec q)} = \dit{\pd{L}{\dot q}}{\xi \ti \vec q} = \ip{\vec q \ti \pd{L}{\dot q}}{\xi} = A_\xi$$
\end{proof}

\subsubsection{Noether Theory and EP Reduction}
We have,
$$ \int_{t_1}^{t_2} L(R,\,\dot R)\,dt = 0 $$
and $L$ is left-invariant and so, $L(SR,\,S\dot R) = L(R,\,\dot R)$
\begin{nthm}[Noethers Theorem]
  Corresponding to each one parameter subgroup of $\SO(3)$, $S(s)$ with $S(0)= e$ and $S'(0) = \hat\xi \in \mathfrak{SO}(3)$, then there is a conserved quantity
  $$ A_\xi := \ip{\Ad^*_{R^\top}\pd{\hat\ell}{\Oh}}{\hat\xi} $$
  with $\di{A_\xi}{t} = 0$ along solutions of the Euler Lagrange Equation.
  $$ \mathcal{E}(R) := \dit\pd{L}{\dot R} - \pd{L}{R} = \vec 0 $$
\end{nthm}
\begin{proof}
  Consider $S(s)$ and differentiate and let $s = 0$ much like before,
  $$ \int_{t_1}^{t_2} \pd{}{\xi_{\mathcal{M}(R)}}\pd{L}{R} + \ip{\pd{L}{\dot R}}{\xi_{\mathcal{M}(\dot R)}} $$
  where here $\xi_{\mathcal{M}}(R) = \hat\xi R$
\end{proof}


We now have a definition,
\begin{ndefi}[Infinitesimal Generator]
  Consider the left action of a Lie group $G$ on the manifold $\mathcal{M}$, $(g, \vec x) \to gx$ ($\vec x \in \mathcal{M}$). Let $\xi \in \mathfrak{g}$ be a vector in the Lie algebra of $G$ and consider one parameter subgroup $$[exp(t\xi) : t \in \R] \subseteq G$$
  Then the orbit of an element $\vec x$ with respect to this subgroup is a smooth map $t \to (exp(t\xi))\vec x$ in $\mathcal{M}$. The infinitesimal generator associated to $\xi$ at $\vec x \in \mathcal{M}$ denoted by $\xi_{\mathcal{M}}(\vec x)$ is the tangent vector (or velocity) to this curve at point $\vec x$,
  $$ \xi_{\mathcal{M}} (\vec x) = \dit\Bigg|_{t=0} (exp(t \xi)\vec x) \in T_{\vec x}\mathcal{M} $$
  this smooth vector field $\xi_{\mathcal{M}} : M \to TM$ and $x \mapsto \xi_{\mathcal{M}}(\vec x)$ is called the infinitesimal generator vector field associated to $\xi$.
\end{ndefi}

Let $G$ be an arbitrary matrix lie group, and let $L$ a left-invariant Lagrangian with variational principle $\int_{t_1}^{t_2} L(g,\,\dot g) = 0$ the reduced system is $L(hg,\, h\dot g)|_{h = g^{-1}} = L(e,\,g^{-1}\dot g) = \ell(\xi) = \ell(g^{-1}\dot g)$.
\begin{nthm}[Noether's Theorem]
  Corresponding to each one-parameter subgroup of $G$, $\chi(s)$ such that $\chi(0) = e$ and $\xi_s(0) = \eta \in \mathfrak{g}$. There is a conserved quantity
  $$ \ip{\Ad^*_{g^{-1}} \pd{\ell}{\xi}}{\eta} $$
\end{nthm}
\begin{proof}
  Done before
\end{proof}

\begin{nprop}
  The left-invariant Lagrangian $L(g,\,\dot g)$ satisfies,
  $$ \dit\pd{L}{\dot g} - \pd{L}{g} = \vec 0  \iff \dit\pd{\ell}{\xi} - \ad^*\pd{\ell}{\xi} = 0$$
\end{nprop}
\begin{proof}
  I think I have done this
  $$ \int_{t_1}^{t_2} \ip{\mathcal{E}}{\d g} = \int_{t_1}^{t_2} \ip{\mathcal{E}(\xi)}{\nu} $$
\end{proof}

\subsection{Diamond Map}
Let $V$ be an $n$-dimensional vector space with dual $V^*$ and pairing $\ip{\vec w}{\vec u}_V$ where $\vec u \in V$ and $\vec w \in V^*$. Let $\mathcal{M}(n,\,\R)$ be a vector space of $n \ti n$ matrices with dual $\mathcal{M}(n,\,\R)^*$ and pairing $\ip{B}{A}_{\mathcal{M}} := \Tr(B^\top A)$ where $A \in \mathcal{M}(n,\,\R)$ and $B \in \mathcal{M}(n,\,\R)$.\\

\noindent
The diamond map is a representation of the transformation of the pairing on $V$ to the pairing on $\mathcal{M}(n,\,\R)$. Let $\vec u \in V$ and $\vec w \in V^*$ and consider the matrices $A \in \mathcal{M}(n,\,\R)$ and $\Lv \in \mathcal{M}(n,\,\R)$ where $A$ is a general matrix and $\Lv$ is a symmetric matrix ($\Lv^\top = \Lv$). The diamond map is defined by $\ip{\vec w}{A\Lv\vec u}_V = \ip{\vec u \diamond \vec w}{\Lv}_{\mathcal{M}}$.
\begin{align*}
  \ip{\vec w}{A\vec u}_V &= \Tr(\vec u \vec w^\top A) \\
  &= \Tr((\vec w \vec u ^\top)^\top A)\\
  &= \ip{\vec w\vec u^\top}{A}_{\mathcal{M}}
\end{align*}
This is for any matrix $A \in \mathcal{M}(n,\,\R)$ and vectors $\vec u \in V$ and $\vec w \in V^*$. We now conclude that,
\begin{align*}
  \ip{\vec w}{A\Lv \vec u}_V &= \Tr(\vec u \vec w^\top A\Lv) \\
  &= \ip{(\vec u\vec w^\top A)^\top}{\Lv}_{\mathcal{M}}\\
  &= \ip{A^\top\vec w\vec u^\top}{\Lv}
\end{align*}
