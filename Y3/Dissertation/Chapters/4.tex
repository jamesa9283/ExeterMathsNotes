% !TEX root = ../notes.tex

\subsection{Calculus of variations}
We are going to consider a continuous level, but you can use discrete level.
% mathscr
\begin{nthm}[The variation principle]
  $$ \L = \int_{t_1}^{t_2} \frac{1}{2}\I\vec \O \cdot \vec\O \,dt $$
  and we find differential equations by letting $\d\L = 0$ but this is subject to $\d\vec\O(t_1) = \d\vec\O(t_2) = \vec 0$
\end{nthm}
and so,
$$ \d\int_{t_1}^{t_2} \frac{1}{2}\I\vec\O \cdot \vec\O\, dt = \int_{t_1}^{t_2} \frac{1}{2}\I\d\vec\O \cdot \vec\O + \frac{1}{2}\I\vec\O \cdot \d\vec\O \, dt  = \int_{t_1}^{t_2}\I\vec\O \cdot \d\vec\O$$
but what is $\d\vec\O$, but remember we have $\Oh$, which is the lie algebra of $\SO(3)$. We said, $\Oh = R^T\dot R = R^{-1}\dot R$. Now we take variations of $\Oh = \vec\O \times \vec X$ and so,
$$ (\d \vec \O)\times\vec X = (\d\Oh) \vec X $$
and so we see that,
$$ \d\Oh = \d(R^{-1}\dot R) = \d R^{-1}\dot R + R^{-1}\d\dot R = 0 $$
as $\d I = \d RR^{-1} + R\d R^{-1}$ and then we see that $R^{-1}\d R R^{-1} + R^{-1}R \d R^{-1} = 0$ and so as $RR^{-1} = I$, $R^{-1}\d RR^{-1} + \d R^{-1} = \vec 0$. We have that $\d R^{-1}\dot R + R^{-1}\d \dot R = \d \Oh$ and $\Oh = R^{-1}\dot R$ where $\Lh = R^{-1}\d R$ and so we sub in,
\begin{align*}
  \d\Oh &= -R^{-1}\d R R^{-1}\dot R + R^{-1}\di{}{t}\d R\\
  &= R^{-1}\d R \Oh + \di{}{t}(R^{-1}\d R) - (\di{}{t}R^{-1})\d R\\
  &= R^{-1}\d R \Oh + \di{}{t}(R^{-1}\d R) + R^{-1}\dot R R^{-1}\d R\\
  &= - \Lh\Oh + \di{}{t}\Lh + \Oh\Lh\\
  &=\Lhd + [\Oh, \Lh]
\end{align*}

\begin{exercise}
  Prove,
  $$ \d\vec\O = \dot\Lambda + (\vec\O \times \vec\Lambda) $$
\end{exercise}
{\color{red} \begin{solution}
  We can use the fact that $\wh{[\vec\O,\,\vec\Lambda]} = [\Oh,\,\Lh]$ and then we can get the required result.
  \begin{align*}
    \d\Oh &= \Lhd + [\Oh,\,\Lh] \\
    &= \Lhd + \wh{[\vec\O,\,\vec\Lambda]} \\
    &= \Lhd + \wh{(\vec\O \ti \vec\Lambda)} \\
    \wh{\d\vec\O} &= \wh{\Ld + (\Ov \ti \Lv)} \\
  \end{align*}
  and so we can see that $\d\Ov = \dot\Lv + (\Ov \ti \Lv)$
\end{solution}
 }
\noindent
Now, let us substitute this back into our variational principle.
\begin{align*}
  \d\int_{t_1}^{t_2} \I \vec\O \cdot \vec\O \,dt &= 0 \\
  \int_{t_1}^{t_2}\I \vec\O \cdot \d\vec\O \,dt &= 0 \\
  \int_{t_1}^{t_2} {\I \vec\O \cdot (\Ld + \vec\O \times \vec\La)\,dt} &= 0\\
  \left[ \I\vec\O \cdot \Lv |_{t_2} - \I\vec\O \cdot \Lv |_{t_1} \right] - \int_{t_1}^{t_2} \di{}{t}(\I\vec\O) \cdot \Lv\,dt + \int_{t_1}^{t_2} (\I\vec\O \times \vec\O)\cdot \Lv &= 0\\
  0 - 0 - \int_{t_1}^{t_2} (- \I\dot{\vec\O} + \I\vec\O \times \vec\O) \cdot \Lv \, dt &= 0
\end{align*}
Hence,
$$ \I\dot{\vec \Lambda} = \I\vec\O \times \vec\O $$
\noindent
We can write the equations by considering the tangent space.

