% !TEX root = ../notes.tex

\subsection{Calculus of variations}
Now we will continue our exploration of pure-applied mathematics by looking at calculus of variations. We are going to consider the mathematics on a continuous level, but you can consider a discrete level. This could be done via discrete differential geometry and is very useful to derive numerical methods subject to some conditions.
% mathscr
\begin{nthm}[Hamilton's Variation Principle]
  Let $G$ be a Lie group and $\mathfrak{g}$ be its associated algebra. We consider a Lagrangian, $L : G \ti \mathfrak{g} \to \mathfrak{g}$. Then Hamilton's Variational Principle holds, that is,
  $$ L = \int_{t_1}^{t_2} \frac{1}{2}\I\vec \O \cdot \vec\O \,dt. $$
  We find differential equations by letting $\d L = 0$ but this is subject endpoint conditions, $\d\vec\O(t_1) = \d\vec\O(t_2) = \vec 0$.
\end{nthm}
\noindent
Now we can take Hamilton's Variational Principle with respect to the Lagrangian we derived in the last section,
$$ \d\int_{t_1}^{t_2} \frac{1}{2}\I\vec\O \cdot \vec\O\, dt = \int_{t_1}^{t_2} \frac{1}{2}\I\d\vec\O \cdot \vec\O + \frac{1}{2}\I\vec\O \cdot \d\vec\O \, dt  = \int_{t_1}^{t_2}\I\vec\O \cdot \d\vec\O$$
We have reached an impassable point, we currently don't know what $\d\vec\O$ is. Remember $\Oh$ is in the Lie algebra of $\SO(3)$. We said that $\Oh = R^T\dot R = R^{-1}\dot R$. Now we take variations of $\Oh = \vec\O \times \vec X$ and so,
$$ (\d \vec \O)\times\vec X = (\d\Oh) \vec X. $$
We can now rewrite this as,
$$ \d\Oh = \d(R^{-1}\dot R) = \d R^{-1}\dot R + R^{-1}\d\dot R = 0 $$
as $\d I = \d RR^{-1} + R\d R^{-1}$. We see that $R^{-1}\d R R^{-1} + R^{-1}R \d R^{-1} = 0$ and so as $RR^{-1} = I$, $R^{-1}\d RR^{-1} + \d R^{-1} = \vec 0$. Therefore, $\d R^{-1}\dot R + R^{-1}\d \dot R = \d \Oh$
and $\Oh = R^{-1}\dot R$ where $\Lh = R^{-1}\d R$ and so we substitute for this quantity,
\begin{align*}
  \d\Oh &= -R^{-1}\d R R^{-1}\dot R + R^{-1}\di{}{t}\d R\\
  &= R^{-1}\d R \Oh + \di{}{t}(R^{-1}\d R) - (\di{}{t}R^{-1})\d R\\
  &= R^{-1}\d R \Oh + \di{}{t}(R^{-1}\d R) + R^{-1}\dot R R^{-1}\d R\\
  &= - \Lh\Oh + \di{}{t}\Lh + \Oh\Lh\\
  &=\Lhd + [\Oh, \Lh].
\end{align*}

\noindent
We now aim to prove,
$$ \d\vec\O = \dot\Lambda + (\vec\O \times \vec\Lambda). $$
We can use the fact that $\wh{[\vec\O,\,\vec\Lambda]} = [\Oh,\,\Lh]$ to get the required result.
\begin{align*}
  \d\Oh &= \Lhd + [\Oh,\,\Lh] \\
  &= \Lhd + \wh{[\vec\O,\,\vec\Lambda]} \\
  &= \Lhd + \wh{(\vec\O \ti \vec\Lambda)} \\
  \wh{\d\vec\O} &= \wh{\Ld + (\Ov \ti \Lv)} \\
\end{align*}
Therefore, $\d\Ov = \dot\Lv + (\Ov \ti \Lv)$. Now, let us substitute this back into our variational principle.
\begin{align*}
  0 &= \d\int_{t_1}^{t_2} \I \vec\O \cdot \vec\O \,dt \\
  0 &= \int_{t_1}^{t_2}\I \vec\O \cdot \d\vec\O \,dt \\
  0 &= \int_{t_1}^{t_2} {\I \vec\O \cdot (\Ld + \vec\O \times \vec\La)\,dt} \\
  0 &= \left[ \I\vec\O \cdot \Lv |_{t_2} - \I\vec\O \cdot \Lv |_{t_1} \right] - \int_{t_1}^{t_2} \di{}{t}(\I\vec\O) \cdot \Lv\,dt + \int_{t_1}^{t_2} (\I\vec\O \times \vec\O)\cdot \Lv\\
  0 &= 0 - 0 - \int_{t_1}^{t_2} (- \I\dot{\vec\O} + \I\vec\O \times \vec\O) \cdot \Lv \, dt.
\end{align*}
Hence,
$$ \I\dot{\vec \Lambda} = \I\vec\O \times \vec\O. $$
We note that can write the equations by considering the tangent space.

