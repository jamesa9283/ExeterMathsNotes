% !TEX root = ../notes.tex

\section{Calculus of variations}
We are going to consider a continuous level, but you can use discrete level.
% mathscr
\begin{nthm}[The variation principle]
  $$ \L = \int_{t_1}^{t_2} \frac{1}{2}\I\vec \O \cdot \vec\O \,dt $$
  and we find differential equations by letting $\d\L = 0$ but this is subject to $\d\vec\O(t_1) = \d\O(t_2) = \vec 0$
\end{nthm}
and so,
$$ \d\int_{t_1}^{t_2} \frac{1}{2}\I\vec\O \cdot \vec\O\, dt = \int_{t_1}^{t_2} \frac{1}{2}\I\d\vec\O \cdot \vec\O + \frac{1}{2}\I\vec\O \cdot \d\vec\O \, dt  = \int_{t_1}^{t_2}\I\vec\O \cdot \d\vec\O$$
but what is $\d\vec\O$, but remember we have $\hat\O$, which is the lie algebra of $\SO(3)$. We said, $\Oh = R^T\dot R = R^{-1}\dot R$. Now we take variations of $\Oh = \vec\O \times \vec X$ and so,
$$ (\d \vec \O)\times\vec X = (\d\Oh) \vec X $$
and so we see that,
$$ \d\Oh = \d(R^{-1}\dot R) = \d R^{-1}\dot R + R^{-1}\d\dot R = 0 $$
as $\d I = \d RR^{-1} + R\d R^{-1}$ and then we see that $R^{-1}\d R R^{-1} + R^{-1}R \d R^{-1} = 0$ and so as $RR^{-1} = I$, $R^{-1}\d RR^{-1} + \d R^{-1} = \vec 0$. We have that $\d R^{-1}\dot R + R^{-1}\d \dot R = \d \Oh$ and $\Oh = R^{-1}\dot R$ where $\Lh = R^{-1}\d R$ and so we sub in,
\begin{align*}
  \d\Oh &= -R^{-1}\d R R^{-1}\dot R + R^{-1}\di{}{t}\d R\\
  &= R^{-1}\d R \Oh + \di{}{t}(R^{-1}\d R) - (\di{}{t}R^{-1})\d R\\
  &= R^{-1}\d R \Oh + \di{}{t}(R^{-1}\d R) + R^{-1}\dot R R^{-1}\d R\\
  &= - \Lh\Oh + \di{}{t}\Lh + \Oh\Lh\\
  &=\Lhd + [\Oh, \Lh]
\end{align*}

\begin{exercise}
  Prove,
  $$ \d\vec\O = \dot\Lambda + (\vec\O \times \vec\Lambda) $$
\end{exercise}

Now, let us substitute this back into our variational principle.
\begin{align*}
  \int_{t_1}^{t_2} \I \vec\O \cdot \vec\O \,dt &= 0 \\
  \int_{t_1}^{t_2}\I \vec\O \cdot \d\vec\O \,dt &= 0 \\
  \int_{t_1}^{t_2} {\I \vec\O \cdot (\dot\La + \vec\O \times \vec\La)\,dt} &= 0\\
  \left[ \I\vec\O \cdot \La |_{t_2} - \I\vec\O \cdot \La |_{t_1} \right] - \int_{t_1}^{t_2} \di{}{t}(\I\vec\O) \cdot \Lambda\,dt + \int_{t_1}^{t_2} (\I\vec\O \times \vec\O)\cdot \vec\Lambda &= 0\\
  0 - 0 - \int_{t_1}^{t_2} (- \I\dot{\vec\O} + \I\vec\O \times \vec\O) \cdot \Lambda \, dt &= 0
\end{align*}
Hence,
$$ \I\dot{\vec \Lambda} = \I\vec\O \times \vec\O $$

\noindent
We can write the equations by considering the tangent space.


\subsection{Euler-Poincare Reduction by Symmetry}
To gain a general idea of how the equations of motion appear for rotational dynamics with symmetry, we consider an arbitrary Lagrangian of this form,
$$ L : T\SO(3) \to \R $$
$$ L = L(R, \dot R) $$
and satisfies,
$$ \d\int_{t_1}^{t_2} L(R, \dot R)\,dt = 0 $$
this means,
\begin{align*}
  \int_{t_1}^{t_2} L(R, \dot R)\,dt &= \int_{t_1}^{t_2} \ip{\pd{L}{R}}{\d R} + \ip{\pd{L}{\dot R}}{\d R}\, dt\\
  &= \ip{\pd{L}{\dot R}}{\d R}\Bigg|_{t_2} - \ip{\pd{L}{\dot R}}{\d R}\Bigg|_{t_1} + \int_{t_1}^{t_2} \ip{\pd{L}{R} - \di{}{t} \pd{L}{\dot R}}{\d R} \,dt
\end{align*}
and so we can notice $\di{}{t}\pd{L}{\dot R} - \pd{L}{R} = 0$
\begin{ndefi}[Left-Symmetric Lagrangian]
  A Lagrangian is said to be left-symmetric or left-invariant under the action of the group of the group of rotations if, $L(\chi R,\, \chi\dot R) = L(R, \dot R)$ $\fa \chi \in \SO(3)$.
\end{ndefi}
We also know $\mathfrak{SO}(3) = T_e\SO(3)$ and we said that $v \in T_eG \implies g^{-1}v \in \mathfrak{g} = T_eG$. We know $\dot R(t) \in T_{R(t)}\SO(3)$ and so we can say $R^{-1}\dot R \in \mathfrak{SO}(3)$.\\

We say
\begin{align*}
  L(R, \dot R) &= L(R^{-1}R, R^{-1}\dot R)\\
  &= \wt{\ell}(R^{-1}\dot R) = \wt{\ell}(\Oh)
\end{align*}
Now we write out Hamilton's principle,
\begin{align*}
  0 &= \d \int_{t_1}^{t_2} \wt{\ell}(\Oh)\,dt\\
  &= \d \int_{t_1}^{t_2} \ell(\vec\O) = \vec{0}\\
\end{align*}
