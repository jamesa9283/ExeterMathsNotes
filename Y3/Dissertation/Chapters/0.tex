% !TEX root = ../notes.tex

\section{Introduction}

We start with a problem. Consider a particle that is half way along a piece of light string which is pulled past its natural extension. The question is, what is the velocity of the particle when it reaches a displacement of zero? Anybody trained in mathematics would start by trying to describe the position of the particle in this system and that is the wrong way to solve this problem.\\

\begin{figure}[!ht]
\centering
\resizebox{0.2\textwidth}{!}{\input{./figures/intro1.pdf_tex}}
\caption{Motivating Problem.}
\end{figure}

\noindent
To solve this problem, the best (and probably only) way is to describe the particle's energy. That is, we know that energy is conserved and so if at some point we know the total energy we can then describe the energy in the system at any point. At $t = 0$ there is zero kinetic energy, as the particle is held at rest. If we assume that the only energies to consider are potential and kinetic, the whole energy of the system can be described by just the elastic potential energy. Then we know that the kinetic energy of the particle at displacement zero is, $\frac{\l x^2}{2l_0}$ where $l_0$ is the natural extension of the spring, $x$ is the elastic extension and $\l$ is the spring constant. Hence, in our case the energy in the string is $\frac{\l (\ell - l_0)^2}{2l_0}$ and so the velocity of the particle at displacement zero is going to be, $v = (\ell - l_0)\sqrt{\frac{\l}{ml_0}}$. In general, we can make these energy arguments about sets of particles in Euclidean space. To do this we go on a little detour to explain the Lagrangian, a useful tool in this area of mathematics.\\

\noindent
Consider some points in Euclidean space, let's call them $\vec r_1, \vec r_2, \dots, \vec r_n$. We seek the energy of each of these particles, which requires us to know the forces applied to each of them. To describe the forces on each of these particles, we will use Newton's Second Law. One of the main equations that we use in mechanics is Newton's Second Law, namely $F = ma$, which can be written as $F = \frac{m(v - u)}{t} \approx \di p t$. After some algebra, we can arrive at an exact differential equation $vdx - pdv = dL$, where $dL$ is some function. Then we can solve this differential equation and arrive at $L = K - V$ where $K$ is kinetic energy and $V$ is potential energy. The Lagrangian is now our invariant, as before we added the potential energy and kinetic energy, here we subtract them and this will be our way to describe systems. The next natural question, is how do we describe systems and that is exactly what we will spend the rest of this dissertation discussing.\\

\noindent
In the following document, there are several, `checkpoints', places where you can take different paths through the dissertation and end up at one of the examples at the end. The following diagram explains the roadmap,
