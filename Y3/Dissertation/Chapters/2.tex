% !TEX root = ../notes.tex


\section{Actions of a Lie Group and Lie Algebra}
In this section we will focus on how our Lie Group will act on our Algebra. We will first use conjugation actions to define our adjoints which will become very useful once we see the Euler Poincare equations.


\begin{ndefi}[Conjugation Action]
  Let $g \in G$, then the operation $I_g : G \to G$ (Inner Automorphism) and so you define it by $h \mapsto ghg^{-1}\quad\fa h \in G$. $I_{gh} = AD_{gh}$.
\end{ndefi}

Take an arbitrary path $h(t) \in {G}$ such that $h(0) = e$ and now $\xi = \dot h(0) \in T_eG$. We now define $Ad_g(\xi) = \di {}{t} I_g h(t)_{t=0} = g \xi g^{-1} \in T_eG$ the adjoint action.

\begin{ndefi}[Adjoint and coadjoint actions of ${G}$ on $\mathfrak{g}$ and $\mathfrak{g}^*$]
  The adjoint action of the matrix group $G$ on it's lie algebra $\mathfrak{g}$ is a map,
  $$ Ad : G \times \mathfrak{g} \to \mathfrak{g} $$
  which is,
  $$ \Ad_g \xi = g\xi g^{-1} $$
  The dual map $\ip {Ad^*_g \mu} \xi = \ip {\mu} {Ad_g\xi}$ where $\mu \in \mathfrak{g}^*$ and $\xi \in T_eG = \mathfrak{g}$. is called the coadjoint map of $G$ on the dual lie algebra $\mathfrak{g}^*$.
\end{ndefi}

We will find that sometimes our classical ideas of vectorspaces doesn't work. Hence, we shall introduce functionals and use them to define dual vector spaces.

\begin{ndefi}[Dual Space for vectors]
  Let $V$ be a finite dimensional vector space, of dimension $n$, over $\R$. The dual vector space is denoted by $V^*$ is the space of all linear functionals from $V \to \R$, $f(v) = a$ where $v \in V$ and $a \in \R$, then also $f(\a v + \b w) = \a f(v) + \b f(w)$ and $\a, \b \in \R$ and $v, w \in V$. Hence, $f(v) = Mv$ we call $M$ the covector such that $Mv \in \R$. The vectorspace of all covectors is the dual space.
  $$ \ip{m}{v} \in \R \quad m \in V^* \quad v \in V $$
\end{ndefi}

\noindent
Now we can see that the dual space is also a vector space so we can use the normal vector space ideas with it,

\begin{nlemma}
  Let $V$ be a vector space of real $n \times n$ real matrices. Then the dual vector space $V^*$ is also a vector space of $n \times n$ matrices and every linear functional $f : V \to \R$ such that,
  $$ f(A) := Tr(B^TA), \quad B \in V^*, A \in V  $$
\end{nlemma}

We need to generalise the idea of a inner product to matrices and here is a particular inner product called  trace pairing. From here on any inner product signs will indicate a trace pairing.

\begin{ndefi}[Trace Pairing]
  For every vector space $V$ of real $n\times n$ matrices with dual $V^*$, then the pairing is,
  $$ \ip B A = Tr(B^TA) = Tr(BA^T) $$
\end{ndefi}

\begin{nprop}
  Suppose $A^T = A$ and $B^T = -B$, then, $\Tr(B^TA) = 0$
\end{nprop}
\begin{proof}
  \begin{align*}
    \Tr(B^TA) &= -\Tr(BA)\\
    &= -\Tr((BA)^T)\\
    &= -\Tr(B^TA^T)\\
    &= -\Tr(A^TB^T)\\
    &= - \Tr(B^TA)
  \end{align*}
\end{proof}

We say that,
\begin{align*}
  \ip {Ad^*_g \mu} \xi &= \ip {\mu} {Ad_g\xi}\\
  &= \ip {\mu} {g\xi g^{-1}}\\
  &= \Tr(\mu^Tg\xi g^{-1})\\
  &= \Tr(\xi g\mu^Tg^{-1})\\
  &= \Tr [(g^T\mu(g^{-1})^T)^T\xi]\\
  &= \ip{g^T\mu (g^{-1})^T} {\xi}\\
  &= \ip{g^T\mu (g^T)^{-1}} {\xi}\\
\end{align*}
