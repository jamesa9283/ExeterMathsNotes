% !TEX root = ../notes.tex

\section{The Hat Map as a Lie Algebra Isomorphisms}
The lie algebra of $\SO(3)$ is the space of skew-symmetric matrices, $\mathfrak{SO}(3)$. Then we can conclude that the Euler-Poincare Equations are written as:
$$ \di {}{t} \pd{\ell}{\Oh} - \ad^*_{\Oh}\pd{\ell}{\Oh} = 0 $$
Let $\vec\Pi$ be any element in $\mathfrak{g}^*$, then $\ad^*$ operator is defined by $\ip{\ad_{\Oh}^*\vec\Pi}{\oh} = \ip{\vec\Pi}{\ad_{\Oh}{\oh}}$ where $\oh \in \mathfrak{g}^*$.
\begin{align*}
  \ip{\ad_{\Oh}^*\vec\Pi}{\oh} &= \ip{\vec\Pi}{\ad_{\Oh}{\oh}}\\
  &= \ip{\vec\Pi}{[\Oh,\,\oh]}\\
  &= \Tr(\vec\Pi^T [\Oh,\,\oh])\\
  &= \Tr(\vec\Pi^T \Oh\oh - \vec\Pi^T\oh\Oh)\\
  &= \vdots\\
  &= \ip{[\vec\Pi,\,\Oh]}{\oh}
\end{align*}

Then, $\ad^*_{\Oh}\vec\Pi = [\vec\Pi,\,\Oh]$. From here we can conclude that, the hap map is a lie algebra isomorphism, i.e., $ {[\Oh,\,\oh] = }\widehat{\vec\O\ti\vec\o} $
\begin{proof}
  Exercise
\end{proof}
We define, $\hat{\cdot} : (\R^3,\, \cdot \ti \cdot) \to (\mathfrak{SO}(3),\, [\cdot,\, \cdot])$. This is the isomorphism.\\

We have described Lagrangians that have left or right invariance. We now look to Lagrangians that have Symmetry breaking parameters, like gravity. If we consider the spherical pendulum, we have defined $\Oh = R^T\dot R$ and we define $\oh = R\vec\O$ and then we can see that $\oh = \dot R R^T = \dot R R^{-1}$ where $R \in \SO(3)$. This doesn't lead to a symmetric Lagrangian but we can still use our theory here. \\

We are going to study rigid body dynamics in the spatial frame. We look firstly to the Lagrangian. We have showed,
$$ L(R,\,\dot R) = \frac{1}{2} \int_\b \rho(\vec X) \norm{\dot R \vec X}^2 d^3\vec X $$
and we showed that $L(R,\, \dot R) = L(e,\, R^{-1}\dot R)$ and then we used Euler Poincare Theorem to show that $\ell = \frac{1}{2}\mathbb{I}\vec\O \cdot \vec\O$. Now assume we would prefer to formulate rigid body dynamics in the spatial frame. We need to consider a $\vec\o$ such that $\oh = \dot R R^{-1}$. We can now prove that, $L(R,\,\dot R) \ne L(R\chi,\,\dot R\chi)$ (right multiplication) hence we have broken symmetry,
\begin{proof}
  Exercise
\end{proof}

Now we seek this Lagrangian,
\begin{align*}
  L(R,\,\dot R) &= \frac{1}{2}\int_{\b}\rho(\vec X)\norm{\dot R \vec X}^2d^3\vec X\\
  &= \frac{1}{2}\int_{\b}\rho(\vec X)\norm{\dot RR^{-1}R \vec X}^2d^3\vec X\\
  &= \frac{1}{2}\int_{\b}\rho(\vec X)\norm{\hat\o R \vec X}^2d^3\vec X\\
  &= \frac{1}{2}\int_{\b}\rho(\vec X)(\o \ti R\vec X) \cdot (\o \ti R\vec X)d^3\vec X\\
  &= \vdots\\
  &= \frac{1}{2}\vec\o \cdot (R\mathbb{I} R^T)\o = L(\vec\o,\,R)
\end{align*}
We define a new parameter, $\mathbb{J} := R\mathbb{I}R^T$ and so $\ell = \ell(\mathbb{J},\,\vec\o) = \frac{1}{2}\vec\o(t) \cdot \mathbb{J}(t) \vec\o(t)$
% J is for spatial frame.

Now we take variations as usual,
$$ \d\int_{t_1}^{t_2} \ell(\J,\,\vec\o) = 0 $$
\begin{align*}
  \d \int_{t_1}^{t_2} \frac{1}{2}\vec \o(t) \cdot \mathbb{J}(t) \vec\o(t) dt = 0
\end{align*}
and we ask what is $\d\J(t)$,
\begin{align*}
  \d\J(t) &= \d(R\I R^T)\\
  &= \d R\I R^T + R\I \d R^T\\
  &= \d R R^{-1}R\I R^T - R\I R^{-1}\d R R^{-1}\\
  &= \Lh \J - \J \Lh\\
  &= [\Lh,\,\J]
\end{align*}
where $\Lh = \d R R^{-1}$

\begin{exercise}
  Prove that $\d \oh = \vec\Lhd + [\Lh,\, \oh]$ and $\d\vec\o = \dot{\vec{\Lambda}} + \vec\Lambda \ti \vec\o$ and then take variations of $\frac{1}{2}\vec\o \cdot \J\vec\o$ and prove that $\di{}{t} (\J\vec\o) = \vec 0$
  and then
 $\di{\J}{t} \oh + \J\vec{\dot{\hat{\o}}} = \vec 0$ and so $\di{\J}{t} = [\oh,\,\J]$.
\end{exercise}
