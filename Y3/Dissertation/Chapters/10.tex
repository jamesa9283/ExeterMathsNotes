% !TEX root = ../notes.tex

\section{Pseudo-Rigid Bodies}
Now let us assume that our body can stretch and sheer, this will be called a psudo rigid body. I have done the following derivations with the assumption that the configuration space we are working in is $\GL^+(3)$, ie. the set of matrices with postive determinant. We make a few assumptions, firstly the moment of inertia tensor is rotationally invariant, it is sufficient that the density function $\rho(\vec X)$ is spherically symmetric. We will also assume that the Lagrangian only depends on the kinetic energy and so we study free ellipsoid motion.\\

We fix a reference configuration via a fixed spatial coordinate system and a moving body coordinate system, both with origin of the fixed point of the body. We will assume that the configuration of the system is a matrix $\vec Q(t) \in \GL^+(3)$ which takes the label $\vec X$ to the spacial position $\vec x(t)$, that is,
$$ \vec x(t, \vec X) = \vec Q(t)\vec X \qquad \dot{\vec x}(t, \vec X) = \dot{\vec Q}\vec X = \dot{\vec Q}(t)\vec Q^{-1}(t)\vec x(t, \vec X) $$
as before let $\rho(\vec X)$ be the density function and $\mathcal{B}$ be the region occupied by the body in it's configuration space. The moment of inertia tensor is assumed to be spherically symmetric, that is,
$$ \int_{\mathcal{B}} \rho(\vec X)\vec X\vec X^T d^3\vec X = kI \qquad k \in \R $$
and $I$ is the identity matrix. We assume without loss of generality that $k = 1$ and so,
$$ \int_{\mathcal{B}} \rho(\vec X)\vec X\vec X^T d^3\vec X = I $$
We now consider the kinetic energy,
\begin{align*}
  K &= \frac{1}{2}\int_{\mathcal{B}} \rho(\vec X) \norm{\dot{\vec x}}\, d^3 \vec X \\
  &= \frac{1}{2}\int_{\mathcal{B}} \rho(\vec X) \norm{\dot{\vec Q}\dot{\vec X}}\, d^3 \vec X \\
  &= \frac{1}{2}\int_{\mathcal{B}} \rho(\vec X) \Tr\left( (\dot{\vec Q}\vec X)(\dot{\vec Q}\vec X)^T \right)\, d^3 \vec X \\
  &= \frac{1}{2}\Tr\left(\dot{\vec Q}\int_{\mathcal{B}} \rho(\vec X) \vec X\vec X^T \, d^3 \vec X\, \dot{\vec Q}^T \right)\\
  &= \frac{1}{2}\Tr\left( \dot{\vec Q}\J\dot{\vec Q}^T \right) \\
  &= \frac{1}{2}\Tr\left( \dot{\vec Q}\dot{\vec Q}^T \right)
\end{align*}
We can notice that this Lagrangian is symmetric and invariant left and right actions, that is, if $L = \frac{1}{2}\Tr\left(\dot{\vec Q}\dot{\vec Q}^T\right)$, then \begin{align*}
  L(g\vec{Q}h, g\dot{\vec Q}h) &= \frac{1}{2}\Tr\left( g\dot{\vec Q}h(g\dot{\vec Q}h)^T \right) \\
  &= \frac{1}{2}\Tr\left( g\dot{\vec Q}hh^T\dot{\vec Q}g^T \right) \\
  &= \frac{1}{2}\Tr\left( \dot{\vec Q}\dot{\vec Q} \right)
\end{align*}
as $g, h \in \SO(3)$.
From Linear Algebra last year we saw that we can decompose a matrix using single value decomposition. That is, take a matrix $A$ and we can represent this as $U\Sigma V$ where $U, V \in O(3)$ and $\Sigma \in \diag^+(3)$. We want $U,V$ to be in $\SO(3)$ and so we now do the following. Take a decomposition of $Q = RAS$ and we know that $\det R = \pm 1$, if $\det R = 1$ leave it as it is, if $\det R = -1$, then we tag on an additional matrix,
$$ M = \begin{pmatrix}
  -1 & 0 & 0 \\ 0 & 1 & 0 \\ 0 & 0 & 1
\end{pmatrix} $$
to $R$ creating $R' = RM$ and similarly for $S' = RS$ if $\det S = -1$. Now we have the following decomposition, $Q = R'MAMS'$, noting that $R', S' \in \SO(3)$, $MAM \in \diag^+(3)$ and $M^2 = I$ and so this makes sense. Now I present a nice example,



\subsection{Noether Theory for general Euler-Poincare reduction}
Let $G$ be an arbitrary matrix Lie group and let $L$ be a left-invariant Lagrangian,
$$ L(hg, h\dot g) = L(g, \dot g) \quad \forall g, h \in G$$
with variational principle
$$ \d \int_{t_1}^{t_2} L(g, \dot g)dt = 0 $$
The reduced system is just $\left.L(hg, h\dot g)\right|_{h = g^{-1}} = L(g^{-1}g, g^{-1}\dot g) := \ell(\xi)$ where $\xi = g^{-1}g$ and $\xi \in \so(3)$.
\begin{nthm}[Euler-Poincare Noether Theorem]
  Corresponding to each one-parameter subgroup of $G$, $\chi(s)$ with $\chi(0) = e$ and $\chi_s (s) = \eta \in \mathfrak{g}$ there is a conserved quantity
  $$ \ip{\Ad^*_{g^{-1}} \pd \ell \xi}{\eta} = K $$
\end{nthm}
\begin{proof}
  Take a one parameter subgroup $\chi(s)$ and multiply this by the Lgrangian left,
  \begin{align*}
    \int_{t_1}^{t_2} L(\chi(s)g, \chi(s)\dot g) = \int_{t_1}^{t_2} L(g, \dot g) \\
  \end{align*}
  Now differentiate wrt s and set $s = 0$ (take first variation).
  $$ \int_{t_1}^{t_2} \left(\ip{\pd L g}{\chi_s(0)g} + \ip{\pd{L}{\dot g}}{\chi_s(0)\dot g}\right)dt = 0 $$
  Integrate by parts,
  $$ 0 = \int_{t_1}^{t_2} \ip{\pd L g - \dit \pd{L}{\dot g}}{\chi_s(0)g} + \left.\ip{\pd{L}{\dot g}}{\chi_s(0)g}\right|_{t_1}^{t_2} $$
  The first part is $0$ as they are just the Euler Legrange equations. Therefore,
  $$ \ip{\pd{L}{\dot g}}{\chi_s(0)g} = K $$
  But we can write $\chi_s(0)g = \eta g = gg^{-1}\eta g = g(g^{-1}\eta g) = g\Ad_{g^{-1}}\eta$. Therefore,
  $$ \ip{\pd{L}{\dot g}}{g\Ad_{g^{-1}}\eta} = K $$
  Now we need to add $\xi$. We want to deform $g_t(t, s)$ but not $g(t)$. Assume that $L(g(t), g_t(t, s)) = L(g^{-1}(t)g(t),  g^{-1}(t)g_t(t, s)) = \ell(g^{-1}(t)g_t(t,s))$. Now differentiate wrt $s$. Now we conclude,
  $$ \ip{\pd{L}{\dot g}}{g_{ts}(t, s)} = \ip{\pd{\ell}{\xi}}{g^{-1}g_{ts}(t, s)} $$
   and set $s = 0$ we get,
   $$ \ip{\pd{L}{\dot g}}{\d g_{t}} = \ip{\pd{\ell}{\xi}}{g^{-1}\d g_{t}(t)} $$
   Therefore,
   $$ \ip{\pd{L}{\dot g}}{\d g_{t}} = \ip{g\pd{\ell}{\xi}}{\d g_{t}(t)} $$
   That is,
   $$ g^{-1}\pd{L}{\dot g} = \pd \ell \xi $$
   Therefore,
   $$ K = \ip{g\pd \ell \xi}{g\Ad_{g^{-1}}\eta} = \ip{\pd \ell \xi}{\Ad_{g^{-1}}\eta} = \ip{\Ad^*_{g^{-1}}\pd \ell \xi}{\eta} $$
\end{proof}
NB! The conserved quantity is the constant that arises from integration by parts.

\begin{eg}
  Now apply this to Pseudo Rigid bodies.
\end{eg}