% !TEX root = ../notes.tex

\section{Summary and Further Work}

We have seen the baseline work for Geometric Mechanics and several related applications. Geometric Mechanics is a big field and so we have only been able to view a certain part of this subject. We can extend the work further to look at Hamiltonians, Lie-Poisson Brackets and their associated reduction. We will spend this final section recapping what we have done over the whole thesis and present some further work that can be done in this area. \\

\noindent
In the first chapter we saw the motivation behind the precursor to Euler-Poincar\'e reduction on Lagrangians - Lagrangian Mechanics. Lagrangian Mechanics lays the foundation of everything we have seen in this thesis. In many problems the equations we derive are very complicated and so we need some method to reduce the equations and make them easier to deal with. This method needs a framework to build off of, which is our Lie groups and Lie algebras. We explored these in Chapter 2. We saw how Lie groups and Lie algebras interweave and this led us to how we used them in Chapter 4. Before we indulged in Euler-Poincar\'e reduction, we took a step back and introduced many useful and simplifying tools in Geometric Mechanics. We met calculus of variations, the body-to-space map and the hat map. Putting this all together we reached Chapter 4 where we met Euler-Poincar\'e reduction and considered it for different types of Lagrangians, left-invariant, right-invariant and with symmetry breaking parameters. We also introduced the idea of a conserved quantity and studied Noether's Theorems. This was then put into practice in Chapter 5 where we looked at some real world examples of Geometric Mechanics, with the spherical pendulum, heavy top and pseudo-rigid bodies.\\

\noindent
As I mentioned above, what we have done is very basic. The area is very rich, looking at Daryl D. Holm's publication list can tell us that. Many of the ideas we have found can be generalised. For example, consider the swinging spring (a spherical pendulum on a spring), then the body-to-space map changes and we get $\vec x(t) = f(t)\vec R(t)\vec X$, where $f(t) : \R_{\ge 0} \to \R_{> 0}$. This changes the reduction to have $f$ involved, however as $\Im f$ is in the dilation group we end up with an Euler-Lagrange equation instead of an Euler-Poincar\'e equation. Furthermore, we can consider fluids and these lead into Geometric Fluid Mechanics, as I call it, which then have Lagrangians dependent on position as well as time. This complexifies the problem and we realise the Hamiltonians and Lie-Poisson brackets are a better way of solving these types of problems. As expected, these lead to a different type of Noether Theorem, namely Noether-Stokes Theorems. In a slightly different vein we can use Geometric Mechanics and Symplectic Geometry to derive Numerical Methods and Symplectic Integrators. I mentioned several times in the thesis that the work that we have done is considered on a continuous level, however, it can also be done on a discrete level via discrete differential geometry.\\

\noindent
In conclusion, I have spent the past year learning Geometric Mechanics with my supervisor and have realised that Geometric Mechanics is more than just an area of maths, it is a collection of tools that we can use in order to solve complex and industrially interesting problems that are pertinent to the modern world. Armed with this knowledge and the ability to learn more about this subject I am now in a position to move forward into a further study, PhD and to even produce publishable work about useful mathematical systems. 