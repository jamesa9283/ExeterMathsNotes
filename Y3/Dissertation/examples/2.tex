% !TEX root = ../notes.tex

\subsection{Heavy Top}

The Heavy Top problem is a rigid body with a fixed point. We are going to study its dynamics in the body frame. That means we consider, $x(t) = R(t)\vec X$ and $R(t) \in \SO(3)$. We know the potential energy of a point mass in a gravitational field is $V = mg\vec e_3 \cdot x(t)$. We can write the following Lagrangian for our problem,
$$ L(R, \dot R) = \int_\mathcal{B} \rho(\vec X) \left( \frac{1}{2}\norm{\dot R \vec X} - g\vec e_3 \cdot R\vec X \right)\,d^3\vec X. $$
\noindent
Now we can verify this Lagrangian is neither left nor right invariant.
\begin{exercise}
  Prove that $L(R,\,\dot R) \ne L(RR^{-1},\,\dot RR^{-1})$ and $L(R,\,\dot R) \ne L(R^{-1}R,\,\dot R^{-1}\dot R)$.
\end{exercise}
We know $\Oh = R^{-1}\dot R$, then we know $L = \frac{1}{2}\Ov \cdot \I\Ov - g\vec e_3 \cdot R\vec\b$ where $\b = \int_B \rho(\vec X)\vec X\, d^3\vec X$ and $\vec \b$ is the centre of mass in the body frame. (We could also use $\b = \vec X_b$). We define $\vec\Gamma = R^{-1}\vec e_3$
and so
$$\ell(\Oh,\,\vec\Gamma) = \frac{1}{2}\Ov(t) \cdot \I \Ov(t) - g\vec\Gamma(t) \cdot \vec X_b$$
% X_b is the center of mass
and we see that $\vec\Gamma(t)$, the gravitational force, is the symmetry breaking parameter. Now we use Hamilton's Variational Principle,
\begin{align*}
  \d\int_{t_1}^{t_2} \ell(\Oh,\,\vec\Gamma) \,dt &= 0\\
  \int_{t_1}^{t_2} \left(- \dit \I\Oh + \I\Ov\ti\Ov\right) \cdot \Lambda\,dt - \int_{t_1}^{t_2} g\vec X_b \cdot \d\vec\Gamma &= 0.
\end{align*}
The first integral contains the terms relating to the rigid body and the second relates to the symmetry breaking term. We also define $\Lh = R^{-1}\d R$. We now look to $\d\vec\Gamma$,
\begin{align*}
  \d\vec\Gamma &= \d R^{-1}\vec e_3 \\
  &= -R^{-1}\d R^{-1}\vec e_3 \\
  &= -\Lh \vec\Gamma\\
  &= - \Lv\ti\vec\Gamma.
\end{align*}
We can now write the following,
$$ \int_{t_1}^{t_2} \left(- \dit \I\Oh + \I\Ov\ti\Ov\right) \cdot \Lambda\,dt + \int_{t_1}^{t_2} -g\ip{\vec X_b}{\Lv\ti\vec\Gamma} = 0 $$
and so we can write,
$$ \int_{t_1}^{t_2} \left[ -\dit\I\Oh + \I\Oh\ti\Oh \cdot \Lv - g(\vec X_b \ti \Gv) \cdot \Lv \right] = 0. $$

\noindent
Now we get,
$$ \I\Od = \I\Oh \ti \Oh + g\Gv\times \vec X_b $$
$$ \Gd = \Gv \ti \Oh. $$

\noindent
We now seek to find the conserved quantity for this system, which we will see to be some type of angular momentum.
\begin{exercise}
  Find $\dit \left( R\I\Oh \right) = gR(\vec\Gamma \ti \vec X_b) \ne 0$. Find what kind of angular momentum is conserved. Find something that is conserved.
\end{exercise}

