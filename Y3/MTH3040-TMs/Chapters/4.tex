% !TEX root = ../notes.tex

\subsection{Subspaces}
If $(X, \T)$ is a topological space and $A$ is any non-empty subset of $X$, then there is a natural way to make $A$ into a topological space,

\begin{ndefi}[Subspace Topology]
  With the above notation, we define
  $$ \T_A = \{U \cap A : U \in \T\} $$
  Thus a subset $V \sub A$ is open in $A$ if and only if there exists an open set $U$ in $X$ with $V = U \cap A$. We call $\T_A$ the subset topology on $A$ induced by $X$.
\end{ndefi}

\noindent
We must check that $\T_A$ satisfies $(T1)-(T3)$.
\begin{enumerate}
  \item $\vn = \vn \cap A$ and $A = A \cap X$ and so they are open sets in $A$.
  \item If $V_1$ and $V_2$ are open in $A$, there are open sets $U_1, U_2 \in X$ where $V_1 = U_1 \cap A$ and $V_2 = U_2 \cap A$. Therefore, $V_1 \cap V_2 = (U_1 \cap A)\cap (U_2 \cap A) = (U_1 \cap U_2)\cap A$ where $U_1 \cap U_2$ is open and so $V_! \cap V_2$ is open.
  \item If $V_i$ where $i \in \cI$ is a family of open sets in $A$, then for each $i$ there is a $U_i \in X$ with $V_i = U_i \cap A$ and,
  $$ \bigcup_{i \in \cI} V_i = \bigcup_{i \in \cI}(U_i \cap A) = \left( \bigcup_{i \in \cI} U_i \right) \cap A $$
  Then since $\left( \bigcup_{i \in \cI} U_i \right)$ is open in $X$, then $\bigcup_{i \in \cI} V_i$ is open in $A$.
\end{enumerate}

\noindent
Here is a remark,
\begin{remark}
   If $(X, d)$ is a metric space, then we have a metric topology on $X$, where a subset $U$ of $X$ is open if and only if for each $x \in U$ we have $B_\e (x) \sub U$ for some $\e > 0$. If $A$ is a non-empty subset of $X$ then the subspace topology on $A$ induced by $X$ is the sae as the topology of the restriction.
\end{remark}

\begin{nlemma}
  Let $A$ be a non-empty subset of a topological space $X$ and let $i : A \to X$ be the inclusion map. Then,
  \begin{enumerate}
    \item $i$ is continuous
    \item For any topological space $Z$ and any function $g : Z \to A$, $g$ is continuous of and only if $i \circ g : Z \to X$ is continuous. This is the \textbf{universal property} for topological spaces.
      \[\begin{tikzcd}
  	Z \\ A & X
  	\arrow["{i \circ g}", from=1-1, to=2-2]
  	\arrow["i"', hook, from=2-1, to=2-2]
  	\arrow["g"', from=1-1, to=2-1]
  \end{tikzcd}\]
    \item The subspace topology on $A$ is the only topology for which (ii) holds for all functions $g$.
  \end{enumerate}
\end{nlemma}
\begin{proof}
  (i), We first prove that $i : A \to X$ is continuous. That is we need to prove for all open sets $U \in X$, $i^{-1}(U)$ is open in $A$. Therefore, take $U \sub X$, then $i^{-1}(U) = \{a \in A : i(a) \in U\} = \{a \in A : a \in U\} = U \cap A$, which is an open set in $A$, by the definition of $\T_A$.\\

  \noindent
  (ii), We just want to show that $g$ is continuous iff $i \circ g$ is continuous.
    \[\begin{tikzcd}
  Z \\ A & X
  \arrow["{i \circ g}", from=1-1, to=2-2]
  \arrow["i"', hook, from=2-1, to=2-2]
  \arrow["g"', from=1-1, to=2-1]
  \end{tikzcd}\]
  For $U \sub X$, then
  \begin{align*}
    (i\circ g)^{-1}(U) &= \{z \in Z : i(g(z)) \in U\} \\
    &= \{z \in Z : g(z) \in U\}
  \end{align*}
  Since $g(z) \in A$, this set can be written as $\{z \in Z : g(z) \in U \cap A\} = g^{-1}(U \cap A)$. If $g$ is continuous, then we could also write $g^{-1}(V)$ is open for every open $V \sub A$, that is also the same as $g^{-1}(U \cap A)$ is open for every open $U \sub A$. We know that is then equivalent to $(i \circ g)^{-1}(U)$ is open for every $U \sub A$, which gives the required result.\\

  \noindent
  (iii), Let $\T'$ be a topology on $A$ such that (ii) holds. So $A$ is a topological space in two ways, $(A, \T_A)$ and $(A, \T')$, we want to prove that $\T_A = \T'$. To do this we make an appropriate choice of $Z$ and $g$. We can choose a function where even though we don't know what $A$ is, we can still make this work. First, let $g = \id_A : (A, \T') \to (A, \T')$. $g$ is continuous as $U \in \T'$, then $g^{-1}(U) = U \in \T'$ is open. By (ii)$\implies$,
  $g \circ i$ is continuous. For open $U \sub X$, $(i \circ g)^{-1}(U)$ is open in $(A, \T')$,
  \begin{align*}
    (i \circ g)^{-1}(U) &= \{ a \in A : i \circ g (a) \in U \}\\
    &= \{a \in A : a \in U\}
    &= U \cap A
  \end{align*}
  Hence, for each $U \sub X$, $U \cap A \in \T'$. Therefore, $\T_A \sub \T'$. We now need the other inclusion. Secondly, let $g = \id_A : (A, \T_A) \to (A, \T')$. Now we consider $i \circ g = i : (A, \T') \to X$ and this is continuous by (i). Therefore, by $(ii) \Leftarrow$, $g$ is continuous. If $V \sub A$ is open in $\T'$ then $g^{-1}(V) = V$ is again in $\T_A$. Therefore, $\T' \sub \T_A$. Hence, $\T' = \T_A$.
\end{proof}

We've seen $\T_A = \{U \cap A : U \in \T\}$ is the only topology such that $\forall g : Z \to A$, $g$ is continuous if and only if $i \circ g$ is continuous. If we try something else,
\[\begin{tikzcd}
	& Z \\
	A & X
	\arrow[from=1-2, to=2-2]
	\arrow[from=2-1, to=2-2]
\end{tikzcd}\]
This doesn't make sense, but what about this,
\[\begin{tikzcd}
	A & X \\
	& Y
	\arrow["i", hook, from=1-1, to=1-2]
	\arrow["h", from=1-2, to=2-2]
	\arrow["{h \circ i}"', dashed, from=1-1, to=2-2]
\end{tikzcd}\]
But there wouldn't be a universal property like that. It is not true that $\forall h : X \to Y$, $h$ is continuous if and only if $h \circ i$ is continuous.

% Lecture 9
\subsection{Products of Topological Spaces}

\begin{ndefi}[Product Topology]
  Let $X$, $Y$ be topological spaces. The \textbf{product topology} on $X\ti Y$ is the topology with basis,
  $$ \mathcal{B} = \{U \ti V : U \text{ open in } X, V \text{ open in } Y\} $$
  That means that a subset of $A$ of $X\to Y$ is open if and only if it is the union of subsets of the form $U\times V$ with $U$ and $V$ open in $X$, $Y$ respectively. Equivalently, $A$ is open for each point $(x, y) \in A$, if there are open sets such that $x \in U \sub X$ and $y \in V \sub Y$.
\end{ndefi}

\begin{remark}
   If we have $\mathcal{B}_X$ and $\mathcal{B}_Y$ for the topologies on $X$, $Y$ then we get the same open sets in $X\ti Y$ if we restrict $U$, $V$ to be sets in $\mathcal{B}_X$, $\mathcal{B}_Y$ respectively.
\end{remark}

\noindent
So a disc in $\R^2$ is an open set, since it can be filled with open rectangles. This isn't finitely many rectangles, but still it can be done. We now check that the product topology gives a usual topology on $\R^2$ formally,
\begin{nlemma}
  Let $X = Y = \R$. Then the product topology on $\R^2$ agrees with the usual topology on $\R^2$.
\end{nlemma}
\begin{proof}[Proof Sketch.]
  The bounded open intervals $B_\e(x) = (x - \e, x + \e)$ for all $x \in \T$ and $\e > 0$ for a basis for the usual topology on $\R$. So $U \sub \R^2$ is open in the product topology if, for all $(x, y) \in U$ there is $\e, \d > 0 $ such that,
  $$ (x, y) \in B_\e(x) \ti B_\d(y) \sub U $$
  Replacing $\e$ and $\d$ with $\min (\e, \d)$ we can assume that $\d = \e$. This agrees with the topology on $\R^2$ given by the metric $d$ where
  $$ d((x, y), (x', y')) = \max(|x - x'|, |y - y'|). $$
  It is not difficult to check that $d$ is a metric on $\R^2$ and that it is Lipschitz equivalent to the Euclidean metric. So it gives the usual topology.
\end{proof}

\noindent
We really should check the product topology gives us a topology.
\begin{enumerate}
  \item $\vn = \vn \ti \vn$ and so it is open in the product topology as it is open in $X$ and $Y$. Similarly for $X\ti Y$ as $X$ is open in $X$ and $Y$ is open in $Y$.
  \item Let $W_1, W_2 \sub X\ti Y$ be open sets in the product topology. We must show that $W_1 \cap W_2$ is an open set in the product topology. Let $(x, y) \in W_1 \cap W_2$, then as $(x, y) \in W_1$ we can find some set $U_1 \ti V_1 \in \mathcal{B}$ with $(x, y) \in U_1\ti V_1 \sub W_1$. Similarly $(x, y) \in U_2 \cap V_2 \sub W_2$.
  Let $U = U_1 \cap U_2$ and $V = V_1 \cap V_2$, which are open. Then, $(x, y) \in U\ti V \sub W_1 \cap W_2$. With $U \ti V \in \mathcal{B}$. So $W_1 \cap W_2$ is the union of sets in $\mathcal{B}$, therefore it is open in the product topology.
  \item Let $W_i$ for $i \in \cI$ be a family of open sets in $X \ti Y$. We must show $\bigcup_{i \in\cI} W_i =: W$ is open. Let $(x, y) \in W$, then $(x, y) \in W_j$ for some $j \in \cI$ where $W_j$ s open. Therefore $(x, y) \in U \ti V \sub W_j$ for some open $U \sub X$ and $V \sub Y$. Then $U \ti V \sub W$,
  so $W$ is open in the product topology.
\end{enumerate}

\begin{remark}
   We can define the product topology of finitely many topological spaces by just iterating the construction. (The product of infinitely many topological spaces).
\end{remark}

\begin{nlemma}
  Let $X$, $Y$ be topological space and let $p_X : X \ti Y \to X$ and $p_Y : X \ti Y \to Y$ be the projection functions:
  $$ p_X((x, y)) = x \qquad p_Y((x, y)) = y $$
  For any topological space $Z$ and any function $f : Z \to X \ti Y$, $f$ is continuous if and only if $p_X \circ f$ and $p_Y \circ f$ are continuous.
    \[\begin{tikzcd}
  	Z \\
  	& {X\times Y} & X \\
  	& Y
  	\arrow["f", from=1-1, to=2-2]
  	\arrow["{p_X}", from=2-2, to=2-3]
  	\arrow["{p_Y}", from=2-2, to=3-2]
  	\arrow[bend left, dotted, from=1-1, to=2-3]
  	\arrow[bend right, dotted, from=1-1, to=3-2]
  \end{tikzcd}\]
  In particular $p_X$ and $p_Y$ are continuous.
\end{nlemma}
\begin{proof}
  If $U \sub X$, then,
  \begin{align*}
    (p_X\circ f)^{-1}(U) &= \{z \in Z : p_X\circ f(z) \in U\} \\
    &= \{z \in Z : f(z) \in U \ti Y\} = f^{-1}(U \ti Y)
  \end{align*}
  ($\Longrightarrow$) If $f$ is continuous, let $U \sub X$ be open. We consider $(p_X \circ f)^{-1}(U) = f^{-1}(U \ti Y)$, this is open in $Z$, as $f$ is continuous and $U \ti Y$ is open in $X \ti Y$. Hence $p_X \circ f$ is continuous. Similarly, $p_Y \circ f$ is continuous.\\

  \noindent
  ($\Longleftarrow$) Suppose that $p_X \circ f$ and $p_Y \circ f$ are continuous. We must show that if $W \sub X\ti Y$ is open, then $f^{-1}(W)$ is open in $Z$. Since any union of open sets in $Z$ is open it's enough to take $W = U \ti V$ with $U \sub X$ and $V \sub Y$ open. Now we consider $f^{-1}(U \ti V)$,
  \begin{align*}
    f^{-1}(U \ti V) &= f^{-1}((U \ti Y) \cap (X \ti V))\\
    &= f^{-1}((U \ti Y)) \cap f^{-1}((X \ti V))\\
    &= (p_X \circ f)^{-1}(U) \cap (p_Y \circ f)^{-1}(V)
  \end{align*}
  This is a union of open sets, and so $f^{-1}(U \ti V)$ is open in $Z$.\\

  \noindent
  Finally, we can see that $p_X$ and $p_Y$ are continuous. Let $Z = X \ti Y$ and $f = \id_{X\ti Y}$, then $(p_X \circ f) (x, y) = (p_X \circ \id_{X \ti Y})(x, y) = p_X(x, y) = x$. Hence, as $f$ is continuous, then $p_X$ is continuous. Similarly for $p_Y$.
\end{proof}