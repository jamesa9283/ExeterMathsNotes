% !TEX root = ../notes.tex

\section{Introduction}
\subsection{Motivation}

In this module we will look at ways to generalise Real Analysis.
\begin{enumerate}
  \item Metric Spaces
  \item Topological Spaces
  \item Measure Spaces
\end{enumerate}
A key idea in Real Analysis is continuity, a function $f : \R \to \R$ is continuous if, given $a \in \R$ given $\e > 0$ there exists some $\d > 0$ so that,
$$ |x - a| < \d \implies |f(x) - f(a)| < \e $$
% continuity diagram
We have seen a version of this for $\R^n \to \R^m$ or $\C \to \C$. This can interpreted as a notion of a distance, we can ensure that the distance between $f(x)$ and $f(a)$ be less than $\e$. Here the distance between real numbers is $|x - y|$. This leads to metric spaces is a set where we have a distance function $d_X(a, b)$ for any points $a, b \in X$.\\

\noindent
Another way to interpret the continuity of $f : \R \to \R$ is to day that for any $U$ in
$\R$, the set,
$$ f^{-1}(U) := \{x \in \R : f(x) \in U\} $$
is also open.
% diagram for U on x^2
We may ask what happens if we choose a $U$ such that $f^{-1}(U) = \varnothing$, but we say that the empty set is open.\\

\noindent
We can talk about continuity without talking about distances, provided that we know what we mean by the idea of open sets. Open sets may not be defined by distance. A space together with a collection of open subsets is a topological space. Metric spaces are topological spaces with a idea of distance.\\

\noindent
Measure spaces are related to length of a subset, and also integration. These are linked since if $A$ is a subset of $\R$ of length $\ell$, then,
$$ \ell = \int_{\R} 1_A(x)\, dx $$
where $1_A : \R \to \R$ is the indicator function,
$$ 1_A \begin{cases}
  1 & \text{if $x \in A$}\\
  0 & \text{if $x \notin A$}
\end{cases} $$
This unproblematic if we have $A = [a, b]$, then we can integrate this nicely,
% add graph
However, if $A = \Q$ it is not clear that we can make sense of this `length' of $\Q$, and the integral is not defined (as a Riemann Integral). Measure Theory provides the theoretical framework for assigning a length to most (but not all, the measureable ones work) subsets of $\R$ and making corresponding integral as Lesbegue integrals. It turns out that $\Q$ has `length' of $0$, so there are way more irrational numbers, and $\Q$ is countable.

\newpage
\subsection{Review of Real Analysis}
For real numbers $a \le b$, we have the open interval,
$$ (a, b) = \{x \in \R : a < x < b\} $$
and closed interval,
$$ [a, b] = \{x \in \R : a \le x \le b\} $$
We can also have the mixed intervals, $(a, b]$ or $(a, \infty)$. \\

\noindent
In general, a subset $U$ is open, if for each $a \in U$ there is some $\e > 0$ such that $(a- \e, a + \e) \subset U$ ($U$ does not contain it's boundary, every point is interior). A closed set is a set where it's complement is open. The empty set and $\R$ are clopen, open and closed.\\

\begin{nlemma}[Triangle Inequality]
  For some $a, b \in R$,
  $$ |a + b| \le |a| + |b| $$
\end{nlemma}
and we can extend this to say $|a - b| \ge ||a| - |b||$. \\


\noindent
Let $A \subset \R$. An upper bound is a number $u$ such that $a \le u$ for all $a \in A$. If $u$ is an upper bound of $a$ then it has many upper bounds, if at least one exists, the set is bounded. A least upper bound or supremum for $A$ is a number $u$ such that,
\begin{enumerate}
  \item $a \le u$ for all $a \in A$
  \item if $u_* < u$ then there is some $a \in A$ with $a > u_*$
\end{enumerate}
If $A$ has a least upper bound $u$, then $u$ might or might not be in $A$. There are similar definitions for greatest lower bound or infimum. A set is bounded, if it is bounded above and below, or there is some $M$ such that $|a| \le M$ for all $a \in A$. An important property of the real numbers is the completeness property: every non-empty set of real numbers which is bounded above has a least upper bound.\\

We say that a sequence converges to $a$, if given $\e > 0$ thiere is some $N \in \N$ such that $|a_n - a| < \e$ for all $n > N$. Then $a$ is the limit of a sequence. A sequence is a bounded if $|a_n| < M$ for all $n$. If $a_n$ is bounded which is monotonically increasing, then it must converge, same for monotonically decreasing. In general a sequence that is bounded, doesn't have to converge. However, a bounded sequence always has a convergence subsequence. \\

\noindent
A function is continuous at a point $a \in \R$, for all $\e > 0$ there is some $\d > 0$ so that $|f(x) - f(a)| < \e$ whenever $|x - a| < \d$. We say that $f$ is continuous if it holds for every $a$. If $f, g : \R \to \R$, then $f \pm g$, $fg$, $\frac{f}{g}$ ($g \ne 0$) are all continuous. Suppose we have a continuous function on a closed and bounded interval
\begin{nthm}[Intermediate Value Theorem]
  Let $f: [a, b] \to \R$ be continuous, for any $v$ between $f(a)$ and $f(b)$, there is at least one $x \in [a, b]$ with $f(x) = v$.
\end{nthm}
% IVT diagram

\begin{nthm}
  Let $f: [a, b] \to \R$ be continuous. Then $f(x)$ is bounded and attains its bounds, i.e. $f$ has a (finite) maximum $M$ and minimum $m$ in $[a, b]$. More precisely $x_{\text{min}}$ and $x_{\text{max}} \in [a, b]$ so that $m = f(x_{\text{min}}) \le f(x) \le f(x_{\text{max}})$ for all $x \in [a, b]$.
\end{nthm}
% Bounds diagram.

\section{Metric Spaces}
We firstly define a metric space,
\begin{ndefi}[Metric Space]
  A metric space, $(X, d)$ consists of a non-empty set $X$ and a function $d : X \ti X \to \R$ satisfying,
  \begin{enumerate}
    \item $d(x, y) \ge 0$ for all $x, y \in X$ and $d(x, y) = 0 \iff x = y$
    \item $d(x, y) = d(y, x)$ for all $x, y \in X$ (symmetry)
    \item $d(x, z) \le d(x, y) + d(y, z)$ (triangle inequality)
  \end{enumerate}
\end{ndefi}

Here are a load of examples,
\begin{eg}
  Take, $X = \R$ and $d_\R(x, y) = |x - y|$. Now, we can probably see normally that the three axioms hold. The first is how we define $|\cdot|$, then $|x - y| = |(-1)(y - x)| = |y - x|$ and the third is the triangle inequality.
\end{eg}
and now for $\R^m$,
\begin{eg}
  If we let $\R^m$ and $d_{\R^m}(x, y) = \sqrt{\sum_{j=1}^n |x_j - y_j|^2}$ where $\vec x, \vec y  \in \R^m$. The axioms hold, as if $d_{\R^m}(\vec x, \vec y) = 0$, then we require that $x_j = y_j$ for all $j$ and so $\vec x = \vec y$. For the second, we can use a similar argument to before as $|x_j - y_j|= |y_j - x_i|$. For the triangle inequality for this metric space, we need to use the Cauchy Schwartz inequality,
  $$ \left| \sum_{j=1}^n a_j\bar{b_j} \right| \le \left(\sum_{j=1}^n |a_j|^2\right)\left(\sum_{j=1}^n |b_j|^2\right) $$
  that is $|\vec a \cdot \bar{\vec b}| < |\vec a|^2|\vec b|^2$.
\end{eg}

We now can look at the taxicab metric,
\begin{eg}
  Take $X = \R^m$ and $d'_{\R^m}(\vec x, \vec y) = \sum_{j=1}^n |x_{i} - y_j|$ for $x, y \in \R^m$. The first two are trivial for $d'$, but the third easier than before,
  $$ \sum_{j=1}^n |x_j - z_j| = \sum_{j=1}^n |x_j - y_j - (y_j - z_j)| \le \sum_{j=1}^n |x_j - y_j| + \sum_{j=1}^n |y_j - z_j| = d'(\vec x, \vec y) + d'(\vec y, \vec z)$$
\end{eg}
For an example not $\R^m$,
\begin{eg}
  Take any $X$ that is non-empty, then
  $$ d(x, y) = \begin{cases}
    1 & \text{if $x \ne y$}\\
    0 & \text{if $x = y$}
  \end{cases} $$
  The first two axioms are clear, then for the third consider $x = z$,
  $$ d(x, z) = 0 \le d(x, y) + d(y, z) $$
  and this is always true. If $x \ne z$, then,
  $$ d(x, z) = 1 \le d(x, y) + d(y, z) $$
  if $x \ne z$, then either $x \ne y$ or $y \ne z$, so the above holds.
\end{eg}

Now for something more abstract,
\begin{eg}
  Consider $\cc[0, 1]$ and let the metric be, $d(f, g) = \max\{f(t) - g(t) : t \in [0,1] \}$. Does this metric make sense? Are they bounded / why does this maximum make sense. This makes sense because of a Theorem in the last lecture. The first two of the conditions follow nicely, then the third,
  \begin{align*}
    |f(t) - h(t)| &= |(f(t) - g(t)) + (g(t) - h(t))|\\
    &\le |f(t) - g(t)| + |g(t) - h(t)|\\
    &= d(f, g) + d(g, h)
  \end{align*}
  and so taking the maximum, we can get that $d(f, h) \le d(f, g) + d(g, h)$.
\end{eg}

We can remark, that this is not the only way to consider the distance between two functions, we could have integrated.

\begin{ndefi}[Subspace]
  A subspace of a metric space $(X, d_X)$, is a non-empty subset $Y$ together with the metric $d_Y$ by restricting $d_X$ to $Y$.
  $$ d_Y(y, y') = d_X(y, y') \qquad \forall y, y' \in Y $$
\end{ndefi}
This is clearly a metric space as if the conditions hold for $X$, they will then hold for $Y$.

\subsection{Continuity in Metric Spaces}
We can talk nicely about continuity in metric space, in a rather obvious way once we realise it's all about distance,
\begin{ndefi}[Limit]
  Let $(X, d)$ be a metric space, then let $(a_n)$ be a sequence of points in $X$. For some $a \in X$ we say that $(a_n)$ converges to $a$, written $a_n \to a$ as $n \to \infty$ if, for any real number $\e > 0$ there exists some $N \in \N$ with $d(a_n, a) < \e$ for all $n > N$. We say that $a$ is the limit of the sequence.
\end{ndefi}

\noindent
This is just a copy of the definition of a limit, just with our metric placed in. Here is an interesting quirk, if we take the discrete metric, then the sequence $\left( \frac{1}{n} \right)$ then this does not converge to zero. For, if we choose $\e > 0$ with $\e < 1$, then $d\left(\frac{1}{n}, 0 \right) > \e$

\begin{ndefi}[Continuity]
  Let $(X, d_X)$ and $(Y, d_Y)$ be metric spaces, then $f : X \to Y$. For $a \in X$, we say that $f$ is continuous at $a$ if, given $\e > 0$, there is some $\d > 0$ so that $d_Y(f(x), f(a)) < \e$ for all $x \in X$ with $d_X(x, a) < \d$. We say $f$ is continuous if it is continuous for every $a$.
\end{ndefi}

We can prove that in the discrete metric then any function $f : X \to Y$ is convergent where $X$ and $Y$ have the discrete metric, just take $\d = 1$.

\subsection{Opens Sets}
We can consider balls, as we have a distance metric we can move forwards to open sets and the required analytic tools.
\begin{ndefi}[Open Ball]
  Let $(X, d)$ be a metric space, for any $a > 0$ and any $a \in X$, the set
  $$ B_\e (a) = \{x \in X : d(x, a) < \e\} $$
  is called an open ball in $X$ of radius $\e$ and center $a$.
\end{ndefi}

\noindent
As a sanity check, when $X = \R$ we get an interval, $(a - \e, a + \e)$
% interval
and with $X = \R^2$ or $\C$, then we see we get an open disc
% open disc

\begin{ndefi}[Open Set]
  A subset $U$ of a metric space $X$ is open if, for every $x \in U$ there is some $\e > 0$ such that
  $$ B_\e(x) \subset U $$
  A subset $V$ is closed if $X \setminus V$ is open.
\end{ndefi}

\noindent
By convention, $\varnothing$ is open and now we prove that the epsilon ball is open.
\begin{nprop}
 For any $a \in X$ and every $\e > 0$ the set $B_\e(a)$ is an open set in $X$.
\end{nprop}
% diagram
\begin{proof}
  Let $x \in B_\e(a)$, then we need to find a $\d > 0$ such that $B_\d(a) \subset B_\e(a)$. Take $\d = \e - d(x, a)$. Then $\e > 0$ and if $y \in B_\d(a)$ then $d(y, a) \le d(y, x) + d(x, a) < \d + d(x, a) = \e$. Thus $y \in B_\e(a)$. This holds for every $y \in B_\d(a)$ and so $B_\d(a) \subset B_\e(a)$.
\end{proof}

\noindent
Here's a slight quirk, if we consider $X$ and $Y \subset X$. If we consider a $U \subset Y$ which is open, this need not be open in $X$. Consider $Y = [0, 1] \subset \R$, and $B_{\frac{1}{2}}(0)$ as our open set, which is just $\{x \in [0, 1] : |x - 0| < \frac{1}{2}\}$. However, in $\R$ this subset is $\left[ 0, \frac{1}{2} \right)$.

\begin{nprop}
   Let $U$ and $V$ be open sets in the metric space $(X, d)$. Then $U \cap V$ is an open set.
\end{nprop}
\begin{proof}
  If $x \in U \cap V$, then there are $\e_1, \e_2 > 0$ such that $B_{\e_1}(x) \subset U$ and $B_{\e_2}(x) \subset V$ and so we just choose $\e = \min(\e_1, \e_2)$. Then $B_\e(x) \subset U \cap V$.
\end{proof}

\noindent
Then by induction we can generalise this,
\begin{nprop}
  The intersection of any finite family of open sets is open, ie. if $n \ge 0$, then $U_1, \dots, U_n$ are open sets then $U_1 \cap U_2 \cap \dots \cap U_n$ is an open set.
\end{nprop}

\noindent
We often write this to mean the above intersection,
$$ \bigcap_{i=0}^n U_i$$
The same works for unions, but we can say more. Suppose we have a family of open sets, indexed by some set $\mathcal{I}$. This means for every $i \in \mathcal{I}$ we have an open set $U_i \subset X$. The indexing set doesn't need to be finite.

\begin{nprop}
   If $U_i$, $i \in \mathcal{I}$ is a family of open sets $\bigcup_{i\in\mathcal{I}} U_i$ is open.
\end{nprop}
\begin{proof}
  Let $U = \bigcup_{i \in \mathcal{I}} U_i$. We need to show that $U$ is open. Let $x \in U$, then $x \in U_i$ for some $i \in \mathcal{I}$. As $U_i$ is open, there is some $\e > 0$ with $B_\e(x) \subset U_i$. As $U_i \sub U$, we have $B_\e(x) \sub U$. Hence $U$ is open.
\end{proof}

\noindent
The intersection of infinitely many open sets, need not be open. Consider,
$$ \bigcap_{n = 1}^\infty \left( -\frac{1}{n}, \frac{1}{n} \right) = \{0\}$$
which is then closed.

Now let us redefine the continuity and convergence in terms of these open sets,
\begin{ndefi}[Limit]
  Let $(a_n)$ be a sequence in a metric space, $(X, d)$ and let $a \in X$. Then $a_n \to a$ as $n \to \infty$ if and only if the following hold,
  \begin{enumerate}
    \item for every open set $U$ containing $a$ there is some $N \in \N$ such that $a_n \in U$ for all $n > N$
  \end{enumerate}
\end{ndefi}
\begin{proof}
  First suppose $a_n \to a$ as $n \to \infty$. We must show that the condition holds. Let $a \in U$, $U$ is open. Then there is some $\e$ with $B_\e(a) \subset U$. As $a_n \to a$ there exists $N \in \N$ with $d(a_n, a) < \e$ for all $n > N$. But then $a_n \in B_\e(a) \sub U$ for all $n > N$ as required.

  Conversely, suppose the conditon holds, then we must show that $a_n \to a$. Let $\e > 0$. Then $B_\e(a)$ is an open set containing $a$, so by the condition there is some $N$ with $a_n \in B_\e(a)$ is an open set containing $a$, so there is some $N$
 with $a_n \in B_e(a)$ for all $n > N$. Hence $d(a_n, a) < \e$ for all $n > N$. This shows $a_n \to a$.
\end{proof}

We can do a similar thing for continuity.
\begin{nprop}
   Let $(X, d_X)$ and $(Y, d_Y)$ be metric spaces and let $f : X \to Y$. Then $f$ is continuous if and only if, for every open set $U$ in $Y$, the set $\{x \in X : f(x) \in U\}$ is an open set in $X$.
\end{nprop}

\noindent
We often use the notation $f^{-1}(U)$ for the set $\{x \in X : f(x) \in U\}$. This is the preimage of the set $U$.  We use this notation even if there is no actual function $f^{-1}$.

\begin{proof}
  Suppose $f$ is continuous, let $U \sub Y$ be open. We must show that $f^{-1}(U)$ is open. If $f^{-1}(U) = \varnothing$, then $f^{-1}(U)$ is open. Let $x \in f^{-1}(U)$, then $f(x) \in U$. Since $U$ is open, there is some $\e > 0$ such $B^Y_\e(f(x)) \sub U$ (with metric $Y$). Since $f$ is continuous, there is some $\d > 0$ so that $d_Y(f(x'), f(x)) < \e$ for all $x'$ such that $d(x, x') < \d$.
  If $x' \in B_\d^X(x)$ then $f(x') \in B^Y_\e(f(x)) \sub U$ and so $x' \in f^{-1}(U)$ and $B^X_\d(x) \sub f^{-1}(U)$. So $f^{-1}(U)$ is open.\\

  Conversely suppose $f^{-1}(U)$ is open for all open $U \sub Y$. Let $x \in X$ and $\e > 0$. Then $U = B^Y_\e(f(x))$ is an open set in $Y$, then $x \in f^{-1}(U)$, which is open in $X$. So there is some $\d > 0$ with $B_\d^Y(x) \sub f^{-1}(U)$. Therefore for all $x' \in B_\d(x)$ where $x' \in f^{-1}(U)$ and so $f(x)\in B_\e^Y(f(x))$, that is forall $x'$ with $d^X(x', x) < \d$ and so we have
  $$ d_Y (f(x'), f(x)) < \e $$
  Hence $f$ is continuous.
\end{proof}
