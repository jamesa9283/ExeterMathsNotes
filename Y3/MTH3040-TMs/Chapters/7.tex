% !TEX root = ../notes.tex

\subsection{Connected Spaces}
Intuitively, a connected topology space doesn't fall apart into two or more pieces. Therefore if we would expect $(0, 1)$ to be connected, but the union $(0, 1)\cup (2, 3)$ is disconnected. However if we take the topologists sine curve,
$$ \{(x, \sin(1/x)) : x > 0\} \cup \{(0, y) : -1 \le y \le 1\} $$
it is less obvious if this should be connected. It turns out that this is connected, but not path-connected.

\begin{ndefi}[Connected]
  A topological space $X$ is connected if there is no surjective continuous function $f : X \to \{0, 1\}$ (where $\{0, 1\}$ has the discrete topology). Otherwise it is disconnected.\\

  \noindent
  We say a non-empty subspace $Y$ of $X$ is connected if it is connected as a topological space with its subspace topology induced by $X$.
\end{ndefi}

\begin{eg}
  $X = (0, 1)\cup (2, 3)$ is disconnected, since the $f : X \to \{0, 1\}$ can be defined as,
  $$ f(x) = \begin{cases}
    0 & x \in (0, 1)\\
    1 & x \in (2, 3)
  \end{cases} $$
  This is surjective and continuous, the preimages $f^{-1}(\{0\}) = (0, 1)$ and $f^{-1}(\{1\}) = (2, 3)$ are both open in $X$, in addition to the basic open set.
\end{eg}

\begin{nthm}\label{thm:intCon}
  Any interval $(a, b)$ with $a < b$ is connected.
\end{nthm}
\begin{proof}
  We will use the Intermediate Value Theorem, whose proof ultimately depends on the completeness of $\R$. Suppose for a contradiction there is a surjective continuous function $f: (a, b) \to \{0, 1\}$ and let $A = f^{-1}(0)$ and $B = f^{-1}(1)$. Then $A, B \ne \vn$ as $f$ is surjective, and $A \cup B = (a, b)$. Also $A, B$ are open since $f$ is continuous. Pick $a_0 \in A$ and $b_0 \in B$. We may assume $a_0 < b_0$.\\

  \noindent
  Then $f$ restricts to a continuous function on the closed interval $[a_0, b_0]$ with $f(a_0) = 0$ and $f(b_0) = 1$. By IVT there is some $x \in [a_0, b_0]$ with $f(x) = \frac{1}{2}$. This is a contradiction as $f$ only takes the values of $0$ or $1$.
\end{proof}

\begin{eg}
  $\Q$ is disconnected. Indeed we can defined $f : \Q \to \{0, 1\}$ by,
  $$ f(x) = \begin{cases}
    0 & x^2 < 2 \\
    1 & x^2 > 2
  \end{cases} $$
  This is well defined, (there is no $x \in \Q$ such that $x^2 = 2$) and it's surjective. The sets,
  $$ f^{-1}(\{0\}) = (- \sqrt 2, \sqrt 2) \cap \Q $$
  $$ f^{-1}(\{1\}) = ((-\infty, -\sqrt 2) \cup (\sqrt 2, \infty))\cap \Q $$
  are open in $\Q$, so $f$ is continuous.
\end{eg}

\begin{ndefi}[Partition]
  A partition of a topological space $X$ is a pair of non-empty open subsets $A, B$ such that $A \cup B = X$ and $A \cap B = \vn$.
\end{ndefi}

\begin{remark}
   If $A, B$ is a partition of $X$ then $A = X \sm B$ and $B = X\sm A$, so $A$ and $B$ are closed aswell as open.
\end{remark}

\begin{remark}
  In Theorem \ref{intCon} we showed $(a, b)$ isn't connected by showing that there is no partition of $(a, b)$. We now show this in general.
\end{remark}

\begin{nlemma}
  For a topological space $X$, the following are equivalent:
  \begin{enumerate}[(i)]
    \item $X$ is connected,
    \item there is no partition of $X$,
    \item the only subsets of $X$ which are both open and closed are $\vn$ and $X$.
  \end{enumerate}
\end{nlemma}
\begin{proof}
  We firstly show $(i)\implies (ii)$. We will prove the contrapositive statement, Suppose $X$ has a partition, then $X = A \cup B$ and $A, B$ are open, non-empty and $A \cap B = \vn$. We want to prove that if $X$ is disconnected. We define $f : X \to \{0, 1\}$ where,
  $$ f(x) = \begin{cases}
    0 & x \in A \\
    1 & x \in B
  \end{cases} $$
  We have a well defined function as $A \cap B = \vn$ and $A \cup B = X$. $f$ is surjective, as $A, B \ne \vn$. $f$ is continuous because $f^{-1}(\{0\}) = A$ and $f^{-1}(\{1\}) = B$, which are open, hence $f$ is continuous. Therefore $f$ is disconnected.\\

  \noindent
  Now we prove $(ii) \implies (iii)$. Let $V \sub X$ be both open and closed and $W = X \sm V$. So $W$ is also both open and closed. We see $V \cup W = X$ and $V \cap W = \vn$. Hence we appear to have produced a partition. Since $X$ has no partition, $V$ and $W$ cannot both be non-empty, so either $V = \vn$ or $W = \vn$ and so the other is just $X$. \\

  \noindent
  (iii) $\implies$ (i). Assume $X$ has no sets which are both open and closed expect $\vn$ and $X$. Let $f : X \to \{0, 1\}$ be continuous. We will show $f$ is not surjective. This means $X$ is connected. Let $A = f^{-1}(\{0\})$ and $B = f^{-1}(\{1\})$, both open as $f$ is continuous. So $A$ is closed as $A = X \sm B$ and $B$ is open. Hence $A$ is open and closed, so either $A = \vn$ or $A = X$. If $A = \vn$ then $X = B$
  and if $A = X$ then $B = \vn$. Hence $f$ is not surjective.
\end{proof}

\begin{nlemma}
  Let $f : X \to Y$ be a continuous function between topological spaces. If $X$ is connected, so is $f(X)$.
\end{nlemma}
\begin{proof}
  Replacing $f$ by the continuous map $f_1 : X \to f(X)$ with $f_1(x) = x$ for all $x$, we may assume $f$ is surjective. We prove the contrapositive, if $f(X)$ is disconnected then $X$ is disconnected. We use use $(ii)$ from the previous lemma.  Suppose $f(X)$ is disconnected, so there is a partition $A, B$ of $f(X)$. Then $f^{-1}(A)$, $f^{-1}(B)$ is a partition of $X$, showing $X$ is also disconnected.
\end{proof}

\subsubsection{Connected Components}
\begin{nlemma}
  Let $X$ be a topological space, let $x \in X$, and let $V_i$ $i \in \cI \ne \vn$, be a family of connected sets with $x \in V_i$ for each $i$. Then $\bigcup_{i \in \cI} V_i$ is connected.
\end{nlemma}
\begin{proof}
  Suppose $V = \bigcup_{i \in \cI} V_i$ has a partition, $A, B$. Without loss of generality $x \in A$. Pick some $y \in B$. Then $y \in V_j$ for some $j$. So the sets $A \cap V_j$ and $B \cap V_j$ contain $x, y$ respectively and are disjoint and open in $V_j$. Hence they form a partition of $V_j$. This is impossible as $V_j$ is connected.
\end{proof}

\begin{ndefi}[Connected Component]
  Let $X$ be a topological space and let $x \in X$. Then the connected components $C_x$ of $x \in X$ is the union of all connected subsets of $X$ containing $x$:
  $$ C_x = \bigcup_{x \in V \sub X, V \text{ connected }} $$
  Then $C_x$ is connected by Lemma 3.46, so $C_x$ is the unique largest connected subset of $X$ containing $x$.
\end{ndefi}
\noindent
Clearly, $X$ is connected if and only if $C_x = X$ for every $x \in X$.
\begin{eg}
  The subspace $X = (0,1) \cup (2, 3)$ of $\R$ has two connected components, if $x \in (0, 1)$ then $C_x = (0, 1)$ and if $x \in (2, 3)$ then $C_x = (2, 3)$.
\end{eg}

\begin{nprop}
  For any $x, y \in X$, either $C_x = C_y$ or $C_x \cap C_y = \vn$.
\end{nprop}
\begin{proof}
  Suppose $C_x \cap C_y \ne \vn$, so there is some $z \in C_x \cap C_y$. As $C_x$ is a connected subset containing $z$, we have $C_x \sub C_z$. Hence $x \in C_z$. Since $z \in C_z$ is connected, this means that $C_z \sub C_x$. So $C_x = C_z$. Similarly $C_y = C_z$. Thus $C_x = C_y$.
\end{proof}

\begin{remark}
   Proposition 3.48 means that $x \in C_y \iff C_x = C_y$. Moreover the relation $x \in C_y$ is an equivalence relation on $X$: it is,
   \begin{enumerate}
     \item reflexive, $x \in C_x$ for all $x$
     \item symmetric, $x \in C_y \implies C_x = C_y \implies C_y = C_x \implies y \in C_x$
     \item transitive, $x \in C_y$ and $y \in C_z$, then $C_x = C_y = C_z$ and so $x \in C_z$.
   \end{enumerate}
\end{remark}

\begin{nprop}
   If $A$ is a connected subset of $X$, then its closure $\bar A$ is also connected.
\end{nprop}
\begin{proof}
  Let $f \to \bar A \to \{0, 1\}$ be a continuous function and let $f_A$ be the restriction to $A$. Then $f_A$ cannot be surjective, as $A$ is connected, so, without loss of generality, $f(a) = 0$ for all $a \in A$.\\

  \noindent
  Then $f^{-1}(1)$ is an open subset of $\bar A$, so $f^{-1}(1) = \bar A \cap U$ for some open subset $U$ of $X$ for which $A \cap U = \vn$. Then $\bar A \cap U = \vn$ aswell ( $X \sm U$ is a closed set containing $A$, so by the definition of $\bar A$ we have $\bar A \sub X \sm U$). Thus $f(x) = 0$ for all $x \in \bar A$, so $f$ is not surjective. Hence $\bar A$ is connected.
\end{proof}

\begin{ncor}
   Connected components are closed. If there are only finitely many of them they are also open.
\end{ncor}
\begin{proof}
  If $C$ is a connected component in $X$ then $\bar C$ is connected, so $\bar C \sub C$. Hence $C = \bar C$ is closed.\\

  \noindent
  If there are only finitely mant of them $C_1, \dots, C_n$ then the complement of each component $C_i$ is the union $\bigcup_{i \ne j} C_j$ of finitely many closed sets, so this complement is closed and $C_i$ is open.
\end{proof}

\noindent
Can we find examples where there are infinitely many closed connected components?
\begin{eg}
  The connected compontents of $\Q$ are singletons. They are closed but not open. To see that a subset $S$ of $\Q$ containing at least two points cannot be connected, let $x, y \in S$ with $x < y$. Choose an irrational number $\a$ with $x < \a < y$. and define $f : S \to \{0, 1\}$ by,
  $$ f(s) = \begin{cases}
    0 & \text{ if } s < \a\\
    1 & \text{ if } s > \a
  \end{cases} $$
  Then $f$ is continuous and surjective.
\end{eg}

\begin{eg}
  Now for the topologists sine curve. We will show it's connected,
  $$ S = \{(x, \sin 1/x) : x > 0\} \cup \{(0, y) : -1 \le y \le 1\}. $$
  This set is the union of two pieces,
  $$ S_1 = \{(x, \sin 1/x) : x > 0\} \qquad S_2 = \{(0, y) : -1 \le y \le 1\} $$
  Since $(0,\infty]$ and $[-1, 1]$ are connected, then $S_1$ and $S_2$ are connected. Thus either $S$ has two connected components or one. However, if $S_1$ and $S_2$ were compontents, they would be open subsets of $S$. But $S_2$ is not open because any open neighbourhood of $(0, 0) \in S_2$ contains points from $S_1$, namely $((n\pi)^{-1}, 0)$ for large enough $n \in \N$. This shows that $S$ is connected. (In fact $S_1$ is open but not closed and $S_2$ is closed but not open).
\end{eg}

\subsection{Path Connected Spaces}
\begin{ndefi}[Path Connected]
  A topological space $X$ is path connected if, for any $x, y \in X$, there is a continuous function $p : [0, 1] \to X$ with $p(0) = x$ and $p(1) = y$. We call $p$ a path from $x$ to $y$.
\end{ndefi}

\begin{eg}
  Any open ball $B_\e(\vec a) \subset \R^n$ for $\e > 0$ is path connected. Indeed, given $\vec x, \vec y \in B_\e(\vec a)$, we can define a path $\vec p$ from $\vec x$ to $\vec y$ by,
  $$ \vec p(t) - (1 - t)\vec x + t\vec y \qquad \text{ for } t\in [0, 1] $$
  We must check that $\vec p (t) \in B_\e(\vec a)$ for all $t$. But, writing $\norm{\vec v}$ for the euclidean norm, we have,
  \begin{align*}
    \norm{\vec p(t) - \vec a} &= \norm{(1 - t)(\vec x - \vec a) + t(\vec y - \vec a)} \\
    &\le \norm{(1 - t)(\vec x - \vec a)} + \norm{t(\vec y - \vec a)} \\
    &= (1 - t)\norm{\vec x - \vec a} + t\norm{\vec y - \vec a}\\
    &< (1 - t)\e + t\e \\
    &= \e
  \end{align*}
\end{eg}

\begin{nlemma}
  A connected open subset $U$ of $\R^n$ is path connected.
\end{nlemma}
\begin{proof}
  Fix some $\vec x \in U$ and let,
  $$ A = \{\vec y \in U : \text{ there is a path from } \vec x \text{ to } \vec y\} $$
  We will show that $A$ and it's complement $B = X \sm A$ are both open. Since $U$ is the disjoint union of $A$ and $B$, and $U$ is connected, this will show that $A = U$.\\

  \noindent
  Let $\vec y \in A$. As $U$ is open, there is $\e > 0$ with $B_\e (y) \sub U$. We will show that $B_\e(y) \sub A$. Let $\vec z \in B_\e(y)$. We will find a path in $U$ connecting $\vec x$ to $\vec z$. Since $y \in A$, there is a path $\vec p : [0, 1] \to U$ connecting $\vec x$ to $\vec y$.
  By a previous example, there is a path $\vec q : [0, 1] \to B_\e(\vec y)$ joining $\vec y$ to $\vec z$. We will concatinate these to form a new path from $\vec x$ to $\vec z$.
  $$ \vec r(t) = \begin{cases}
    p(2t) & \text{ if } 0 \le t \le \frac{1}{2}\\
    q(2t - 1) & \text{ if } \frac{1}{2}\le t \le 1.
  \end{cases} $$
  This is well defined since $\vec p(1) = \vec q(0) = \vec y$, and it is continuous since $\vec p$ and $\vec q$ are. Moreover, $\vec r(t) \in U$ for all $t \in [0, 1]$. Hence $\vec z \in A$. Since this works for all $\vec z \in B_\e (\vec y)$, this shows that $A$ is open.\\

  \noindent
  We now show that $B$ is also open by a similar argument. Let $\vec y \in B$. There is some $\e> 0$ with $B_\e(y) \sub U$. Let $\vec z \in B_\e(y)$. We must show that there is no path in $U$ from $\vec x$ to $\vec z$. If we had such a path $\vec p$, we could find a path $\vec q$ from $\vec y$ to $\vec z$ as above and then define,
  $$ \vec r(t) = \begin{cases}
    p(2t) & \text{ if } 0 \le t \le \frac{1}{2}\\
    q(2t - 1) & \text{ if }\frac{1}{2}\le t \le 1
  \end{cases} $$
  This gives a path in $U$ from $\vec x$ to $\vec y$, which is impossible since $\vec y \notin A$.
 \end{proof}

 \begin{nlemma}
   Any path connected topological space is connected.
 \end{nlemma}
 \begin{proof}
   Let $X$ be a path connected topological space, and let $x, y \in X$. We will show that $y$ is in the connected component $C_x$ of $x$. Then $C_x = X$ so $X$ is connected.\\

   \noindent
   Since $X$ is path connected, there is a path $[0, 1] \to X$ with $p(0) = x$ and $p(1) = y$. Now $[0, 1]$ is connected so, so by a lemma above, $p([0, 1]) = \{p(t) : t \in [0, 1]\}$ is a connected subset of $X$ containing $x$ and $y$.\\

   \noindent
   Since $C_x$ is the unique largest connected subset of $X$ containing $x$, it follows that $y \in C_x$ as required.
 \end{proof}

\noindent
 Now the question is, is the converse true? It isn't true in general. A counter example is the topologists since curve,
 \begin{eg}
   The topologists sine curve isn't connected.
 \end{eg}
 \begin{proof}
   Let $A = (1/\pi, 0)$, i.e. the last point on the curve with $y = 0$ and let $B = (0, 0)$. Suppose for a contradiction that there is a path $\vec p \in S$ with $\vec p(0) = A$ and $\vec p(1) = B$. We will write $\vec p(t) = (p_1(t), p_2(t))$, so $p_1$ and $p_2$ are the projections of $\vec p$ to the $x$-axis of $\vec p$ onto the $x$ and $y$ axis. Thus $p_1$ and $p_2$ are continuous. \\
   It is possible that the path reaches the $y$-axis before $t = 1$, so let $t_0$ be the first time it happens. More precisely, let,
   $$ T = \{ t \in [0, 1] : p_1(t) = 0\}. $$
   Then $1 \in T$ and $T$ is bounded below, so it has infinimum $t_0$. As $p_1$ is continuous, $p_1(t_0) = 0$. As $p_1(0) > 0$, we have $0 < t_0 \le 1$ and $p_1(t) > 0$ if $t \in [0, t_0)$.\\
   Since $\vec p$ is continuous, there is some $\e > 0$ so that if $t_0 - \e < t < t_0$ then $\norm{\vec p(t) - \vec p(t_0)} < 1$. In particular, if $t_0 - \e < t < t_0$ then $|p_2(t) - p_2(t_0)| < 1$. Now $p_1$ is continuous on the interval $[t_0 - \frac{1}{2}\e, t_0]$, so by IVT, $p_1$ takes all the values between $p_1(t_0) = 0$ ad $p_1(t_0 - \frac{1}{2}\e) > 0$. \\
   Thus for large enough $n \in \Z$ we can find $t', t'' \in [t_0 - \frac{1}{2}\e, t_0]$ and $p_1(t') = [(2n + \frac{1}{2})\pi]^{-1}$ and $p_1 (t'') = [(2n + \frac{3}{2})\pi]^{-1}$. Then $p_2(t') = \sin (1/p_1(t')) = +1$ and $p_2(t'') = -1$.
   This can't happen as $|p_2(t) - p_2(t_0)| < 1$.
  \end{proof}

\section{Metric Spaces II}
\subsection{Complete Metric Spaces}
We will now return to metric spaces and talk about Cauchy Sequences,
\begin{ndefi}[Cauchy Sequence]
  A sequence $(a_n)_{n \ge 1}$ in a metric space $(X, d)$ is a Cauchy Sequence if for every $\e > 0$, there is an $N \in \N$ with $d(a_m, a_n) < \e$ for all $m,n > N$.
\end{ndefi}

\noindent
An important result is that,
\begin{nthm}[]
  A sequence in $\R$ converges if and only if it is a Cauchy Sequence.
\end{nthm}
\noindent
It is easy to show the forward direction, but is hard to show the backward direction. It requires the completeness property of $\R$. Can we prove this without the completeness property? Well, no, on $\Q$ this theorem doesn't hold. We will turn things on their head and define completeness from Cauchy Sequences,
\begin{ndefi}[Complete Metric Space]
  A metric space $(X, d)$ if every Cauchy sequence in $X$ converges to an element of $X$.
\end{ndefi}

\begin{eg}
  $\R$ is a complete metric space.
\end{eg}
\begin{eg}
  $\Q$ isn't complete, we can find a Cauchy sequence that converges to $\sqrt 2$, but not a rational number.
\end{eg}
\begin{eg}
  Recall that $\mathcal{C}[0, 1] = \{f : [0, 1] \to \R : f \text{ continuous}\}$. Then $\mathcal{C}[0, 1]$ is complete in the sup metric,
  $$ d(f, g) = \sup_{t \in [0, 1]} |f(t) - g(t)|. $$
\end{eg}
\begin{proof}
  Let $f_1, f_2, \dots$ be a Cauchy sequence in $\mathcal{C}[0, 1]$. Given some $\e >0$ there is an $N$ such that for all $m, n > N$, $|f_m(t) - f_n(t)| < \e$ for all $t \in [0, 1]$. \\

  \noindent
  Let $s \in [0, 1]$. Then, given a $\e > 0$ there is $N$ so that, for all $m, n> N$,
  $$|f_n(s) - f_m(s)| \le d(f_n, f_m) < \e$$
  so $(f_n(s))_{n=1}^\infty$ is a Cauchy sequence in $\R$. Therefore, it has a unique limit $f(s)$. We have now defined a function $f : [0, 1] \to \R$ which is our candidate for the limit of the given sequence. We need to show,
  \begin{enumerate}
    \item $f \in \mathcal{C}[0, 1]$,
    \item $f_n \to f$ with respect to $d$.
  \end{enumerate}
  To show that $f$ is continuous. We fix a $s \in [0, 1]$ and show $f$ is continuous at $s$. We choose $\e > 0$ and $N'$  such that,
  $$ \max_{t \in [0, 1]} |f_n(t) - f_m(t)| < \frac{1}{4}\e \quad \forall m, n > N' $$
  Choose some $\d > 0$ so,
  $$ |f_{N'+1}(t) - f_{N'+1}(s)| < \frac{1}{4}\e \quad\forall t \in [0, 1] \text{ with } |t - s|< \d $$
  For $m > N'$ we have and $|t - s| < \d$,
  \begin{align*}
    |f_m(t) - f_m(s)| &\le |f_m (t) - f_{N' + 1}(t)| + |f_{N'+1} - f_{N'+1}(s)| + |f_{N'+1}(t) - f_m(s)|\\
    &< \frac{1}{4}\e + \frac{1}{4}\e + \frac{1}{4}\e \\
    &= \frac{3}{4}\e.
  \end{align*}
  Since $f(t) = \lim_{m \to \infty} f_m(t).$ Taking limits as $m \to \infty$. We get,
  $$ |f(t) - f(s)| \le \frac{3}{4}\e < \e  $$
  for all $t \in [0, 1]$ with $|t - s| < \d$. So $f$ is continuous at $s$. This holds for all $s \in [0, 1]$. Therefore, $f \in \mathcal{C}[0, 1]$. We next show $f_n \to f$ in the metric $d$. i.e,
  $$ \sup_{t \in [0, 1]} |f_n(t) - f(t)| \to 0 \text{ as } n \to \infty. $$
  Given $\e > 0$, pick $N$ such that $\max_{t \in [0, 1]} |f_n(t) - f_m(t)| < \frac{1}{2}\e$ for all $n, m > N$. Fix $n > N$ and let $m \to \infty$. Therefore,
  $$ \max_{t \in [0, 1]} |f_n(t) - f(t)| \le \frac{1}{2}\e < \e$$
  That is, $d(f_n, f) < \e$ for all $n > N$, and so $f_n \to f$. Our Cauchy sequence $f_1, f_2, \dots$ converges to $f$.
\end{proof}

\noindent
We now show that the same space of functions, with a different metric, is not complete.
\begin{eg}
  $\mathcal{C}[0, 1]$ is not complete in the $L^1$ metric,
  $$ d_1(f, g) = \int_0^1 |f(t) - g(t)|dt $$
\end{eg}
\begin{proof}
  We give a Cauchy sequence of functions for $d_1$ which does not converge to any function in $\mathcal{C}[0, 1]$. For $n \ge 1$, define,
  $$ f_n(t) = \begin{cases}
    0 & \text{if } 0 \le t \le \frac{1}{2} - \frac{1}{2n}\\
    n(t - \frac{1}{2} - \frac{1}{2n}) & \text{ if} \frac{1}{2} - \frac{1}{2n }\le n \le \frac{1}{2} + \frac{1}{2n} \\
    1 & \text{ if } \frac{1}{2} + \frac{1}{2n} \le t \le 1
  \end{cases} $$
  Then $f_n \in \cc [0, 1]$. If $N \in \N$ and $m \ge n \ge N$ then $f_n(t) = f_m(t)$ unless $\frac{1}{2} - \frac{1}{2n} \le t \le \frac{1}{2} + \frac{1}{2n}$, and $|f_n(t) - f_m(t)| \le 1$ for all $t$. Hence,
  $$ d_1(f_m, f_n) \le \frac{1}{n} < \frac{1}{N} \text{ if } m,n > N $$
  This shows that $(f_n)$ is a Cauchy sequence with respect to $d_1$. \\

  \noindent
  We will show that the sequence $(f_n) \in \cc[0, 1]$ does not converge to $d_1$ to any function in $\cc[0, 1]$. It is clear that $f$ is `trying to converge' to a function $f$ with,
  $$ f(t) = \begin{cases}
    0 & \text{if }t < \frac{1}{2} \\
    1 & \text{if }t > \frac{1}{2}.
  \end{cases} $$
  This function can't be in $\cc[0, 1]$ whether we assign to $t = \frac{1}{2}$.\\

  \noindent
  Since $f$ is not in our metric space $(\cc[0, 1], d_1)$, this is not quite enough to show the sequence does not converge, perhaps maybe to some other limit $g \in \cc[0, 1]$. Suppose for a contradiction that such a continuous $g$ does exist. We first show that if $0 < t < \frac{1}{2}$ then $g(t) = 0$. Suppose there is some $t_0 \in (0, \frac{1}{2})$ with $|g(t_0)| = \a > 0$. Since $g$ is continuous, there is some $\d > 0$ with $|g(t) - g(t_0)| < \frac{1}{2}\a$
  for all $t$ with $|t - t_0| < \d$. Choosing $\d$ even smaller if necessary, we may assume that $(t_0 - \d, t_0 + \d) \sub (0, \frac{1}{2})$.\\
  If $|t - t_0| < \d$ we then have $|g(t)| \ge \frac{1}{2}\a$. For any $n$ large enough that $\frac{1}{2} - \frac{1}{2n} > t + \d$ we have,
  $$ d_1(f_n, g) = \int_0^1 |f_n(t) - g(t)|dt \le \int_{t_0 - \d}^{t_0 + \d} |0 - g(t)|dt \ge (2\d)\frac{1}{2}\a = \d\a $$
  Then $d_1(f_n, g)$ doesn't tend to zero as $n \to \infty$. This shows that $g(t) = 0$ for all $t \in (0, \frac{1}{2})$. By continuity, $g(0) = 0$. As similar argument shows $g(t) = 1$ for all $t \in (\frac{1}{2}, 1]$. Thus we have our required contradiction.
\end{proof}