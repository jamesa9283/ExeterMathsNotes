% !TEX root = ../notes.tex

\noindent
\subsection{Nulls Sets and Outer Measure}
Null sets are sets that we can assign a length of $0$. We can `over-estimate' the length of a set by covering with a countable union of open sets. If we can make the over-estimate arbitrarily small, then our set should have length zero,
\begin{ndefi}[Null Set]
  A subset $A$ of $\R$ is a null set if, given $\e > 0$, there is a countable family of open intervals $I_n$, $n \ge 1$ such that,
  $$ A \sub \bigcup_{n=1}^\infty I_n \text{ and } \sum_{n=1}^\infty m(I_n) < \e $$
  where, for an open interval $I = (u, v)$ with $u \le v$ we define $m(I) = v - u$.
\end{ndefi}
\begin{eg}
  Given any finite set $A = \{a_1, a_2, \dots, a_k\}$ is a null set. For a given $\e > 0$ we pick $\d > 0$ with $2k\d < \e$. Define $I_n$ by,
  $$ I_n = \begin{cases}
    (a_n - \d, a_n + \d) & \text{ if } 1 \le n \le k\\
    \vn & \text{ if } n > k
  \end{cases} $$
  Then $a_n \in I_n$ for $1 \le n \le m$, so,
  $$ A\sub \bigcup_{n=1}^\infty I_n $$
  and,
  $$ \sum_{n=1}^\infty m(I_n) = \sum_{n=1}^k 2\d < \e. $$
\end{eg}

\begin{eg}
  The middle third cantor set $C$ is a null set. Recall $C = \bigcap_{k=0}^\infty C_k$ where $C_k$ is the disjoint union of $2^k$ closed intervals, each of length $3^{-k}$. We choose some $\a$ with $\frac{1}{3} < \a < \frac{1}{2}$. Given $\e > 0$, let $k$ be large enough that $(2\a)^k < \e$. For each of the $2^k$ closed intervals $J$ of length $3^{-k}$ making up $C_k$,
  pick an open interval $I \supset J$ of length $\a^k$, and label these intervals $I_1, I_2, \dots, I_r$ with $r = 2^k$. For $n > r$, let $I_n = \vn$. Then,
  $$ C \subset C_k \sub \bigcup_{n=1}^\infty I_n $$
  and
  $$ \sum_{n=1}^\infty m(I_n) = \sum_{n=1}^r \a^k = r\a^k = (2\a)^k < \e $$
  This shows that $C$ is indeed a null set.
\end{eg}

\begin{nlemma}
  If $A$ and $B$ are null sets, so is $A \cup B$.
\end{nlemma}
\begin{proof}
  Given $\e > 0$, we can find open intervals $I_n$ and $J_n$ for $n \ge 1$ such that,
  $$ A \sub \bigcup_{n=1}^\infty I_n \text{ and } \sum_{n=1}^\infty m(I_n) < \frac{1}{2}\e, $$
  and,
  $$ B \sub \bigcup_{n=1}^\infty J_n \text{ and } \sum_{n=1}^\infty m(J_n) < \frac{1}{2}\e. $$
  Let $W_1, W_2, \dots$ be the sequence of intervals $I_1, J_1, I_2, J_2, \dots$. Then,
  $$ A \cup B \sub \bigcup_{n=1}^\infty W_n = \bigcup_{n=1}^\infty I_n \cup \bigcup_{n=1}^\infty J_n $$
  and,
  $$ \sum_{n=1}^\infty W_n = \sum_{n=1}^\infty I_n + \sum_{n=1}^\infty J_n < \e $$
  We can rearrange these sums since it is absolutely convergent.
\end{proof}

\noindent
It follows by induction that the union of finitely many null sets is a null set. More generally,
\begin{nlemma}
  The union of countably many null sets is a null set.
\end{nlemma}
\begin{proof}
  Let $A_1, A_2, \dots$ be null sets, Given $\e > 0$ for each $j \ge 1$, there is a countable collection of open sets $I_1^{(j)}, I_2^{(2)}, \dots$ such that
  $$ A_j \sub \bigcup_{n=1}^\infty I_{n}^{(j)} \text{ and } \sum_{n=1}^\infty m(I_n^{(j)}) < 2^{-j}\e.$$
  The collection of intervals $I_n^{(j)}$ for all $j$, $n \ge 1$ is countable. We can list it is $I_1^{(1)}, I_1^{(2)}, I_2^{(1)}, I_1^{(3)}, I_{2}^{(2)}, I_3^{(1)}, \dots$. Label these intervals $W_1, W_2, \dots$. Then we have,
  $$ \bigcup_{j=1}^\infty A_j \sub \bigcup_{j=1}^\infty\bigcup_{i=1}^\infty I_n^{(j)} = \bigcup_{n=1}^\infty W_n$$
  and,
  \begin{align*}
    \sum_{j=1}^\infty m(W_n) &= \sum_{j=1}^\infty \sum_{i=1}^\infty m(I_n^{(j)}) \\
    &< \sum_{j=1}^\infty 2^{-j}\e = \e.
  \end{align*}
  (We can rearrange the sum since it's absolutely convergent.) Hence $\bigcup_{j=1}^\infty A_j$ is a null set.
\end{proof}

\begin{eg}
  $\Q$ is a null set. In $\Q$ is countable, so we may list it's elements $q_1, q_2, \dots$. Then $\Q$ is the union of countably many null sets $\{q_n\}$ for $n \ge 1$.
\end{eg}

\noindent
Now we seek to look at the Outer Measure. We wanted the following properties,
\begin{enumerate}
  \item $m((a, b)) = m([a, b]) = b - a$ if $b \ge a$,
  \item $m(A) \in [0, \infty]$,
  \item $m(A \cup B) \le m(A) + m(B)$ for all set $A, B$,
  \item More generally we should expect $m$ to respect countably infinite unions.
  \item $m$ should be translation invariant, $m(A + c) = m(A)$.
\end{enumerate}

\noindent
We will now consider an $m^*$ which turns out to have all the properties above apart from the equality conditions (not listed). We will apply the same over-estimate strategy for null sets and define $m^*(A)$ for an arbitrary $A$ as the best over estimate. Consider the set of all possible over-estimates,
$$ Z(A) = \left\{ \sum_{n=1}^\infty m(I_n) : I_1, I_2 \dots \text{ are open intervals with } A \sub \sum_{n=1}^\infty I_n \right\} $$
Clearly $\infty \in Z(A)$. Either $Z(A) = \{\infty\}$ or $Z(A)\sm \{\infty \}$ is a non-empty subset of $\R$ bounded below by $0$. Hence $Z(A)$ has an infinimum, which may be $\infty$.

\begin{ndefi}[Outer Measure]
  The outer measure of $A$ is $m^*(A) = Z(A)$
\end{ndefi}

\noindent
This means that for any family $I_1, I_2, \dots$ of open intervals covering $A$, we have,
$$ m^*(A) \le \sum_{n=1}^\infty m(I_n) $$
but, given $\e > 0$, there is such a family with,
$$ \sum_{n=1}^\infty m(I_n) \le m^*(A) + \e $$
In summary, we have defined $m^*(A) \in [0, \infty]$ for every subset $A \sub \R$, and shown that is has the following properties,
\begin{enumerate}
  \item If $A$ is an interval, then $m^*(A)$ is the length of $A$.
  \item $m^*$ is countably subadditive,
  $$ m^*\left( \bigcup_{n=1}^\infty A_n \right) \le \sum_{n=1}^\infty m^*(A_n) $$
  \item $m^*$ is translation-invariant, $m^*(A + c) = m^*(A)$.
\end{enumerate}

\noindent
However, the example above shows that $m^*$ cannot be additive on countable disjoint unions. That is,
$$ m^*\left( \bigcup_{n=1}^\infty A_n \right) = \sum_{n=1}^\infty m^*(A_n) \text{ if } A_i \cap A_j = \vn \text{ when } i \ne j $$
cannot always hold. (Indeed, it doesn't always hold even for finite unions.)


\subsection{Lebesgue Measurable Sets}
\begin{nthm}[]
  $m^*$ satisfies the countable subadditity property,
  $$ m^*\left( \bigcup_{j=1}^\infty A_j \right) \le \sum_{j=1}^\infty m^*(A_j). $$
\end{nthm}
\begin{proof}
  The argument is similar to the proof that a countable union of null sets is a null set. Given $\e > 0$ for each $j \ge 1$, there is a countable collection of open intervals $I_1^{(j)}, I_2^{(j)}, \dots$, such that
  $$ A_j \sub \bigcup_{n=1}^\infty I_n^{(j)} \text{ and } \sum_{m=1}^\infty m(I_n^{(j)}) \le m^*(A_j) + 2^{-j-1}\e.$$
  Then,
  $$ \bigcap_{j=1}^\infty A_j \sub \bigcap_{j=1}^\infty \bigcap_{n=1}^\infty I_n^{(j)} $$
  and,
  \begin{align*}
    \sum_{j=1}^\infty \sum_{n=1}^\infty m(I_n^{(j)}) &\le \sum_{j=1}^{\infty} (m^*(A_j) + 2^{-j-1}\e)\\
    &= \left( \sum_{j=1}^\infty m^*(A_j) \right) + \e \\
  \end{align*}
  As this holds for all $\e$, then the result follows.
\end{proof}

\noindent
Recall the for a bounded open interval $I = (a, b)$ we have defined $m(I) = b - a$, the length in the usual sense. The definition of $m^*$ is comparable with this.

\begin{nlemma}
  Let $I \sub \R$ be an interval. Then $m^*(I) = m(I)$.
\end{nlemma}
\begin{proof}
  Lengthy but trivial, so omited.
\end{proof}

\noindent
So the outer measure has,
\begin{enumerate}
  \item the length of an interval,
  \item sub-additive property,
  \item Translation Invariance.
\end{enumerate}

\noindent
\subsection{Lebesgue Measure}
We now describe a large class $\mathcal{M}$ of subsets where $m^*$ behaves nicely.
\begin{ndefi}[Lesbegue Measurable]
  We say $E\sub \R$ is (Lesbegue) Measurable if, for all $A\sub \R$,
  $$ m^*(A) = M^*(A \cap E) + M^*(A\cap E^c) $$
  We write $\mathcal{M}$ for the collection of measurable sets of $\R$.
\end{ndefi}

\noindent
\begin{remark}
   By sub-additivity, we always have,
   \begin{align*}
     m^*(A) &= m^*((A\cap E) \cup (A\cap E^c))\\
     &\le m^*(A \cap E) + m^*(A \cap E^c),
   \end{align*}
   so to check that $E\in \mathcal{M}$ it is enough to show,
   $$ m^*(A\cap E) + m^*(A \cap E^c) \le m^*(A). $$
\end{remark}

\noindent
\begin{remark}
   Some properties of $\mathcal{M}$ follows easily from the properties of $m^*$,
   \begin{enumerate}
     \item If $E$ is a null set, then $E\in \mathcal{M}$. Indeed $m^*(A \cap E) \le m^*(E) = 0$, ad similarly for $m^*(A \cap E^c)$. In particular $\vn \in \mathcal{M}$.
     \item If $E \in \mathcal{M}$, then $E^c \in \mathcal{M}$.
     \item $\mathcal{M}$ is translation invariant. If $E\in \mathcal{M}$ then $E + t \in \mathcal{M}$.
   \end{enumerate}
\end{remark}

\begin{nthm}[]
  $\mathcal{M}$ admits countable unions and intersections. If $E_1, E_2, \dots \in \mathcal{M}$ then,
  $$ \bigcup_{n=1}^\infty E_n \in \mathcal{M} \text{ and } \bigcap_{n=1}^\infty E_n \in \mathcal{M} $$
  \noindent
  Moreover, if $E_i \cap E_j = \vn$ for $i \ne j$, then
  $$ m^*\left( \bigcup_{n=1}^\infty E_n \right) = \sum_{n=1}^\infty m^*(E_n)$$
\end{nthm}
\begin{proof}
  Omitted
\end{proof}

\begin{nthm}[]
  Every interval is in $\mathcal{M}$
\end{nthm}
\begin{proof}
  Omitted.
\end{proof}
\begin{ncor}
   Every subset of $\R$ is in $\mathcal{M}$, and every closed subset of $\R$ is in $\mathcal{M}$.
\end{ncor}
\begin{proof}
  As $E \in \mathcal{M} \iff E^c \in \mathcal{M}$, it suffices to consider the open subsets. By the theorems it suffices to show that any open set $U$ is a countable union of open intervals.\\

  \noindent
  Now $U$ is the disjoint union of it's connected components. If $x \in U$ then the connected components $C_x$ of $U$ containing $x$ is the open interval $(a, b)$, where $a = \inf \{a' : (a', x] \sub U\}$ and $b = \sup \{b' : [x, b') \sub U\}$.\\

  \noindent
  Thus each connected component contains rational numbers. As $\Q$ is countable and the components are disjoints, it follows that there are at most countable many components, so $U$ is indeed a union of countably man open intervals.
\end{proof}

\begin{ndefi}[Lebesgue]
  For $E\in \mathcal{M}$ we define $m(E) = m^*(E)$. Then $m$ is the Lebesgue measure on $\R$.
\end{ndefi}

\noindent
We now have achieved our aim as far as possible. We have assigned a `length' for $m(A)$ for any reasonable subset $A \sub \R$ with some of the properties we wanted.

\begin{remark}
   We have done this just for $\R$, but we can do this similarly for $\R^2$ and moreover for $\R^n$.
\end{remark}

\subsection{$\s$-algebras and measure spaces}
We now set up some abstract ideas to talk about measures more generally.
\begin{ndefi}[$\s$-algebra]
  A $\s$-algebra $\mathcal{B}$ on a set $X$ is a faimly of subsets such that,
  \begin{enumerate}
    \item $\vn \in \mathcal{B}$,
    \item if $A \in \mathcal{B}$ then $X\sm A \in \mathcal{B}$,
    \item if $A_1, A_2, \dots$ is a countable sequence of sets in $\mathcal{B}$, then,
    $$ \bigcup_{n=1}^\infty A_n \in \mathcal{B} $$
  \end{enumerate}
\end{ndefi}

\begin{eg}
  On any set $X$, the powerset $\mathcal{P}(X)$ of $X$ is a $\s$-algebra.
\end{eg}

\begin{eg}
  Let $X = \{a, b, c, d\}$ then one possible $\s$-algebra is,
  $$ \{\vn, \{a, b\}, \{c, d\}, X\} $$
\end{eg}

\begin{remark}
   If $\mathcal{B}$ is a $\s$-algebra then $X \in \mathcal{B}$ since $X = X\sm \vn$.\\

   \noindent
   Also for a countable sequence we have,
   $$ \bigcap_{n=1}^\infty A_n \in \mathcal{B} $$
   as,
   $$ \bigcup_{n=1}^\infty (X \sm A_n) = X \sm \bigcap_{n=1}^\infty A_n  \in \mathcal{B}$$
   \noindent
   Of course, this also holds for finite unions and intersections. We just take $A_n = \vn$ for all but finitely many $n$.
\end{remark}

\begin{remark}
   A family of set $\mathcal{B}$ satisfying the definition for a $\s$-algebra, except that only finite unions are allowed is called an algebra on $X$.
\end{remark}

\begin{nprop}
   Let $\{\mathcal{B}_i : i \in \mathcal{I}\}$ be any set of $\s$-algebras on $X$. Then their intersection,
   $$ \mathcal{B} = \bigcap_{i \in \mathcal{I}} \mathcal{B}_i $$
   is also a $\s$-algebra on $X$.
\end{nprop}
\begin{proof}[]
  We check the three conditions from the definition.
  \begin{enumerate}
    \item As $\vn \in \mathcal{B}_i$ for all $i$, then $\vn \in \mathcal{B}$,
    \item If $A \in \mathcal{B}$,, then $A \in \mathcal{B}_i$ for each $i$, so $X\sm A \in \mathcal{B}_i$. Thus $X\sm A \in \mathcal{B}$.
    \item If $A_1, A_2, \dots \in \mathcal{B}$, then $A_1, A_2, \dots \in \mathcal{B}_i$ for each $i$, so $\bigcup_{n=1}^\infty A_n \in \mathcal{B}_i$ for each $i$. Thus $\bigcup_{n=1}^\infty A_n \in \mathcal{B}$.
  \end{enumerate}
\end{proof}

\noindent
Why is this important? We can make the following defintion,
\begin{ndefi}[$\s$-algebra generated by a set]
  Let $\mathcal{E}$ be any family of subsets of a set $X$. Then the $\s$-algebra generated by $\mathcal{E}$ is the smallest $\s$-algebra containing $\mathcal{E}$.
\end{ndefi}

\noindent
Further we define the Borel Algebra,
\begin{ndefi}[Borel Algebra]
  Let $(X, \T)$ be a topological space. The Borel algebra on $X$ is the $\s$-algebra generated by $\T$. It's elements are called Borel subsets of $X$.
\end{ndefi}

\begin{eg}
  On $\R$ the open sets are precisely unions of open intervals $(a, b)$. So the Borel Algebra of $\R$ are any sets that can be built up from open intervals by taking complements and countable infinite unions and intersections.
\end{eg}

\begin{eg}
  The set $\mathcal{M}$ of measurable functions in $\R$ is a $\s$-algebra. This follows from the properties given above. Since open intervals are in $\mathcal{M}$, it follows that $\mathcal{M}$ contains the Borel algebra on $\R$.
\end{eg}

\begin{remark}
   It turns out that not every null set is a Borel set. Since null sets are measurable, this means that there are measurable sets which are not Borel sets. However, every Borel set is contained in a measurable set with the same outer measure.
\end{remark}

\begin{eg}
  Here is a sketch of a contruction of a null set which isn't Borel. We can write each $x \in [0, 1]$ as a binary expansion $x = \sum_{n=1}^\infty a_n/2^n$, with $a_n \in \{0, 1\}$. Choose the non-terminating expansion unless $x = 0$. Then the expansion is unique. \\

  \noindent
  Define $b_n = 2a_n \in \{0, 2\}$ and set $f(x) =\sum_{n=1}^\infty b_n/3^n$. This maps $[0, 1]$ bijectively to the middle-third cantor set. Then $f(S)$ is a null set. In particular, it is Lebesgue measurable.\\

  \noindent
  Now suppose $f(S)$ is a Borel set. Now $f: [0, 1] \to [0, 1]$ is injective and increasing, and the preimage of an interval under $f$ is an interval. Since unions and complements behave nicely under preimages, the preimage of a Borel set is a Borel set. This is $f(S)$ is a Borel set, so is $S$, and this is a contradiction.
\end{eg}

\begin{ndefi}[Measure / Measure Space]
  Let $X$ be a set and $\mathcal{B}$ a $\s$-algebra on $X$. A measure is a function $\mu : \mathcal{B} \to [0, \infty]$ such that,
  \begin{enumerate}
    \item $\mu(\vn) = 0$,
    \item for any countably infinite family of set $A_1, A_2, \dots \in \mathcal{B}$ with $A_i \cap A_j = \vn$ for $i \ne j$, we have,
    $$ \mu \left( \bigcup_{n=1}^\infty A_n \right) = \sum_{n=1}^\infty \mu(A_n)$$
  \end{enumerate}
  We call the triple $(X, \mathcal{B}, \mu)$ a measure space. Further $\mu(X) < \infty$ and a probability measure if $\mu(X) = 1$.
\end{ndefi}

\begin{eg}
  The triple $(\R, \mathcal{M}, m)$ is a measure space, where $\mathcal{M}$ is the set of measurable sets, and $m$ the Lebesgue measure. \\

  \noindent
  The tripe $(\R, \mathcal{B}, m)$, where $\mathcal{B}$ is the Borel algebra. These two measure spaces are different since $\mathcal{B} \ne \mathcal{M}$.
\end{eg}

\begin{eg}
  For any set $X \ne \vn$ and any $p \in X$ we have the Dirac measure at $p$,
  $$ \mu_p(A) = \begin{cases}
    1 & \text{ if } p \in A\\
    0 & \text{ if } p \notin A
  \end{cases} $$
  on the $\s$-algebra $\mathcal{P}(X)$.
\end{eg}

\begin{ndefi}[Measurable Function]
  Let $(X, \mathcal{B}, \mu)$ be a measure space, let $E \sub \mathcal{B}$ and let $f: E \to \R$ be a function. We stay $f$ is a measurable function if, for each interval $I \sub \R$, the set $f^{-1}(I) \sub E$ is measurable.
\end{ndefi}

\begin{remark}
 There are functions $f : \R \to \R$ which are Lebesgue measurable but not Borel measurable. Take the indicator function of a set which is Lebesgue measurable but not Borel measurable. 
\end{remark}