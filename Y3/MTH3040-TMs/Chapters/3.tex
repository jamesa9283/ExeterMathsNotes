% !TEX root = ../notes.tex

We can compare two topologies on the same set $X$.
\begin{ndefi}[Coarse/Fine]
  Let $\T_1, \T_2$ be topologies on $X$. We say $\T_1$ is coarser than $\T_2$ (or weaker) if every open set of $\T_1$ is an open set in $\T_2$. We also say that $\T_2$ is finer than $\T_1$.
\end{ndefi}
On any $X$, the coarsest topology is the indiscrete topology and the finest is the discrete topology.

\begin{eg}
  Let $X = \{1, 2\}$, we can ask what are the topologies on $X$? The subsets of $X$ are $\vn$, $\{1\}$, $\{2\}$ and $\{1, 2\}$. Any topology of $X$ contains $\vn$ and $\{1, 2\}$ so the possible topologies are $\T_1 = \{\vn, \{1, 2\}\}$ (indiscrete topology), $\T_2 = \{\vn, \{1\}, \{1, 2\}\}$, $\T_3 = \{\vn, \{2\}, \{1, 2\}\}$, $\T_2 = \{\vn, \{1\}, \{2\}, \{1, 2\}\}$ (discrete topology).\\

  \noindent
  We can say that $\T_1$ is coarser than $\T_2$, $\T_3$ and $\T_4$. $\T_2$ is finer than $\T_1$ and coarser than $\T_4$, similarly for $\T_4$. We say $\T_4$ is finer than $\T_1$, $\T_2$ and $\T_3$. $\T_2$ and $\T_3$ are not comparable as neither is coarser than the other.
\end{eg}

\subsection{Closed Sets in a TS}
\begin{ndefi}[Closed]
  A subset $A$ of a topological space $X$ is closed if its complement $X\sm A$ is open.
\end{ndefi}
Note that $\vn$ and $X$ are closed. So a set can be both open and closed. It is also to have a set that is neither. USing demorgans laws for sets,
$$ \bigcup_{i \in \cI} X \sm U_i = X\sm \left( \bigcap_{i \in \cI} U_i \right) \qquad \bigcap_{i \in \cI} X \sm U_i = X \sm \left( \bigcup_{i \in \cI} U_i \right) $$
and the properties of open sets, we can show

\begin{nprop}
   \begin{enumerate}
     \item An arbitrary intersection of closed sets is closed
     \item A finite union of closed sets is closed.
   \end{enumerate}
\end{nprop}
\begin{proof}
  (i) Let $C_i$ for $i \in \cI$ be an arbitrary collection of closed sets in $X$. Then,
  $$ X\sm \left( \bigcap_{i \in \cI} U_i \right) = bigcup_{i \in \cI} X \sm U_i $$
  Since the sets $X\sm C_i$ are open, so is their union. Hence $\bigcap_{i \in \cI} C_i$ is closed.\\

  \noindent
  (ii) \textbf{Exercise}
\end{proof}

\noindent
Again, the union of an infinite family of closed sets need not be closed.

\subsection{Convergence and Continuity}
\begin{ndefi}[Limit of a sequence]
  Let $a_n$, $n \ge 1$ be a sequence of points in a topological space $X$. We say that $a_n$ converges to a point $a \in X$, written $a_n \to a$ as $n \to \infty$, if, for every open set $U$ of $X$ with $a \in U$, there is some $N \in \N$ such that $a_n \in U$ for all $n > N$.
\end{ndefi}

\begin{eg}
  Let $X$ be a topological space with the indiscrete topology (the only open sets are $\vn$ and $X$). Then every sequence $(a_n)$ in $X$ converges to every point $a \in X$. For, given an open set $U$ containing $a$, we must have $U = X$, and then $a_n \in X$ for all $n$.
\end{eg}

\begin{remark}
   If $X$ is a metric space, viewed as a topological space with topology given by it's metric, then the two definitions agree.
\end{remark}

\begin{ndefi}[Continuous]
  A function $f : X \to Y$ between topological spaces is continuous if, for every open set $U$ of $Y$, the subset $f^{-1}(U)$ is an open subset $X$.
\end{ndefi}

\begin{eg}
  Let $f : \R \to \R$,
  $$ \begin{cases}
    1 & x \ge 0\\
    0 & x < 0
  \end{cases} $$
  is not continuous since, for the open set $U = \left( \frac{1}{2}, \frac{3}{2} \right)$ we have $f^{-1}(U) = [0, \infty)$
\end{eg}

Here's a slightly more interesting example,
\begin{eg}
  Let $X = (\R, \T_d)$ and let $Y = (\R, \T_u)$ where $\T_d$ is the discrete topology and $\T_u$ is the usual topology on $\R$. Let $f : X \to Y$ and $f: Y \to X$ be the identity map on $\R$. \\

  Then $f$ is continuous, for if $U \sub Y$ is open then $f^{-1}(U) = U$ is certainly open in $X$. However $g$ is not continuous, the set $V = \{0\}$ is open in $Y$ (because every set is open in $Y$) but $f^{-1}(V)$ is not open in $X$ (since $\{0\}$ is not an open set in the usual topology.)
\end{eg}

\begin{nlemma}[]
  If $f : X \to Y$ and $g : Y \to Z$ are continuous maps between topological spaces, then $g \circ f : X \to Z$ is continuous
\end{nlemma}
\begin{proof}
  Let $U$ be an open set in $Z$. Then $g^{-1}(U)$ is an open set in $A$ since $g$ is continuous, and therefore $f^{-1}(g^{-1}(U))$ is an open set in $X$ since $f$ is continuous. But,
  \begin{align*}
    f^{-1}(g^{-1}(U)) &= \{x \in X : f(x) \in g^{-1}(U)\}\\
    &= \{ x \in X : g(f(x)) \in U\} = (g \circ f)^{-1}(U)
  \end{align*}
  Hence $g\circ f$ is continuous.
\end{proof}

Continuous functions should be thought of as the structure-preserving functions between topological spaces, in the same as we have homomorphisms between groups, and linear maps between vector spaces. An isomorphism of topological spaces is called a homeomorphism.

\begin{ndefi}[Homeomorphism]
  A homeomorphism between topological dpaces $X$ and $Y$ as a continuous function $f : X \to Y$ which is bijective and whose inverse function $f^{-1} : Y \to X$ is also continuous. We say that $X$ and $Y$ are homeomorphic if there is a homeomorphism between them.
\end{ndefi}

\begin{eg}
  The intervals $(0, 1)$ and $(0, \infty)$ in $\R$ (usual topology) are homeomorphic. Indeed, consider $f : (0,1) \to (0, \infty)$ with
  $$ f(x) = \frac{1 - x}{x} $$
  This is well defined and continuous, and is bijective with continuous inverse $g : (0, \infty) \to (0, 1)$ with,
  $$ g(y) = \frac{y}{1 + y} $$
\end{eg}
The inverse of a homeomorphism is again, a homeomorphism, but a continuous bijection is not necessarily a homeomorphism.

\begin{eg}
  We have seen that $(\R, \T_d) \to (\R, \T_u)$ is a continuous bijection whose inverse is not continuous. So it is not a homeomorphism.
\end{eg}

% Lecture
\subsection{Interior and Closure}

Here's a defintion
\begin{ndefi}[Interior]
  Let $X$ be a topological space. For any $A \sub X$, the interior of $A$, written $A^\circ$, is the union of all open subsets of $X$ contained in $A$,
  $$ A^\circ = \bigcup_{U \text{ open;} U \sub A} U $$
\end{ndefi}

\begin{nprop}
   \begin{enumerate}
     \item $A^\circ$ is the (unique) largest open subset contained in $A$, that is $A^\circ$ is an open set, $A^\circ \sub A$ and if $U$ is open and $U \sub A$ then $U \sub A^\circ$.
     \item For $x \in X$ we have $x \in A^\circ$ $\iff$ there exists an open set $U$ with $x \in U \sub A$
     \item $A^\circ = A \iff$ $A$ is open.
   \end{enumerate}
\end{nprop}
\begin{proof}
  \begin{enumerate}
    \item $A^\circ$ is a union of open sets, so it is open. If $U$ is open and $U \sub A$ then $U$ is one of the sets in the union, so $U \sub A^\circ$.
    \item If $x \in A^\circ$ then $x \in U$ for some $U$, so $x \in U \sub A$. Conversely if $x \in U \sub A$ for some $U$, then $U$ is one of the sets in the union and so $x \in U^\circ$.
    \item If $A^\circ = A$, then $A$ is open as $A^\circ$ by $(i)$. Conversely, if $A$ is open, then it is clearly the largest open set contained in $A$, so $A^\circ = A$ by $(i)$.
  \end{enumerate}
\end{proof}

\begin{ndefi}[Closure]
  Let $X$ be a topological space. For any $A \sub X$, the closure of $A$, written $\bar A$ is the intersection of all closed subsets of $X$ which contain $A$:
  $$ \bar A = \bigcap_{C \text{ closed; } A \sub C} C $$
\end{ndefi}
\begin{nprop}
 \begin{enumerate}
   \item $\bar A$ is the (unique) smallest closed subset containing $A$. That is, $\bar A$ is a closed set, $A \sub \bar A$, and if $C$ is closed and $A\sub C$ then $\bar A \sub C$.
   \item For $x \in X$ we have $x \in \bar A \iff$ there is no open set $U$ with $x \in U$ and $U \cap A = \vn$
   \item $\bar A = A \iff A$ is closed.
 \end{enumerate}
\end{nprop}
\begin{proof}
  Exercise
\end{proof}

\noindent
Here is an application,
\begin{ncor}
   Let $X$ be any topological space and let $S$ be a subset $X$. Let $(a_n)$ be a sequence in $X$ with $a_n \in S$ for all $n$. If $a_n$ converges to some point $a \in X$ then $a \in \bar S$.
\end{ncor}
\begin{eg}
  Let $X = \R$ and $S = \R^+$. Let $a_n = \frac{1}{n} \in S$ and $a_n \to 0$ but $0 \in S$, but $0 \in \bar S$.w
\end{eg}

\begin{proof}
  Recall that $x \in \bar S$ if and only if there no open $U \ni x$ with $U \cap S = \vn$. Suppose $U \sub X$ is open and suppose $a \in U$. Then there is some $N$ so $a_n \in U$ for some $n > N$. Therefore, $a_n \in U \cap S$ for all $n > N$. So $U \cap S \ne \vn$. Hence $a \in \bar S$.
\end{proof}
\newpage
\subsection{Hausdorff Spaces}
\begin{wrapfigure}{r}{0.4\textwidth}
  \centering
  \resizebox{0.4\textwidth}{!}{\input{./figures/Hausdorff.pdf_tex}}
  \caption{Hausdorff Spaces}
  \vspace{-20pt}
\end{wrapfigure}
\begin{ndefi}[Hausdorff]
  A topological space is Hausdorff if, given any points $x, y \in X$ with $x \ne y$, there exists open sets $U, V \in X$ with $x \in U$, $y \in V$ and $U \cap V = \vn$.
\end{ndefi}

We ask two questions, are metric spaces Hausdorff and are all topological spaces Hausdorff?
\begin{eg}
  Every metric space $(X, d)$ is Hausdorff. If $x, y \in X$ and $x \ne y$. Let $\e = d(x, y)$, so $\e > 0$. Then $B_{\frac{\e}{2}}(x)$ and $B_{\frac{\e}{2}}(y)$ are disjoint open sets containing $x, y$ respectively.
\end{eg}

\begin{eg}
  Let $X$ be a set with at least two elements and let $X$ have the indiscrete topology. Then $X$ is not hausdorff.\\
  Indeed, take $x, y \in X$ with $x \ne y$. The only open set that contains $x$ is $X$, this also contains $y$. Therefore, it cannot be Hausdorff.
\end{eg}

\begin{remark}
   It follows that any topological space that is not Hausdorff, it cannot comes from a metric. For example, there is no metric on a set $X$ with $|X| \ge 2$ which gives rise to the indiscrete topology.
\end{remark}

\begin{nprop}
  In a Hausdorff space, any sequence can converge to at most one point.
\end{nprop}
\begin{proof}
  Suppose $a_n \to a$ and $a_n \to b$. We must prove that $a = b$. Suppose $a \ne b$, let $U, V$ be open sets with $a \in U$, $b \in V$ and $U \cap V = \vn$. There exists an $N_1$ such that $a_n \in U$ for all $n > N_1$ and there is also an $N_2$ such that $a_n \in V$ for all $n > N_2$. For $n > \max(N_1, N_2)$, then $a_n \in U\cap V$ and so $U \cap V \ne \vn$ - Contradiction. Therefore, $a = b$.
\end{proof}

\begin{nprop}
   If $f : X \to Y$, $X, Y$ are topological spaces, is injective and continuous and $Y$ is Hausdorff. Then $X$ is Hausdorff.
\end{nprop}
\begin{proof}
  Let $x_1 \ne x_2$ where $x_1, x_2 \in X$. Let $y_1 = f(x_1)$ and $y_2 = f(x_2)$. Then $y_1 \ne y_2$ as $f$ is injective. As $Y$ is hausdorff, then there are open sets $U, V \sub Y$ with $y_1 \in U$ and $y_2 \in V$ with $U \cap V \ne \vn$. Then $x_1 \in f^{-1}(U)$ and $x_2 \in f^{-1}(V)$, these preimages are open subsets in $X$ as $f$ is continuous.
  If $x \in f^{-1}(U) \cap f^{-1}(V)$, therefore $f(x) \in U \cap V$, with is $\vn$ - Contradiction. So $f^{-1}(U) \cap f^{-1}(V) = \vn$. Therefore $X$ is Hausdorff.
\end{proof}

\begin{ncor}
   If $X$ and $Y$ are homeomorphic then $X$ is Hausdorff if and only if $Y$ is Hausdorff.
\end{ncor}
\begin{proof}
  Let $f : X \to Y$ be a homeomorphism. Then we apply the previous proposition as we have two continuous injections, $f : X \to Y$ and $f^{-1} : Y \to X$ which gives us both directions.
\end{proof}