% !TEX root = ../notes.tex

\begin{ncor}
   Let $f : X \to X'$ and $g : Y \to Y'$ be continuous functions and define $f \ti g : X\ti Y \to X'\ti Y'$ by $(f \ti g)(x, y) = (f(x), g(y))$. Then $f \ti g$ is continuous.
\end{ncor}
\begin{proof}
  We have $p_{X'}\circ (f \ti g) = f\circ p_x : X \ti Y \to X$, since both functions take $(x, y)$ to $f(x)$.
  \[\begin{tikzcd}
  	{X\times Y} & {X'\times Y'} \\
  	X & Y
  	\arrow["{f\times g}", from=1-1, to=1-2]
  	\arrow["f", from=2-1, to=2-2]
  	\arrow["{p_X}"', from=1-1, to=2-1]
  	\arrow["{p_{X'}}", from=1-2, to=2-2]
  \end{tikzcd}\]
  Since $p_X$ is continuous (Lemma 3.24) and $f$ is continuous, it follows that their composite $f \circ p_X$ is continuous. Hence $p_{X'} \circ (f \ti g)$ is continuous. Similarly, so is $fp_{Y'}\circ (f \ti g)$. so by Lemma 3.24, $f\ti g$ is continuous.
\end{proof}

\begin{ncor}
  For any topological space $X$, the diagonal map $\D : X \to X\ti X$, $\D(x) = (x, x)$, is continuous.
\end{ncor}
\begin{proof}
  Let $p_1, p_2$ be the projections from $X\ti X$ to the first and second factors. Then $p_1 \circ \D$ and $p_2\circ \D$ coincide with the identity function $id_X : X \to X$ (which is most certainly continuous). So by Lemma 3.24 $\D$ is continuous.
\end{proof}

\begin{ncor}
   For continuous functions $f, g : X \to \R$, the functions $f \pm g$, $fg$ etc. are continuous.
\end{ncor}
\begin{proof}
  Let $m : \R \to \R \to \R$ be the multiplication function $m(x, y) = xy$. We know that this is continuous. Now $fg : X \ti X \to \R$ is the composite of the continuous maps $\D : X \ti X \to X$. $f \times g : X\ti X \to \R\ti \R$ and $m : \R \ti \R \to \R$, so it is continuous. The cases for $f + g$, and $f - g$ are similar.
\end{proof}

\subsection{Compact Spaces}
We look back to the function on the closed and bounded interval. We need both `closed' and `bounded'. We can find a continuous function $(0, 1) \to \R$ defined by $\frac{1}{x}$ and $[0, \infty)$ defined by $x$ that have no maximum.\\

\noindent
It is also important that we work with $\R$ not $\Q$. On the closed bounded subset $S = [1, 2] \cap \Q$, the function,
$$ f(x) = \frac{1}{x^2 - 2} $$
is well defined as $x^2 \ne 2$ but has no maximum or minimum. The essential property of $[a, b] \sub \R$ which makes this work is \textbf{compactness}.

\begin{ndefi}[Open Cover, Compact]
  Let $X$ be a topological space and let $A$ be any subset of $X$,
  \begin{enumerate}
    \item An \textbf{open cover} of $A$ in $X$ is a family of open sets $U_i$, $i \in \cI$ such that,
    $$ A \sub \bigcup_{i \in \cI} U_i $$
    \item $A$ is \textbf{compact} if every open cover $U_i$, $i \in \cI$ has a finite subcover, that is there are $i_1, \dots i_m \in \cI$ with,
    $$ A \sub U_{i_1} \cup U_{i_2} \cup \dots \cup U_{i_n} $$
  \end{enumerate}
\end{ndefi}

\begin{remark}
  Taking $A$ to be $X$ itself, $X$ is compact if, for any family of open sets $U_i$, $i \in \cI$ with,
  $$ X = \bigcup_{i \in \cI} U_i $$
  we have,
  $$ X = \bigcup_{j=1}^n U_{i_j}  $$
  for some finite subcover $\{i_1, i_2, \dots, i_n\}$ of $\cI$.
\end{remark}

\begin{eg}
  The intervals $U_n = (n - 1, n + 1)$ in $\R$ for $n \in \Z$ form an open cover of $\R$,
  $$ \bigcup_{n \in \Z} U_n = \R $$
  but we cannot write $\R$ as a finite union of these intervals. Hence $\R$ is not compact.
\end{eg}

\begin{eg}
  Any finite topological space $X = \{x_1, \dots, x_n\}$ is compact. If we have the open cover,
  $$ X = \bigcup_{i \in \cI} U_i $$
  then, $1 \le j \le n$ we can pick $i_j \in \cI$ with $x_j \in U_{i_j}$. This means that $U_{i_1}\cup \dots \cup U_{i_n}$.
\end{eg}

\subsubsection{Compact subsets of $\R$}
Our aim in this section is to show that a subset of $\R$ is compact if and only if it is closed and bounded. So for example $[0, 1] \cup [2, 3]$ is compact. The hardest part of this is to show that a closed interval $[a, b]$ is compact. This is the Heine-Borel Theorem. We will prove some easier results in greater results first,

\begin{nlemma}
  Let $(X, d)$ be a metric space. Then any compact subset $A$ of $X$ is bounded.
\end{nlemma}
\begin{proof}
  For $n \ge 1$, let $U_n = \{y \in X : d(y,x) < n\}$, the open ball with center $x$ and radius $n$. For every $y \in X$ we can find $n > d(y, x)$, so
  $$ X = \bigcup_{n=1}^\infty U_n, $$
  and we have,
  $$ A \sub \bigcup_{n=1}^\infty U_n $$
  that is, it's an open cover. Since $A$ is compact, $A$ is contained in the union of finitely many of these sets $U_{n_1}, \dots, U_{n_k}$. Taking $R = \max\{n_1, \dots, n_k\}$ we have $A \sub U_R$, so $d(a, x) < R$ for all $a \in A$. Hence $A$ is bounded.
\end{proof}

\noindent
A compact subset of a metric space is also closed. In fact this holds in any Hausdorff space.

\begin{nlemma}
  A compact subset $C$ of a Hausdorff space $X$ is closed.
\end{nlemma}
\begin{proof}
  We have to show that $D = X\sm C$ is open. We will show that, for each $x \in D$, there is an open set $U_x$ with $x \in U_x \sub D$. Then $D = \bigcup_{x \in D} U_x$ is open.\\

  \noindent
  Let $x \in D$. Since $X$ is Hausdorff, for each $y \in C$, we can find open sets $A_y$ and $B_y$ with $x \in A_y$ and $y \in B_y$ and $A_y \cap B_y = \vn$. Then $C \sub \bigcup_{y \in C} B_y$. Since $C$ is compact, there are $y_1, y_2, \dots, y_n \in C$ with $C \sub \bigcup_{k=1}^n B_{y_k}$. However,
  $$ \left( \bigcup_{k=1}^n B_{y_k} \right) \cap \left( \bigcap_{k=1}^n A_{y_k} \right) = \vn $$
  so $\bigcap_{k=1}^n A_{y_k}$ is a subset of $D$ containing $x$. Moreover, $U$ is open since is it is the intersection of finitely many open sets.
\end{proof}

\noindent
Combining the two previous results, we get that,
\begin{ncor}
  A compact subset of a metric space is closed and bounded. In particular, any compact subset of $\R$ is closed and bounded.
\end{ncor}

\noindent
So far we have proven only that finite sets are compact,
\begin{nthm}[Heine-Borel Theorem]
  Let $a, b \in \R$ with $a < b$. Then the closed, bounded interval $[a, b]$ is compact.\label{thm:HeineBorel}
\end{nthm}

\noindent
The proof uses the completeness property of $\R$, every non-empty subset of $\R$ which is bounded above has a least upper bound.

\begin{proof}
  Let $U_i$, $i \in \cI$ be an open cover of $[a, b]$. We must show it has a finite subcover. Consider the set,
  $$ S = \{x\in [a, b] : [a, x] \sub U_{i_1} \cup U_{i_2} \cup \dots \cup U_{i_n} \text{ for some } i_1, \dots, i_n \in \cI\} $$
  We will now show the following,
  \begin{enumerate}[(i)]
    \item $S$ is bounded above by $b$,
    \item $S \ne \vn$,
    \item the least upper bound of $S$, is $b$,
    \item $b \in S$.
  \end{enumerate}
  \noindent
  This then means that $[a, b] \sub U_{i_1}\cup \dots \cup U_{i_n}$ for some $i_1, \dots, i_n$ as required.\\

  \noindent
  (i), This is obvious, then if $x \in S$, then $x \le b$.\\

  \noindent
  (ii), We seek an element that is in $S$. It makes sense to let this be $a$. Since $[a, b] \sub \bigcup_{i \in \cI} U_i$, there is some $i_1$ with $a \in U_{i_1}$. Then $[a, a] \sub U_{i_1}$, so $a \in S$.\\

  \noindent
  (iii), Let $c$ be the least upper bound of $S$. Then $a \le c \le b$. We have seen that $a \in U_{i_1}$ for some $i_1$. As $U_{i_1}$ is open, we have
  $$ [a, a+\e ) \in U_{i_1} \text{ for some }\e > 0 $$
  so $a + \frac{1}{2}\e \in S$. Thus $c > a$.\\

  \noindent
  Suppose $a < c < b$. We have $c \in U_i$ for some $i$. As $U_i$ is open, then there is some $\e > 0$ with $(c - \e, c + \e)\sub U_i$. As $c - \e$ is not an upper bound of $S$, we can find some $x \in S$ with $x > c - \e$. Then,
  $$ [a, x] \sub U_{i_1}\cup \dots \cup U_{i_n} $$
  for some $i_1, \dots, i_n$. Hence,
  $$ [a, c - \e] \sub U_{i_1} \cup \dots \cup U_{i_n} $$
  and so,
  $$ [a, c + \e) \sub U_{i_1} \cup \dots \cup U_{i_n}\cup U_i $$
  This shows that $c + \frac{1}{2}\e \in S$, contradicting the choice of $c$ as the least upper bound for $S$. Hence $c = b$.\\

  \noindent
  (iv), $b \in U_i$, for some $i$, so $(b - \e, b] \subset U_i$ for some $\e > 0$. As $b$ is the least upper bound for $S$, we can find some $x > b - \e$ with $x \in S$ with $x \in S$, so
  $$ [a, x] \sub U_{i_1}\cup \dots \cup U_{i_n} $$
  for some $i_1,\dots, i_n$. Then,
  $$ [a, b] = [a, x] \cup (b - \e, b] \sub U_{i_1}\cup \dots \cup U_{i_n} \cup U_i $$
  Hence $b \in S$. These give us that $[a, b]$ is compact.
\end{proof}

\noindent
\textbf{Note:} An open bounded interval isn't compact. Take $(-1, 1) = \bigcup_{n \ge 1} \left( -1 + \frac{1}{n}, 1 - \frac{1}{n} \right)$. You can't get away with finitely many of these, it's an open cover where no finite subcover will work.

\begin{nlemma}[]
  Let $C$ be a compact subset in a topological space $X$ and let $A$ be a closed and bounded subset of $X$ with $A\sub C$. Then $A$ is compact.\label{lemma:7.9}
\end{nlemma}
\begin{proof}
  Let $U_i$ for $i \in \cI$ be open sets such that $A\sub \bigcup_{i \in \cI} U_i$. We must show that $A$ is contained in the union of finitely many of the $U_i$. Let $B = X\sm A$. Then $B$ is open in $X$, since $A$ is closed. We have,
  $$ C \sub X = B \cup \bigcup_{i \in \cI} U_i $$
  Since $C$ is compact, there are $i_1, \dots, i_n \in \cI$ so that,
  $$ C \sub B \cup \bigcup_{k=1}^n U_{i_k} $$
  As we know that as $B = X\sm A$ and $C \sub X$, then $A \sub \bigcup_{k=1}^n U_{i_k}$, as required.
\end{proof}

\begin{ncor}
   A subset of $\R$ is compact if and only if it is closed and bounded.
\end{ncor}
\begin{proof}
  We saw that a compact subset must be closed and bounded. Conversely, let $A$ be a closed, bounded subset of $R$. Since $A$ is bounded, there is some $R > 0$ so that $|a| < R$ for all $a \in A$. Then $A \sub [-R, R]$. The closed, bounded interval $[-R, R]$ is compact by Theorem \ref{thm:HeineBorel}. So $A$ is compact by Lemma \ref{lemma:7.9}.
\end{proof}

\subsection{Middle Cantor Set}

Compact sets can be a lot more complicated than we give them credit for. Consider the following construction of the Middle Third Cantor Set. Start with $A_0 = [0, 1]$, then divide it into three pieces and remove the middle third, that is remove $(\frac{1}{3}, \frac{2}{3})$. Hence, $A_1 = [0, \frac{1}{3}]\cup [\frac{2}{3}, 1]$. Then do it again, $A_2 = [0, \frac{1}{9}] \cup [\frac{2}{9}, \frac{1}{3}] \cup [\frac{2}{3}]\cup [\frac{8}{9}, 1]$.
Hence we will have $A_n$ as a union of $2^n$ closed intervals of length $3^{-n}$. We further define,
$$ A = \bigcap_{n=0}^\infty A_n $$
Then we can look at the length, then find that the length is zero, but it still an infinite set. We find $A$ is closed and bounded. Hence $A$ is compact.\\
\begin{figure}[!ht]
\centering
\begin{tikzpicture}[decoration=Cantor set,very thick]
  \draw decorate{ (0,0) -- (3,0) };
  \draw decorate{ decorate{ (0,-.5) -- (3,-.5) }};
  \draw decorate{ decorate{ decorate{ (0,-1) -- (3,-1) }}};
  \draw decorate{ decorate{ decorate{ decorate{ (0,-1.5) -- (3,-1.5) }}}};
\end{tikzpicture}
\caption{Middle Third Cantor Set}
\end{figure}