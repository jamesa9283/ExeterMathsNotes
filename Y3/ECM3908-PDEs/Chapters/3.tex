% !TEX root = ../notes.tex

\section{Canonical Form of first and second order PDEs}
\subsection{Coordinate Method for the first order constant coefficient PDEs}

Consider $au_x + bu_y = 0$ where $a, b$ are constant. We can represent this as $(a b) \cdot \grad u$, if we let $\vec v = (u v)$ and so $\grad u \cdot v = 0$. This is the directional derivative of $u(x, y)$ in the direction of the vector of coefficients $(a b)$, and it's zero. Now let us change the coordinates, such that the coordinates have their axis is parallel to $(a b)$. Then $(\xi, \eta) = \left[ (x, y) \cdot (a b),\, (x, y) \cdot (b -a) \right]$. Therefore, $\xi = ax + by$ and $\eta = bx - ay$.

\begin{eg}
  Solve $au_x + bu_y + cu = g(x, y)$ using the coordinate method. We let $\xi = ax + by$ and $\eta = bx - ay$. Then we can write out $u_x = u_\xi\xi_x + u_\eta\eta_x = au_\xi + bu_\eta$ and $u_y = u_\xi\xi_y + u_\eta\eta_y = bu_\xi - au_\eta$. Now we substitute these into our equation and get, $(a^2 + b^2)u_\xi + cu = g(\xi, \eta)$. We can now solve this with an integrating factor and get that,
  $$u(\xi, \eta) = f(\eta)e^{-\frac{c}{a^2 + b^2}\xi} + e^{-\frac{c}{a^2 + b^2}}\int e^{-\frac{c}{a^2 + b^2}\xi} g(\xi, \eta)\, d\xi$$
\end{eg}

\begin{exercise}
  Find the general solution of $-3u_x + 4u_y + 5u = e^{x + 3y}$.
\end{exercise}
\begin{solution}
  $u(x, y) = e^{\frac{1}{2}x - y}\left( f(4x + 2y) + \frac{1}{15}e^{\frac{1}{2}x + 4y} \right)$
\end{solution}

\subsection{Classification of second order linear PDEs.}

Let us write the general form of a second order PDE,
$$ A(x, y)u_{xx} + B(x, y)u_{xy} + C(x, y)u_{yy} + D(x, y)u_x + E(x, y)u_y + F(x, y)u = G(x, y) $$

We want to classify the PDE, then we can find the right method to solve them. The classification is inspired by how we classify a quadratic curve in analytic geometry, we write the following,
$$ Ax^2 + Bxy + Cy^2 + Dx + Ex + F = 0 $$
$$ \begin{cases}
  B^2 - 4AC > 0 & \text{hyperbola}\\
  B^2 - 4AC = 0 & \text{parabola} \\
  B^2 - 4AC < 0 & \text{ellipse} \\
\end{cases} $$

\noindent
A similar thing happens, but the signs of the coefficients may change in the domain of solution. However, we assume that this case will never happen, so it's classification holds for the whole domain of solution. The same definition as for quadratic curve works for PDEs. \\

\noindent
This classification is on an invariant that is invariant under a change of coordinates. So we consider some change of variables that is guided by the characteristics. Let $\xi = \xi(x, y)$ and $\eta = \eta(x, y)$. We assume that $(x, y) \mapsto (\xi, \eta)$ is one-to-one, that is $J \ne 0$. Let us input this change of coordinates, $u_x = u_\xi\xi_x + u_\eta\eta_x$ and $u_y = u_\xi\xi_y + u_\eta\eta_y$ and $u_{xx} = u_{\xi\xi}\xi_x^2 + u_{\xi\eta}\xi_x\eta_x + u_\xi\xi_{xx} + u_{\eta\eta}\eta_x^2  + u_{\xi\eta}\xi_x\eta_x + u_\xi\xi_{xx} $
and after a load of laborious maths we get,
$$ A^*u_{\xi\xi} + B^*u_{\eta\xi} + C^*u_{\eta\eta} + D^*u_\xi + E^*u_\eta + F^*u = G^* $$
where $A^* = A\xi_x^2 + B\xi_x\xi_y + C\xi_y^2$, $B^* = 2A\xi_x\eta_x + B(\xi_x\eta_y + \xi_y\eta_x) + 2C\xi_y\eta_y$, $C^* = A\eta_x^2 + B\eta_x\eta_y + C\eta_y^2$, $D^* = A\xi_{xx} + B\xi_{xy} + C\xi_{yy} + D\xi_x + E\xi_y$, $E^* = A\eta_{xx} + B\eta_{xy} + C\eta_{yy} + D\eta_x + E\eta_y$, $F^* = F$ and $G^* = G$
we see that this PDE has the same form. If we consider the discriminant of the PDE, this is ${B^*}^2 - 4A^*C^*$, and if we substitute the definitions, we can prove that this is just $J^2(B^2 - 4AC)$ and so the sign of the discriminant is invariant under this change of coordinates. Hence, we can write the PDE in the form,
$$ Au_{xx} + Bu_{xy} + C_{yy} = H(x, y, u_x, u_y) $$

\begin{eg}
  The wave equation $u_{xx} - u_{tt} = 0$ is hyperbolic.
\end{eg}
\begin{eg}
  The Laplace equation $u_{xx} + u_{yy} = 0$ is elliptic.
\end{eg}
\begin{eg}
  The diffusion / heat equation $u_{t} - u_{xx} = 0$ is parabolic.
\end{eg}
\begin{eg}
  Classify,
  \begin{itemize}
    \item $u_{xx} - u{xy} = 0$, this is hyperbolic
    \item $4u_{xx} + 6u_{xy} + 9u_{yy} = 0$, this is elliptic.
  \end{itemize}
\end{eg}

\subsection{Cannonical Forms of the semilinear second order PDE.}


\textbf{PDE Problem: }We have a system or singular PDEs and we know of different types of boundary conditions; Dirichlet, Neumann and Robin boundary conditions, or it can be of mixed type.\\

\noindent
Assume that the general form of our PDE is,
$$ A(x, y)u_{xx} + B(x, y)u_{xy} + C(x, y)u_{yy} = H(x, y, u_x, u_y) $$
the cauchy data is to specify $\pd u {\vec n}$ and $u(x, y)$ on some curve $\La$ in the $(x, y)$ - plane. Let $x = x_0(s)$, $y = y_0(s)$, $u = u_0(x_0(s), y_0(s)) = u_0(s)$ and $\pd u {\vec n} = v_0(s)$. However it's easier to write the last in other terms as $\di {u_0} s = \pd u x \di {x_0} s + \pd u y \di {y_0} s$ and $v_0(s) = \grad u \cdot {\vec n} = \pd u x$. What is $\vec n$?
We know that $(\di {x_0} s\, \di {y_0} s)$ is tangent to the curve $\La$, hence we let $\vec n = (\di{y_0} s, -\di{x_0} s)$. Hence we can write that $v_0(s) = (\pd u x, \pd u y) \cdot (\di{y_0} s, -\di{x_0} s) = \pd u x \di{y_0}s - \pd u y \di{x_0} s$. Therefore we can write the cauchy data as, $x = x_0(s)$, $y = y_0(s)$, $u = u_0(s)$, $\pd u x = p_0(s)$ and $\pd u y = q_0(s)$.\\

\noindent
We now want a condition for a unique solution in neighbourhood of the initial curve, and then to find the characteristics of the equation. Consider $\di p s = \pdd u x \di {x_0} s + \pdxy u \di {y_0} s$ and $\di q s = \pdd u x + \pdxy u \di y s$. We now have our second order PDE, $Au_{xx} + Bu_{xy} + Cu_{yy} = H$. Now let us write these in matrix form,
$$ \begin{pmatrix}
  A & B & C \\
  \di {x_0} s & \di {y_0} s & 0 \\ 0 & \di {x_0} s & \di {y_0} s
\end{pmatrix} \begin{pmatrix}
  u_{xx} \\ u_{xy} \\ u_{yy}
\end{pmatrix} = \begin{pmatrix}
  H \\ \di p s \\ \di q s
\end{pmatrix} $$
Here we want $\hat A \ne 0$. Now we want the determinant of the coefficiet matrix,
$$ |\hat A| = A\left(\di {y_0} s\right)^2 + B\di {x_0} s \di {y_0} s + C\left(\di {x_0} s\right)^2 $$
if the characteristics are parameterised by $\t$ then from $*$, we can say that
$$ A\left(\di y \t\right) - B\di x \t \di y \t + C\left(\di x \t\right)^2 =0 $$
Hence, after some manipulation,
$$ A\left(\di y x\right)^2 - B\di y x + C = 0 $$
or $A\l^2 - B\l + C = 0$ and this is the characteristic equation of the second order semi-linear PDE.

We have already shown that a second order PDE is invariant under change of coordinates and achieved,
$$ A^*u_{\xi\xi} + B^*u_{\xi\eta} + C^*u_{\eta\eta} = H(\xi, \eta, y, u_\xi, u_{\eta}) $$
We want to choose characteristics such that $A^* = C^* = 0$, however we know that $A^* = A\xi_x^2 + B\xi_x\xi_y + C\xi_y^2$ and that $C^* = A\eta_x^2 + B\eta_x\eta_y + C\eta_y^2$. Now we can write them in the form,
$$ A\g_{x}^2 + \g_x\g_y + c\g_y^2 = 0 $$
and so we get,
$$ A\left(\frac{\g_x}{\g_y}\right)^2 + B\left(\frac{\g_x}{\g_y}\right) + C = 0 $$
Now assume that one of them is a constant. Hence we differentiate, $d\g = \g_xdx + \g_ydy = 0$ and so $\di y x = - \frac{\g_x}{\g_y}$ and so, $A(\di y x) - B\di y x + C = 0$ and we recover the characteristic equation. Hence we now want to find roots of this equation and get,
$$ \di y x = \frac{B \pm \sqrt{B^2 - 4AC}}{2A} $$
and so we have two equations for the characteristics and so we integrate to find the equations for characteristic curves. Here $\xi = \phi(x, y)$ and $\eta = \psi(x, y)$.

Let us consider three different cases, \\
\addcontentsline{toc}{subsubsection}{Hyperbolic Canonical Form}\textbf{Hyperbolic: }$A^*u_{\xi\xi} + B^*u_{\xi\eta} + C^*u_{\eta\eta} = H^*$ and so we have $B^2 - 4AC > 0$ and so we get,
$$ B^*u_{\xi\eta} = H^*$$
or,
$$ \frac{\partial^2 u}{\partial \xi \partial \eta} = \frac{H^*}{B^*} = H $$
This is the first canonical form for hyperbolic PDEs. We can find the second order, let $\a = \xi + \eta$ and $\b = \xi - \eta$. Therefore $u_{\xi} = u_\a \a_\xi + u_\b \b_\xi = u_\a + u_\b$ and $u_{\xi\eta} = u_{\a\a} - u_{\a\b} + u_{\b\a} - u_{\b\b} = u_{\a\a} - u_{\b\b} = H_2(\a, \, u_\a, u_\b)$. This is the second canonical form of hyperbolic PDEs.\\

\noindent
\addcontentsline{toc}{subsubsection}{Parabolic Canonical Form}\textbf{Parabolic Equations: }Here we have $B^2 - 4AC = 0$ and so we have a repeated root. Therefore, $\xi = c_1$ or $\eta = c_2$. Therefore, if $A^* = A\xi_x^2 + B\xi_x\xi_y + c\xi_y^2 = (\sqrt A \xi_x + \sqrt c \xi_y)^2 = 0$. Now consider $B^* = 2A\xi_x\eta_x + B(\xi_x\eta_y + \xi_y\eta_x) + 2C\xi_y\eta_y = 2(\sqrt A\xi_x + \sqrt C \xi_y)(\sqrt A \eta_x + \sqrt C\eta_y) = 0$ as $(\sqrt A \xi_x + \sqrt C\xi_y) = 0$.
Hence, $A^* = B^* = 0$ and so we get the PDE of the form,
$$u_{\eta\eta} = \frac{H^*}{C^*} = H_3(\xi, \eta, u_\xi, u_\eta)$$
 This is the canonical form for parabolic PDEs. If you set $C^* = 0$, you just get $u_{\xi\xi} = H_3(\xi, \eta, u_\xi, u_\eta)$. Similarly to before, if you solve $\di y x = \frac{B }{2A}$ and call $\xi$ the solution, then let $\eta = y$ you get a one-to-one mapping. Similarly, letting $\xi = x$ gives the same outcome.

\noindent
\addcontentsline{toc}{subsubsection}{Elliptic Canonical Form}\textbf{Ellipic PDEs:} The characteristic equation is,
$$ A\left(\di y x\right)^2 - B\left(\di y x\right) + C = 0 $$
We know that $B^2 - 4AC < 0$, let the roots be $\hat \a(x, y) \pm i \hat \b(x, y)$, now we integrate this and get $\xi = \a(x, y) + i \b(x, y)$ and $\eta = \a(x, y) - i\b(x, y)$. Now we let $\a = \frac{1}{2}(\xi + \eta)$ and $\b = \frac{1}{2i}(\xi - \eta)$. Now we reduce using this change of transformation. The reduced PDE is,
$$ A^{**}u_{\a\a} + B^{**}u_{\a\b} + C^{**}u_{\b\b} = H^{**}(\a, \b, u, u_\a, u_{\b}) $$
What happens when we set $A^* = C^* = 0$. We substitute in for the derivatives into the equation for $A^*$,
\begin{align*}
  A^* &= A(\a_x + i\b_x) + B(\a_x + i\b_x)(\a_y + i\b_y) + C(\a_y + i\b_y)^2\\
  &= A(\a_x^2 + B\a_x\a_y + C\a_y^2) - (A\b_x^2 + B\b_x\b_y + C\b_y^2) + i\left[ 2A\a_x\b_x + B[\a_x\b_y + \b_x\a_y\right] + 2C\a_y\b_y ]
\end{align*}
and the same thing for $C^*$ and get that, $C^* = A(\a_x^2 + B\a_x\a_y + C\a_y^2) - (A\b_x^2 + B\b_x\b_y + C\b_y^2) - i\left[ 2A\a_x\b_x + B[\a_x\b_y + \b_x\a_y] + 2C\a_y\b_y \right]$ and finally
\begin{align*}
  B^* &= 2A\xi_x\eta_x +B(\xi_x\eta_y + \xi_y\eta_x) + 2C\xi_y\eta_y \\
  &= 2A\a_x\b_x + B[\a_x\b_y + \b_x\a_y] + 2C\a_y\b_y
\end{align*}

Hence, $A^{**} - C^{**} + iB^{**} = 0$ and also $A^{**} - C^{**} - iB^{**} = 0$. Therefore, $A^{**} = C^{**}$ and $B^{**} = 0$. Hence, we can write,
$$ u_{\a\a} + u_{\b\b} = \frac{H^{**}}{A^{**}} = H_4(\a, \b, u, u_\a, u_\b)  $$
This is the canonical form of Elliptic PDEs.

