% !TEX root = ../notes.tex
w
\begin{nthm}[Cauchy Problem for first order PDEs]
  Suppose $C$ is a given curve in the $(x, y)$-plane with it's parametric equation, $x = x_0(t)$ and $y = y_0(t)$ where $t \in I \sub \R$ and derivatives $x_0(t)$ and $y_0'(t)$ are piecewise continuous such that they satisfy $x_0'^2 + y_0'^2 \ne 0$. Suppose that $u = u_0(t)$ is a given function on the curve $C$. Then there exists a solution $u = u(x, y)$ of the equation,
  $$ F(x, y, u, u_x, u_y) = 0 $$
  in the domain $D \sub \R^2$ containing the curve $C$ for all $t \in I$. $u(x, y)$ satisfies $u(x_0(t), y_0(t)) = u_0(t)$ for all values of $t \in I$.
\end{nthm}

Now for a lot of examples,
\begin{eg}
  Find the general solution of the PDE, $xu_x + yu_y = u$. We let $a = x$, $b = y$ and $c = u$, hence,
  $$ \frac{dx}{x} = \frac{dy}{y} = \frac{du}{u} $$
  and now we split and solve to get $y = c_1x$ and $u = c_2x$. Hence, the solution is $f\left( \frac{y}{x}, \frac{u}{x} \right) = 0$. We could have written this as $\frac{u}{x} = F\left( \frac{y}{x} \right)$ or $u(x, y) = xF\left( \frac{y}{x} \right)$.
\end{eg}

\begin{eg}
  Obtain the general solution of the linear equation $xu_x + yu_y = nu$ where $n$ is a constant. Here we do the same thing as above,
  $$ \frac{dx}{x} = \frac{dy}{y} = \frac{du}{nu} $$
  and we get the solution $u(x, y) = x^nF\left( \frac{y}{x} \right)$
\end{eg}

\begin{eg}
  Find the general solution of $x^2u_x + y^2u_y = (x + y)u$. Here the characteristic is,
  $$ \frac{dx}{x^2} = \frac{dy}{y^2} = \frac{du}{(x + y)u}$$
  The first function is easy to construct, we find that $\frac{1}{y} - \frac{1}{x} = c_1$ and the second can be found from
  \begin{align}
    (x + y)udx &= x^2du \\
    (x + y)udy &= y^2du
  \end{align}
  and then solving. Hence $\frac{x-y}{u} = c_2$. Then we can say the solution is $f(\frac{y - x}{xy}, \frac{x - y}{u}) = 0$ or $u(x, y) = (x - y)h(\frac{y - x}{xy})$
\end{eg}
\begin{exercise}
  Verify the solution.
\end{exercise}

\begin{eg}
  Obtain the general solution of the linear equation $u_x - u_y = 1$ with the Cauchy data $u(x, 0) = x^2$. We find the characteristics,
  $$ \frac{dx}{1} = \frac{dy}{-1} = \frac{du}{1} $$
  and so we find that $y + x = c_1$ and $u - y = c_2$. Therefore, $u(x, y) = -y - F(x + y)$ and using the Cauchy data we can get that $u(x, y) = (x + y)^2 - y$
\end{eg}

\noindent
\textbf{Review:} We have $a(x, y, u) \pd u x + b(x, y, u)\pd u y = c(x, y, u)$ and we write this as $(a, b, c) \cdot (u_x, u_y, u_z) = 0$ and wrote $u(x, y)$ as the third coordinate and then considered the level surface $f(x, y, u) = u(x, y) - u$ and got that $\nab f = (u_x, u_y, -1)$ and hence concluded that $(a, b, c) \cdot \nab f = 0$ recovers our PDE. $\nab f$ is perpendicular to the solution surface, and $(a, b, c)$ is tangent to the surface and some curve in the surface must have tangent vector $(a, b, c)$ which we call the characteristic curve.
$$ (a, b, c) = \left(\di x t, \di y t, \di u t\right) $$
and this yielded a way to solve the PDE. How do we now parmaterise this solution surface?\\

\newpage
\subsection{Characteristic Projections}

\noindent
We shall now introduce characteristic projections. Suppose that $u(x, y)$ is specified along some curve $\La$ in the $(x, y)$-plane we then have $u = u_0(s)$ when $x = x_0(s)$ and $y = y_0(s)$ where $s$ parameterises $\La$ in 3D, $(x_0(s), y_0(s), u_0(s))$ is our initial curve.\\

\begin{wrapfigure}{r}{0.5\textwidth}
  \centering
  \vspace{-20pt}
  \resizebox{0.5\textwidth}{!}{\input{./figures/ccurves.pdf_tex}}
  \caption{Geometric Interpretations.}
  \vspace{-20pt}
\end{wrapfigure}


Characteristics pass through this curve and they are tangent to $(a, b, c)$, so
$$ \di x \tau = a, \quad \di y \tau = b, \quad \di u \tau = c $$
with initial conditions $x = x_0(s)$, $y = y_0(s)$ and $u = u_0(s)$ at $\tau = 0$. Then we know that the parameterised surface will be $(x(s, \tau), y(s, \tau), u(s, \tau))$and these are the parametric equations of the solution surface.

\begin{eg}
  Solve $\pd u x + \pd u y = 1$ subject to the boundary data $u = 0$ when $x + y = 0$. We can solve this by setting up characteristics,
  $$ \di x \tau = 1 \quad \di y \quad \di y \tau = 1 \quad \di u \tau = 1 $$
  Then we need initial conditions so we will find solutions depending on the parameter $s$, so our intial conditions are,
  $$ x = s \quad y = -s \quad u = 0 \quad \text{at $\tau = 0$} $$
  and then we solve this system and get,
  $$ x(\t) = \t + s \quad y(\t) = \t - s \quad u(\t) = \t $$
  Now we can eliminate $\t$ and get the solution as $x + y = 2u$ or $u = \frac{x + y}{2}$
\end{eg}
\begin{eg}
  Solve the PDE, $u_t + uu_x = 1$ for $u = u(x, t)$ in $t > 0$ subject to the initial condition $u = x$ at $t = 0$. We have the characteristics of,
  $$ \di t \tau = 1 \quad \di x \tau = u \quad \di u \tau = 1 $$
  We can solve the first and third very quickly, more specifically, $t(\tau) = \tau + c_1$ and $u(\tau) = \tau + c_3$, now given the data we can form some initial conditions, $x = s$, $u = s$ for $\tau = 0$. Hence, we have $t = \tau$ and $u = \tau + s$. Now we solve the second equation by plugging in $u$, $\di x \tau = \tau + s$ and now we can solve them, $x(\tau) = \frac{1}{2}\tau^2 + s\tau + s$, or $s = \frac{x(\tau) - \frac{1}{2}\tau^2}{\tau + 1}$.
  Now we plug in the other solutions and find,
  $$ u(x, t) = \frac{\frac{1}{2}t^2 + t + x}{t + 1} $$
\end{eg}

We start with another example,
\begin{eg}
  Find the solution of $u(x + y)u_x + u(x - y)u_y = x^2 + y^2$ with the Cauchy data $u = 0$ on $y = 2x$. We start with the characteristics,
  $$ \frac{dx}{u(x + y)} = \frac{dy}{u(x - y)} = \frac{du}{x^2 + y^2} $$ we can verify that $ydy + xdy - udu = 0$. Now we can turn this equation into an exact equation,
  $$ d\left(xy - \frac{1}{2}u^2\right) = xdy + ydx - udu = 0 $$
  and so, $xy - \frac{1}{2}u^2 = c_1 = \phi(x, y, u)$. We now need to find a $\psi$ function, consider $xdx - ydy - udu = 0$ is satisfied by our characteristics and so we can form another exact equation,
  $$ d\left( \frac{1}{2}x^2 - \frac{1}{2}y - \frac{1}{2}u\right) = 0 $$
  and so, $\psi = x^2 - y^2 - u^2$. Therefore the general solution is,
  $$ u(x, y) = f\left(2xy - u^2, x^2 - y^2 - u^2\right) = 0 $$
  Now we consider the Cauchy data, $u(x, 2x) = 0 = f(4x^2, -3x^2)$ and so $\frac{1}{2}c_1 = -\frac{1}{3}c_2$, hence, $\frac{1}{4}\left( 2xy - u^2 \right) = -\frac{1}{3}\left(x^2- y^2 - u^2\right)$ and so $7u^2 = 6xy + 4(x^2 + y^2)$
\end{eg}

Now we revisit the Cauchy Theorem in a slightly different form.
\begin{nthm}[Cauchy Problem (revisited)]
  Suppose that $x_0(t)$, $y_0(t)$ and $u_0(t)$ are continuously differentiable functions of $t$ in $t \in [0, 1]$. Further suppose that $a(x, y, u)$, $b(x, y, u)$ and $c(x, y, u) \in \cc^1$ with respect to their arguments t respect to some domain $D$ of the $(x, y, u)$-space. $\La : x = x_0(t) \quad y = y_0(t) \quad u = u_0(t)$ for $t \in [0, 1]$ and $y'_0(t)a(x_0, y_0, u_0) - x'_0(t)b(x_0, y_0, z_0) \ne 0$ then there exists a unique solution $u = u(x, y)$ of the quasi-linear PDE $au_x + bu_y = c$.
\end{nthm}

This condition assures that the intial curve is not in the same direction as the solution as they should arise from the Cauchy data. Another interpretation of this is there is a, one to one mapping from $x, y, z$ to $t, \tau, s$. We can now state the Cauchy-Kowalevski Theorem,
\begin{nthm}[Cauchy-Kowaleski]
  A necessary condition for a unique solution $u(x, y)$ to exist in a neighbourhood $\La$ is for the first derivative of $u(x, y)$ to be determined on $\La$
\end{nthm}

\noindent
We say that along the curve if $\La$ is parameterised by $s$, then $u_0(x_0(s), y_0(s)) = u_0(s)$. Any point of this curve has a projection onto the $(x,y)$-plane. We want to find the first order derivatives of $u$,
$$ \di {u_0} s = \pd {u_0} x \di {x_0} s + \pd{u_0} y\di{y_0} s $$
and the PDE says,
$$ a\pd u x + b\pd u y = c $$
Now, let us put these in a matrix and find a condition,
$$ \begin{pmatrix}
  a & b \\ \di{x_0} s & \pd {y_0} s
\end{pmatrix} \begin{pmatrix}
  \pd u x \\ \pd u y
\end{pmatrix} = \begin{pmatrix}
  c \\ \di{u_0} s
\end{pmatrix}$$
Hence, we need the determinant to be non-zero and so $a\di{y_0} s - b\di{x_0} s \ne 0$. This is just what we said in the Cauchy Problem Theorem. Here is the formal statement,
\begin{nthm}[]
  The PDE $a\pd u x + b\pd u y = c$ has a unique analytical solution in some neighbourhood of $\La$, provided $a, b$ and $c$ are analytic and satisfy $a \di{y_0} s - b\di{x_0} s \ne 0$
\end{nthm}

Some more examples,
\begin{eg}
  Solve the following PDE, $\pd u t + xu\pd u x = u$ for $u(x, t)$ in $t > 0$ subject to the cauchy data $u = x$ at $t = 0$ for $0 < x < 1$. We can write the characteristics as,
  $$ \frac{dt}{1} = \frac{dx}{xu} = \frac{du}{u} $$
  or,
  $$ \di{t}{\t} = 1 \qquad \di x \t = xu \qquad \di u \t = u $$
  We can solve the first of the second form of characteristics, with the initial curve of $t = 0, x = s, u = s$ and we can say that $t = \t$. Now we aim to solve $dt = \frac{du}{u}$ which yields a solution of the form $u(s, t)= f(s, t)e^t$ which when we use our intial curve gives us a specific solution of $u(s, t) = se^t$. Now we can solve $\di x \t = xu$ by plugging in our solution for $u(s, t)$ and so we solve, $\di x t = xse^t$ which we can separate and then solve. This yields a solution of $x(s, t) = se^{s(e^t - 1)}$ for $0 < s < 1$. We can now talk about the domain of definition, which will be $0 < x < e^{e^t - 1}$.

  Now we aim to eliminate $s$, so find a solution in terms of physical variables, we can see quickly that $s = ue^{-t}$ and so after substituting it in we get that $x = ue^{u - t - ue^{-t}}$
\end{eg}

Here is another example,
\begin{eg}
  $\pd u x + \pd u y = u^3$ subject to $u = y$ on $x = 0$ for $0 < y < 3$. We write the characteristics,
  $$ \di x \t = 1 \qquad \di x \t = 1 \qquad \di u \t = u^3 $$
  We can also write the initial curve of $x = 0, y = s, u = s$. This leads to us being able to write $x(\t) = \t$ and $y(\t) = \t + s$. Now we aim to solve $\di u \t = u^3$, we aim to solve it by seperation of variables so let $u(s, \t) = A(s)B(\t)$. Now, $\di u \t = A(s)\di B \t$, as we are considering $\di u \t = u^3$ we say that $u^3 = A(s)\di B \t$ and so, $\frac{dB}{B^3} = A^2 d\t$ and integrating we reach that $B(\t) = \frac{1}{\sqrt{c - 2A^2(s)\t}}$. Hence, $u(s, \t) = \frac{A(s)}{\sqrt{c - 2A^2(s)\t}}$.
  Now consider the intial curve and we get that $A(s) = s\sqrt c$ and so,
  $$ u(s, t) = \frac{s}{\sqrt{1 - 2s^2\t}} $$
  Finally, by considering the fact that $x = \t$ and $y = \t + s$, we can say that $s = y - x$ and the the implicit solution is,
  $$ u = \frac{y - x}{\sqrt{1 - (y-x)^2x}} $$
  An interesting feature of this solution is that it blows up if $1 - (y - x)^2x = 0$, that is along the line $y = x + \frac{1}{\sqrt{2x}}$ the solution does not exist and it's a singularity. The domain of solution is $x < y < x + 3$. Hence we can plot the domain of solution,
\end{eg}

\subsection{Canonical Form for Linear First Order Equations}
Assume that the PDE is linear,
$$ a(x, y)\pd u x + b(x, y)\pd u y + c(x, y)u = d(x, y) $$
We aim to reduce this PDE to an ODE an integrate this ODE. This reduction is guided by the characteristics of this PDE. Assume that the characteristics are, $\xi = \xi(x, y)$, $\eta = \eta(x, y)$. This will be just a coordinate change, and this is a one-to-one mapping between $(x, y) \mapsto (\xi, \eta)$, that is $J = \xi_x\eta_y - \xi_y\eta_x \ne 0$. We want to substitute for these partial differentials.
$$ u_x = u_\xi\xi_x + u_\eta\eta_x \qquad u_y = u_\xi\xi_y + u_\eta\eta_y$$
Now we substitute this into our linear PDE,
$$ (a\xi_x + b\xi_y)u_\xi + (a\eta_x + b\eta_y)u_\eta + cu = d $$
Let $A = a\xi_x + b\xi_y$ and $B = a\eta_x + b\eta_y$, we get that $Au_\xi + Bu_\eta + cu = d$. If $B = 0$ we have an ODE. If $B = 0$, then $a\eta_x + b\eta_y = 0$. Our characteristics of our original equation is, $\frac{dx}{a} = \frac{dy}{b}$. The level curves of $B = 0$ are always characteristics of the original first order PDE. Hence, $\eta(x, y) = c$ and so $d\eta = 0 = \eta_x dx + \eta_ydy = 0$ or $\eta_x + \eta_y \di y x = 0$. Now we substitute into the characteristcs, $a\eta + b\eta_y = 0$.
This tells us that the general condition is just the characteristics. For the second characteristic we need to choose a one parameter family of curves such that the jacobean is zero. Hence we set $\xi = x$ (or $\xi = y$), then the jacobean is non-zero, or $\xi(x, y) = c$ and choose $\eta = y$ to satisfy $J \ne 0$. As $B = 0$ we can rewrite the PDE as,
$$ u_\xi + \frac{c}{A}u = \frac{d}{A} $$
and we call this the cannonical form of this PDE and it is an ODE, hence we can just integrate it and get the required solution.

\begin{eg}
  Reduce, $u_x - u_y = u$. The characteristics are,
  $$ \frac{dx}{1} = \frac{dy}{-1} = \frac{du}{u} $$
  as $dx = -dy$ we let $x + y = \xi$, now set $\eta = y$. Hence we can say that $J \ne 0$. $u_x = u_\xi$ and $u_y = u_\xi + u_\eta$. We substitute this into the PDE, $u_x - u_y = -u_\eta = u $ or $u_\eta + u = 0$ and hence $u(\xi, \eta) = f(\xi)e^{-\eta}$ or $u(x, y) = f(x + y)e^{-y}$
\end{eg}