%%
%% This is file, `definitions.tex',
%% generated with the extract package.
%%
%% Generated on :  2022/01/11,10:33
%% From source  :  notes.tex
%% Using options:  active,generate=testoutput,extract-env={ndefi},extract-cmd={section,subsection,subsubsection}
%%
\documentclass{article}
\def\npart {3}
\def\nterm {Autumn Term}
\def\nyear {2021}
\def\nlecturer {Professor Mohamed Sa\"{i}di}
\def\ncourse {Groups, Rings and Fields}

\makeatletter
\ifx \nauthor\undefined
  \def\nauthor{James Arthur}
\else
\fi

\author{Based on lectures by \nlecturer \\\small Notes taken by \nauthor}
\date{\nterm\ \nyear}

\usepackage[utf8x]{inputenc}
\usepackage{alltt}
\usepackage{amsfonts}
\usepackage{amsmath}
\usepackage{amssymb}
\usepackage{amsthm}
\usepackage{booktabs}
\usepackage{caption}
\usepackage{color}
\usepackage{enumitem}
\usepackage{fancyhdr}
\usepackage{fullpage}
\usepackage{graphicx}
\usepackage{mathdots}
\usepackage{mathtools}
\usepackage{microtype}
\usepackage{multirow}
\usepackage{listings}
\usepackage{pdflscape}
\usepackage{pgfplots}
\usepackage{siunitx}
\usepackage{slashed}
\usepackage{tabularx}
\usepackage{tikz}
\usepackage{tikz-cd}
\usepackage{tkz-euclide}
\usepackage[normalem]{ulem}
\usepackage[all]{xy}
\usepackage{imakeidx}

\setlength{\headheight}{20pt}
\setlength{\headsep}{10pt}

% lstLean

\definecolor{keywordcolor}{rgb}{0.7, 0.1, 0.1}   % red
\definecolor{commentcolor}{rgb}{0.4, 0.4, 0.4}   % grey
\definecolor{symbolcolor}{rgb}{0.0, 0.1, 0.6}    % blue
\definecolor{sortcolor}{rgb}{0.1, 0.5, 0.1}      % green

\def\lstlanguagefiles{lstlean.tex}
\lstset{language=lean}


\makeindex[intoc, title=Index]
\indexsetup{othercode={\lhead{\emph{Index}}}}

\ifx \nextra \undefined
  \usepackage[pdftex,
    hidelinks,
    pdfauthor={James Arthur},
    pdfsubject={Exeter Maths Notes: Year \npart\ - \ncourse},
    pdftitle={Year \npart\ - \ncourse},
  pdfkeywords={Exeter Mathematics Maths Math \npart\ \nterm\ \nyear\ \ncourse}]{hyperref}
  \title{Year \npart\ --- \ncourse}
\else
  \usepackage[pdftex,
    hidelinks,
    pdfauthor={James Arthur},
    pdfsubject={Exeter Maths Notes: Year \npart\ - \ncourse\ (\nextra)},
    pdftitle={Year \npart\ - \ncourse\ (\nextra)},
  pdfkeywords={Exeter Mathematics Maths Math \npart\ \nterm\ \nyear\ \ncourse\ \nextra}]{hyperref}

  \title{Year \npart\ --- \ncourse \\ {\Large \nextra}}
  \renewcommand\printindex{}
\fi

\pgfplotsset{compat=1.12}

\pagestyle{fancyplain}
\ifx \ncoursehead \undefined
\def\ncoursehead{\ncourse}
\fi

\lhead{\emph{\nouppercase{\leftmark}}}
\ifx \nextra \undefined
  \rhead{
    \ifnum\thepage=1
    \else
      \npart\ \ncoursehead
    \fi}
\else
  \rhead{
    \ifnum\thepage=1
    \else
      \npart\ \ncoursehead \ (\nextra)
    \fi}
\fi
\usetikzlibrary{arrows.meta}
\usetikzlibrary{decorations.markings}
\usetikzlibrary{decorations.pathmorphing}
\usetikzlibrary{positioning}
\usetikzlibrary{fadings}
\usetikzlibrary{intersections}
\usetikzlibrary{cd}
\usetikzlibrary{matrix}

\tikzset{
  mycross/.pic={
    \draw[pic actions, rotate=45]
      (-3pt,0) -- (3pt,0)
      (0,-3pt) -- (0,3pt);
  },
}


\newcommand*{\Cdot}{{\raisebox{-0.25ex}{\scalebox{1.5}{$\cdot$}}}}
\newcommand {\pd}[2][ ]{
  \ifx #1 { }
    \frac{\partial}{\partial #2}
  \else
    \frac{\partial^{#1}}{\partial #2^{#1}}
  \fi
}
\ifx \nhtml \undefined
\else
  \renewcommand\printindex{}
  \DisableLigatures[f]{family = *}
  \let\Contentsline\contentsline
  \renewcommand\contentsline[3]{\Contentsline{#1}{#2}{}}
  \renewcommand{\@dotsep}{10000}
  \newlength\currentparindent
  \setlength\currentparindent\parindent

  \newcommand\@minipagerestore{\setlength{\parindent}{\currentparindent}}
  \usepackage[active,tightpage,pdftex]{preview}
  \renewcommand{\PreviewBorder}{0.1cm}

  \newenvironment{stretchpage}%
  {\begin{preview}\begin{minipage}{\hsize}}%
    {\end{minipage}\end{preview}}
  \AtBeginDocument{\begin{stretchpage}}
  \AtEndDocument{\end{stretchpage}}

  \newcommand{\@@newpage}{\end{stretchpage}\begin{stretchpage}}

  \let\@real@section\section
  \renewcommand{\section}{\@@newpage\@real@section}
  \let\@real@subsection\subsection
  \renewcommand{\subsection}{\@ifstar{\@real@subsection*}{\@@newpage\@real@subsection}}
\fi
\ifx \ntrim \undefined
\else
  \usepackage{geometry}
  \geometry{
    papersize={379pt, 699pt},
    textwidth=345pt,
    textheight=596pt,
    left=17pt,
    top=54pt,
    right=17pt
  }
\fi

\usepackage{hyperref}
\hypersetup{
    colorlinks,
    citecolor=black,
    filecolor=black,
    linkcolor=black,
    urlcolor=black
}

\ifx \nisofficial \undefined
\let\@real@maketitle\maketitle
\renewcommand{\maketitle}{\@real@maketitle\begin{center}\begin{minipage}[c]{0.9\textwidth}\centering\footnotesize These notes are not endorsed by the lecturers, and I have modified them (often significantly) after lectures. They are nowhere near accurate representations of what was actually lectured, and in particular, all errors are almost surely mine.\tableofcontents\end{minipage}\end{center}}
\else
\fi

% Theorems
\theoremstyle{definition}
\newtheorem*{aim}{Aim}
\newtheorem*{axiom}{Axiom}
\newtheorem*{claim}{Claim}
\newtheorem*{cor}{Corollary}
\newtheorem*{conjecture}{Conjecture}
\newtheorem*{defi}{Definition}
\newtheorem*{eg}{Example}
\newtheorem*{exe}{Exercise}
\newtheorem*{fact}{Fact}
\newtheorem*{law}{Law}
\newtheorem*{lemma}{Lemma}
\newtheorem*{notation}{Notation}
\newtheorem*{prop}{Proposition}
\newtheorem*{question}{Question}
\newtheorem*{rrule}{Rule}
\newtheorem*{thm}{Theorem}
\newtheorem*{assumption}{Assumption}

\newtheorem*{remark}{Remark}
\newtheorem*{warning}{Warning}
\newtheorem*{exercise}{Exercise}

\newtheorem{nthm}{Theorem}[section]
\newtheorem{nlemma}[nthm]{Lemma}
\newtheorem{nprop}[nthm]{Proposition}
\newtheorem{ncor}[nthm]{Corollary}
\newtheorem{ndefi}[nthm]{Definition}

\renewcommand{\labelitemi}{--}
\renewcommand{\labelitemii}{$\circ$}
\renewcommand{\labelenumi}{(\roman{*})}

\let\stdsection\section
\renewcommand\section{\newpage\stdsection}

\newcommand\qedsym{\hfill\ensuremath{\square}}
% Strike through
\def\st{\bgroup \ULdepth=-.55ex \ULset}


\tikzcdset{scale cd/.style={every label/.append style={scale=#1},
    cells={nodes={scale=#1}}}}

%%%%%%%%%%%%%%%%%%%%%%%%%
%%%%% Maths Symbols %%%%%
%%%%%%%%%%%%%%%%%%%%%%%%%

% Matrix groups
\newcommand{\GL}{\mathrm{GL}}
\newcommand{\Or}{\mathrm{O}}
\newcommand{\PGL}{\mathrm{PGL}}
\newcommand{\PSL}{\mathrm{PSL}}
\newcommand{\PSO}{\mathrm{PSO}}
\newcommand{\PSU}{\mathrm{PSU}}
\newcommand{\SL}{\mathrm{SL}}
\newcommand{\SO}{\mathrm{SO}}
\newcommand{\Spin}{\mathrm{Spin}}
\newcommand{\Sp}{\mathrm{Sp}}
\newcommand{\SU}{\mathrm{SU}}
\newcommand{\U}{\mathrm{U}}
\newcommand{\Mat}{\mathrm{Mat}}

% Matrix algebras
\newcommand{\gl}{\mathfrak{gl}}
\newcommand{\ort}{\mathfrak{o}}
\newcommand{\so}{\mathfrak{so}}
\newcommand{\su}{\mathfrak{su}}
\newcommand{\uu}{\mathfrak{u}}
\renewcommand{\sl}{\mathfrak{sl}}

% Special sets
\newcommand{\C}{\mathbb{C}}
\newcommand{\CP}{\mathbb{CP}}
\newcommand{\GG}{\mathbb{G}}
\newcommand{\N}{\mathbb{N}}
\newcommand{\Q}{\mathbb{Q}}
\newcommand{\R}{\mathbb{R}}
\newcommand{\RP}{\mathbb{RP}}
\newcommand{\T}{\mathbb{T}}
\newcommand{\Z}{\mathbb{Z}}
\renewcommand{\H}{\mathbb{H}}

% Brackets
\newcommand{\abs}[1]{\left\lvert #1\right\rvert}
\newcommand{\bket}[1]{\left\lvert #1\right\rangle}
\newcommand{\brak}[1]{\left\langle #1 \right\rvert}
\newcommand{\braket}[2]{\left\langle #1\middle\vert #2 \right\rangle}
\newcommand{\bra}{\langle}
\newcommand{\ket}{\rangle}
\newcommand{\norm}[1]{\left\lVert #1\right\rVert}
\newcommand{\normalorder}[1]{\mathop{:}\nolimits\!#1\!\mathop{:}\nolimits}
\newcommand{\tv}[1]{|#1|}
\renewcommand{\vec}[1]{\boldsymbol{\mathbf{#1}}}
\newcommand{\ip}[2]{\left\langle #1\,, #2 \right\rangle}

% not-math
\newcommand{\bolds}[1]{{\bfseries #1}}
\newcommand{\cat}[1]{\mathsf{#1}}
\newcommand{\ph}{\,\cdot\,}
\newcommand{\term}[1]{\emph{#1}\index{#1}}
\newcommand{\phantomeq}{\hphantom{{}={}}}
% Probability
\DeclareMathOperator{\Bernoulli}{Bernoulli}
\DeclareMathOperator{\betaD}{beta}
\DeclareMathOperator{\bias}{bias}
\DeclareMathOperator{\binomial}{binomial}
\DeclareMathOperator{\corr}{corr}
\DeclareMathOperator{\cov}{cov}
\DeclareMathOperator{\gammaD}{gamma}
\DeclareMathOperator{\mse}{mse}
\DeclareMathOperator{\multinomial}{multinomial}
\DeclareMathOperator{\Poisson}{Poisson}
\DeclareMathOperator{\var}{var}
\newcommand{\E}{\mathbb{E}}
\newcommand{\Prob}{\mathbb{P}}

% Algebra
\DeclareMathOperator{\adj}{adj}
\DeclareMathOperator{\Ann}{Ann}
\DeclareMathOperator{\Aut}{Aut}
\DeclareMathOperator{\Char}{char}
\DeclareMathOperator{\disc}{disc}
\DeclareMathOperator{\dom}{dom}
\DeclareMathOperator{\fix}{fix}
\DeclareMathOperator{\Hom}{Hom}
\DeclareMathOperator{\id}{id}
\DeclareMathOperator{\image}{image}
\DeclareMathOperator{\im}{im}
\DeclareMathOperator{\tr}{tr}
\DeclareMathOperator{\Tr}{Tr}
\newcommand{\Bilin}{\mathrm{Bilin}}
\newcommand{\Frob}{\mathrm{Frob}}

% Others
\newcommand\ad{\mathrm{ad}}
\newcommand\Art{\mathrm{Art}}
\newcommand{\B}{\mathcal{B}}
\newcommand{\cU}{\mathcal{U}}
\newcommand{\Der}{\mathrm{Der}}
\newcommand{\D}{\mathrm{D}}
\newcommand{\dR}{\mathrm{dR}}
\newcommand{\exterior}{\mathchoice{{\textstyle\bigwedge}}{{\bigwedge}}{{\textstyle\wedge}}{{\scriptstyle\wedge}}}
\newcommand{\F}{\mathbb{F}}
\newcommand{\G}{\mathcal{G}}
\newcommand{\Gr}{\mathrm{Gr}}
\newcommand{\haut}{\mathrm{ht}}
\newcommand{\Hol}{\mathrm{Hol}}
\newcommand{\hol}{\mathfrak{hol}}
\newcommand{\Id}{\mathrm{Id}}
\newcommand{\lie}[1]{\mathfrak{#1}}
\newcommand{\op}{\mathrm{op}}
\newcommand{\Oc}{\mathcal{O}}
\newcommand{\pr}{\mathrm{pr}}
\newcommand{\Ps}{\mathcal{P}}
\newcommand{\pt}{\mathrm{pt}}
\newcommand{\qeq}{\mathrel{``{=}"}}
\newcommand{\Rs}{\mathcal{R}}
\newcommand{\Vect}{\mathrm{Vect}}
\newcommand{\wsto}{\stackrel{\mathrm{w}^*}{\to}}
\newcommand{\wt}{\mathrm{wt}}
\newcommand{\wto}{\stackrel{\mathrm{w}}{\to}}
\renewcommand{\d}{\mathrm{d}}
\renewcommand{\P}{\mathbb{P}}
%\renewcommand{\F}{\mathcal{F}}

\newcommand{\fa}{\forall\,}
\newcommand{\ex}{\exists\,}
\renewcommand{\l}{\lambda}

% LA

\newcommand{\fs}{\sum_{i=1}^n}
\newcommand{\vV}{\vec v \in V}
\renewcommand{\v}{\vec v}

\let\Im\relax
\let\Re\relax

\DeclareMathOperator{\area}{area}
\DeclareMathOperator{\card}{card}
\DeclareMathOperator{\ccl}{ccl}
\DeclareMathOperator{\ch}{ch}
\DeclareMathOperator{\cl}{cl}
\DeclareMathOperator{\cls}{\overline{\mathrm{span}}}
\DeclareMathOperator{\coker}{coker}
\let\ker\relax
\DeclareMathOperator{\ker}{Ker}
\DeclareMathOperator{\conv}{conv}
\DeclareMathOperator{\cosec}{cosec}
\DeclareMathOperator{\cosech}{cosech}
\DeclareMathOperator{\covol}{covol}
\DeclareMathOperator{\diag}{diag}
\DeclareMathOperator{\diam}{diam}
\DeclareMathOperator{\Diff}{Diff}
\DeclareMathOperator{\End}{End}
\DeclareMathOperator{\energy}{energy}
\DeclareMathOperator{\erfc}{erfc}
\DeclareMathOperator{\erf}{erf}
\DeclareMathOperator*{\esssup}{ess\,sup}
\DeclareMathOperator{\ev}{ev}
\DeclareMathOperator{\Ext}{Ext}
\DeclareMathOperator{\fst}{fst}
\DeclareMathOperator{\Fit}{Fit}
\DeclareMathOperator{\Frac}{Frac}
\DeclareMathOperator{\Gal}{Gal}
\DeclareMathOperator{\gr}{gr}
\DeclareMathOperator{\hcf}{hcf}
\DeclareMathOperator{\Im}{Im}
\DeclareMathOperator{\Ind}{Ind}
\DeclareMathOperator{\Int}{Int}
\DeclareMathOperator{\Isom}{Isom}
\DeclareMathOperator{\lcm}{lcm}
\DeclareMathOperator{\length}{length}
\DeclareMathOperator{\Lie}{Lie}
\DeclareMathOperator{\like}{like}
\DeclareMathOperator{\Lk}{Lk}
\DeclareMathOperator{\Maps}{Maps}
\DeclareMathOperator{\orb}{orb}
\DeclareMathOperator{\ord}{ord}
\DeclareMathOperator{\otp}{otp}
\DeclareMathOperator{\poly}{poly}
\DeclareMathOperator{\rank}{rank}
\DeclareMathOperator{\rel}{rel}
\DeclareMathOperator{\Rad}{Rad}
\DeclareMathOperator{\Re}{Re}
\DeclareMathOperator*{\res}{res}
\DeclareMathOperator{\Res}{Res}
\DeclareMathOperator{\Ric}{Ric}
\DeclareMathOperator{\rk}{rk}
\DeclareMathOperator{\Rees}{Rees}
\DeclareMathOperator{\Root}{Root}
\DeclareMathOperator{\sech}{sech}
\DeclareMathOperator{\sgn}{sgn}
\DeclareMathOperator{\snd}{snd}
\DeclareMathOperator{\Spec}{Spec}
\DeclareMathOperator{\spn}{span}
\DeclareMathOperator{\stab}{stab}
\DeclareMathOperator{\St}{St}
\DeclareMathOperator{\supp}{supp}
\DeclareMathOperator{\Syl}{Syl}
\DeclareMathOperator{\Sym}{Sym}
\DeclareMathOperator{\vol}{vol}
\DeclareMathOperator{\range}{Range}
\DeclareMathOperator{\Null}{null}

\pgfarrowsdeclarecombine{twolatex'}{twolatex'}{latex'}{latex'}{latex'}{latex'}
\tikzset{->/.style = {decoration={markings,
                                  mark=at position 1 with {\arrow[scale=2]{latex'}}},
                      postaction={decorate}}}
\tikzset{<-/.style = {decoration={markings,
                                  mark=at position 0 with {\arrowreversed[scale=2]{latex'}}},
                      postaction={decorate}}}
\tikzset{<->/.style = {decoration={markings,
                                   mark=at position 0 with {\arrowreversed[scale=2]{latex'}},
                                   mark=at position 1 with {\arrow[scale=2]{latex'}}},
                       postaction={decorate}}}
\tikzset{->-/.style = {decoration={markings,
                                   mark=at position #1 with {\arrow[scale=2]{latex'}}},
                       postaction={decorate}}}
\tikzset{-<-/.style = {decoration={markings,
                                   mark=at position #1 with {\arrowreversed[scale=2]{latex'}}},
                       postaction={decorate}}}
\tikzset{->>/.style = {decoration={markings,
                                  mark=at position 1 with {\arrow[scale=2]{latex'}}},
                      postaction={decorate}}}
\tikzset{<<-/.style = {decoration={markings,
                                  mark=at position 0 with {\arrowreversed[scale=2]{twolatex'}}},
                      postaction={decorate}}}
\tikzset{<<->>/.style = {decoration={markings,
                                   mark=at position 0 with {\arrowreversed[scale=2]{twolatex'}},
                                   mark=at position 1 with {\arrow[scale=2]{twolatex'}}},
                       postaction={decorate}}}
\tikzset{->>-/.style = {decoration={markings,
                                   mark=at position #1 with {\arrow[scale=2]{twolatex'}}},
                       postaction={decorate}}}
\tikzset{-<<-/.style = {decoration={markings,
                                   mark=at position #1 with {\arrowreversed[scale=2]{twolatex'}}},
                       postaction={decorate}}}

\tikzset{circ/.style = {fill, circle, inner sep = 0, minimum size = 3}}
\tikzset{scirc/.style = {fill, circle, inner sep = 0, minimum size = 1.5}}
\tikzset{mstate/.style={circle, draw, blue, text=black, minimum width=0.7cm}}

\tikzset{eqpic/.style={baseline={([yshift=-.5ex]current bounding box.center)}}}
\tikzset{commutative diagrams/.cd,cdmap/.style={/tikz/column 1/.append style={anchor=base east},/tikz/column 2/.append style={anchor=base west},row sep=tiny}}

\definecolor{mblue}{rgb}{0.2, 0.3, 0.8}
\definecolor{morange}{rgb}{1, 0.5, 0}
\definecolor{mgreen}{rgb}{0.1, 0.4, 0.2}
\definecolor{mred}{rgb}{0.5, 0, 0}

\def\drawcirculararc(#1,#2)(#3,#4)(#5,#6){%
    \pgfmathsetmacro\cA{(#1*#1+#2*#2-#3*#3-#4*#4)/2}%
    \pgfmathsetmacro\cB{(#1*#1+#2*#2-#5*#5-#6*#6)/2}%
    \pgfmathsetmacro\cy{(\cB*(#1-#3)-\cA*(#1-#5))/%
                        ((#2-#6)*(#1-#3)-(#2-#4)*(#1-#5))}%
    \pgfmathsetmacro\cx{(\cA-\cy*(#2-#4))/(#1-#3)}%
    \pgfmathsetmacro\cr{sqrt((#1-\cx)*(#1-\cx)+(#2-\cy)*(#2-\cy))}%
    \pgfmathsetmacro\cA{atan2(#2-\cy,#1-\cx)}%
    \pgfmathsetmacro\cB{atan2(#6-\cy,#5-\cx)}%
    \pgfmathparse{\cB<\cA}%
    \ifnum\pgfmathresult=1
        \pgfmathsetmacro\cB{\cB+360}%
    \fi
    \draw (#1,#2) arc (\cA:\cB:\cr);%
}
\newcommand\getCoord[3]{\newdimen{#1}\newdimen{#2}\pgfextractx{#1}{\pgfpointanchor{#3}{center}}\pgfextracty{#2}{\pgfpointanchor{#3}{center}}}

\newcommand\qedshift{\vspace{-17pt}}
\newcommand\fakeqed{\pushQED{\qed}\qedhere}

\def\Xint#1{\mathchoice
   {\XXint\displaystyle\textstyle{#1}}%
   {\XXint\textstyle\scriptstyle{#1}}%
   {\XXint\scriptstyle\scriptscriptstyle{#1}}%
   {\XXint\scriptscriptstyle\scriptscriptstyle{#1}}%
   \!\int}
\def\XXint#1#2#3{{\setbox0=\hbox{$#1{#2#3}{\int}$}
     \vcenter{\hbox{$#2#3$}}\kern-.5\wd0}}
\def\ddashint{\Xint=}
\def\dashint{\Xint-}

\newcommand\separator{{\centering\rule{2cm}{0.2pt}\vspace{2pt}\par}}

\newenvironment{own}{\color{gray!70!black}}{}

\newcommand\makecenter[1]{\raisebox{-0.5\height}{#1}}

\mathchardef\mdash="2D

\newenvironment{significant}{\begin{center}\begin{minipage}{0.9\textwidth}\centering\em}{\end{minipage}\end{center}}
\DeclareRobustCommand{\rvdots}{%
  \vbox{
    \baselineskip4\p@\lineskiplimit\z@
    \kern-\p@
    \hbox{.}\hbox{.}\hbox{.}
  }}
\DeclareRobustCommand\tph[3]{{\texorpdfstring{#1}{#2}}}
\makeatother


\title{Groups, Rings and Fields Definitions}

\begin{document}
\maketitle


\section[Basics of Groups]{\Sectionformat {Basics of Groups}{1}}

\begin{ndefi}[Group]
  $G$ is a nonempty set and endowed with a composition rule $(\cdot)$. We denote this $(G, \cdot)$. $(\cdot)$ is well defined, so we can associate another element $a \cdot b \in G$ and $a \cdot b$ is unique. $(\cdot)$ must be associative,
  $$ a \cdot (b \cdot c) = (a \cdot b) \cdot c $$
  The brackets are irrelevant when combining more than two elements. We also have \textbf{natural element}, so,
  $$ c \cdot e_G = c = e_G \cdot c $$
  There are also inverses, so,
  $$ a \cdot a^{-1} = e_G = a^{-1} \cdot a $$
  So the inverse naturalises the element.
\end{ndefi}

\begin{ndefi}[k-cycle]
  A $k$ cycle, $\sigma = (a_1, a_2, \dots, a_k) \in S_n$  is a permutation,
  $$ \begin{pmatrix}
    a_1 & a_2 & \dots & a_{k-1} & a_k \\
    a_2 & a_3 & \dots & a_k & a_1
  \end{pmatrix} $$
\end{ndefi}

\begin{ndefi}[Dihedral Group]
  Let us take the $n$-gon ($n \ge 3$) and depending on when $n$ is odd or even we have a vertex along with the vertex one, you get them lying on the y-axis. Then you get all the rotations symmetries in the plane, which maps the $n$-gon to itself. There are $2n$ of them, the rotation clockwise with angle $\frac{2\pi}{n}$, there are $n$ of these. Then we have the elements where we flip the shape, $s$, first where $s^2 = 1$.
  $$ D_{2n} = \{1, r, r^2, \dots, r^{n-1}, s, sr, sr^2, \dots, sr^{n-1} \} $$
  Then\marginnote{\emph{Lecture 2}}[0mm] this is our $2n$ elements. This is indeed a group with composition of rotations and $n \ge 3$ then the group isn't abelian. We also have the interesting rule which spits out the non-commutative behavior,
  $$ sr^i = r^{-i}s = r^{n-i}s $$
\end{ndefi}

\subsection[Subgroups and Orders]{\Sectionformat {Subgroups and Orders}{2}}

\begin{ndefi}[Subgroup]
  A subgroup, $H \subset G$, of a group $(G, \cdot)$,
  \begin{itemize}
    \item $\forall x, y \in H, x \cdot y \in H$
    \item $\forall x \in H, x^{-1} \in H$
  \end{itemize}
\end{ndefi}

\begin{ndefi}[Order of an element]
  Let $G$ be a group and $a \in G$. The order of $a$ is,
  $$ \ord (a) = \min \{n \ge 1 : a^n = e_G\} $$
\end{ndefi}

\begin{ndefi}[Generator]
  If $G$ is a group, $a\in G$, the subset $H = \{a^n : n \in \Z\}$ of $G$ consisting of all powers of the element $a$ is a subgroup, and is called the cyclic subgroup of $G$ generated by $a$, and $a$ is called a generator of $H$. The subgroup is denoted by $\gen a$.
\end{ndefi}

\begin{ndefi}[Cyclic Group]
  A group $G$ is called cyclic if $\ex a \in G$ such that $G = \gen a$ equals the (sub)group generated by $a$.
\end{ndefi}

\begin{ndefi}[Product of Groups]
  Let $(G, \circ)$ and $(H, *)$ be two groups. We define a new group $(G \times H, \cdot)$ called the product group of $G$ and $H$, as follows,
  $$ G \times H = \{(g, h) : g \in G, h \in H\} $$
  is the set-theoretic product of $G$ and $H$. The composition law $(\cdot)$ is defined by,
  $$ (g_1, h_1) \cdot (g_2, h_2) = (g_1 \circ g_2, h_1 * h_2) $$
  The from this, the rest of the group axioms follow trivially.
\end{ndefi}

\subsection[Homomophism]{\Sectionformat {Homomophism}{2}}

\begin{ndefi}[Homomophism]
  Let there be a group $(G, \circ)$ and $(H, *)$ and define a homomophism from $G \to H$ which satisfy,
  \begin{enumerate}
    \item For $g_1, g_2 \in G$, $f(g_1 \circ g_1) = f(g_1) * f(g_2)$
    \item $f(e_G) = e_H$
  \end{enumerate}
\end{ndefi}

\begin{ndefi}[Image]
  Let $f : G \to H$ be a homomorphism, we define the image as,
  $$ \Im f = \{ h \in H \m \ex g \in G, h = f(g)\} $$
\end{ndefi}

\begin{ndefi}[Kernel]
  Let $f : G \to H$ be a homomorphism, we define the kernel as,
  $$ \ker f = \{ g \in G \m f(g) = e_G \} $$
\end{ndefi}

\section[Cosets and Normal Subgroups]{\Sectionformat {Cosets and Normal Subgroups}{1}}

\begin{ndefi}[Relation]
  $x \sim y \implies x^{-1}y = h\in H$
\end{ndefi}

\begin{ndefi}[Left Coset]
  We define the left coset as this equivalence relation.
\end{ndefi}

\subsection[Normal Subgroups]{\Sectionformat {Normal Subgroups}{2}}

\begin{ndefi}[Normal Subgroup]
  A subgroup $H$ of $G$ is called normal if,
  $$ xH = Hx = \{h'x : h' \in H\} \qquad \fa x \in G $$
\end{ndefi}

\begin{ndefi}[Conjugate]\marginnote{\emph{Lecture 6}}[0mm]
  Two elements $g,h \in G$ if we can find a $x \in G$ such that,
  $$ g = x h x^{-1} $$
  and we call it the conjugate of $g$ by $x$.
\end{ndefi}

\begin{ndefi}[Signature]
  If we consider $\e : S_n \to \{\overline{0}, \overline{1}\}$ and consider a new map, $\s \mapsto \e(\s)$ where we define,
  $$ \e(\s) = \begin{cases}
    0 & \text{if $\s$ is even}\\
    1 & \text{if $\s$ is odd}
  \end{cases} $$
\end{ndefi}

\subsection[Quotient Groups]{\Sectionformat {Quotient Groups}{2}}

\begin{ndefi}[Quotient Group Law]
  We define a composition law $(\cdot)$ on the set of left cosets $G/H$ by,
  \begin{align*}
    (\cdot) : G/H \times G/H \to G/H\\
    (xH, yH) \mapsto xH \cdot yH = xyH
  \end{align*}
\end{ndefi}

\subsubsection[First Isomorphism Theorem]{\Sectionformat {First Isomorphism Theorem}{3}}

\section[Group Actions]{\Sectionformat {Group Actions}{1}}

\begin{ndefi}[Group Action]
  Let $(G, *)$ be a group and a set $A$. A group action is a map,
  $$ (\cdot) : G \times A \to A $$
  $$ (g, a) \mapsto g \cdot a $$
  satisying,
  \begin{align}
    (g_1 * g_2) \cdot a &= g_1 \cdot (g_2 \cdot a) \quad \fa g_1, g_2 \in G, \quad a \in A\\
    e_G \cdot a &= a \quad \fa a\in A
  \end{align}
\end{ndefi}

\begin{ndefi}[Action by left multiplication]
  Consider $(\cdot) : G \times G \to G$ and define $(h, g) \mapsto h \cdot g = h * g$. Axiom $(1)$ is satisfied,
  $$ (h_1 * h_2) \cdot g = (h_1 * h_2) * g = h_1 * (h_2 * g) = h_1 . (h_2.g) $$
  and axiom $(2)$ is also satisfied.
\end{ndefi}

\begin{ndefi}[Action by conjugation]
  A group $(G, *)$ acts on itself defined by $(h, g) \mapsto (h \cdot g) = h * g * h^{-1}$. Now check the axioms,
  \begin{align*}
    (h_1 * h_2) \cdot g &= (h_1 * h_2) * g * (h_1 * h_2)^{-1}\\
    &= (h_1 * h_2) * g * (h_2^{-1} * h_1^{-1})\\
    &= h_1 * (h_2 * g * h_2^{-1}) * h_1^{-1}\\
    &= h_1 \cdot (h_2 \cdot g)\\
  \end{align*}
  The second axiom is also satisfied.
\end{ndefi}

\begin{ndefi}[Permutation Representation]
  Let $(S_A, \circ)$ be the group of all bijections from $A \to A$; $S_A$ is the group of symmetries of $A$, the group law is just composition of bijections. The map,
  $$ \tau : G \to S_A $$
  is defined by,
  $$ \t(g) = \t_g $$
  is a group homomorphism,
  \begin{align*}
    \t (g_1 * g_2) (a) &= (g_1 * g_2)\cdot a \\
    &= g_1 \cdot (g_2 \cdot a)\\
    &= \t_{g_1} (\t_{g_2}(a)) \\
    &= (\t(g_1) \circ \t(g_2))(a)
  \end{align*}
  and we call $\tau$ the permutation representation associated to the action $(\cdot)$.
\end{ndefi}

\begin{ndefi}[Kernel of representation]
  The kernel of $\t : G \to S_A$
  $$ \ker \t = \{g\in G : \t_g = \id_A\} = \{g \in G : g \cdot a = a\} $$
  is just the kernel of the representation $\t$. If we find $\ker \t = \{e_G\}$, or $\t$ is injective, we say $(\cdot)$ is faithful.
\end{ndefi}

\subsection[Stabilisers and Orbits]{\Sectionformat {Stabilisers and Orbits}{2}}

\begin{ndefi}[Stabiliser]
  We define the following set called the stabiliser
  $$ \stab (a) = \{g \in G : g \cdot a = a\} $$
\end{ndefi}

\begin{ndefi}[Orbit]
  Let $a \in A$. The equivalence class of $a$ for the relation $\sim$ is,
  $$ \overline a = \{b \in A : \ex g \in G, b = g \cdot a\} = \{g \cdot a : g \in G\} $$
  is called the orbit of a, for the given action, ad is denoted $\orb (a)$.
\end{ndefi}

\begin{ndefi}[Regular permutation representation]
  If $g \in G$, $\rho(g)$ is the permutation defined for $i,j \in \{1, \dots, n\}$ by,
  $$ \rho(g)(i) = j \qquad \text{if $g * g_i = g_j$} $$
\end{ndefi}

\section[Class Equation]{\Sectionformat {Class Equation}{1}}

\subsection[Normalisers, Centralisers and Centers]{\Sectionformat {Normalisers, Centralisers and Centers}{2}}

\begin{ndefi}[Normaliser]
  The stabiliser of the above group action,
  $$ N_G(A) = \{g \in G : gAg^{-1} = A\} $$
\end{ndefi}

\begin{ndefi}[Centraliser]
  We say that the kernel of the $\phi_A$ is the centraliser,
  $$C_G(A) = \ker \phi_A = \bigcap_{a \in A}\stab(a) = \{g \in N_A(a) L gag^{-1} = a,\,\fa a \in A\} $$
  and these are just all the commuting elements.
\end{ndefi}

\begin{ndefi}[Center of $G$]
  The center is a normal abelian subgroup of $G$ such that,
  $$ Z(G) = \{g \in G : gh = hg,\, \fa h \in G\} $$
\end{ndefi}

\subsection[The Class Equation]{\Sectionformat {The Class Equation}{2}}

\subsubsection[Conjuagacy Classes of $S_n$]{\Sectionformat {Conjuagacy Classes of $S_n$}{3}}

\subsection[Simple Groups]{\Sectionformat {Simple Groups}{2}}

\begin{ndefi}[Simple Groups]
  $G$ is simple if the only normal subgroups of $G$ are $H = G$ and $H = \{e_G\}$.
\end{ndefi}

\section[Sylow's Theorems]{\Sectionformat {Sylow's Theorems}{1}}

\begin{ndefi}[$p$-group]
  Let $p$ be a prime number. A group of cardinality $p^t$ for some $t \ge 1$ is called a $p$-group. A non-trivial subgroup of a $p$-group is a $p$-group.
\end{ndefi}

\begin{ndefi}[Sylow $p$-group]
  If we consider a group $G$, such that $|G| = m \cdot p^r$, then the subgroups $H_i$, cardinality $|H_i| = p^r$ is called the Sylow $p$-groups.
\end{ndefi}

\begin{ndefi}[Fixed Point]
  Consider a group $H$ acting on a set $X$ and take a $x \in X$. Then if $h \cdot x = x$ for all $h \in H$, then we say that $x$ is a fixed point of the action.
\end{ndefi}

\subsection[Proof of Sylow I]{\Sectionformat {Proof of Sylow I}{2}}

\subsection[Proof of Sylow II]{\Sectionformat {Proof of Sylow II}{2}}

\subsection[Proof of Sylow III]{\Sectionformat {Proof of Sylow III}{2}}

\subsection[Classifying groups through Sylow]{\Sectionformat {Classifying groups through Sylow}{2}}

\section[Polynomials]{\Sectionformat {Polynomials}{1}}

\begin{ndefi}[Polynomial]
  A polynomial with coefficients in $\Q$ is an infinite sequence
  $$ (a_0,\,a_1,\,\dots,\,a_n,\,\dots) $$
  such that $\ex N \ge 0$ with $a_i =0\,\fa i \ge N$
\end{ndefi}

\begin{ndefi}[Division]
  Let $f, g \in \QX$. We say that $g$ divides $f$ if $\ex h \in \QX$ such that\footnote{I hold that this is NOT a definition, it is a lemma as this can be proved.}
  $$ f(X) = g(X)h(X) $$
\end{ndefi}

\begin{ndefi}[Irriducible]
  Let $f \in\QX$ be a non constant polynomial, $f \in \QX\sm\Q$ and $\deg(f) \ge 1$. We say that $f$ is irreducible if whenever $f(X) = g(X)h(X)$ then either $g$ or $h$ is a unit.
\end{ndefi}

\section[Rings and Fields]{\Sectionformat {Rings and Fields}{1}}

\begin{ndefi}[Commutative Ring]
  If we have a set $(R,\,+,\,\ti)$, then the following is true,
  \begin{enumerate}
    \item $(R,\,+)$ is an abelian group.
    \item $\ti$ must be commutative and associative
    \item Addition and multiplication are distributive, ie.
    $$ a \ti (b + c) = a \ti b + a \ti c $$
  \end{enumerate}
\end{ndefi}

\begin{ndefi}[Zero Divisor]
  Let $R$ be a ring. An element $a \in R \sm \{0_R\}$ is called a zero divisor if $\ex b \in R \sm \{0\}$ such that
  $$ ab = 0_R $$
\end{ndefi}

\begin{ndefi}[Unit]
  Assume $R$ has an identity $1$. An element $u \in R$ is called a unit if $\ex v \in R$ such that $uv = 1$. We denote $v$ by $u^{-1}$ and call it the inverse of $u$.
\end{ndefi}

\begin{ndefi}[Group of Units]
  Let $R$ be a ring with identity $1$. The set of units of $R$ is denoted
  $$ R^\ti = \{u \in R : u \text{ is a unit}\} $$
\end{ndefi}

\begin{ndefi}[Integral Domain]
  A ring is called an integral domain if it has no zero divisors
\end{ndefi}

\begin{ndefi}[Field]
  A ring $F$ with identity is called a field if $F^\ti = F \sm \{0\}$, or $F$ is a field if every non zero element of $F$ is a unit.
\end{ndefi}

\begin{ndefi}[Finite field with $p$ elements]
  Let $p$ be a prime integer. The field $\Z/p\Z$ is denoted $\F_p$ and called a finite field with $p$ elements.
\end{ndefi}

\begin{ndefi}[Subring]
  A subset $S$ of a ring $R$ is called a subring if $(S,\,+)$ is a subgroup of $(R,\,+)$ and $S$ is closed under multiplication.
\end{ndefi}

\section[Ring Homorphisms and Ideals]{\Sectionformat {Ring Homorphisms and Ideals}{1}}

\begin{ndefi}[Ring Homomorphism]
  Let $R$ and $S$ be rings. A map $\phi : R \to S$ is called a ring homomorphism if it satisfies,
  \begin{enumerate}
    \item $\phi(a + b) = \phi(a) + \phi(b)\quad\fa a b \in R$
    \item $\phi(ab) = \phi(a)\phi(b)\quad\fa a b \in R$
    \item $\phi(1_R) = 1_S$
  \end{enumerate}
  In addition, $\phi(0_R) = 0_S$ and $\phi(-a) = -\phi(a)$ for all $a \in A$.
\end{ndefi}

\begin{ndefi}[Kernel]
  Let $\phi : R \to S$ be a ring homomorphism. We define,
  $$ \ker \phi = \{r \in R : \phi(r) = 0_S\} $$
  and call this set the kernel.
\end{ndefi}

\begin{ndefi}[Image]
  Let $\phi : R \to S$ be a ring homomorphism. We define,
  $$ \Im \phi = \{s \in S : \ex r \in R, \phi(r) = s\} $$
  and call this set the image.
\end{ndefi}

\begin{ndefi}[Ideal]
  Let $R$ be a ring. A subset $I \subset R$ is called an ideal if the following hold,
  \begin{enumerate}
    \item $(I,\,+)$ is a subgroup of $(R,\,+)$
    \item $\fa a \in I, b \in R$, it holds that $ab \in I$. Or $I$ is closed under multiplication by arbitrary elements in $R$.
  \end{enumerate}
\end{ndefi}

\subsection[Construction of the quotient ring]{\Sectionformat {Construction of the quotient ring}{2}}

\begin{ndefi}[Addition and Multiplication of ideals]
  Suppose we have two ideals, $I$ and $J$ and we define addition,
  $$ I + J = \{ a + b : a \in I, b \in J\} $$
  and the product,
  $$ IJ = \left\{\sum_{i=1}^m a_ib_j : a \in I,\,b\in J\, m \ge 1 \right\} $$
\end{ndefi}

\begin{ndefi}[Principal Ideals]
  Let $R$ be a ring,
  \begin{enumerate}
    \item An ideal $I \in R$ is called principal if $I = (a)_R$ is generated by one element $a \in R$, called a generator of $R$.
    \item Let $a, b \in R$ and define $(a,\,b)_R = \{ac + bd : c, d \in R\}$. Then $(a,\,b)_R$ is an ideal called the ideal generated by $a$ and $b$.
  \end{enumerate}
\end{ndefi}

\section[Prime and Maximal Ideals]{\Sectionformat {Prime and Maximal Ideals}{1}}

\subsection[Maximal Ideals]{\Sectionformat {Maximal Ideals}{2}}

\begin{ndefi}[Proper Ideal]
  Let $R$ be a ring, then an ideal of $R$ is proper if $I \subsetneq R$.
\end{ndefi}

\begin{ndefi}[Maximal Ideal]
  A proper ideal $M \subsetneq R$ is called a maximal ideal if the only ideals of $R$ containing $M$ are $M$ and $R$.
\end{ndefi}

\subsection[Prime Ideals]{\Sectionformat {Prime Ideals}{2}}

\begin{ndefi}[Prime Ideals]
  A proper ideal $P \subsetneq R$ is a prime ideal if for some $a, b \in R$ and $ab \in I$, then $a \in I$ or $b \in I$
\end{ndefi}

\subsection[Field of Fractions]{\Sectionformat {Field of Fractions}{2}}

\subsection[Chinese Remainder Theorem]{\Sectionformat {Chinese Remainder Theorem}{2}}

\begin{ndefi}[Comaximal]
  We call two ideal comaximal, if $I + J = R$, if two ideals are comaximal, then $I + J = IJ$.
\end{ndefi}

\section[Divisibility and Factorisation]{\Sectionformat {Divisibility and Factorisation}{1}}

\begin{ndefi}[Norm]
  A norm is map from an integral domain to $\N$,
  $$ N : R\setminus \{0_R\} \to \N $$
  We call a norm multiplicative if $N(ab) = N(a)N(b)$ for all $a, b \in R\setminus \{0\}$.
\end{ndefi}

\begin{ndefi}[Euclidean Domain]
  We say that $R$ is a euclidean domain if $N$ is a norm and we have $a \in R$ and $0_R \ne b \in R$ then we can write $a = bq + r$ where $q \in R$ and $r = 0_R$ or $N(r) < N(b)$.
\end{ndefi}

\begin{ndefi}[Principal Ideal Domain]
  Let $R$ be an integral domain, then $R$ is a PID if it has the property if every ideal of $R$ is principal, ie.
  $$ I = (a)_R = \{ab : b \in R\} $$
\end{ndefi}

\begin{ndefi}[Divisble]
  Let $R$ be a ring, then let $a, b \in R$. Then we say $b$ divides $a$ means that $a = bc$ for some $c \in R$. We write $b/a$.
\end{ndefi}

\begin{ndefi}[Greatest Common Divisor]
  A greatest commmon divisor of $a$ and $b$, say $d$ is described as,
  \begin{enumerate}
    \item $d/a$ and $d/b$
    \item If $d'/a$ and $d'/b$, then $d'/d$
  \end{enumerate}
\end{ndefi}

\begin{ndefi}[Irreducible, Prime, Associate]
  Let $R$ be an integral domain,
  \begin{enumerate}
    \item An element $r \in R \setminus \{0, R^\ti\}$ is called irreducible if $ab \in R$ then $a \in R^\ti$ or $b \in R^\ti$.
    \item An element $p \in R\setminus\{0, R^\ti\}$ is called prime if $p/ab$ then $p/a$ or $p/b$
    \item Two elements $a, b \in R$ are associate we write $a \sim b$ if $a = bu$ where $u \in R^\ti$
  \end{enumerate}
\end{ndefi}

\begin{ndefi}[Unique Factorisation Domain]
  A UFD is an integral domain $R$ in which every element $r \in R\setminus \{0, R^\ti\}$ has the following properties,
  \begin{enumerate}
    \item $r = p_1p_2\dots p_n$ is a product of irreducible elements $p_1, p_2, \dots, p_n$.
    \item The above factorisation is unique up to associates if $r = q_1q_2 \dots q_m$ is another factorisation of $r$ as a product of irreducible elements $q_1, q_2, \dots, q_m$ then $n = m$ and after possible renumbering the factors $p_i \sim q_i$ for all $i$.
  \end{enumerate}
\end{ndefi}

\end{document}
