% !TEX root = ../notes.tex

\section{Basics of Groups}
We\marginnote{\emph{Lecture 1}}[0mm] start by defining a group, it is an example of an algebraic structure.
\begin{ndefi}[Group]
  $G$ is a nonempty set and endowed with a composition rule $(\cdot)$. We denote this $(G, \cdot)$. $(\cdot)$ is well defined, so we can associate another element $a \cdot b \in G$ and $a \cdot b$ is unique. $(\cdot)$ must be associative,
  $$ a \cdot (b \cdot c) = (a \cdot b) \cdot c $$
  The brackets are irrelevant when combining more than two elements. We also have \textbf{natural element}, so,
  $$ c \cdot e_G = c = e_G \cdot c $$
  There are also inverses, so,
  $$ a \cdot a^{-1} = e_G = a^{-1} \cdot a $$
  So the inverse naturalises the element.
\end{ndefi}

\noindent
If we just have a group usually $a \cdot b \ne b \cdot a$, if $a \cdot b = b \cdot a$ are called abelian or commutative groups. This is in reference to the mathematician Abel.

\noindent
If $G$ is finite as a set, then we can say that $G$ is a finite group and we denote the size or cardinality of $G$ as $|G|$, sometimes this is said to be the order. The cardinality can be infinite.\\

\noindent
\begin{eg}
  We know a very important group, the group of integers $\Z$. This set is infinite as $n \ne n + 1$ and the composition law is $+$ and we know that it's associative and natural element of $0$ and each element $n$ has an inverse of $-n$. We can also say,
  $$ k_1 + k_2 = k_2 + k_1 $$
  and so we have an infinite abelian group.
\end{eg}

\begin{eg}
  We can also consider groups of integers module $n$, denoted,
  $$ \Z_n = \{[0]_n, [1]_n, \dots, [n-1]_n\} $$
  where we have modulo classes (see Number Theory notes week 2). We can say, if $[k]_n = [l]_n$ if and only if $n \m k - l$. Also if you have $[k_1]_n$ and $[k_2]_n$, then $[k_1]_n + [k_2]_n = [k_1 + k_2]_n$. We have to check if this addition is well defined and it is, as you can just multiply by a constant as $[k+ rn]_n = [k]_n$. This is also a group with natural element of $[0]_n$ the inverse of $[k]_n$ is just $[-k]_n$ as $[k]_n + [-k]_n = [0]_n$. This is a finite abelian group and $|\Z_n| = n$.
\end{eg}

There is two worlds, non-commutative and commutative. Nature is not commutative, things aren't that nice. Our best example of the non-commutative group is the group of permutations. Let $n \in \Z^+$ and then let there be a set $S_n = \{1, 2, \dots, n\}$ and consider all possible bijections $\sigma$ from that set to itself. As these are finite sets and of the same cardinality, it suffices to check it's injective.
$$ \begin{pmatrix}
  1 & 2 & \dots & n-1 & n\\
  \sigma(1) & \sigma(2) & \dots & \sigma(n-1) & \sigma(n)
\end{pmatrix} $$
saying this is a bijection says the bottom row, given they are integers from 1 to $n$, appear only once, they don't appear twice.

\begin{eg}
  Let us take $S_4$, then we can take an element,
  $$ \sigma = \begin{pmatrix}
    4 & 3 & 2 & 1
  \end{pmatrix} $$
  and we can call this $\sigma$ and is an element of the group.
\end{eg}

New question, what is $|S_n|$, how many $\sigma$ are there? It's $n!$.

\begin{proof}
  Define $\sigma$ and you have to consider $\sigma(1)$ and theres $n$ possibilities, then for $\sigma(2)$ theres $n-1$ possibilities, then we can't use $\sigma(1)$ or $\sigma(2)$ and hence theres $n - 2$ possibilities for $\sigma(3)$ and so on. So we have,
  $$ n (n - 1) \cdot (n - 2) \cdot (n - 3) \dots 2 \cdot 1 = n! $$
\end{proof}

We can form a group where the composition is just $\circ$ on our set of bijections $\sigma$. If we take a $\sigma \circ \tau$ then this is also a bijection into $S_n$. This is associative and we get a natural element of $\id_{S_n}$. Then every bijection has an inverse $\sigma^{-1}$, which is unique. What is $\sigma^{-1}$, just reverse the order of the rows,
$$ \sigma^{-1} = \begin{pmatrix}
  1 & 2 & 3 & 4
\end{pmatrix} $$

This group is non-commutative if $n \ge 3$ then $S_n$ is not commutative. If we an integer $1 \le k \le n$ and take $k$ elements $\{a_1, a_2, \dots, a_k\} \subset \{1, 2, 3, \dots, n\}$. Then we define
\begin{ndefi}[k-cycle]
  A $k$ cycle, $\sigma = (a_1, a_2, \dots, a_k) \in S_n$  is a permutation,
  $$ \begin{pmatrix}
    a_1 & a_2 & \dots & a_{k-1} & a_k \\
    a_2 & a_3 & \dots & a_k & a_1
  \end{pmatrix} $$
\end{ndefi}

A k-cycle is a permutation and a bijection as you only write each number from $1$ to $n$ once. The $1$-cycle is just the identity. The $2$-cycle is the transposition. Then onwards it just shifts elements around. We can count the number of $k$-cycles, which is,
$$ \frac{n(n-1) \dots (n + k -1)}{k} $$

We can now see the dihedral group $D_{2n}$,
\begin{ndefi}[Dihedral Group]
  Let us take the $n$-gon ($n \ge 3$) and depending on when $n$ is odd or even we have a vertex along with the vertex one, you get them lying on the y-axis. Then you get all the rotations symmetries in the plane, which maps the $n$-gon to itself. There are $2n$ of them, the rotation clockwise with angle $\frac{2\pi}{n}$, there are $n$ of these. Then we have the elements where we flip the shape, $s$, first where $s^2 = 1$.
  $$ D_{2n} = \{1, r, r^2, \dots, r^{n-1}, s, sr, sr^2, \dots, sr^{n-1} \} $$
  Then\marginnote{\emph{Lecture 2}}[0mm] this is our $2n$ elements. This is indeed a group with composition of rotations and $n \ge 3$ then the group isn't abelian. We also have the interesting rule which spits out the non-commutative behavior,
  $$ sr^i = r^{-i}s = r^{n-i}s $$
\end{ndefi}

We can describe the group by it's elements and it's composition rule. We can define $D_4$ quite nicely,
$$ D_{4} = \{1, r, s, sr\} $$
and we find this to be commutative. Hence, $D_4$ is abelian.

\begin{nlemma}
  The following are true:
  \begin{itemize}
    \item The natural element is unique
    \item The inverse of each element is unique
    \item $(ab)^{-1} = b^{-1}a^{-1}$
    \item $au = av \implies u = v$ and $ub = vb \implies u = v$.
    \item Exponentiation makes sense
    \item Associativity means that any string of elements combined with the composition rule can be done in any order.
  \end{itemize}
\end{nlemma}

\subsection{Subgroups and Orders}


\begin{ndefi}[Subgroup]
  A subgroup, $H \subset G$, of a group $(G, \cdot)$,
  \begin{itemize}
    \item $\forall x, y \in H, x \cdot y \in H$
    \item $\forall x \in H, x^{-1} \in H$
  \end{itemize}
\end{ndefi}

This leads to also us being able to say $x \cdot x ^{-1} = e_G \in H$, so the natural element must also be in $H$.

\begin{eg}
  \begin{itemize}
    \item $(G, \cdot)$ is a subgroup of itself.
    \item We can take the trivial subgroup $\{e_G\}$.
    \item Given a $m \in \Z$ the subset $m\Z = \{mk : k \in \Z\}$ of integers is a subgroup of $(\Z, +)$.
    \item If we take $\{1, r, r^2, \dots, r^{n-1}\}$ this is a subgroup of $D_{2n}$.
  \end{itemize}
\end{eg}

\begin{ndefi}[Order of an element]
  Let $G$ be a group and $a \in G$. The order of $a$ is,
  $$ \ord (a) = \min \{n \ge 1 : a^n = e_G\} $$
\end{ndefi}
If you never reach the natural element, we call $\ord a$ to be infinite.

\begin{nlemma}
  The following are true,
  \begin{itemize}
    \item $\ord a = 1$ if and only if $a = e_G$
    \item Let $0 \ne n \in \Z$, then $\ord n = \infty$
    \item Every element in a finite group must have finite order. As if the order was infinite, then you must have infinitely elements, namely, $\{1, a, a^2, a^3, \dots, a^i, a^{i+1}, \dots\}$ which are all distinct and so $G$ cannot be finite.
    \item Consider some $k = \ord a < \infty$ and $n \ge 1$ with $a^n = e_G$, then $k \m n$
    \begin{proof}
      We have instantly that $n \ge k$ and now let $n = tk + r$ with $0 \le r < k$. Then, $a^n = a^{tk+r} = a^{tk} \cdot a^r = (a^k)^ta^r = e_G^ta^r = a^r = e_G$. Hence, we can say that $r = 0$ as $n$ is the smallest number such that $a^n = e_G$.
    \end{proof}
  \end{itemize}
\end{nlemma}

If we consider the symmetric group, then we can say,
\begin{nlemma}
  Let $n \ge k \ge 1$ and $\sigma = (a_1, a_2, \dots, a_k) \in S_n$ and is a k-cycle. Then $\ord \sigma = k$. Further, if $\sigma \in S_n$ then one can write $\sigma = \tau_1 \circ \tau_2 \circ \dots \circ \tau_m$ and we can find the order of this disjoin composition of cycles. We find that this is, $\ord (\lcm(\tau_i))_{i=0}^m$
\end{nlemma}

\begin{remark}
   Disjoint cycles commute and the decomposition is unique.
\end{remark}\marginnote{\emph{Lecture 3}}[0mm]

\begin{nlemma}
  If we take $\Z_n$, then we can take the order of say $[k]$, then we say that,
  $$ \ord [k] = \frac{n}{\gcd (n, k)} $$
\end{nlemma}

\begin{ndefi}[Generator]
  If $G$ is a group, $a\in G$, the subset $H = \{a^n : n \in \Z\}$ of $G$ consisting of all powers of the element $a$ is a subgroup, and is called the cyclic subgroup of $G$ generated by $a$, and $a$ is called a generator of $H$. The subgroup is denoted by $\gen a$.
\end{ndefi}


\begin{ndefi}[Cyclic Group]
  A group $G$ is called cyclic if $\ex a \in G$ such that $G = \gen a$ equals the (sub)group generated by $a$.
\end{ndefi}

\begin{nlemma}
  If a group is generated by $a$, it is also generated by $a^{-1}$
\end{nlemma}
\begin{proof}
  If we have any $a$, then we can write this: $a = (a^{-1})^{-1}$ and so the generator is not unique.
\end{proof}

\noindent
We notice that this works because we can cycle around $n$ and this can be proved using Euclidean division.


\begin{eg}
  \begin{itemize}
    \item $\Z = \gen 1$, is an infinite cyclic group generated by $1$. NB! Here $a^n = a \cdot n$
    \item on a similar note, $\Z_n = \gen {[1]_n}$. However, we can go further! If $k \ge 1$, with $\gcd (k, n) = 1$, then $\Z_n = \gen {[k]_n}$ is also generated by $[k]_n$. This is proved as $\ord [k]_n = \frac{n}{\gcd (k, n)} = n$ and so the order is the group and so $H = \gen k = \Z_n$.
    \item We can talk about $H = \gen {(1234)}$, which is a cyclic subgroup of $S_4$.
  \end{itemize}
\end{eg}

\begin{ndefi}[Product of Groups]
  Let $(G, \circ)$ and $(H, *)$ be two groups. We define a new group $(G \times H, \cdot)$ called the product group of $G$ and $H$, as follows,
  $$ G \times H = \{(g, h) : g \in G, h \in H\} $$
  is the set-theoretic product of $G$ and $H$. The composition law $(\cdot)$ is defined by,
  $$ (g_1, h_1) \cdot (g_2, h_2) = (g_1 \circ g_2, h_1 * h_2) $$
  The from this, the rest of the group axioms follow trivially.
\end{ndefi}

\begin{nlemma}
  Let $(G, \circ)$ and $(H, *)$ be groups. If $G$ and $H$ are abelian, then so is $G \times H$. If both $G$ and $H$ are finite, then so is $G \times H$ and $|G \times H| = |G||H|$
\end{nlemma}

\begin{proof}
  Assume that $G, H$ are abelian, and $g_1, g_2 \in G$ and $h_1, h_2 \in H$ then
  $(g_1, h_1) \cdot (g_2, h_2) = (g_1 \circ g_2, h_1 * h_2) = (g_2 \circ g_1, h_2 * h_1) = (g_2, h_2) \cdot (g_1, h_1)$, hence abelian. If both groups are finite, then the number of elements in $G \times H$ is the same as the number of pairs of elements and so that must be $|G| \times |H|$.
\end{proof}
