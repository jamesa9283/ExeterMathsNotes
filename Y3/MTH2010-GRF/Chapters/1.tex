% !TEX root = ../notes.tex

\section{Basics of Groups}
We start by defining a group, it is an example of an algebraic structure.
\begin{ndefi}[Group]
  $G$ is a nonempty set and endowed with a composition rule $(\cdot)$. We denote this $(G, \cdot)$. $(\cdot)$ is well defined, so we can associate another element $a \cdot b \in G$ and $a \cdot b$ is unique. $(\cdot)$ must be associative,
  $$ a \cdot (b \cdot c) = (a \cdot b) \cdot c $$
  The brackets are irrelevant when combining more than two elements. We also have \textbf{natural element}, so,
  $$ c \cdot e_G = c = e_G \cdot c $$
  There are also inverses, so,
  $$ a \cdot a^{-1} = e_G = a^{-1} \cdot a $$
  So the inverse naturalises the element.
\end{ndefi}

\noindent
If we just have a group usually $a \cdot b \ne b \cdot a$, if $a \cdot b = b \cdot a$ are called abelian or commutative groups. This is in reference to the mathematician Abel.

\noindent
If $G$ is finite as a set, then we can say that $G$ is a finite group and we denote the size or cardinality of $G$ as $|G|$, sometimes this is said to be the order. The cardinality can be infinite.\\

\noindent
\begin{eg}
  We know a very important group, the group of integers $\Z$. This set is infinite as $n \ne n + 1$ and the composition law is $+$ and we know that it's associative and natural element of $0$ and each element $n$ has an inverse of $-n$. We can also say,
  $$ k_1 + k_2 = k_2 + k_1 $$
  and so we have an infinite abelian group.
\end{eg}

\begin{eg}
  We can also consider groups of integers module $n$, denoted,
  $$ \Z_n = \{[0]_n, [1]_n, \dots, [n-1]_n\} $$
  where we have modulo classes (see Number Theory notes week 2). We can say, if $[k]_n = [l]_n$ if and only if $n \m k - l$. Also if you have $[k_1]_n$ and $[k_2]_n$, then $[k_1]_n + [k_2]_n = [k_1 + k_2]_n$. We have to check if this addition is well defined and it is, as you can just multiply by a constant as $[k+ rn]_n = [k]_n$. This is also a group with natural element of $[0]_n$ the inverse of $[k]_n$ is just $[-k]_n$ as $[k]_n + [-k]_n = [0]_n$. This is a finite abelian group and $|\Z_n| = n$.
\end{eg}

There is two worlds, non-commutative and commutative. Nature is not commutative, things aren't that nice. Our best example of the non-commutative group is the group of permutations. Let $n \in \Z^+$ and then let there be a set $S_n = \{1, 2, \dots, n\}$ and consider all possible bijections $\sigma$ from that set to itself. As these are finite sets and of the same cardinality, it suffices to check it's injective.
$$ \begin{pmatrix}
  1 & 2 & \dots & n-1 & n\\
  \sigma(1) & \sigma(2) & \dots & \sigma(n-1) & \sigma(n)
\end{pmatrix} $$
saying this is a bijection says the bottom row, given they are integers from 1 to $n$, appear only once, they don't appear twice.

\begin{eg}
  Let us take $S_4$, then we can take an element,
  $$ \sigma = \begin{pmatrix}
    4 & 3 & 2 & 1
  \end{pmatrix} $$
  and we can call this $\sigma$ and is an element of the group.
\end{eg}

New question, what is $|S_n|$, how many $\sigma$ are there? It's $n!$.

\begin{proof}
  Define $\sigma$ and you have to consider $\sigma(1)$ and theres $n$ possibilities, then for $\sigma(2)$ theres $n-1$ possibilities, then we can't use $\sigma(1)$ or $\sigma(2)$ and hence theres $n - 2$ possibilities for $\sigma(3)$ and so on. So we have,
  $$ n (n - 1) \cdot (n - 2) \cdot (n - 3) \dots 2 \cdot 1 = n! $$
\end{proof}

We can form a group where the composition is just $\circ$ on our set of bijections $\sigma$. If we take a $\sigma \circ \tau$ then this is also a bijection into $S_n$. This is associative and we get a natural element of $\id_{S_n}$. Then every bijection has an inverse $\sigma^{-1}$, which is unique. What is $\sigma^{-1}$, just reverse the order of the rows,
$$ \sigma^{-1} = \begin{pmatrix}
  1 & 2 & 3 & 4
\end{pmatrix} $$

This group is non-commutative if $n \ge 3$ then $S_n$ is not commutative. If we an integer $1 \le k \le n$ and take $k$ elements $\{a_1, a_2, \dots, a_k\} \subset \{1, 2, 3, \dots, n\}$. Then we define
\begin{ndefi}[k-cycle]
  A $k$ cycle, $\sigma = (a_1, a_2, \dots, a_k) \in S_n$  is a permutation,
  $$ \begin{pmatrix}
    a_1 & a_2 & \dots & a_{k-1} & a_k \\
    a_2 & a_3 & \dots & a_k & a_1
  \end{pmatrix} $$
\end{ndefi}

A k-cycle is a permutation and a bijection as you only write each number from $1$ to $n$ once. The $1$-cycle is just the identity. The $2$-cycle is the transposition. Then onwards it just shifts elements around. We can count the number of $k$-cycles, which is,
$$ \frac{n(n-1) \dots (n + k -1)}{k} $$

We can now see the dihedral group $D_{2n}$,
\begin{ndefi}[Dihedral Group]
  Let us take the $n$-gon and depending on when $n$ is odd or even we have a vertex along with the vertex one, you get them lying on the y-axis. Then you get all the rotations symmetries in the plane, which maps the $n$-gon to itself. There are $2n$ of them, the rotation clockwise with angle $\frac{2\pi}{n}$, there are $n$ of these. Then we have the elements where we flip the shape, $s$, first where $s^2 = 1$.
  $$ D_{2n} = \{1, r, r^2, \dots, r^{n-1}, s, sr, sr^2, \dots, sr^{n-1} \} $$
  Then this is our $2n$ elements. This is indeed a group with composition of rotations and $n \ge 3$ then the group isn't abelian. We also have the interesting rule which spits out the non-commutative behavior,
  $$ sr^i r^{-i}s = r^{n-i}s $$
\end{ndefi}
