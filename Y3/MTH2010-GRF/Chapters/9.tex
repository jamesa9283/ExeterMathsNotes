% !TEX root = ../notes.tex

Now for the first isomorphism theorem for rings,\marginnote{\emph{Lecture 21}}[0mm]
\begin{nthm}[The First Isomorphism Theorem for Rings]
  Let $\phi : R \to S$ be a surjective ring homomorphism. Then $R/\ker\phi \cong S$.
\end{nthm}
Here is what the theorem says pictorially,
\[\begin{tikzcd}
	R & S \\
	{R/\ker f}
	\arrow["f", from=1-1, to=1-2]
	\arrow["\phi"', from=1-1, to=2-1]
	\arrow["\pi"', from=2-1, to=1-2]
\end{tikzcd}\]

where we say that $\pi$ is a ring isomorphism. Hence, we just need to justify this, as most of it follows from the FIT for Groups.
\begin{proof}
  It suffices to prove that $\pi(\bar a\bar b) = \pi(\bar a)\pi(\bar b)$. Firstly we know $\pi(\bar a\bar b) = \pi(\bar{ab}) = f(ab) = f(a)f(b) = \pi(\bar a)\pi(\bar b)$. Hence we have proved the FIT for rings.
\end{proof}

Ideals can be added and multiplied,
\begin{ndefi}[Addition and Multiplication of ideals]
  Suppose we have two ideals, $I$ and $J$ and we define addition,
  $$ I + J = \{ a + b : a \in I, b \in J\} $$
  and the product,
  $$ IJ = \left\{\sum_{i=1}^m a_ib_j : a \in I,\,b\in J\, m \ge 1 \right\} $$
\end{ndefi}
\begin{proof}
  We aim to prove addition. So we have to prove that it is closed under addition, so suppose that $a_1,\,a_2 \in I$ and $b_1,\,b_2 \in J$, then,
  \begin{align*}
    (a_1 + b_1) + (a_2 + b_2) &= \underbrace{(a_1 + a_2)}_{\in I} + \underbrace{(b_1 + b_2)}_{\in J} \in I + J\\
  \end{align*}
  We also note that $0 \in I + J$ as we can write it as $0_R = 0_R + 0_R$. The inverse of an element, $(a + b)^{-1} = \underbrace{a^{-1}}_{\in I} + \underbrace{b^{-1}}_{\in J}$. Hence it's a subgroup under addition, so now take an element $a+b \in I + J$, then prove that take a $c \in R$ and then prove $(a + b)c \in I$. Now we distribute, $(a + b)c = ac + bc$, but we know that $a \in I$ and $b \in J$ and so $ac \in I$ and $bc \in J$. Hence, $ac + bc \in I + J$.\\

  \noindent
  Now for the product, so we show it's a subgroup. Firstly, zero, $0_R 0_R = 0_R$ and moreover $0_Ra = 0_R$. Now we show it's closed under addition, we show that $a_ib_i \in I$ where $a_i \in I$ and $b_i \in J$ and $c_id_i \in J$ where $c_i \in I$ and $d_i \in I$, then we consider $a_ib_j + c_id_j$, then we know this in $IJ$ as this is just a finite sums of terms where the first term is in $I$ and the second in $J$. Now we consider $(a_ib_i + c_id_i)^{-1} = (a_ib_i)^{-1} + (c_id_i)^{-1}$ and again this is a finite sum of the required form. Hence this is then just in $IJ$. We now seek to prove the other property, if $a_ib_i \in IJ$ and $c \in R$, then $c (a_ib_i) = \underbrace{(ca_i)}_{\in I}b_i \in IJ$.
\end{proof}

\begin{remark}
   It only really makes sense to consider finite sums in groups and rings as we have no sense of a limit as a group does not come inbuilt with limit points. See a topological space (I think).
\end{remark}
\marginnote{\emph{Lecture 22}}[0mm]
\begin{ndefi}[Principal Ideals]
  Let $R$ be a ring,
  \begin{enumerate}
    \item An ideal $I \in R$ is called principal if $I = (a)_R$ is generated by one element $a \in R$, called a generator of $R$.
    \item Let $a, b \in R$ and define $(a,\,b)_R = \{ac + bd : c, d \in R\}$. Then $(a,\,b)_R$ is an ideal called the ideal generated by $a$ and $b$.
  \end{enumerate}
\end{ndefi}

Here is a lemma about inclusion,
\begin{nlemma}
  Let $R$ be a ring,
  \begin{enumerate}
    \item $(b)_R \subset (a)_R$ if and only if $b = ac$ for some $c \in R$.
    \item If $R$ is an integral domain with identity, then,
    $$ (a)_R = (b)_R \iff a = bu \quad u \in R^\ti $$
  \end{enumerate}
\end{nlemma}
\begin{proof}
  ($\longleftarrow$) If $b = b1_R \in (b)_R$ and if it is contained in $(a)_R$, then every element of $b$ must be a multiple of $a$ and so $b = ac$ for some $c \in R$.\\
  ($\longrightarrow$) If $b = ac$, then we show $(b)_R \subset (a)_R$, so take an element $br \in (b)_R$ for some $r \in R$. This is an element as $b = (ac)r = a(cr) \in (a)_R$ as it's a multiple of $a$. \\

  \noindent
  ($\longleftarrow$), if $a= bu$, then $(a)_R \subset (b)_R$ from the previous part. However, $b = au^{-1}$ and so $(b)_R \subset (a)_R$ and so $(a)_R = (b)_R$.\\
  ($\longrightarrow$) If $(a)_R = (b)_R$, we weaken our statement by using that $(a)_R \subset (b)_R$, hence $a = bu$ for some $u \in R$, now we show that $u$ is a unit. We use the other inclusion, $(b)_R \subset (a)_R$ and so $b = ad$. We use these two identies, $a = (ad)u$, and so $a - (ad)u = 0_R$. We now distributivity, $a(1_R - du) = 0$ and as $R$ is integral, either $a = 0$ or $du = 1_R$, but $a$ is non-zero, so $du = 1_R$, hence, $d = u^{-1}$ and $u$ is a unit.
\end{proof}

\begin{eg}{(Non-Principal Ideals)}
  Let $R = \Z[X]$ and $I = (2,\,X)_{\Z[X]} =\{2h(X) + Xg(X) : h, g \in \Z[X]\}$ an ideal. Claim that $I$ is not principal. We argue by contradiction, assume that it is principal, $I = (f(X))_{\Z[X]}$ is pincipal generated by some $f(X) \in \Z[X]$. Then, $2 = 2 \ti 1 + X \ti 0 \in I$. So, $\ex p(X) \in R$ such that $2 = f(X)p(X)$. In particular $0 = \deg 2 = \deg f + \deg g$.
  which implies that $\deg f = \deg g = 0$ and $f$ is a constant polynomial. Further, since $f$ divides $2$ it holds that $f \in \{1, -1, 2, -2\}$. If $f = \pm 1$, then $1 = 2h(X) + Xg(X)$ for some $h,g \in \Z[X]$ which is impossible. So consider $f = \pm 2$, but $X = 0 \ti 2 + 1 \ti X \in I$. Hence, $X = 2q(X)$ for some $q(X) \in \Z[X]$ which again is impossible.
\end{eg}

\begin{eg}{(Construction of a surjective homomorphism with a given kernel)}
  Let $R = \Z[\sqrt{-3}] = \{a + b\sqrt{-3} : a, b \in \Z\}$ and $J = (2, 1 + \sqrt{-3})_R$. We want to construct a surjective homomorphism from $R$ to the finite field with kernel $J$. We want to map $2 \to 0_F$ and $1 + \sqrt{-3} \to 0$. We think about the image of an element $a + b\sqrt{-3} \to f(a + b\sqrt{-3}) = f(a) + f(b)f(\sqrt{-3})$.
  To make a homomorphism, we must just take the class of that integer as that's the only was to make a homomorphism in $\F_2$. We now consider $f(\sqrt{-3})$, we want to map this to $1$, which we can as $-3 = \bar 1$ and so $f(a + b\sqrt{-3}) = f(a) + f(b)$.
  Consider the map,
  $$f : \Z[\sqrt{-3}] \to \F_2$$
  $$ f(a + b\sqrt{-3}) = \bar a + \bar b $$
  Our map is surjective as we have infinitely many elements mapping to $\bar 0$ and infinitely many mapping to $\bar 1$, forming a dichotemy. Now to check it's a homomorphism
  \begin{align*}
    f\left[ (a + b\sqrt{-3}) + (c + d\sqrt{-3}) \right] &= f\left[ (a + c) + (b + d)\sqrt{-3} \right] \\
    &= \bar{a + c} + \bar{b + d} \\
    &= \bar a + \bar c + \bar b + \bar d \\
    &= (\bar a + \bar b) + (\bar c + \bar d)\\
    &= f(a+ b\sqrt{-3}) + f(c + d\sqrt{-3})
  \end{align*}
  Now for multipicity,
  \begin{align*}
    f\left[ (a + b\sqrt{-3})(c + d\sqrt{-3}) \right] &= f((ac - 3bd) + (ad + bc)\sqrt{-3})\\
    &= \bar{ac - 3bd} + \bar{ad + bc} \\
    &= \bar{ac + bd} + \bar{ad + bc}\\
    &= \bar a\bar c + \bar b\bar d + \bar a\bar d + \bar b\bar c\\
    &= (\bar a + \bar b)(\bar c + \bar d) \\
    &= f(a + b\sqrt{-3})f(c + d\sqrt{-3})\\
  \end{align*}
  So $f$ is a ring homomorphism, so we now show that $\ker f = (2,\,1+\sqrt{-3})_R$. Clearly, $J = (2, 1 + \sqrt{-3})_R \in \ker f$ as $f(2) = f(1 + \sqrt{-3}) = \bar 0$. Now for the other inclusion. Let $a + b\sqrt{-3} \in \ker f$ such that $f(a + b\sqrt{-3}) + \bar a + \bar b = \bar 0$ and so $a - b = 2k$ for some $k \in \Z$. This $a + b \sqrt{-3} = 2k + b + b\sqrt{-3} = 2k + b(1 + \sqrt{-3}) \in J$, as required.
\end{eg}
