% !TEX root = ../notes.tex

Recall\marginnote{\emph{Lecture 10}}[0mm] the left regular representation of $G$ on itself by left multiplication. Assume $G$ is of finite cardinality $n$. If we lavel the elements of $G$ as $\{g_1, \dots, g_n\}$ the regular representation defined a faithful permutation representation (an injective homomorphism),
$$ \rho : G \to S_n $$
called the regular permutation representation defined as followed,
\begin{ndefi}[Regular permutation representation]
  If $g \in G$, $\rho(g)$ is the permutation defined for $i,j \in \{1, \dots, n\}$ by,
  $$ \rho(g)(i) = j \qquad \text{if $g * g_i = g_j$} $$
\end{ndefi}
The permutation representation $\rho$ depends on the give labelling of the elements of $G$. In particular, since $\ker \rho = \{e_G\}$ we obtain by the FIT that $G$ is isomorphic to it's image $\r(G)$; a subgroup of $S_n$. Hence, we obtain,
\begin{nthm}[Cayley's Theorem]
  A finite group of cardinality $n$ is isomorphic to a subgroup of $S_n$.
\end{nthm}
Next, we define the left action of a group $G$ on the set of left cosets of a given subgroup. Let $H$ be a subgroup of $G$ and $A = (G/H)_{left}$ the set of left cosets of $H$. The group $G$ acts on $A$ by
$$ g \cdot (g'H) = (g * g')H $$
This action is called the action of $G$ on the left cosets of $H$ by left multiplication. If $|G : H| = m$ is finite, and we lavel the elementsof $A$ as $\{g_1H, \dots, g_mH\}$, then the above representation defines a homomorphism
$$ \t : G \to S_m $$
as follows: if $g \in G$, $\t(g)$ is the permutation defined for $i,j \in \{1, \dots, m\}$ by
$$ \t(g)(i) = j \qquad \text{if $g \cdot g_jH = (g*g_i)H = g_jH$}$$
The permutation representation $\t$ depends on the given labelling of the elements of $A$.\\

\noindent
Let $\t_H : G\to S_{G/H}$ be the permutation representation associated to the action of $G$ by left multiplication on the left cosets of $H$. Thus if $g \in G$,
$$ \t_H(g) : G/H \to G/H $$
is the bijection defined by,
$$ \t_H(g)(g'H) = (g*g')H $$

\begin{nthm}
  The following hold,
  \begin{itemize}
    \item $G$ acts transitively on $G/H$.
    \item The stabiliser of $e_GH$ is the subgroup $H$.
    \item $\ker (\t_H) = \bigcap_{x \in G} xHx^{-1}$, and $\ker (\t_H)$ is the largest normal subgroup of $G$ contained in $H$.
  \end{itemize}
\end{nthm}
\begin{proof}
  \begin{itemize}
    \item Let $aH, bH \in G/H$ and $g = b * a^{-1}$. Then
    \begin{align*}
      g \cdot(aH) &= (b * a^{-1})\cdot aH\\
      &= (b * a * a^{-1})H \\
      &= bH
    \end{align*}
    \item The stabiliser of $e_GH$ is
    \begin{align*}
      \{g \in G : g.(e_GH) = gH = H\} &= \{g \in G : gH = H\}\\
      &= H
    \end{align*}
    \item By definition,
    \begin{align*}
      \ker (\pi_H) &= \{g \in G : g \cdot (xH) = xH,\, \fa x \in G\}\\
      &= \{g \in G : (g*x)H = xH,\, \fa x \in G\}\\
      &= \{g \in G : (x^{-1}*g*x)H = H,\,x\in G\}\\
      &= \{g \in G : x^{-1}*g*x \in H,\,\fa x \in G\}\\
      &= \{g \in G : g \in xHx^{-1},\, \fa x \in G\}\\
      &= \bigcap_{x \in G} xHx^{-1}
    \end{align*}
    Further $\ker (\pi_H)$ is a normal subgroup of both $G$ and $H$. Now let $N$ be a normal subgroup of $G$ contained in $H$ then $N = xHx^{-1} \subset xHx^{-1},\,\fa x \in G$ hence, $N \subset \bigcap_{x \in G} xHx^{-1} = \ker (\pi_H)$. This shows that $\ker (\pi_H)$ is the largest subgroup of $G$ contained in $H$.
  \end{itemize}
\end{proof}

\begin{ncor}
   Let $G$ be a finite group of cardinality $n$ and $p$ the smallest prime number dividing $n = |G|$, then any subgroup of $G$ of index $p$ is normal. In particular, if $G$ has a subgroup of index $2$ then this subgroup must be normal.
\end{ncor}
\begin{proof}
  Let $H \le G$ and then $\pi_H$ is the permutation representation by the multiplication of left cosets of $H$ in $G$. Let $K = \ker pi_H$ and so $|G : K| = |G : H||H : K| = pm$.\\

  \noindent
  Since $H$ has $p$ left cosets, $G/K$ is isomorphic to a subgroup of $S_p$ ($\pi_H(G)$) by the FIT. By Lagrange’s Theorem, $pm = |G/K| \m |S_p| = p!$. Thus $m \m \frac{p! }{p} = (p-1)!$. But all prime factors of $(p-1)! < p$ and by the minimality of $p$, every possible prime divisor of $m \ge p$. This forces $m =1$ as $m \m |G|$, so $H = K$ is a normal subgroup of $G$ (since $K$ is normal): the equality $|H : K| = 1$ just means that $H = K$.
\end{proof}

\section{Class Equation}\marginnote{\emph{Lecture 11}}[0mm]
\subsection{Normalisers, Centralisers and Centers}

We are going to consider the class equation which relates to conjuagtion. We are going to consider the subsets of $G$, $\S(G) = \{A \sub G\}$, these arent necessarily subgroups, they are just subsets. Then we have the following action,
$$ (\cdot) : G \times \S(G) \to \S(G)$$
$$ (g,\,A) \mapsto gAg^{-1} = \{gag^{-1} : a \in A\}$$
If you have a group action, you have a stabiliser and an orbit.
$$ \stab(A) = \{g \in G : gAg^{-1} = A\} $$
and this is the normaliser.
\begin{ndefi}[Normaliser]
  The stabiliser of the above group action,
  $$ N_G(A) = \{g \in G : gAg^{-1} = A\} $$
\end{ndefi}
The normaliser is a subgroup of $G$ as it is just a stabiliser. The normaliser of $a$ acts on $A$ itself,
$$ \phi_A : N_G(A) \times A \to A $$
$$ (g, a) \mapsto gag^{-1}$$
this is a group action and we are interested in this group action. We can go deeper and find the stabiliser of $\phi_A$,
$$ \orb(a) = \{gag^{-1} : g \in N_G(A)\} $$
$$ \stab(a) = \{g \in N_G : gag^{-1}=a \iff ga = ag\} $$
Hence the stabiliser is just the commuting elements of this. Hence, we now look towards the kernel of $\phi_A$ and we say that,
\begin{ndefi}[Centraliser]
  We say that the kernel of the $\phi_A$ is the centraliser,
  $$C_G(A) = \ker \phi_A = \bigcap_{a \in A}\stab(a) = \{g \in N_A(a) L gag^{-1} = a,\,\fa a \in A\} $$
  and these are just all the commuting elements.
\end{ndefi}
and we know that
\begin{nlemma}
  The $C_G(A)$ is always a normal subgroup of $N_G(A)$.
\end{nlemma}
Now we ask, what happens when $A = G$ and so we ask, what is $N_G(G)$? $G$, and what is $C_G(G)$? Well we write it out,
$$ Z(G) = \{g \in G : gh = hg,\, \fa h \in G\} $$
and this is called the center of $G$.
\begin{ndefi}[Center of $G$]
  The center is a normal abelian subgroup of $G$ such that,
  $$ Z(G) = \{g \in G : gh = hg,\, \fa h \in G\} $$
\end{ndefi}

\begin{eg}
  If $G$ is abelian, then $Z(G) = G$, so we are only interested when $G$ is not abelian.
\end{eg}

We also note that the center is contained in the centraliser of every subset of $A$. The center is the intersection of all subsets of $A \in G$.\footnote{check this}

\subsection{The Class Equation}
Let us consider $g \in G$ and the subset $\{g\} \sub G$, then,
$$ N_G(\{g\})=C_G(\{g\}) = \{h \in G : hgh^{-1} = g\} = \{h \in G : hg = gh\} $$
This is then the subgroup of elements that commute with $g$. We note that $C_G(g) = \stab(g)$ is precisely the stabiliser of $g$ under the conjugation action of $G$ onto itself. The orbit of $\{g\}$ under conjuagtion is,
$$ \orb(g) = \{hgh^{-1} : h \in G\} $$
and consists of all elements of $G$ which are conjugate to $g$. \\
We note that $\orb(g) = \{g\} \iff hgh^{-1} = g$ for all $h \in G$ and this is equivalent to $g \in Z(G)$. Thus,
$$ |\orb(g)| = 1 \iff g \in Z(G) $$
Now assume that $G$ is finite. The orbit-stabiliser theorem states that,
$$ |\orb(g)| = \frac{|G|}{|C_G(g)|} $$
The congugacy classes of elements of $G$ form a partition of $G$
$$ G = \bigcup_{g \in G} \orb(g) $$
where the union is disjoint. By the above discussion, we have,
$$ Z(G) = \bigcup_{g \in G,\,|\orb(g)|=1}\orb(g) $$
where the union is over all of these elements of $G$ with $|\orb(g)|=1$. Let $\{\orb(g_1),\dots, \orb(g_r)\}$ be the distinct conjuagacy classes of $G$ that are \textbf{not} contained in $Z(G)$. Then,
$$ G = Z(G)\bigcup \left(\bigcup_{i=1}^r \orb(g_i)\right)$$
Counting the number of elements of $G$, and soncisdering the relation $|\orb(g_i)| = |G : C_G(g_i)|$, we find the class equation:

\begin{nthm}[The Class Equation]
  Let $G$ be a finite group and $\{\orb(g_1),\dots, \orb(g_r)\}$ be the distinct conjuagacy classes of $G$ which are \textbf{not} contained in $Z(G)$, then,
  $$ |G| = |Z(G)| + \sum_{i=1}^r |G : C_G(g_i)| $$
\end{nthm}
\marginnote{\emph{Lecture 12}}[0mm]

\begin{nthm}[]
  If $p$ is a prime and $|G| = p^m$ then $Z(G)$ is non-trivial.
\end{nthm}
\begin{proof}
  Considering the class equation, we know $p \m |G|$ and we claim $p \m |G : C_G(g_i)|$, this is true as $|G : C_G(g_i)| \m |G|$ but as $|G| = p^m$ we must know that $|G : C_G(g_i)|$ is a power of $p$ and so $p \m |G : C_G(g_i)|$ and so hence know that $p \m Z(G)$ and it must be non-trivial.
\end{proof}

Here a weaker version of sylows theorem,
\begin{nthm}[Cauchy's Theorem]
  Let $G$ be a finite group and $p$ a prime number which divides $|G|$. Then there exists an element of $G$ of order $p$, and a subgroup of $G$ of cardinality $p$.
\end{nthm}

\subsubsection{Conjuagacy Classes of $S_n$}
We will now look to find all of the cycles that commute with some cycle $\s$.\\

Take $1 \le m \le n \in \Z$ and say $\s = (a_1\,a_2\,\dots\,a_m) \in S_m$. There are $\frac{n(n-1)\dots (n+m-1)}{m}$ and this is $|\orb (\s)|$ and so using orbit-stabiliser theorem,
$$ \frac{n!}{|C_{S_n}(\s)|} = \frac{n(n-1)\dots (n+m-1)}{m} $$
and so we get that,
$$ |C_{S_n}(\s)| = (n-m)!m $$
We can determine this centraliser in a nicer way. The centraliser is just $\{\t\in S_n : \t\s = \s\t\}$ and we know that $\{\s^i\t,\, \}$ where $0 \le i \le m- 1$ and $\t$ is disjoint. Looking at the cardinality of this set, we find that it's just $m(n - m)!$, which says that we must just have the centraliser as $C_{S_n}(\s) = \{\s^i\t\}$.

\subsection{Simple Groups}
\begin{tcolorbox}
  \center Simple groups are not simple.
\end{tcolorbox}

\begin{ndefi}[Simple Groups]
  $G$ is simple if the only normal subgroups of $G$ are $H = G$ and $H = \{e_G\}$.
\end{ndefi}

\begin{nthm}
  $A_5$ is a simple subgroup.
\end{nthm}
\begin{proof}
  We consider all the different cycles in $A_5$, you have
  $$ 1 \qquad (1\,2\,3) \qquad (1\,2\,3\,4\,5) \qquad (1\,2)(3\,4) $$
  Now we count the amount of each cycle. There are one $1$-cycles. There are 20 $3$-cycles, 24 $5$-cycles and 15 of the rest. Now let us find their orbits. \\
  We consult the orbit-stabiliser theorem. We see that,
  $$ |\orb_{A_5}((1\,2\,3))| = \frac{|A_5|}{|C_{A_5}((1\,2\,3))|}$$
  We can find the centraliser in $S_5$ so we then take that and consider $C_{A_5}((1\,2\,3)) = C_{S_5}((1\,2\,3)) \cap A_5$ and so we see that $C_{A_5}((1\,2\,3)) = 3$. Hence,
  $$ |\orb_{A_5}((1\,2\,3))| = \frac{|A_5|}{|C_{A_5}((1\,2\,3))|} = \frac{60}{3} = 20 $$
  and so all $3$-cycles conjugate in $S_5$ are conjugate in $A_5$.\\
  We now do the same for the $5$-cycles. We find that $|\orb_{A_5}((1\,2\,3\,4\,5))|$. Then we see that there are two conjugacy classes with both cardinality 12.\\
  Now for the last type. We can check that there is no odd permutation such that it commutes. Hence the cardinality of the centraliser of this is even and it divides $|A_5|$ and so we see it must be 15.\\

  Now suppose $H$ is normal, now it must be the union of the conjuagacy classes. Find ways to sum, $\{1, 12, 12, 15, 20\}$ to make $60$. There is only way to do this, by adding them all together. Hence it's either $1$ or $60$ and so $A_5$ is simple.
\end{proof}
