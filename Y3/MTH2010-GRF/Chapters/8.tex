% !TEX root = ../notes.tex

\begin{eg}
  What are the units in $\Z$? We need $ab = 1$ and so the units are $\pm 1$ and so $\Z^\ti = \{-1,\,1\} = \gen{-1} \cong \Z/2\Z$.\\
\end{eg}

We now define an integral domain,
\begin{ndefi}[Integral Domain]
  A ring is called an integral domain if it has no zero divisors
\end{ndefi}

a few motivating examples are that $\Z$ is an integral domain and $\Z/n\Z$ is also an integral domain if and only if $n$ is prime.

\begin{ndefi}[Field]
  A ring $F$ with identity is called a field if $F^\ti = F \sm \{0\}$, or $F$ is a field if every non zero element of $F$ is a unit.
\end{ndefi}

\begin{nlemma}
  $\Z/n\Z$ is a field if and only if $n$ is a prime integer.
\end{nlemma}
\begin{proof}
  If $\bar k \in \Z/n\Z$ then $\bar k \in (\Z/n\Z)^\ti$ if and only if $\gcd(k,\,n)=1$. Hence every non zero element is a unit if and only if $n$ has no positive divisor $k$ such that $1 < k < n$, which is precisely the definition of $n$ being a prime.
\end{proof}

and moreover, we have a special name for this field,
\begin{ndefi}[Finite field with $p$ elements]
  Let $p$ be a prime integer. The field $\Z/p\Z$ is denoted $\F_p$ and called a finite field with $p$ elements.
\end{ndefi}

Now we can link integral domains and fields,
\begin{nlemma}
  A field is necessarily an integral domain.
\end{nlemma}
\begin{proof}
  Exercise
\end{proof}

Now for something that really should have been defined before,
\begin{ndefi}[Subring]
  A subset $S$ of a ring $R$ is called a subring if $(S,\,+)$ is a subgroup of $(R,\,+)$ and $S$ is closed under multiplication.
\end{ndefi}

We also introduce that $\Z[i] = \{a + ib : a, b \in \Z\}$ are called the Gaussian Integers.
