% !TEX root = ../notes.tex

\section{Sylow's Theorems}\marginnote{\emph{Lecture 13}}[0mm]
Suppose we have a finite group $G$, $|G| = n$. We know if we have a subgroup, then the cardinality of the subgroup we know this divides $n$. Sylows Theorems regards the converse of this, if we have some $k \m n$, is there some $H$ such that $|H| = k$. Sylow's Theorems provides an answer, but only for positive powers of primes. \\

Suppose we have some prime $p$, such that $|G| = p^r \cdot m$, where $r \ge 1$ and $\gcd(m,\, p) = 1$. If we have a group that has cardinality that is some $p^r$, this is a $p$-group. If we have some $p$-group, then any subgroup is also a $p$-group. The subgroups of some group $G$, that are $p$-groups, are called the $p$-subgroups of $G$. More formally,
\begin{ndefi}[$p$-group]
  Let $p$ be a prime number. A group of cardinality $p^t$ for some $t \ge 1$ is called a $p$-group. A non-trivial subgroup of a $p$-group is a $p$-group.
\end{ndefi}

These $p$-subgroups, which have maximal cardinality, ie. if $|H| = p^r$, then we call this the Sylow $p$-subgroup.
\begin{ndefi}[Sylow $p$-group]
  If we consider a group $G$, such that $|G| = m \cdot p^r$, then the subgroups $H_i$, cardinality $|H_i| = p^r$ is called the Sylow $p$-groups.
\end{ndefi}
\begin{remark}
   We can consider a Sylow $p$-group for each prime in the prime decomposition of $|G|$.
\end{remark}

If we consider the set of all Sylow $p$-groups,
$$ \Syl_p(G) = \{H \subset G : |H| = p ^ r\} $$
and we consider $|\Syl_p(G)|$, we say $|\Syl_p(G)| = n_p(G)$

Sylow's Theorem tells us that for any power of $p^i \m |G|$ there exists a subgroup with cardinality $p^i$, in particular there exists $p$-Sylow Subgroups, ie. $n_p(G)$ is non-empty.

\begin{eg}
  Let's consider $S_3$ and we know $|S_3| = 6$ and so we can consider a $3$-Sylow subgroup. We get $\{1,\,(1\,2\,3),\,(1\,3\,2) \}$, ie. the subgroup generated by a $3$-subgroup. Now we consider the $2$-Sylow subgroups, there are $3$ of them. There are something very special about the number of these subgroups.
\end{eg}

There are four statements,
\begin{nthm}[Sylows Theorems]
  Let $p$ be a prime such that $p \m |G|$,
  \begin{enumerate}
    \item $n_p(G)$ is nonempty, there exists subgroups of cardinality $p^i$ for every $p^i \m |G|$. Every $p$-subgroup is contained a $p$-Sylow subgroup. In fact we find that a $p$-Sylow subgroup is just the elements of order $p^i$.
    \item Any two $p$-Sylow groups must be conjugate. They form an orbit on $G$ by the action of conjuagation.
    \item \begin{enumerate}
      \item The number of $p$-Sylow subgroups $\c 1 \mod p$. It also says that the number of Sylow subgroups must divide $m$.
      \item The number of $p$-Sylow subgroups is just the index of the normaliser of the $p$-sylow subgroups.
    \end{enumerate}
  \end{enumerate}
\end{nthm}

We now are going to try and focus on the proof of each of these. We need more than orbit-stabiliser theorem. Let us consider a finite $p$-group which acts on a finite set. We consider the congruence and point formula.\\

Suppose we have a group $|H| = p^r$ and $H$ acts on a finite set, then let's look at what happens to the orbits of some $x \in X$.
$$ \orb (x) = \{h \cdot x : h \in H\} $$
and we know
$$ |\orb(x)| = \frac{|H|}{|\stab(x)|} $$
let us see what happens, the first possibility is that $|\orb(x)| > 1$, this means that $\stab(x) \ne H$ and $p \m |\orb(x)|$ as $|H| = p^r$. Conversely, if $|\orb(x)| = 1$, then $\stab(x) = H$, hence $h \cdot x = x$ for any $h \in H$. If you have a group action and an element that satisfies this, then we call it the fixed point.
\begin{ndefi}[Fixed Point]
  Consider a group $H$ acting on a set $X$ and take a $x \in X$. Then if $h \cdot x = x$ for all $h \in H$, then we say that $x$ is a fixed point of the action.
\end{ndefi}
We know that,
$$ X = \bigcup_{x \in X}{\orb(x)} $$
and now we can distinguish the elements of cardinaliy one and not one.
$$ X = \bigcup_{|\orb(x)| = 1}{\orb(x)} + \bigcup_{|\orb(x)| > 1}{\orb(x)} $$
Then, of course as $X$ is finite we can count this, we get the cardinality of the set of fixed points in the first union. In the second union, we get a sum of some powers of $p$.
$$ X = |\{x \in X : \fa h \in H, h \cdot x = x\}| + \bigcup_{|\orb(x)| > 1}{\orb(x)} $$
Then we consider this $\mod p$, then the second union disappears, for conveniance, let us call $\Fix_H(X) = \{x \in X : \fa h \in H, h \cdot x = x\}$
$$ |X| \c |\Fix_H(G)| \mod p $$
This is the congruence point formula, or fixed point congruence formula.
\begin{nthm}[Fixed Point Congruence]
  Let $H$ be a finite $p$-group acting on a finite set $X$ and $\Fix_H(X)$ is the subset of fixed points of $X$ under this action. Then,
  $$ |X| \c |\Fix_H(G)| \mod p $$
\end{nthm}
\begin{ncor}
  Let $H$ be a finite $p$-group on a finite set $X$. Suppose $\gcd(|X|,\,p) = 1$, then there exists fixed points.
\end{ncor}

\subsection{Proof of Sylow I}
\begin{proof}[\textbf{Proof of Sylow I}]
  We will prove a stronger result, take any $1 \le i \le r-1$ and take a subgroup of $G$ of cardinality of $p^i$. There exists some $H \subset H' \subset G$ and $|H'| = p^{i+1}$. We will prove this by induction on $i$.\\
  For $i = 1$, this is just Cauchy's Theorem. This implies $\ex H = \gen{x} \subset G$ and $|H| = p$.\\
  Now we assume that there is a $H'$ such that $H\subset H' \subset G$ where $|H| = p^r$, and hence prove that there exists some other $H''$ that satifies the chain condition but $|H''| = p^r$. To do this we consider the action,
  $$ H \times G/H \to G/H $$
  $$ (h, gH) \mapsto hgH $$
  and now we consider the fixed points of this action. Hence,
  $$ |G/H| \c |\Fix_H(G/H)|\mod p $$
  Let $gH \in \Fix_H(G/H)$. Then $\fa h \in H$ we have,
  \begin{align*}
    &hgH &= gH \iff hg \in gH \iff g^{-1}hg \in H\\
    &g^{-1}Hg \subset H \iff g^{-1}Hg = H \iff g \in N_g(H)
  \end{align*}
  where the forth step follows from $|g^{-1}Hg| = |H|$. Thus,
  $$ \Fix_H(G/H) = \{gH : g \in N_G(H)\} = N_G(H)/H $$
  and so,
  $$ |G/H| \c |N_G(H)/ H|\mod p $$
  We know that $|H| = p^i$ and then $p \m |G/H| = p^{r-i}m$ it then divides $|N_G(H)/H|$ by the fixed point congruence formula. Then we consult Cauchy's Theorem, $N_G(H)/H$ has a subgroup of order $p$ which is of the form $\hat H/H$ where $\hat H \subset N_G(H)$ containing $H$. Thus, we can say $|\hat H/H| = p$ and so, $|H'| = p|H| = p^{i+1}$.\\
  Then we can continue this, until we reach a subgroup of order $G$ of order $p^r$ which must be a $p$-sylow subgroup of $G$.
\end{proof}

\subsection{Proof of Sylow II}\marginnote{\emph{Lecture 14}}[0mm]
Sylow II says that all $p$-sylow subgroups are conjugate.
\begin{proof}[\textbf{Proof of Sylow II}]
  Let $P$ and $Q$ be Sylow subgroups of $G$, we show there exists some $g \in G$ such that $Q = pPg^{-1}$. Consider the action,
  $$ Q \times G/P \to G/P $$
  $$ (g',\,gP) \mapsto g'gP $$
  Since $Q$ is a finite $p$- group we have the fixes point congruence,
  $$ |G/P| \m |\Fix_Q(G/P)|\mod p $$
  As $P$ is a Sylow $p$-group of $G$, $|G/P| = m$ is not $0 \mod p$, hence $|\Fix_Q(G/P)|$ is not $0\mod p$. The later implies that $g'g \in gP$ for all $g' \in Q$, which implies $g' \in gPg^{-1}$ for all $g' \in Q$ hence $Q \subset gPg^{-1}$. This implies the equality $Q = gPg^{-1}$ since both $Q$ and $gPg^{-1}$ have the same cardinality $p^r$. Hence, $P$ and $Q$ are conjugate.
\end{proof}

\subsection{Proof of Sylow III}
Sylow III says that the number of Sylow subgroups is congruent to $1\mod p$.
\begin{proof}[\textbf{Proof of Sylow III}]
  We know that there are at least one sylow subgroups. Let us take $P \in \Syl_p(G)$. Consider the action,
  $$ P \times \Syl_p(G) \to \Syl_p(G) $$
  $$ (g,\,Q)\mapsto gQg^{-1} $$
  due to Sylow II we know that $gQg^{-1}$ is also a $p$-subgroup. We apply the fixed point congruence theorem,
  $$ |\Syl_p(G)| \c |\Fix_P(\Syl_p(G))| \mod p $$
  Sylow III says that $|\Fix_P(\Syl_p(G))| \c 1 \mod p$, but in fact $|\Fix_P(\Syl_p(G))| = 1$. Hence we claim,
  \begin{claim}
    $\Fix_P(\Syl_p(G)) = \{p\}$
  \end{claim}
  \begin{proof}
    Let $Q \in \Fix_{P}(\Syl_p(G))$, this means that $gQg^{-1} = Q$, $\fa g \in P$. This just means that $g \in N_G(Q)$ and so $P \subset N_G(Q)$, but also we know $Q \subset N_G(Q)$ and $Q$ is always normal in it's normaliser. Apply Sylow II, as we know that both $P$ and $Q$ are $p$-Sylow subgroups of the normaliser, we can say that under the normaliser $P = hQh^{-1}$ for some $h \in N_Q(G)$ and hence as $Q$ is normal, then $hQh^{-1} = Q = P$. Hence, the only fixed point is just $P$. Hence, we have proved our claim.\\

    Note when we proved Sylow III, we considered an action by conjuagation, but when we proved Sylow II we considered left multiplication. Here we will consider a slightly different action.
    $$ G\times \Syl_p(G) \to \Syl_p(G) $$
    $$ (g, G) = gGg^{-1} $$
    Let $P \in \Syl_p(G)$, we consider $\orb(P) = \{g \in G : gPg^{-1}\}$. The orbit is contained in $\Syl_p(G)$ as it's just lots of $p$-Sylow subgroups. In Sylow II, we proved that any two $p$-Sylow subgroups are conjuagate, and so we can say that these two subgroups are going to be equal. Hence, there is only one orbit for all $p$-Sylow subgroups. Now apply Orbit-Stabiliser Theorem,
    $$ |\orb(g)| = \frac{|G|}{|\stab (g)|} $$
    and so as we know that $|G| = p^r \cdot m$ and $\gcd(p, m)= 1$ and so by Sylow III, we know that $\gcd(|n_p(G)|,\, p) = 1$. Here we have an integer which is coprime to $p$ which is coprime to $|n_p(G)|$, hence it must divide $m$.
  \end{proof}
  Hence, we have proved that $\Fix_P(\Syl_p(G))| = 1$ and that $n_p(G) \m m$ so we have proved Sylow III.
\end{proof}
In fact we can go further and say precisely that $n_p(G)$ is the number of $p$-Sylow groups. We aim to show that $n_p(G) = \frac{|G|}{|N_G(P)|}$
\begin{proof}[\textbf{Proof of Sylow III $\dagger$}]

  Let $P \in \Syl_p(G)$, then there is only one orbit by conjugation and so we use a the Orbit Stabiliser Theorem, we know that $\stab(g) = N_G(P)$. This then says that
  $$ n_p(G) = |\Syl_p(G)| = |\orb(P)| = \frac{|G|}{|\stab_G(\{P\})|} = \frac{|G|}{|N_G(P)|} $$
\end{proof}

\noindent
Let us take a $p$-Sylow subgroup of a group $G$, which is a subgroup of the normaliser of $G$. This subgroup is normal in the normaliser. We proved $n_p(G) \m m$, which is the index. If we consider $|G/N_G(P)| = \b$, we know that $\b \m m$. In Sylow III we proved that $n_p(G) \m m$ and in dagger we proved that $n_p(G) = \b$.

\begin{ncor}
  Let $P$ be a Sylow $p$-subgroup of a finite group $G$. Then $P$ is the unique subgroup of $G$ (ie. $n_p(G) = 1$) if and only if $P$ is a normal subgroup of $G$.
\end{ncor}
\begin{proof}
  Note that every conjugate $gPg^{-1}$ of $P$ is a Sylow $p$-group, and the converse holds by (Sylow II). Thus,
  $$ \Syl_p(G) = \{gPg^{-1} : g \in G\} $$
  Now $n_p(G) = 1$ ($\Syl_p(G) = \{P\}$) if and only if $gPg^{-1} = P$ for all $g \in G$, and so $P$ is normal.
\end{proof}

Again, the Sylows Theorems aren't interesting when the group is abelian as $gPg^{-1} = gg^{-1}P = P$. This means that finite abelian group is easy to understand.\\

An interesting theorem we never got to cover is that,
\begin{nthm}[Structure Theorem for finite abelian groups]
  Every finite group is isomorphic to a product of finite cyclic groups.
  $$ G \cong \Z^r \oplus \Z/a_1\Z \oplus \Z/a_2\Z \oplus \dots \oplus \Z/a_n\Z $$
  then this is unique up to isomorphism modulo the following condition; there exists some unique integers such that,
  $$ a_1 \m a_2 \m \dots \m a_n $$
\end{nthm}
\begin{eg}
  \begin{itemize}
    \item Let $G$ be a finite abelian group, then for any prime $p \m |G|$, there exists only one $p$-Sylow subgroup.
    \item Take $\Z/2^2\Z$ and $\Z/3^3\Z$ and consider the product. We get an abelian group, with cardinality of $2^2 \times 3^3$ for a $2$-subgroup is $2$ or $4$ and for a $3$-subgroup, we must have $3,6,9$ and so the $2$-sylow must have cardinality $4$ and the $3$-sylow subgroup must have cardinality $9$. We can go further and say that there is only $1$ $2$-sylow and it is $\Z/2^2\Z \times \{0\}$ and the $3$-sylow is just $\{0\}\times\Z/3^3\Z$
    \item Sylows Theorems only tell us about groups with cardinality with one prime, it can't go further. There is no more general theorem past Sylow Theorems.
  \end{itemize}
\end{eg}

\subsection{Classifying groups through Sylow}\marginnote{\emph{Lecture 15}}[0mm]
We want to use Sylow's Theorems to classify small groups. Cosnider the following example, $|G| = p$, then we know that it's cyclic and we know the subgroup. If we go forward and consider $|G| = pq$, where $p$ and $q$ are distinct. Assume $p < q$ wlog, if we look to the number of $q$-Sylow subgroups. We know $n_q(G) = 1 + tq$ where $t \ge 0$, but we also know that $n_q(G) \m p$. As $q > p$, it cannot divide $p$ and so we know that $t =0$ and so if $Q \in \Syl_q(G)$ then $Q$ must be normal.\\
If we consider $p$, then we know that $n_p(G) = 1 + sp$ and we know that $n_p(G) \m q$, but $q$ is a prime, so $n_p(G) = 1$ or $n_p(G) = q$. If $n_p(G) = q = 1 +sp$ and so $q \c 1 \mod p$. Hence we either have one normal subgroup or $p \m q - 1$.

\begin{eg}
  If $p = 5$ and $q = 13$, we know that the $13$-Sylow subgroup must be unique and the $5$-Sylow subgroup is also unique as $13 \not\c 1 \mod 5$.\\

  \noindent
  However, if we take $p = 3$ and $q = 7$, then the $7$-Sylow subgroup is unique and normal and the $7$-Sylow subgroup is not necessarily unique as $7 \c 1 \mod 3$.\\

  \noindent
  If we take $\Z/p\Z \times \Z/q\Z$ here $n_p(G) = n_q(G) = 1$. If we have a group of cardinality of $21$ and has 7 $3$-Sylow subgroups can't be abelian. This can be constructed through semi-direct product.
\end{eg}

\noindent
Let us consider a general group of cardinality $12$. Consider $|G| = 2^2 \times 3$, we can say that $n_3(G) \c 1 \mod 3$ and it must divide $4$ and so it's either $1$ or $4$. If it's one then $G$ has a normal sylow subgroup. If it's $4$, then $G$ is isomorphic to $A_4$.
\begin{claim}
  Let $G$ be a group, if $|G|= 12$ and $G$ has four $3$-Sylow subgroups then it is isomorphic to $A_4$.
\end{claim}
\begin{proof}
  We will have four $3$-sylow subgroups, $\Syl_3(G) = \{P_1,\,P_2,\,P_3,\,P_4\}$ and we know they form an orbit under conjuagation.
  $$ G\times \Syl_3(G) \to \Syl_3(G)$$
  $$ (g, P_i) \mapsto gP_ig^{-1} $$
  If we label the three subgroups, we get a permutation representation, hence a homomorphism, $\phi : G \to S_4$ such that $g \mapsto \s_g$. We also know that $n_3(G) = \frac{|G|}{|N_G(P_i)|} = 4$ but also we know that $P_i \subset N_G(P_i)$ both of them has three elements and so they must be equal, $P_i = N_G(P_i)$. We consider $\ker\phi$, we know this is just the intersection of the stabilisers. However, the stabilisers are just the normalisers. Hence,
  $$ \ker \phi = \bigcap_{i=1}^4 N_G(P_i) = \bigcap_{i=1}^4 P_i $$
  $P_i$ are distinct and their cardinality is a prime. Hence, their intersection must be trivial.
  $$ \bigcap_{i=1}^4 P_i = e_G $$
  Hence, $\phi$ is injective and so by FIT we have that $G \cong \phi(G)$ as $G$ has eight elements of order $3$ and hence so does $\phi(G)$. All three cycles in $S_4$ are in $A_4$. So consider $\phi(G) \cap A_4$, $8 \le |\phi(G) \cap A_4| \m 12$. Hence $|\phi(G) \cap A_4| = 12$, therefore $\phi(G) \cong A_4$. By transitivity of isomorphisms as $G \cong \phi(G)$, then $G \cong A_4$.
\end{proof}

More generally, $|G| = p^2 q$ where $p$ and $q$ are distinct primes. We first consider $q < p$, then $n_p(G) \c 1 \mod p \implies n_p(G) = 1 + tp$ and $n_p(G) \m q$ and these both imply that $t = 0$. Hence there is a unique $p$-Sylow subgroup that is normal. Now suppose $p < q$, then consider $n_q(G) = 1 + sq$ and $q \m p^2$. Hence $n_q(G) = 1, p$ or $p^2$ either $s = 0$, hence we have a unique Sylow subgroup. If $s > 0$ then we have that $q = p^2$ as $q > p$. Hence $sq = p^2 - 1 = (p+1)(p-1)$. Hence $q \m p + 1$. Hence $q = p + 1$. So $p =2$ and $q = 3$. Hence we have a group of cardinality $12$. Therefore $n_q(G) = 1$ or $p = 2$ and $q = 3$ and $|G| = 12 \implies G \cong A_4$.
