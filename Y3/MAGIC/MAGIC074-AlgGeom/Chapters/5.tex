% !TEX root = ../notes.tex

\section{Chapter 2 - Noetherian Spaces}
We recall the definition of a Noetherian Ring. Let $A$ be a (comm) ring with elements $f_1, \dots, f_n \in A$ then the ideal generated by $f_1, \dots, f_n$ is $I = \gen{ f_1, f_2, \dots, f_n} = \{a_1f_1 + \dots a_nf_n\} \sub A$. An ideal $A \sub A$ is finitely generated if and only if we can fine $f_1, \dots f_n \in A$ such that $I = \gen{f_1, \dots, f_n}$.
\begin{ndefi}[Noetherian Ring]
  $A$ is Noetherian if every ideal in $A$ is finitely generated.
\end{ndefi}

\noindent
This is equivalent to $A$ having the asending chain condition. That is, every chain of ascending ideals $I_1 \sub I_2 \sub \dots \sub I_n \sub A$ becomes stationary, i.e. $\ex N$ such that $I_n = I_N$ for all $n \ge N$.
\begin{proof}
  Exercise
\end{proof}

\begin{eg}
  Consider $A = k[X]$, we use the Euclidean Algorithm for polynomials and show that every ideal in $k[X]$ can be generated by one element $f(X)$, $I = \gen{f(X)}$. Therefore, $k[X]$ is Noetherian.
\end{eg}

\noindent
A non-example,
\begin{eg}
  Consider $A = k[X_1, \dots]$ is not Noetherian. Consider the ideal generated by all of the $X_i$'s.
\end{eg}

\begin{exercise}
  If $\phi : A \to B$ is a surjective ring homomorphism and $A$ is Noetherian, then $B$ is Noetherian.
\end{exercise}

\begin{nthm}[Hilbert Basis Theorem]
  If $A$ is Noetherian, then $A[X]$ is Noetherian.
\end{nthm}
\noindent
This tells us that if $k$ is a field, then $k[X]$ is Noetherian, and so therefore $k[X_1, \dots, X_n]$ is Noetherian. If $A = k[a_1, \dots, a_n]$ is a finitely generated $k$-algebra, then we have a surjective algebra homomorphism $\phi : k[X_1, \dots, X_n] \to A$ where it's defined by $X_i \mapsto a_i$. Then we use Exercise 2, to see that $A$ is Noetherian. This current isn't geometric, we will discuss the geometric interpretations now.\\

\noindent
\begin{ndefi}[Topological Noetherian Space]
  $A$ topological space $X$ is Noetherian if every descending chain of closed subsets $Z_1 \supseteq Z_2 \supseteq Z_3 \supseteq \dots \supseteq Z_n$ becomes stationary.
\end{ndefi}
\noindent
This can be seen to be equivalent to saying that every chain of open subsets becomes stationary or every non-zero collection of closed subsets of $X$ has a minimal element.\\

\noindent
Now let $(V, k[V])$ be an affine variety, and $Z \sub V$ a subset. We define $I(Z) = \{f \in k[V] : f(x) = 0 \, \forall x \in Z\}$. If $Z \sub V$, then $Z = \V(S)$ for some $S \sub k[V]$, since $S \sub I(Z) \implies \V(I(Z)) \sub \V(S) = Z$, clearly $\V(S) \sub \V(I(Z))$. Therefore $\V(S) = \V(I(Z))$. Therefore the vanishing set of a subset of $k[V]$ is an ideal. \\

\noindent
If $Z_1 \supseteq Z_2 \supseteq \dots$ is a descending chain of closed subsets in $V$, then obtain thr ascending chain of ideals in $k[V]$:
$$ I(Z_1) \sub I(Z_2) \sub \dots \sub k[V] $$
Since $k[V]$ is Noetherian, then there must be some $N$ such that for all $n \ge N$, $I(Z_n) = I(Z_N)$ and so $\V(I(Z_n)) = \V(I(Z_N))$ and so $\V$ is a Noetherian space.\\

\noindent
\subsection{Irriducible Components}
Here is some motivation. Consider $W = V[X^2YZ + Y^2Z - YZ^2] \sub \A^3$, this is two planes intersecting with a curve as you can factor this to $YZ(X^2 + Y - Z)$. This has three components, $I(W) = \gen Y \cap \gen Z \cap {X^2 + Y - Z}$

\begin{ndefi}[Irreducible]
  A topological space $X$ is called irreducible if it cannot be written as a union $X = Y \cup Z$ for some proper subsets $Y, Z \subsetneq Z$
\end{ndefi}

\begin{exercise}
  This for open subsets.
\end{exercise}
\begin{exercise}
  An affine variety $V$ is irreducible if and only if $k[V]$ is a integral domain.
\end{exercise}
\noindent
A closed subset $Z \sub V$ is closed iff $I(Z)$  is a prime ideal in $k[V]$

\begin{remark}
   If $X$ is a Noetherian space, then it may be written as a finite union of closed subsets.
\end{remark}

\begin{ndefi}[irredundant union]
  Let $X$ be a set and we have some subsets $X_1, \dots, X_n \sub X$ such that $X = \bigcup_{i=1}^n X_i$. This is an irredundant union of $\forall i \ne j$, $X_i \not\sub X_j$.
\end{ndefi}
\begin{eg}
  Let $X = \V(\gen X) \cup \V(\gen Y) \cup \V(\gen {X, Y}) \sub \A^2$. This is irredundant as $\V(\gen{X, Y}) \sub \V(\gen X)$.
\end{eg}


\begin{nthm}
  Let $X$ be Noetherian, then $X$ can be written as a finite irredundant union of $X_1, \dots, X_n$ of closed irreducible subsets $X = \bigcup_{i=1}^n X_i$
\end{nthm}
\begin{proof}
  Ommited, too similar to primary decomposition.
\end{proof}

\begin{ndefi}[Irreducible Compontents]
  These $X_i$'s are called the irreducible components of $X$
\end{ndefi}
\begin{eg}
  Thinking back to the motivating example, the irreducible components are $\V(Y)$, $\V(Z)$ and $\V(X^2 - Y + Z)$.
\end{eg}
\begin{eg}
  Let $V = \V(XZ, YZ) \sub \A^3$, this is just $\V(Z) \cup \V(X, Y)$. This shows us that it's not necessarily the same dimension!
\end{eg}
\noindent
We now want to get `algebra-geometry dictionary', which are the Hilbert's Nullstellensatz,


\subsection{Integral Extension + Nullstellensatz}

\begin{ndefi}[Integral element]
  Let $A$ be a subring of $B$, then an element $g \in B$ is \textbf{integral} over $A$ if there is some $n > 0$ and soe $f_1, \dots, f_n \sub A$ such that,
  $$ g^n + fg^{n-1} + \dots + f_n = 0 $$
  That is, $g$ satisfies monic equations over $A$. That is, $\ex F \in A[X]$ such that $F(g) = 0$ where $F$ is monic.
\end{ndefi}

\begin{eg}
  Let $A = k[X^2, X^3]$, where $B = k[X]$. Then $X \in B$ is integral over $A$ as $F(T) = T^2 - X^2$.
\end{eg}
