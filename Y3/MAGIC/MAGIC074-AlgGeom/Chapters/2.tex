% !TEX root = ../notes.tex

We defined the most important algebraic variety last time and then we defined an algebraic set, $\V(S)$. We start with some remarks from last time. There is some issues about $k$, we assumed that we could take any $k$. If $k$ is finite, then $p = p^e$, then for $a \in \F_p$ then $a^q = a$ by Euler Fermat Theorem. Then $f(X) = X^q - X \in \F_q[X]$, evaluates to $0$ forall $a \in k$, but $f$ is not the zero polynomial. So we have problems with finite fields.\\

\noindent
The other issues, is more geometric. If we have $\R$, then $X^2 + 1 \in k[X]$ doesn't have any zeros. Hence, we need to work with algebraically closed sets (another example is the Whitney Umbrella).
\begin{center}
  \textbf{From now on, we consider $k = \bar k$, here $k$ is algebraically closed.}\\
  \textbf{Also, all rings in the course are commutative, and contain $1$ and all ring homomorphisms take $1$ to $1$. }
\end{center}

\section{Affine Varieties}

Today, we give a definition of an affine variety that is dependant of the embedding in $\A^n$. Therefore, we need to define an algebra,
\begin{ndefi}[Algebra]
  Let $k$ be a field and $A$ be a ring, that is also a $k$-vector space. Then $A$ is a $k$-algebra if $\l \cdot (ab) = (\l \cdot a) \cdot b$, for all $\l \in k$ and $a,b \in A$
\end{ndefi}

The trivial example is $k$ is a $k$-algebra. The second example is $A = k[X]$. The third is that $k$ being any field and $V$ a set, let $A := \map (V, K)$, this is a $k$-algebra as $A$ is a ring with $(f + g)(v) = f(v) + g(v)$ and $(f \cdot g)(v) = f(v) \cdot g(v)$. $A$ is a $k$-vector space as $(\l \cdot f)(v) = \l \cdot f(v)$ for all $\l \in k$, for all $v \in V$.\\

We now need morphisms,
\begin{ndefi}[$k$-algebra homomorphism]
  Let $A, B$ be $k$-algebras. A map $\phi : A \to B$ is a morphism between $k$-algebras if it is a ring homomorphism and a $k$-linear map. We write,
  $$ \hom_{k-alg}(A, B) = \{k\text{-alg homom from} A \to B\} $$
\end{ndefi}

\begin{ndefi}[Subalgebra]
  Let $C \sub A$, $C$ is a subalgebra if $C$ is a subring and a $k$-subspace.
\end{ndefi}

If $A = k[X]$ and $B = k$ are $k$-algebras, then
$$ \hom_{\text{k-alg}} (k[X], k) \ni \phi $$
Then $\phi$ is determined by $\phi(X) = a \in k$. Have a bijection $a \in k$, then we can associate a $\phi : k[X] \to K$ to it. We can associate $a \mapsto (\phi_a : X \mapsto a)$. We will see that in more generalaity that
$$ \hom_{k-alg}(k[X], k) = \A^1(k) $$

In this course we will see that considering all algebras is too much, but there is one that is enough to describe what we want. We want to look at the right type of algebras, more specifically the finitely generated $k$-algebra
\begin{ndefi}[Finitely generated $k$-algebra]
  $A$ is finitely generated if $A = k[a_{1}, a_2, \dots a_n]$ for some finite set $S = \{a_1, a_2, \dots, a_n\}\sub A$.
\end{ndefi}
Then we define a morphism $\phi : k[X_1, \dots, X_n] \to A = k[a_{1}, \dots, a_n]$, we define $X_i \mapsto a_i$ for all $i$. Now we see that $\phi$ is surjective and so by the First Isomorphism Theorem for $k$-algebras we get $k[X_1, \dots, X_n]/ \ker \phi \cong A$. We know $\ker \phi$ is an ideal in $k[X_1, \dots, X_n]$ and so finitely generated $k$-algebras are the same, in a bijection of rings $k[X_1, \dots, X_n]/ I$.

If $A \sub \map (V, K)$ be a subalgebra and $x \in V$, then there is always a $k$-algebra homomorphism $\e_x : A \to k$ where $\e_x (f) \mapsto f(x)$. This $\e_x$ is the evaluation homomorphism at the element $x$. Now assume $k = \bar k$ (algebraically closed), then,
\begin{ndefi}[Affine $k$-variety]
  An affine $k$-variety is a pair $(V, A)$, where $V$ is a set and $A\sub \map(V, K)$ is a finitely generated sub-algebra such that
  $$ V \to \hom_{k-alg}(A, k) $$
  $$ x \mapsto \e_x $$
  is a bijection.
\end{ndefi}

This means, the elements of $V$ correspond one to one with $k$-algebra homomorphisms from $A \to k$. Here is an example, Consider the pair $(\A^n(k), k[X_1, \dots, X_n])$ this an affine variety. $A$ is finitely generated by $X_1, \dots, X_n$. The $X_i$ are defined coordinate function $X_i(x_1, x_2, \dots x_n) = x_i$, we now show this is a bijection. Assume we have $(x_1, \dots, x_n)$ and $(y_1, \dots, y_n) \in \A^n$ and $\e_x = \e_y$. Then, $\e_x(X_i) = \e_y(Y_i)$ for all $i$. Then by Exercise 1, $(x_1,  \dots, x_n) = (y_1, \dots, y_n)$. Hence, it is injective.
We now show surjectivity, $\phi \in \hom_{k-alg}(A, k)$. Set $x_i := \phi(X_i)$ for all $i$ and $X \in A^n$. Then $\phi(X_i) = x_i = \e_x(X_i)$ for all $i$. Since, $X_i$ generate $A$, we must have $\phi(f) = \e_x(f)$ for all $f \in A$, and hence $\phi = \e_x$. Therefore we have surjectivity. In total, $\A^n \to \hom_{k-alg}(k[X_1, \dots, X_n], k)$ is a bijection and $(\A^n, k[X_1, \dots, X_n])$ is an affine variety.

We say $V$ is an affine variety to mean, we consider the pair $(V, A)$ for $A := k[V]$ is the coordinate algebra of $A$. We can now do the Zariski Topology of an affine variety.

\subsection{Zariski Topology of an affine variety}
Let $(V, A)$ be an affine variety. Then $S \sub A$ define
$$ \V(S) := \{x \in V : f(x) = 0 \,\forall f \in S\} $$
\begin{exercise}
  Show that $\V(S)$ $S \sub A$ form the closed sets of a topology on $V$, the Zariski Topology.
\end{exercise}

\begin{nprop}
  Let $W \sub V$, where $(V, A)$ is an affine variety, and $W$ is closed. Then $W$ itself is an affine variety.
\end{nprop}
