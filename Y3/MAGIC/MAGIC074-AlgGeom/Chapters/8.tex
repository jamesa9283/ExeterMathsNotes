% !TEX root = ../notes.tex

\noindent
\begin{proof}[(3)]
  We know $k[V]$ is a finitely generated algebra, by some $g_1, \dots, g_r$. Let $Z = \gen{g_1, \dots, g_r}$. Consider $\phi : T_xV \to \Hom_k (Z, k) = Z^*$. We consider $\a \mapsto \a|_Z$, the restriction. Then $\phi$ is injective, consider $\left.\a\right|_Z = 0$, then $\a(g_i) = 0$ and if it's zero on the generators then it's zero in the whole space. Hence $\a = 0$.Since $\dim_k (\Hom_k (Z, k)) < \infty$, it is finite. Then since $\phi$ is injective, then $\phi(T_xV) \le Z^*$ and so it has finite dimension.
  Therefore, if $k[V]$ is generated by $r$ elements then $\dim_k (T_xV) \le r$ for all $x \in V$.
\end{proof}

\begin{eg}
  Let $(V, k[V]) = (\A^n, k[X_1, \dots, X_n])$. Let $a = (a_1, \dots, a_n) \in \A^n$. We define $\a_i : k[V] \to k$ where $f \mapsto \left. \pd f {X_i}\right|_a$ (this is the formal derivative and evaluation at $a$). We know $\a_i \in T_a\A^n$, these are $k$-linear and they have a Leibnitz rule, which is just the product rule. We also know that the $\a_i$ are $k$-linearly independent and so $n \le \dim T_aV \le n$.
\end{eg}

\subsection{Morphisms}
Let $\phi : V \to W$ be the morphism of affine varieties. We take a point $x \in V$, then any $\a \in T_xV$ is a derivation $a : k[V] \to k$ and also we have $\phi^\sharp : k[W] \to k[V]$. Then we compose $a \circ \phi^sharp : k[W] \to k \in T_{\phi(x)}W$. We see we get a $k$-linear map,
$$ d\phi_x : T_x V \to T_{\phi(x)}W $$
defined by,
$$ \a \mapsto a \circ \phi^\sharp $$
Where we call $d\phi_x$ is the differential of $\phi$ at the point $x$. Here is a another fact,
\begin{fact}
  If we have $\phi : V \to W$ morphism of affine varieties such that $\phi^\sharp : k[W] \to k[V]$ is surjective. Then for all $x \in V$ such that $d \phi_x : T_xV \to T_{\phi(x)}W$ is injective and,
  \begin{equation}
    \Im d\phi_x = \{\b \in T_{\phi(x)}W : \b(I) = 0\} \tag{($*$)}
  \end{equation}
  where $I = \ker \phi$
\end{fact}
\begin{proof}
  Check
\end{proof}
\noindent
The application is as follows. Let $\phi : V \to \A^n$ is the inclusion ($V$ is closed in $\A^n$). Assume $I(V) = \gen{f_1, \dots, f_r} \in k[X_1, \dots, X_n]$. Then we know $k[V] = k[X_1, \dots, X_n]/ I$. Then $\phi^\sharp : k[X_1, \dots, X_n] \to k[X_1, \dots, X_n]/I$ is surjective. Let $a = (a_1, \dots, a_n) \in V$, then if
$ \a \in T_aV $, then if $\a$ vanishes on $f_1, \dots, f_n$, then $\a$ vanishes on $I$. This then gives us that $\dim T_aV = \dim \{a \in T_a\A^n : \a(f_i) = 0\}$. From the example, we know a basis of $T_a\A^n$:
$$ \a_i = \left.\pd{}{X_i}\right|_a \qquad \forall i = 1,\dots, n $$
This then means that $T_a\A^n$ is the $k$-span of the $\a_i$. This is just,
$$ T_a\A^n \ni \sum_{i=1}^n c_i\a_i $$
We have $d\phi_a : T_aV \to T_a\A^n$ and then $(*)$. We have image,
\begin{align*}
  \Im (d\phi_a) &= \{\b \in T_a\A^n : \b(I) = 0\} \\
  &= \{\sum_{i=1}^n c_i\a_i : \sum_{i=1}^n c_i\a_i (f_j) = 0 \}
  &= \{\sum_{i=1}^n c_i\a_i : \sum_{i=1}^n c_i\left.\pd {f_j}{X_i}\right|_{a} = 0 \}
\end{align*}
Then we now have linear equations. Hence we consider the jacobean, $A = (a_{ij})_{i = 1, \dots n \\j = 1, \dots, r} \in M_{n \ti n}(k[\A^n])$. Then we have $a_{ij} = \pd {f_j}{X_i}$. We have $\dim T_aV$ is just the dimension of the space row vectors $c$ such that $cA = 0$. Hence we have that the dimension is $n - \rank (A)$. Hence we come to fact 3,
\begin{fact}
  If we have a closed subset $V \sub \A^n$, given by $I = I(V) = \gen{f_1, \dots, f_n}$. For $a \in V$, we have $\dim T_aV = \rank{\left.\pd {f_j}{X_i}\right|_a}$
\end{fact}
We consider the cusp $V = V(X^3 - Y^2)$. We very obviously have some interesting things happening at zero. We consider the jacobean,
$$ \mathcal{J} = \begin{pmatrix}
  \left.3X^2\right|_a \\
  \left.-2Y\right|_a
\end{pmatrix} $$
If we consider $(0, 0)$, then we get the zero matrix and so $\rank \mathcal{J} = 0$. Hence $\dim T_{(0, 0)}V = 2$. If $a = (a_1, a_2) \ne (0, 0)$. Then we see,
$$ \mathcal{J} = \begin{pmatrix}
  3a_1^2 \\
  -2a_2
\end{pmatrix} $$
Therefore, we have at least one non-zero entry and so $\rank A = 1$, and so $\dim T_a V = 1$. Hence, the origin is special.

\begin{remark}
  For any $m \in \N$, then $\dim T_aV \ge m$ if and only if $\left.n - \rank A \right|_a \ge n$, or $\left.\rank A\right|_a \le n - m$. That is, all $(n - m + 1) \ti (n - m + 1)$ minors of Jacobean vanish at $a$. Moreover, $\forall m : \{a \in V : \dim T_aV \ge m\}$ is a closed set!
\end{remark}

\begin{fact}% Fact Four
  Let $V$ be an affine variety and $m \ge 0$. Then $\{a \in V : \dim T_aV \le m\}$ is an open set.
\end{fact}

Now we have to suppose that $V$ is reducible. This is because otherwise there is some problem at the intersection of the irreducible components. Let $d := \min_{a \in V} \{\dim T_aV\}$ for $0 \le d \le n$ where $V \sub \A^n$.
\begin{ndefi}[Smooth]
  We call $a\in V$ smooth (or nonsingular) if $\dim T_aV = d$.
\end{ndefi}
We know by fact four, that the set of smooth points is open. If $a \in V$ is singular, if $a$ is not smooth.
$$ \Sing V = \{a \in V : a \text{ is singular }\} = \{a \in V : \dim T_a V > d\} $$
This is the singular locus of $V$ and is closed. We call $V$ smooth if $\sing V \ne \vn$. We finally state a Theorem,
\begin{nthm}
  Let $V$ be an affine irreducible variety, and $x \in V$ be a smooth point. Then $\dim V = \dim T_x V$
\end{nthm}