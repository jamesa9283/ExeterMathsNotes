% !TEX root = ../notes.tex

\section{Introduction}
What is this course about?\marginnote{\emph{Lecture 1}}[0mm]
\begin{nthm}[Invariance of domain]
  If $n \ne m$, then $\R^n$ and $\R^m$ are not homeomorphic
\end{nthm}

\noindent
This is hard to prove, there is no way to directly prove this without a big deal of theory, but soon we will be able to do that. Now consider, $\SO(3)$ and $p = \{\text{trace-one projectors in } \R^4\} = \{A \in M_4(\R) : A^T = A = A^2, \, \tr A = 1\}$ and now $S^3 = \{x \in \R^4 : x_1^2 + x_2^2 + x_3^2 + x_4^2 = 1\}$ and $\RP^3 = S^3 \sm \sim$, where $x \sim (\pm x)$, so all $S^3$ with opposite points glued together.\\

\begin{nthm}
  $\SO(3)$, $P$, $\RP^3$ are homeomorphic to eachother, but not $S^3$.
\end{nthm}

\noindent
This is similar to the first theorem that we wrote, but in this case the spaces aren't the most natural or easiest to consider. We need some more thoughts to prove this. We can prove positive statements, that is, producing a homeomorphisms. To prove a negation, that is harder and we need more theory. Here is another theorem,

\begin{nthm}[Brouwer Fixed Point Theorem]
  Let $f : [0, 1]^n \to [0, 1]^n$ be continuous. Then there exists some $a \in [0, 1]^n$ with $f(a) = a$
\end{nthm}

\noindent
How can we prove this? Well we need the fact that $S^{n-1}$ is not contractible.

\begin{nthm}[Bosak-Ulam]
  Say $f : S^n \to S^m$ is odd (or antipodal) if $f(-x) = f(x)$ for all $x$. There is no odd continuous function $S^n \to S^m$ if $n > m$.
\end{nthm}

\noindent
We have said we need several complicated methods to do things in Algebraic Topology. Here we are interested in cohomology. In any space $X$ there is a ring $H^*(X)$ (the cohomology ring of $X$), this will be hard to construct and not instantly obvious. This is a graded ring, there are abelian groups $H^i(X)$ for $i \in \Z$ (but with $H^i(X) = 0$ if $i < 0$) and a product rule defining $ab \in H^{i + j}(X)$ for $a \in H^i(X)$ and $b \in H^j(X)$ and an identity $1 \in H^0(X)$. This is nearly a commutativity, we have a graded commutativity, $ba = (-1)^{ij}ab$ if $a \in H^i(X)$ and $b \in H^j(X)$. Any continuous map from $X \to Y$ provides a ring homomorphism $f^* = H^*(Y) \to H^*(X)$.\\

\noindent
Here is the most basic example, take $X = \R^n \sm 0$ with $n > 1$. There is an element $u_{n-1} \in H^{n-1}(\R^n\sm 0)$ such that $H^*(\R\sm 0) = \Z \oplus \Z{u_{n-1}}$ and so $H^0(\R^n\sm 0) = \Z$ and $H^{n-1}(\R\sm 0) = \Z u_{n-1}$ and $H^k(\R^n\sm 0) = 0$ for $k \ne 0, n-1$. What is the ring structure?
$p \ti (qu_{n-1}) = (pq)u_{n-1}$, the only thing of interest is $u_{n-1}^2 \in H^{2n - 2}(\R^n \sm 0) = 0$ and so $u_{n-1}^2 = 0$ and so there is no space for anything `interesting'. So, if $n \ne m$ then $H^*(\R \sm 0)$ and $H^*(\R^m \sm 0)$ are not isomorphic as graded rings and so therefore $\R^n\sm 0$ is not homeomorphic to $\R^m\sm 0$. Then after a few more steps, invariance of domain follows from this.\\

\noindent
Here is an outline programme for the course,
\begin{enumerate}
  \item Define $H^*(X)$ and prove key properties (Eilenberg-Steener Axioms)
  \item Use ES axioms to calulate $H^*(\R\sm 0)$ and $H^*(S^{n-1})$
  \item Deduce invariance of domain. Brouwers fixed point, fundemental theorem of algebra.
  \item Calculate $H^*(X)$ for various interesting spaces $X$ developing additional techniques along the way.
\end{enumerate}
As an example of $(iv)$, consider $\C^n$ and $F_n(\C) = \{(z_1, z_2, \dots, z_n) : z_i \ne z_j \text{ for } i \ne j\} \subset \C^n$, now define $f_{ij} : F_n(\C) \to S^1$ by $f_{ij}(z) = \frac{z_i - z_j}{|z_i - z_j|}$ and this makes sense as $z_i \ne z_j$. Recall, $H^1(S^1) = H^1(\C\sm 0) = \Z u_1$ and so put $a_{ij} = f^{*}_{ij}(u_1)$ (where is a ring homomorphism as cohomologies are a thing). Here are the facts,
\begin{itemize}
  \item $a_{ij} = a_{ji}$
  \item $a_{ij}^2 = 0$
  \item $a_{ij}a_{jk} + a_{jk}a_{ki} + a_{ki}a_{ij} = 0$ for all distinct $i, j, k$.
  \item $H^*(F_n(\C))$ is the free graded-commutative ring generated by the elements $a_{ij}$ subject to these relations (and give a basis for $H^*(F_{n}(\C))$ over $\Z$).
\end{itemize}

\begin{exercise}
  Understand $H^*(F_3(\C))$ and $H^*(F_4(\C))$ from this description, for example $H^1(F_3(\C)) = \Z[a_{12}, a_{13}, a_{23}] \cong \Z^3$. We look at the second cohomology group we can take products of these so we guess $a_{12}a_{13}, a_{12}a_{23}, a_{13}a_{23}$ as generators, but $a_{12}a_{23} + a_{23}a_{31} + a_{31}a_{12} = a_{12}a_{23} - a_{13}a_{23} - a_{12}a_{13} = 0$ and so we can remove $a_{13}a_{23}$ as it's just $a_{12}a_{13} - a_{12}a_{23}$ and so $H^2(F_3(\C)) = \Z[a_{12}a_{13}, a_{12}a_{23}]$. Furthermore $H^k(F_3(\C)) = 0$ for $k > 2$.
\end{exercise}


\section{Defining Cohomology}

We now define these cohomology groups,
\begin{itemize}
  \item For every space $X$, we define a differential graded ring $C^*(X)$
  \item For every DGR, $A^*$, we define a cohomology ring $H^*(A^*)$
  \item We define $H^*(X) = H^*(C^*(X))$.
\end{itemize}
We start with $(ii)$,
\begin{ndefi}[Cochain Complex]
  A cochain complex is a sequence of abelian groups $U^k$ (for $k \in \Z$, but often $U^k = 0$ for $k < 0$) together with some homomorphisms $d : U^k \to U^{k+1}$ satisfying $D^2 = 0$, ($U^{k-1} \to U^k \to U^{k+1} = 0$)
\end{ndefi}

\begin{ndefi}[Differential Graded Ring]
  A DGR is a cochain complex $A^*$ with a product rule $ab \in A^{i+j}$ for $a \in A^i$ and $b \in A^j$ and an element $1 \in A^0$ satisfying $1a = a1 = a$, $(ab)c = (ab)c$, $a(b + c) = ab + ac$, $(a + b)c = ac + bc$ and satisfies some Liebnitz rule $d(a, b) = d(a)b + (-1)^ia \cdot d(b)$ for $a \in U^i$ and $b \in U^j$ (The $(-1)^j$ is the dizul sign rule. It says if you swap two odd things a minus sign pops out.)
\end{ndefi}

\begin{eg}\marginnote{\emph{Lecture 2}}[0mm]
  $\Omega^*(M)$ is the de Rham complex of smooth differential forms on a smooth manifold.
\end{eg}

\begin{eg}
  Let $A^* = \Z[x] \oplus \Z[x]a$ with $x \in A^2$ and $a \in A^1$. This means that $A^{2n} = \Z x^n$ and $A^{2n+1} = \Z x^n a$ (for $n \ge 0$). We also need some differential, we let $d(a) = x$, this only gives a differential of one class but if forces everything else, as $d^2(a) = d(d(a)) = d(x) = 0$ and this forces $d(x^n) = 0$ by Liebnitz rule, what about $d(x^n a) = 0 + x^nx = x^{n + 1}$.
\end{eg}

Now we define the homology of the DGR. Given a DGR ring $A^*$, put $Z^* = Z^*(A^*) = \ker d$, or more specifically $Z^i = \ker (d : A^i \to A^{i+1}) = \{a \in A^i : da = 0\}$ and $B^*(A^*) = \im A^*$. We claim that,
\begin{claim}
  $Z^*$ is a subring of $A^*$ and $B^*$ is a two-sided ideal in $Z^*$.
\end{claim}
\begin{proof}
  We need to check it's a subset of $Z^*$. If $b \in B^*$ then $b = d(x)$ for some $x$, therefore $d(b) = d^2(x) = 0$ and so $b \in Z^*$. Hence $B^* \sub Z^*$.\\

  \noindent
  The Liebnitz rule gives $d(1) = d(1 \times 1) = d(1) \times 1 + 1 \times d(1) = 2d(1)$ and so $d(1) = 0$. Hence $1 \in Z^0$. Suppose that $a, b \in Z^*$. Then $d(ab) = d(a)b + ad(b) = 0 + 0 = 0$. So $Z^*$ contains $1$, closed under multiplication and closed under addition and subtraction. Therefore it is a subgroup. \\

  \noindent
  Suppose $a \in Z^*$ and $b \in B^*$ so $b = d(y)$ for some $y$. Then $d(ay) = d(a)y \pm ad(y) = 0 \pm ab$ and so $ab in \im(d) = B^*$. So $B^*$ is closed under left multiplication by $Z^*$, so it is a left ideal. Similarly if $a \in B^*$ and $b \in Z^*$ then $ab \in B^*$.
\end{proof}



\begin{ndefi}[Cohomology]
  $H^* = H^*(A^*) = Z^*(A^*)\sm B^*(A^*) = \ker d \sm \im d$
\end{ndefi}

\textbf{Terminology: }$Z^*$ is the subring of cocycles, $B^*$ is the ideal of coboundaries and $H^*$ is the cohomology ring, for $f \in Z^i$ write $[f] = $coset $f + B^i \in H^i$. (So $[f_0] = [f_1]$ if and only if $f_0 - f_1 \in B^i$ if and only if $f_0 - f_1 = d(x)$ for some $x$)

\begin{eg}
  Let $A^* = \Z[x] \oplus \Z[x]a$ with $d(x^n) = 0$ and $d(x^na) = x^{n+1}$. Then $Z^* = \Z[x]$ and $B^* = \Z[x]x = \spn \{x^{n+1} : n \ge 0\}$. Therefore, $H^* = \Z[x] \sm (x\Z[x]) = \Z$ ie. $H^0 = \Z$ and $H^i = 0$ for $i \ne 0$
\end{eg}

\begin{eg}
  If $A^* = \Omega^*(M)$ then $H^*(A^*)$ is the de Rahm cohomology of $M$.
\end{eg}

The next step for each space $X$, define a DGR $C^*(X)$. Then define $H^*(X) = H^*(C^*(X))$. For a bit of motivation we define $H^0(X)$. Introduce an equivalence relation as $X$ by $x \sim y$ if and only if there is a continuous path $u : [0, 1] \to X$ with $u(0) = x$ and $u(1) = y$.

In the diagram $a_0 \sim a_1$, $c_0 \sim c_1$ but $a_0 \not\sim c_0$. We define $\fg X = X \sm \sim [A, B, C]$. Then $H^0(X)$ will be $\text{map}(\fg X, \Z)$ (which is $\Z^3$). OR, $H^0(X) = \{f : X \to \Z : f(u_1) = f(u_0) \text{ for all paths} u \in X\}$.
Let us tidy this definition $S_0(X) = X$ and $S_1(X)$ is the set of paths in $X$, $C^0(X)$ is the set of maps from $S_0(X)$ to $\Z$ and $C^1(X)$ is the set of maps from $S_1(X)$ to $\Z$. Given some $f \in C^0(X)$, that is some $f : X \to \Z$ we want to define $df \in C^1(X)$, that is $df : S_1(X) \to \Z$. For every $u \in S_1(X)$ we need to define $(df)(u) \in \Z$. We put $(df)(u) = f(u(1)) - f(u(0))$.
Now
\begin{align*}
  Z^0 &= \ker (d : C^0(X) \to C^1(X))\\
  &= \{f : X \to \Z : (df)(u) = 0 \,\,\forall u\} \\
  &= \{f : X \to \Z : f(u(1)) = f(u(0)) \,\,\forall u\}
\end{align*}
Also $C^k(X) = 0$ for all $k < 0$ therefore $B^0(X) = 0$ and $H^0 = Z^0\sm B^0 = Z^0 = \map (\fg X,\, \Z)$ as before.\\

\noindent
We have only defined two of the $C^k$'s so we need the rest of them and the differentials.
\begin{ndefi}[Standard k-simplex]
  $\Delta_k = \{(t_0, t_1, \dots, t_k) \in \R^k : t_i \ge 0 \, t_0 + t_1 + \dots + t_k = 1\}$
\end{ndefi}

\begin{eg}
  $$ \Delta_0 = \{t_0 : t_0 \ge 0 \, t_0 = 1\} = \{1\} $$
  $$ \Delta_1 = \{ (t_0, t_1) : t_0, t_1 \ge 0 \, t_1 = 1 - t_0 \} $$
  We see that $\Delta_1$ is a line, the $\Delta_2$ is just a plane and $\Delta_3$ is a tetrahedron and so on.
\end{eg}

\textbf{NB!} For $0 \le i \le k$ let $e_i$ be the $i^{th}$ basis representation so $e_i \in \Delta_k$

\begin{ndefi}
  $S_k(X)$ is the set of all continuous maps $u : \Delta_k \to X$
\end{ndefi}

\begin{ndefi}[Cochain Group]
  $C^*(X) = \map(S_k(X),\, \Z)$
\end{ndefi}

\noindent
We now want to make this into a DGR and so we want to define a differential and then a product. We now need to consider the faces the simplices. We define $\d_1, \d_2, \d_3 : \D_1 \to \D_2$ by $\d_1(t_0, t_1) = (0, t_0, t_1)$, $\d_2(t_0, t_1) = (t_0, 0, t_1)$ and $\d_3(t_0, t_1) = (t_0, t_1, 0)$. They insert $0$ into position $i$. Recall that $e_0 = (1, 0, 0)$, $e_1 = (0, 1, 0)$ and $e_2 = (0, 0, 1)$ and so it's the oppposite case. Hence the $\d_i$'s give the edge opposite $e_i$. More generally, let us define,
$$ \d_i : \D_k \to \D_{k+1} \quad 0 \le i \le k+1 \qquad \d_i = t \text{ with $0$ in posiiton} i $$
Here is a formula, to be explained next time,
\begin{ndefi}[]
  $d : C^k(X) \to C^{k+1}(X)$ where
  $$ (df)(u) = \sum_{i=0}^{n+1} (-1)^i f(u \circ \d_i) $$
  for all $u : \D_{k+1} \to X$.
\end{ndefi}