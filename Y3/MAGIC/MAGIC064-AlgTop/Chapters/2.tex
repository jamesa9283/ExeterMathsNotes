% !TEX root = ../notes.tex

\noindent
Suppose $f \in C^kX$, so $f : S_kX \to \Z$, that is, $f(u) \in \Z$ for all $u : \D_k \to X$/ Suppose we have $S_{k+1}X$ that is $\D_{k+1}\to X$ so $v \circ S_i : \D_k \to X$ and so $f(v \circ S_i) \in \Z$, put $(df)(v)$ as we defined in the last lecture. So $df \in C^{k+1}(X)$.\\

Recall we spoke briefly about chain complexes and cochain complexes, for these we need $d^2 = 0 : C^{k-1}(X) \to C^{k+1}(X)$. Note
$$(d^2f)(v) = \sum_{j=0}^{k+1} (-1)^j (df)(v \circ \d_j) = \sum_{i=0}^{k} \sum_{j=0}^{k+1} (-1)^{i+j} f(v \circ \d_j \circ \d_i) $$
\newpage
\noindent
Now we look at these terms and find the following diagram.

\begin{wrapfigure}{r}{0.3\textwidth}
  \centering
  \vspace{-20pt}
  \resizebox{0.3\textwidth}{!}{\input{./figures/tetra.pdf_tex}}
  \caption{Tetetrahedron}
  \vspace{-20pt}
\end{wrapfigure}
\noindent
 Consider a tetrahedron, the $\d_i$ give ud the faces and the $\d_i\circ \d_j$ give us the edges. We can look at the arrows and the orientations of the faces. If we consider the ways we go around the faces, we find they go in opposite ways around and so they cancel out.

% add tetrahedron diagram
\begin{nlemma}
  if $0 \le j \le i \le k$, then$\begin{tikzcd}
	{\Delta_{k-1}} & {\Delta_{k}} & {\Delta_{k+1}}
	\arrow["\d_i", from=1-1, to=1-2]
	\arrow["\d_j"', from=1-2, to=1-3]
\end{tikzcd}$ is the same as$\begin{tikzcd}
	{\Delta_{k-1}} & {\Delta_{k}} & {\Delta_{k+1}}
	\arrow["\d_j", from=1-1, to=1-2]
	\arrow["\d_{i+1}"', from=1-2, to=1-3]
\end{tikzcd}$ or $\d_j\d_i = \d_{i+1}\d_j$.
\end{nlemma}
\begin{proof}
  Consider $x = (x_0, \dots, x_{k-1}) \in \D_{k-1}$. Then $\d_i x = x$ with $0$ inserted in position $i$. Therefore $\d_j\d_i x = \d_i x$ with a $0$ inserted into position $j \le i$ which pushes the first zero to position $i+1$. $\d_jx = x$ with $0$ inserted in position $j$ and $\d_{i+1}\d_j x$ with $0$ inserted in position $i+1 > j$. Therefore the first $0$ doesn't move, so it's just $x$ with zeros at positions at $j$ and $i+1$ which is just $\d_i\d_j x$.
\end{proof}

\begin{eg}
  Let $k = 6$, $j = 2$ and $i = 4$. Then
  \begin{align*}
    \d_2\d_4(x_0, \dots, x_5) &= \d_2(x_0, x_1, x_2, x_3, 0, x_4, x_5)\\
    &= (x_0, x_1, 0, x_2, x_3, 0, x_4, x_5)
  \end{align*}
  But also,
  \begin{align*}
    \d_5\d_2 &= \d_5(x_0, x_1, 0, x_2, x_3, x_4, x_5)\\
    &= (x_0, x_1, 0, x_2, x_3, 0, x_4, x_5)
  \end{align*}
  which are the same.
\end{eg}
\noindent
Note: in $(d^2f)(v)$, then $f(v \circ \d_j\circ \d_i)$ occurs with a sign $(-1)^{i+j}$ and $f(v \circ \d_{i+1} \circ \d_j)$ occurs withw $-(-1)^{i+j}$ and these two terms are the same and so they cancel with their difference in sign. Thus, $d^2f = 0$. Thus, the abelian group $C^kX$ and the maps $d : C^kX \to C^{k+1}X$ form a cochain complex.\\

Recall the definition of a cohomology of a cochain complex, $U^*$,
\begin{itemize}
  \item $Z^k = \ker (d : U^k \to U^{k+1})$
  \item $B^k = \im(d : U^{k-1} \to U^k)$
\end{itemize}
and $d^2 = 0$ implies that $B^k \le Z^k$ and so we can form a quotient $H^k = Z^k / B^k = H^*(U^*)$. We define $H^k(X) = H^k(C^k(X)) = \frac{\ker (d : U^k \to U^{k+1})}{\im(d : U^{k-1} \to U^k)}$. This has a ring structure and so $1 \in C^0(X) = \map(S_0(X), \Z) = \map(X, \Z)$ and so we let $1$ be the constant map sending everything in $X$ to $1$. Then $(d1)(u) = 1(u \circ \d_0) - 1(u \circ \d_1) = 0$ for $u : \D_1 \to X$. So $1 \in Z^0(X)$ and so $[1] = 1 + B^0(X) \in H^0(X)$ (but $B^0(X) = 0$).\\

\noindent
Now to define the product, suppose $f \in C^n(X)$ and $g \in C^m(X)$, ie. $f(u), g(v) \in \Z$ for all $u : \D_n \to X$ and $v : \D_m \to X$. We want to find $fg \in C^{n+m}(X)$ and so we need to define $(fg)(w) \in \Z$ for all $w : \D_{m+n} \to \Z$. We define
$\begin{tikzcd}
	{\Delta_n} & {\Delta_{n+1}} & {\Delta_m}
	\arrow["\lambda", from=1-1, to=1-2]
	\arrow["\rho"', from=1-3, to=1-2]
\end{tikzcd}$ where,
$$ \l(x_0, x_1, \dots, x_n) = (x_0, x_1, \dots, x_n, 0, \dots, 0) $$
$$ \rho(y_1, y_2, \dots, y_m) = (0, \dots, 0, y_0, y_1, \dots, y_m) $$
So $w \circ \l : \D_n \to X$ and $w \circ \rho : \D_m \to X$, so $f(w \circ \l) \in \Z$ and $g(w \circ \rho) \in \Z$. Therefore $(fg)(w) = f(w\circ \l) g(w \circ \rho) \in C^{n+m}(X)$. This is easily seen to distribute over addition.

\begin{exercise}
  This product is associative, with $1$ as a $2$-sided unit $1f = f = f1$ and also $d(fg) = d(f)g + (-1)^n fd(g) \in C^{n+m+1}(X)$ (so the Liebnitz formula holds)
\end{exercise}

\noindent
Therefore $C^*(X)$ is a differential graded ring. Therefore, $H^*(X)$ has a well defined ring structure. This isn't hard, what is substantially harder is $ab = (-1)^{nm}ba$ for $a \in H^n(X)$ and $b \in H^m(X)$, we will prove this later. We now have a definition of a cohomology ring, this isn't useful though. We can't calculate the cohomology of each space. We proved that $H^0(X) = \map (\fg X, \Z)$. Now let us focus on $H^1(X)$.\\

\noindent
\begin{figure}[!ht]
\centering
\input{./figures/homotopy_sq.pdf_tex}
\caption{Homotopy Square}
\end{figure}

\noindent
Suppose $X$ has a basepoint $x_0$, the loops at $x_0$ are maps $u : \D_1 \to X$ with $u(e_0) = u(e_1) = x_0$ so $u \in S_1X$. Suppose $f \in C^1X$ so $f(u) \in \Z$ for any such $u$. Suppose $f \in B^1X$, that is $f = dg$ for some $g \in C^0X$, ie $g : X \to \Z$. Then for any $u : \D_1 \to X$ we have $f(u) = (dg)(u) = g(u(e_1)) - g(u(e_0))$, if $u$ is a loop, then $f(u) = g(x_0) - g(x_0) = 0$.\\

\noindent
So for $a \in H^1X$ we have a well-defined $a(u) \in \Z$ given by $a(u) = f(u)$ for any $f \in Z^1X$ with $a = [f]$. But $\pi_1(X)$ is loops up to homotopy relation and points, suppose $h : [0, 1]^2 \to X$ is a homotopy relative to endpoints between $u$ and $b$, divide this into two triangles $p, q : \D_2 \to X$ suppose
$f \in Z^1(X)$ so $df= 0$ in $C^2X$ so $df(p) = df(q) = 0$ and using this we ca check that $f(v) = f(u)$. So $a \in H^1X$ the integer $a(u)$ depends only on the homotopy class of $u$, the corresponding element of $\pi_1(X)$. Using this we have a well-defined map from $\a : H_1X \to \Hom(\fg X, \Z)$ and it's a fact that if $X$ is path connected, then $\a$ is an isomorphism.