% !TEX root = ../notes.tex

\section{Divisibility}

\subsection{Division Algorithm}

\begin{ndefi}[Well Ordering Principle]
  Every non-empty subset of $\N_0$ contains a least element
\end{ndefi}

\begin{nthm}[Division Algorithm]
  Given a $a\in\Z$ and a $b\in\N_1$ there exists unique integers $q$ and $r$ satisfying $a = bq + r$ and $0\le r < b$.
\end{nthm}
The proof splits into uniqueness and existence.

\begin{proof}
  We shall first prove existence, define $S := \{a - xb : x \in \Z \quad \text{ and } a - xb \ge 0 \}$. We know $S \ne 0$ since,
  \begin{itemize}
    \item if $a \ge 0$, then choose $m = 0$, them $a - mb = a \ge 0$
    \item if $a < 0$, then let $a = m$, so $a - mb = a - ab = (-a)(b-1) \ge 0$ since $-a>0$ and $b > 0$\footnote{You absolute plank, there doesn't exist any numbers between $0$ and $1$ in $\Z$, so $b > 0$ is the same as $b \ge 1$}
  \end{itemize}
  Hence $S$ is non-empty subset of $\N_0$ and so by the well ordering principle $S$ must contain a least element $r \ge 0$. Since $r \in S$, then we have there exists a $q \in\Z$ such that $a - qb = r$ and so $a = qb + r$. Now it remains to check that $r < b$, so assume for a contradiction that $r \ge b$, then let there be a $r_1 = r -b \ge 0$. Then,
  $$ a = qb + r = qb + (r_1 + b) = (q+1)b + r_1 $$
  and so $a - (q+1) b = r_1 \in S$ and is smaller than $r$, a contradiction.\\

  Now let us show uniqueness, assume that there exist another pair $q', r'$ such that $a = q'b+r'$ where $0 \le r' < b$. Then form $a = a + qb + r = q'b + r'$ we have that, $(q - q')b = r' - r$. If $q = q'$, then we must have $r = r'$, suppose for a contradiction that this isn't true, then,
  $$ b \le |q - q'||b| = |r - r'| $$
  However, since $0 \le r, r' < b$ and so $|r - r'| < b$ which gives a contradiction.
\end{proof}

\subsection{Greatest Common Divisor}
Let us start with a theorem.
\begin{nthm}
  Let $a, b \in \Z$, $\ex d \in \N_0$ and non-unique $x, y \in \Z$ such that,
  \begin{enumerate}
    \item $d \m a$ and $d \mid b$
    \item and if $e \in \Z$, $e \mid a$ and $e \mid b$, then $e \mid d$
    \item $d = ax + by$
  \end{enumerate}
\end{nthm}

\begin{proof}
  If $a = b = 0$, then $d = 0$\\
  Suppose that $a \ne b \ne 0$, then let
  $$ S := \{am + bn : m,n \in \Z \text{ and } am + bn > 0\} $$
  Now $a^2 + b^2 > 0$ so $S$ is non-empty and a subset of $\N_1$. Hence, by the Well ordering principle then there must be some minimum element $d$. Then we can write $d = ax + by$ by definition of $S$.\\

  \noindent
  By the division Algorithm, $a = qs + r$ for some $q, r \in \Z$ with $0 \le q  < d$. Suppose for a contradiction that $r \ne 0$. Then,
  $$ 0 < r = a - qd = a - q(ax + by) = (1- qx)a - qby $$
  Hence, $r \in S$. But $r < d$, contradiciting the minimality of $d$ in $S$. So we must have $r = 0$, i.e $d \m a$. The same works for $d \m b$.\\

  \noindent
  Suppose that $e \in \Z$, $e \m a$ and $e \m b$. Then $e$ divides any linear combination of $a$ and $b$, so $e \m d$. Suppose that $e \in \N_1$ also satisfies $(i)$ and $(ii)$. Then, $e \m d$ and $d \m e$ and so $d = \pm e$, but $d, e \ge 0$ and so $d = e$. Thus $d$ is unique.
\end{proof}

\noindent
Note that this is a standard trick to prove that integers divide, by just proving that $r = 0$ by contradiction.

\begin{ncor}
  If $a, b \in \Z$ then there exists a unique $d \in \N_1$ such that.
  \begin{enumerate}
    \item $d \m a$ and $d \m b$
    \item if $e \in \Z$, then $e \m a$ and $e \m b$ then $e \m d$
  \end{enumerate}
\end{ncor}

\begin{proof}
  The existence of a $d$ is given by the theorem. In the proof of uniqueness we only use $(i)$ and $(ii)$.
\end{proof}

\begin{ndefi}[Greatest Common Divisor]
  Let $a, b \in \Z$. Them $d$ of the previous corollary is just the greatest common divisor of $a$ and $b$, written $\gcd (a, b)$. Also sometimes seen as $\hcf (a, b)$.
\end{ndefi}

If $\gcd(a, b) = 1$, then $a$ and $b$ are coprime.

\begin{identity}[Bezouts Identity]
  Given $a, b \in \Z$ there exist $x, y \in \Z$ such that $\gcd(a, b) = ax + by$.
\end{identity}

\begin{nprop}
  Let $a, b, c \in \Z$, then,
  \begin{enumerate}
    \item $\gcd(a, b) = \gcd(b, a)$
    \item $\gcd(a, \gcd(b, c)) = \gcd(\gcd(a, b), c)$
    \item $\gcd(ac, bc) = |c|\gcd(a, b)$
    \item $\gcd(1, a) = \gcd(a, 1) = a$
    \item $\gcd(0, a) = \gcd(a, 0) = |a|$
    \item $c \m \gcd(a, b)$ if and only if $c \m a$ and $c \m b$
    \item $\gcd(a + cb, b) = \gcd(a, b)$
  \end{enumerate}
\end{nprop}

Then we can consider the following remark,
\begin{remark}
   Note that $\gcd(a, b) = 0$ if and only if, $a = b = 0$. Otherwise, $\gcd(a, b) \ge 1$.
\end{remark}

\begin{proof}
  Checking these properties are pretty simple, for $(vi)$ just use Bezouts. \\

  We shall prove $(iii)$, so let $d = \gcd(a, b)$ and $e = \gcd(ac, bc)$. By $(vi)$, $cd \m e = gcd(ac, bc)$ since $cd \m ac$ and $cd \m bc$. Then by Bezouts, there exists $x, y \in \Z$ such that $d = ax + by$. Then,
  $$ cd = acx + bcy $$
  and as $e \m ac$ and $e \m bc$ and so by linearity we have $e \m cd$. Therefore, $|e| = |cd|$ and so, $e = |c|d$.\\

  Now, let's prove $(vii)$, let $e = \gcd(a + bc, b)$ and $f = \gcd(a, b)$. Then $e \m (a + bc)$ and $e \m b$. Thus by linearity, we have $e \m a$. Hence, $e \m a$ and $e \m b$ so by property $(vi)$, we have $e \m f$. Similarly we can get that $f \m a + bc$ and $f \m b$ and so again my $(vi)$ we have $e = f$ as $f, e \ge 0$.
\end{proof}

\begin{nlemma}[Euclids Lemma]
  Let $a, b, c \in \Z$. If $a \m bc$ and $\gcd(a, b) = 1$, then $a \m c$.
\end{nlemma}

\begin{proof}
  Suppose that $a \m bc$ and $\gcd(a, b) = 1$. By Bezouts, we get that for some $x, y \in \Z$ we get $1 + ax + by$. Hence, $c = acx + bcy$, but $a \m acx$ and $a \m bcy$, so $a \m c$ by linearity.
\end{proof}

\begin{nthm}[Solubility of linear equations in $\Z$]
  Let $a, b, c \in \Z$. The equation,
  $$ ax + by = c $$
  is soluble with $x, y \in \Z$ if and only if $\gcd(a, b) = c$
\end{nthm}

\begin{proof}
  Let $d = \gcd(a, b)$. Then $d \m a$ and $d \m b$ so if there exists $x, y \in \Z$ such that $c = ax + by$ then $d \m c$ by linearity of divisibility. Now, suppose that $d \m c$. Then we can write $c = qd$ for some $q \in \Z$. By Bezouts, there exists some $x', y' \in \Z$ such that $d = ax' + by'$. Hence, $c = qd = aqx' + bqy'$ and so $x = qx'$ and $y = qy'$ gives a suitable solution.
\end{proof}

\subsection{Euclids Algorithm}

\begin{nthm}[Euclids Algorithm]
  Let $a, b \in \N_1$ with $a > b > 0$ and $b \nmid a$. Let $r_0 = a$, $r_1 = b$ and apply the division Algorithm repeatedly to obtain a sequence of remainders defined sucessively,
  \begin{align*}
    r_0 &= r_1q_1 + r_2 && 0 < r_2 < r_1\\
    r_1 &= r_2q_2 + r_3 && 0 < r_3 < r_2\\
    &\vdots\\
    r_{n-2} &= r_{n-1}q_{n-1} + r_n && 0 < r_{n} < r_{n-1}\\
    r_{n-1} &= r_nq_n + r_{n+1} && r_{n+1}=0
  \end{align*}
  Then the last non-zero remainder, $r_n$ is the $\gcd(a, b)$.
\end{nthm}

\begin{proof}
  There is a stage at which $r_{n+1} = 0$ because the $r_i$ are strictly decreasing non-negative integers. We have,
  \begin{align*}
    \gcd(r_i, r_{i+1}) &= \gcd(r_{i+1}q_{i+1} + r_{i+2}r_{i+1})\\
    &= \gcd(r_{i+2}r_{i+1}) \\
    &= \gcd(r_{i+1}, r_{i+2})
  \end{align*}
  Applying this result repeatedly,
  \begin{align*}
    \gcd(a, b) &= \gcd(r_0, r_1)\\
    &= \gcd(r_2, r_3)\\
    &= \dots\\
    &= \gcd(r_{n-1}, r_n)\\
    &= r_n
  \end{align*}
  Where the last equality is because $r_n \m r_{n-1}$
\end{proof}

\begin{remark}
   One can also use Euclids Algorithm to find the $x, y\in \Z$ Bezouts Identity state to exist by working backwards. These aren't unique.
\end{remark}

\subsection{Extended Euclidean Algorithm}

Instead of doing Euclids, and working backwards we can compute our bezouts $x, y$ during euclids. This is the extended Euclids Algorithm. This time we are going to define sequnces of integers $x_i$ and $y_i$, such that $r_i = ax_i + by_i$. Recall that $r_n$ is the last non-zero remainder and that $r_n = \gcd(a, b)$. Therefore $\gcd(a, b) = r_n = ax_n + by_n$ and so $(x, y) := (x_n, y_n)$.\\

We have that $r_0 = a$ and $r_1 = b$. Hence, we see $r_0 = 1 \times a + 0 \times b$ and $r_1 = 0 \times a + 1 \times b$, and so we set $(x_0, y_0) := (1, 0)$ and $(x_1, y_1) := (0, 1)$. So, now we consider for $i \ge 2$ we have a pair $(x_j, y_j)$ for $j < i$. Then $r_{i-2} = r_{i-1}q_{i-1} + r_i$ and so,
\begin{align*}
  r_i &= r_{i-2} - r_{i-1}q_{i-1}\\
  &= (ax_{i-2} + by_{i-2}) + (ax_{i-1} + by_{i-1})q_{i-1}\\
  &= a(x_{i-2} - x_{i-1}q_{i-1}) + b(y_{i-2} - y_{i-1}q_{i-1})\\
\end{align*}
Thus we set $x_i := x_{i-2} - x_{i-1}q_{i-1}$ and $y_i := y_{i-2} - y_{i-1}q_{i-1}$. These can be defined recursively this way.
$$ (x_i, y_i) := (x_{i-2}, y_{i-2}) - q_{i-1}(x_{i-1}, y_{i-1}) $$

\begin{eg}
  We compute $\gcd(841, 160)$ use Extended Euclidean Algorithm.
  % Please add the following required packages to your document preamble:
% \usepackage{graphicx}
\begin{table}[!ht]
\centering
\begin{tabular}{llllllllll}
$i$ & $r_{i-2}$            &   & $r_{i-1}$            &          & $q_{i-1}$ &   & $r_i$ & $x_i$ & $y_i$ \\ \hline
0 &                       &   &                       &          &            &   & 841  & 1    & 0    \\
1 &                       &   &                       &          &            &   & 160  & 0    & 1    \\
2 & 841                   & = & 160                   & $\times$ & 5          & + & 41   & 1    & -5   \\
3 & 160                   & = & 41                    & $\times$ & 3          & + & 37   & -3   & 16   \\
4 & 41                    & = & 37                    & $\times$ & 1          & + & 4    & 4    & -21  \\
5 & 37                    & = & 4                     & $\times$ & 9          & + & 1    & -39  & 205  \\
6 & \multicolumn{1}{r}{4} & = & \multicolumn{1}{r}{1} & $\times$ & 4          & + & 0    &      &      \\
  & \multicolumn{1}{r}{}  &   &                       &          &            &   &      &      &
\end{tabular}%
\end{table}

Therefore, $\gcd(841, 160) = 1 = 841 \times (-39) + 160 \times 205$.
\end{eg}
