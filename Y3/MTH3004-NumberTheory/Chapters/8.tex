% !TEX root = ../notes.tex

\section{Quadratic Residues, Legendre Symbols, Euler Criterion and Gauss' Lemma}
We will study the theory of congruences modulo an odd prime $p$. By completing the square we can reduce any quadratic residue to,
$$ x^2 \c a \mod p $$

\begin{nlemma}
  Let $p$ be an odd prime and $a \in \Z$. Consider,
  \begin{equation}
    x^2 \c a \mod p
  \end{equation}
  if $p \m a$, then $(1)$ is equivalent to $x \c 0 \mod p$. Otherwise if $p \nmid a$ and $(1)$ has one solution, then $x \c b \mod p$ then $p \nmid b$ and $x \c -b$ is another, different solution.
\end{nlemma}
\begin{proof}
  If $x \c 0 \mod p$, then clearly $x^2 \c 0 \mod p$. The converse follows from Euclid's Lemma for primes.\\
  Now suppose $p \nmid a$ and $b^2 \c a \mod p$, then clearly $-b$ is also a solution to this equation. If $b \c - b \mod p$ and so $b \c 0 \mod p$. But then $a \c b^2 \c 0 \mod p$ - Contradiction as $a \nmid b$.
\end{proof}

\begin{ndefi}[Quadratic Residue]
  Let $p$ be an odd prime and $a \in \Z$ such that $p \nmid a$. Then $a$ is a Quadratic Residue mod $p$ if $\ex x \in \Z$ such that $x^2 \c a \mod p$ and $a$ is a Quadratic Non-Residue if not.
\end{ndefi}

\begin{nprop}
  Let $p$ be an odd prime. Then every reduced residue system mod $p$ contains exactly $\frac{(p-1)}{2}$ quadratic residues and $\frac{(p-1)}{2}$ quadratic non-residues mod $p$. The quadratic residue belong to the residue classes contaning,
  $$ 1^2,\,2^2,\,\dots,\,\left(\frac{(p-1)}{2}\right)^2 $$
\end{nprop}
\begin{proof}
  First show that the list of numbers are distinct mod $p$. If $x^2 \c y^2 \mod p$ where $1 \le x, y \le \frac{p-1}{2}$ then $(x+y)(x-y) \c 0 \mod p$. But, $1 < x+ y < p$ so $x + y$ is coprime to $p$. So by the Cancellation Law, we must have $x - y \c 0 \mod p$ and so $x \c y \mod p$ and as $|x - y| < p$, then $x = y$. The remaining squares are,
  $$ \left(\frac{p+1}{2}\right)^2,\,\left(\frac{p+3}{2}\right)^2,\,\dots,\, (p - 2)^2,\, (p-1)^2 $$
  but $(p - k)^2 \c (-k)^2 \c k^2 \mod p$ for every $k \in \Z$ with $1 \le k \le \frac{(p - 1)}{2}$, these are then congruent to,
  $$ \left(\frac{p-1}{2}\right)^2,\,\left(\frac{p-3}{2}\right)^2,\,\dots,\, 2^2,\, 1^2 $$
  this is our original list. Hence, there are precisely $\frac{p-1}{2}$ quadratic residues mod $p$ and so there are $\frac{p-1}{2}$ quadratic non-residues mod $p$.
\end{proof}

\subsection{Legendre Symbol}
\begin{ndefi}[Legendre Symbol]
  Let $p$ be an odd prime. For any $a \in \Z$, we define the Legendre Symbol to be,
  $$ \gls = \begin{cases}
    +1 & \text{$p \nmid a$ and $a$ is a quadratic residue mod $p$} \\
    -1 & \text{$p \nmid a$ and $a$ is not a quadratic residue mod $p$}\\
    0 & \text{$p \m a$}
  \end{cases}$$
\end{ndefi}

\begin{remark}
   By an earlier lemma, we see that $x^2 \c a \mod p$ has precisely $\gls + 1$ distinct solutions mod $p$
\end{remark}

\begin{remark}
   We always have $\ls 1 p = 1$. Moveover, if $a, b \in \Z$ such that $a \c b \mod p$. Then, $\ls a p \c \ls b p$. This is sometimes known as periodicity.
\end{remark}

\begin{eg}
  If $m \in \Z$ with $p \nmid m$, then $\ls {m^2} p = 1$.
\end{eg}

\subsection{Eulers Criterion}

\begin{nlemma}
  Let $p$ be an odd prime and let $g$ be a primitive root mod $p$. Let $a \in \Z$ with $p \nmid a$. Then $a \c g^k \mod p$ for some $k \in \Z$ and $a$ is a quadratic residue mod $p$ if and only if $k$ is even.
\end{nlemma}
\begin{proof}
  First note that a primitive root $g$ mod $p$ exists by an earlier Corollary, so $a \c g^k \mod p$ for some $k \in \Z$. Suppose $k \in \Z$ is even. Then $k = 2j$ and so $a \c (g^j)^2 \mod p$. Thus $a$is a quadratic residue mod $p$. Suppose conversely $a$ is a quadratic resdiue mod $p$. Then $a \c b^2 \mod p$ for some $b \in \Z$ and $p \nmid b$.
  Then $b \c g^r$ for some $r \in \Z$ and so $g^k \c (g^r)^2 \c g^{2r} \mod p$. By an earlier proposition, we can say $k \c 2r \mod p-1$ by an earlier proposition since $\ord_p(g) = \phi(p) = p-1$. So $k \c 2r \mod 2$ since $2 \c (p-1)$. Hence $k \c 0 \mod 2$ and is even.
\end{proof}

\begin{nthm}[Eulers Criterion]
  If $p$ is an odd prime and $a \in \Z$ then
  $$ \gls \c a^{\frac{p-1}{2}}\mod p $$
\end{nthm}
\begin{proof}
  This is obvious if $p \m a$. So suppose $p \nmid a$. Let $g$ be a primitive root mod $p$. Then there exists some $k \in \Z$ such that $a \c g^k \mod p$. Since $\ord_p(g) = p - 1$ we have $g^{p-1} \c 1 \mod p$ and $g^{\frac{p-1}{2}}\not\c 1 \mod p$. Since $p$ is an odd prime we have, $g^{\frac{p-1}{2}}\c \pm 1 \mod p$.
  Therefore, $g^{\frac{p-1}{2}} \c -1 \mod p$. Then,
  $$ a^{\frac{p-1}{2}} = \left(g^k\right)^{\frac{p-1}{2}} \c \left(g^{\frac{p-1}{2}} \right)^k \c (-1)^k \mod p $$
  The result now follows from the previous lemma.
\end{proof}

Now for an alternative proof,
\begin{proof}[alternative proof]
  Again, we may suppose $p \nmid a$. Suppose that $\gls = 1$. Then $\ex b \in \Z$ with $p \nmid b$ such that $a \c b^2 \mod p$. Thus by FLT we have,
  $$ a^{\frac{p-1}{2}} \c \left(b^2\right)^{\frac{p-1}{2}} \c b^{p-1} \c 1 \c \gls \mod p $$
  Now suppose that $\gls = -1$ and consider the polynomial
  $$ f(x) = x^{\frac{p-1}{2}}-1 $$
  since $f$ has degree $\frac{p-1}{2}$, hence by Lagranges Polynomial Congruence Theorem,
  $$ f(x) \c 0 \mod p $$
  has $\frac{p-1}{2}$ solutions. But we have shown by that the quadratic residues mod $p$ are solutions and there are $\frac{p-1}{2}$ of them. Hence, none of the quadratic non-residues are solutions and so $a^{\frac{p-1}{2}} \not\c 1 \mod p$. But by FLT we have $a^{p-1} \c 1 \mod p$ and we can say that $a^{\frac{p-1}{2}} \c \pm 1 \mod p$. Therefore,
  $$ a^{\frac{p-1}{2}} \c -1 \c \gls \mod p $$
  This completes the proof.
\end{proof}

\begin{nthm}[Multiplicity of Legendre's Symbol]
  Let $p$ be an odd prime and $a, b \in \Z$. Then $\ls{ab} p = \ls a p \ls b p$
\end{nthm}
\begin{proof}
  If $p \m a$ or $p \m b$, then $p \m ab$ so $\ls{ab} p = 0$ and so either $\ls a p = 0$ and $\ls b p = 0$, the result is proved.\\
  No suppose $p \nm a$ and $p \nm b$. Then by Euclids lemma for primes we have $p \nm ab$. Moreover, Eulers Criterion tells us,
  \begin{align*}
    \ls{ab} p &= (ab)^{\frac{p-1}{2}} \\
    &= a^{\frac{p-1}{2}}b^{\frac{p-1}{2}} \\
    &= \ls a p \ls b p \mod p
  \end{align*}
  and both sides are $\pm 1$. If they were different, we could have $1 \c -1 \mod p$ which means $p \m 2$ - Contradiction. $p$ is odd.
\end{proof}

Now for another theorem,
\begin{nthm}
  If $p$ is an odd prime then,
  $$ \ls{-1} p = (-1)^{\frac{p-1}{2}} = \begin{cases}
    1 & p \c 1 \mod 4 \\
    -1 & p \c 3 \mod 4
  \end{cases} $$
  In other words, $x^2 \c -1 \mod p$ is soluble if and only if $p \c 1 \mod 4$.
\end{nthm}
\begin{proof}
  By Eulers Criterion,
  $$ \ls{-1} p \c (-1)^{\frac{p-1}{2}} \mod p $$
  and both sides are $\pm 1$. Then as above if they were different, $p \c 2$ and so they are the same as $p$ is an odd prime.
\end{proof}

\begin{nthm}
  There are infinitely many primes $p$ with $p \c 1 \mod 4$.
\end{nthm}
\begin{proof}
  If suffices to prove the for any $N \in \N$ there exists a prime $p$ with $p > N$ and $p \c 1 \mod 4$. Let $M = (2(N!))^2 + 1$. If $p$ is a prime with $p \le N$ then $M \c 1 \mod p$ so $p \nm M$. Let $p$ be a prime factor of $M$, then $p > N$. As $M$ is odd, $p$ is also odd. Then we have $(2(N!))^2 \c -1 \mod p$ and so the congruence $x^2 \c -1 \mod p$ is soluble and so $p \c 1 \mod 4$ by the previous theorem.
\end{proof}

\subsection{Gauss' Lemma}
We make the defintion,
\begin{ndefi}[]
  Let $a \in \Z$ and $n \in \N$. We write $\l(a, n)$ for the unique integer such that $a \c \l(a, n)\mod n$ and $0 \le \l(a, n) < n$, ie. $\l(a, n)$ is the remainder of the division algorithm applied to $a$ and $n$.
\end{ndefi}

We note this isn't standard notation, but it is useful. Now we move onto Gauss' Lemma,
\begin{nthm}[Gauss' Lemma]
  Let $p$ be an odd prime and let $a \in \Z$ with $p \nm a$. Then,
  $$ \ls a p = (-1)^\La \qquad \La = \#\{ j \in \N : 1 \le j \le \frac{p-1}{2},\, \l(aj, p) > \frac{p}{2}\}$$
\end{nthm}

\begin{eg}
  Let $p = 13$ and $a = 5$.\\
  If $j = 1$, then $\l(5, 13) = 5 < \frac{13}{2}$\\
  If $j = 2$, then $\l(10, 13) = 10 < \frac{13}{2}$... and so on. We find that $\La = \#\{2, 4, 5\} = 3$ and so $\ls 5 13 = -1$.
\end{eg}

\begin{proof}
  Let $S_a = \{aj : 1 \le j \le \frac{p-1}{2}\}$ and we define,
  $$ \{r_i\}_{i=1}^m = \{\l(aj, p) : aj \in S_a,\, 0 \le \l(aj, p) < \frac{p}{2}\} $$
  $$ \{s_i\}_{i=1}^n = \{\l(aj, p) : aj \in S_a,\, \frac{p}{2} \le \l(aj, p) < {p}\} $$
  so that $n = \La$. Note that $\l(aj, p) \ne \frac{p}{2}$ since $\frac{p}{2} \notin \Z$ and $\l(aj, n) \ne 0$ since $p \nmid a$ and $p \nmid j$. Also note that $j_1 \ne j_2$ then $\l(aj_1, p) \ne \l(aj_2, p)$ since,
  \begin{align*}
    \l(aj_1, p) = \l(aj_2, p) \implies aj_1 &\c aj_2 \mod p \\
    \implies j_1 &\c j_2 \mod p\\
    \implies j_1 &= j_2 && \text{since $0 < j_1, j_2 < p$.}
  \end{align*}
  Hence, $m + n = \#S_a = \frac{p-1}{2}$ as we proved that there isn't anything at the end points and distinct $j$'s return distinct values. We claim that,
  $$ \{r_1, \dots, r_m, (p - s_1), \dots, (p - s_n)\} = \{1, 2, \dots, \frac{p-1}{2}\} $$
  It is clear that $r_i, (p - s_i) \in \{1,\,2\,\dots, \frac{p-1}{2}\}$ and there are $\frac{p-1}{2}$ elements so it suffices to prove that they are all different. We have already show that $r_i \ne r_j$ and $s_i \ne s_j$ for $i \ne j$. To show that $r_i \ne p - s_j$ we argue by contradiction. If $r_i + s_i = p$, let $r_i = \l(aj_1, p)$ and $s_i = \l(aj_2, p)$. Then,
  \begin{align*}
    r_i + s_j &= p \\
    &= l(aj_1, p) + \l(aj_2, p) \\
    &\c aj_1 + aj_2 \mod p\\
    &\c a(j_1 + j_2) \mod p
  \end{align*}
  Hence, $a(j_1 + j_2) \c 0 \mod p$. So by Euclids Lemma for primes, either $p \m a$ or $p \m j_1 + j_2$. However, $p \nm a$ and $2 \le j_1 + j_2 \le p -1$ so then $p \nm j_1 + j_2$ - Contradiction. Therefore $r_i \ne p - s_j$. \\
  Now on the one hand we have,
  \begin{align*}
    r_1r_2 \dots r_m(p-s_1)(p-s_2) \dots (p - s_n) &= 1 \ti 2 \ti \dots \ti \frac{p-1}{2} = \left(\frac{p-1}{2}\right)!\\
    &= r_1r_2\dots r_m s_1s_2 \dots s_n (-1)^n \mod p
  \end{align*}
  On the other hand, by the definition of $r_i$ and $s_j$,
  \begin{align*}
    r_1r_2 \dots r_ms_1s_2\dots s_n &= \prod_{j=1}^{\frac{p-1}{2}} \l(aj, p) \\
    &= \prod_{j=1}^{\frac{p-1}{2}} (aj) \\
    &\c a^{\frac{p-1}{2}}\left(\frac{p-1}{2}\right)! \mod p
  \end{align*}
  Therefore,
  $$ \left(\frac{p-1}{2}\right)! \c (-1)^n a^{\frac{p-1}{2}}\left(\frac{p-1}{2}\right)! \mod p  $$
  Now since $p \m \left(\frac{p-1}{2}\right)!$, the cancellation law for congruences shows that,
  $$ 1 \c (-1)^n a^{\frac{p-1}{2}} \mod p  $$
  Now we rearrange and get $a^{\frac{p-1}{2}} \c (-1)^n \mod p$ and so $\ls a p \c (-1)^n \mod p$ by Euler's Criterion. Then both sides are $\pm 1$, if they were different, then we get that $2 \m p$ - Contradiction. Therefore $\ls a p = (-1)^\La$ as required.
\end{proof}

\begin{ndefi}[Floor Function]
  For any $x \in \R$ we set $\floor{x} := \max\{n \in \Z \}$
\end{ndefi}

\begin{ncor}
 If $p$ is an odd prime, then,
 $$ \ls 2 p = (-1)^{\frac{p^2-1}{8}} = \begin{cases}
   1 & p \c \pm 1 \mod 8\\
   -1 & p \c \pm 3 \mod 8
 \end{cases} $$
\end{ncor}
\begin{proof}
  Apply Gauss' Lemma for $a=2$,
  $$ \ls 2 p = (-1)^\La $$
  where $\La = \#\{1 \le j \le \frac{p-1}{2} : \l(2j, p) > \frac{p}{2}\}$. Note that for $1 \le j \le \frac{p-1}{2}$ we have $2 \le 2j \le p -1$ and so $\l(2j, p) = 2j$. Moreover, $2j < \frac{p}{2}$ if and only if $j < \frac{p}{4}$ and $\frac{p}{2} < 2j < p$ if and only if $\frac{p}{4} < j < \frac{p}{2}$.
  It follows that, $\La = \#\{j \in \N : \frac{p}{4} < j < \frac{p}{2}\}$. We can calculate this,
  \begin{align*}
    \#\{j \in \N : \frac{p}{4} < j < \frac{p}{2}\} &= \#\{j \le \frac{p-1}{2}\} - \#\{j < \frac{p}{4}\} \\
    &= \frac{p-1}{2} - \floor{\frac{p}{4}}
  \end{align*}
  Since $p$ is odd, then one of the following must occur,
  \begin{enumerate}
    \item $p = 8k+1 \implies \frac{p-1}{2} = 4k, \floor{\frac{p}{4}} = 2k \implies \La = 2k$.
    \item $p = 8k+3 \implies \frac{p-1}{1} = 4k+1, \floor{\frac{p}{4}} = 2k \implies \La = 2k+1$.
    \item $p = 8k+5 \implies \frac{p-1}{1} = 4k+2, \floor{\frac{p}{4}} = 2k+1 \implies \La = 2k+1$.
    \item $p = 8k+7 \implies \frac{p-1}{1} = 4k+3, \floor{\frac{p}{4}} = 2k+1 \implies \La = 2k+2$
  \end{enumerate}
  Hence, $(-1)^\La = +1 \iff p = 8k+1 \text{ or } p = 8k + 7$
  We note that if $p = 8k + r$, then,
  $$ \frac{p^2 - 1}{8} = \frac{r^2 + 16rk + 64k^2 - 1}{8} = \frac{r^2 - 1}{8} + 2(kr + 4k^2) \c \frac{p^2 - 1}{8} \mod 2$$
  By checking cases of $r = \pm 1, \pm 3$ and see,
  $$ \frac{p^2 - 1}{8} = \begin{cases}
    0 \mod 2 & p \c \pm 1 \mod 8 \\
    1 \mod 2 & p \c \pm 3 \mod 8
  \end{cases} $$
  and the result follows.
\end{proof}

We now prove that there are more infinitely many primes!
\begin{nthm}
  There are infinitely many primes $p$ with $p \c -1 \mod 8$
\end{nthm}
\begin{proof}
  It suffices to prove that for any $N \in \N$ there exists a prime $p$ with $p > N$ and $p \c -1 \mod 8$. Let $M = 8(N!)^2 - 1$. If $p$ is a prime with $p \le N$ then $M \c -1 \mod p$ and so $p \nm M$. Let $p$ be a prime factor of $M$, then $p$ is odd and $p > N$. Moreover,
  $$ (4(N!))^2 \c 16(N!)^2 \c 2M + 2 \c 2 \mod p $$
  Thus, $\ls 2 p = 1$ and so $p \c \pm 1\mod 8$ by the Corollary above. But if all the prime factors of $M$ were congruent to $1 \mod 8$, then $M \c 1 \mod 8$, which is not the case. Therefore, $M$ must have at least one prime factor $p$ with $p \c -1 \mod 8$ and $p > N$.
\end{proof}

\begin{nlemma}
  Let $p$ be an odd primes and $a \in \Z$ and $p \nm a$. Then,
  $$ \ls a p = (-1)^t  \qquad t = \sum_{k=1}^\frac{p-1}{2} \floor{\frac{ak}{p}}$$
\end{nlemma}
\begin{proof}
  Recall the notation from Gauss' Lemma. Here, $\l(aj, p) = aj - pk$ for some $k \in \Z$ such that $0 \le aj - pk < p$. It follows that $k \le \frac{aj}{p} < k+1$ and hence $k = \floor{\frac{aj}{k}}$. We therefore deduce that $\l(aj, p) = aj - p \floor{\frac{aj}{k}}$. Now we recall $r_i$ and $s_i$. Then,
  \begin{align*}
    \sum_{i=1}^m r_i + \sum_{i=1} s_i &= \sum_{j=1}^{\frac{p-1}{2}} \l(aj, p)\\
    &= \sum_{j=1}^{\frac{p-1}{2}} \left( aj - p\floor{\frac{aj}{p}} \right)
  \end{align*}
  Since $a$ and $p$ are odd, then,
  \begin{equation}
    \sum_{j=1}^{\frac{p-1}{2}} j - \sum_{j=1}^{\frac{p-1}{2}}\floor{\frac{aj}{p}} \c \sum_{i=1}^m r_i + \sum_{i=1} s_i \mod 2 \tag{$*$}
  \end{equation}
  Now we recall from Gauss' Lemma that,
  $$ \{r_1, \dots, r_m, (p - s_1), \dots, (p - s_n)\} = \{1, 2, \dots, \frac{p-1}{2}\} $$
  Thus,
  $$ \sum_{i=1}^m r_i + np + \sum_{i=1}^n s_i \c \sum_{j=1}^{\frac{p-1}{2}} j \mod 2 $$
  and so we now can say,
  $$ \sum_{i=1}^m r_i + \sum_{i=1}^n s_i \c n +  \sum_{j=1}^{\frac{p-1}{2}} j \mod 2 $$
  Then comparing with $(*)$,
  $$ n \c \sum_{j=1}^{\frac{p-1}{2}}\floor{\frac{aj}{p}} \mod 2 $$
  and so the result follows from Gauss' Lemma.
\end{proof}
