% !TEX root = ../notes.tex

\section{Primes and Congurences}

We start by defining primes and composite numbers,
\begin{ndefi}[Prime]
  A number $p \in \N_1$ with $p > 1$ is prime if and only if it's only divisors are $1$ and $p$, i.e.
  $$ n \m p \implies n = 1 \text{ or } n = p $$
\end{ndefi}
\begin{ndefi}[Composite Numbers]
  A number $n \in \N_1$ with $n > 1$ is composite if and only if it is not prime, i.e.
  $$ n = ab \qquad 1 < a, b \in \N  $$
\end{ndefi}
One is neither composite nor prime.

\begin{nprop}
  If $n \in \N_1$ with $n > 1$, then $n$ has a prime factor.
\end{nprop}
\begin{proof}
  Use strong induction, so assume for $1 < m < n$ where $m \in \N_1$ that $m$ has a prime factor. \\
  Case (i): If $n$ is prime, then $n$ is a prime factor of $n$.\\
  Case (ii): If $n$ is composite, then $n = ab$ where $a, b > 1$ and so, $1 < a < n$. By the induction hypothesis, there is a prime $p$ such that $p \m a$. Hence, $p \m a$ and $a \m n$ so, by transitivity $p \m n$.
\end{proof}

\begin{nprop}
   If $1 < n \in \N_1$, then we can write $n = p_1p_2\dots p_k$ where $k \in \N_1$ and $p_i$ are primes.
\end{nprop}
\begin{proof}
  If $n$ is prime, then the result is clear. So suppose that $n$ is composite. Then $n$ must have a prime factor, so $n = p_1n_1$ where $1 < n_1 \in \N_1$. If $n_1$ is prime, we are done. If $n_1$ is composite, then we can write $n_1 = p_2n_2$ and so on... This process terminates as $n > n_1 > n_2 > \dots > 1$. Hence after at least $n$ steps we obtain a prime factorisation of $n$.
\end{proof}

\begin{eg}
  $$ 666 = 3 \times 222 = 3 \times 2 \times 111 = 3 \times 2 \times 3 \times 37 $$
\end{eg}

\begin{nthm}
  There are infinitely many primes
\end{nthm}

\begin{proof}[Euclid's Proof]
  For a contradiction, assume there are finitely many primes, $\{p_1, p_2, p_3, \dots, p_n\}$ and that is a complete list. Consider $N := p_1p_2\dots p_n + 1 \in \N$. Then $N > 1$ so by the first proposition, $N$ has a prime factor $p$. However, every prime is one of the elements of the list, so $p = p_i$. Hence, $p_i \m (p_1p_2\dots p_n)$ so $p \m (N - 1)$. However, $p \m N$ and we can write $1 = N - (N - 1)$, so $p \m 1$, which is a contradiction.
\end{proof}

\subsection{Fundemental Theorem of Arithmetic}

\begin{nlemma}
  Let $n \in Z$, then if $p \nm n$ then $\gcd(p,n)=1$
\end{nlemma}
\begin{proof}
  Let $d = \gcd(p, n)$. Then $d \m p$ so by definition of prime either $d = 1$ or $d = p$. But $d \m n$ so $d \ne p$ because $p \nm n$. Hence, $d = 1$.
\end{proof}

\begin{nthm}[Euclid's Lemma for Primes]
  Let $a, b \in \Z$ and $p$ be a prime. If $p \m ab$, then $p \m a$ or $p \m b$.
\end{nthm}
\begin{proof}
  Assume $p \m ab$ and that $p \nm a$. We shall prove $p \m b$. By Lemma, $\gcd(p, a) = 1$, so by Euclid's lemma, $p \m b$.
\end{proof}

\begin{remark}
   Euclid's Lemma for primes immediately generalises to several factors.
\end{remark}

\begin{ndefi}[]
  Let $n\in \N_1$ and $p$ be a prime. Then,
  $$ v_p (n) := \max \{k \in \N \cup \{0\} : p^k \m n\}  $$
  In other words, $k$ is the unique non-negative integer such that $p^k \m n$ but $p^{k+1} \m n$. Equivalently, $v_p (n) = k$ if and only if $n = p^k n'$ where $n' \in \N$ and $p \nm n'$.
\end{ndefi}

\begin{eg}
  We can see that,
  \begin{itemize}
    \item $v_2(720) = 4$ as $2^4 \m 720$ but $2^5 \nm 720$
    \item $v_3 (720) = 2$ as $3^2 \m 720$ but $3^3 \nm 720$
    \item $v_5 (720) = 1$ as $5^1 \m 720$ but $5^2 \nm 720$
    \item if $p \ge 7$, then $v_p (720) = 0$ as $p \nm 720$.
  \end{itemize}
\end{eg}

\begin{nlemma}
  Let $n,m \in \N_1$ and $p$ be a prime. Then $v_p(mn) = v_p(m) + v_p(n)$
\end{nlemma}
\begin{proof}
  Let $k = v_p(m)$ and $\ell = v_p(n)$. Then we write $m = p^km'$ where $p\nm m'$ and $n = p^\ell n'$ where $p \nm n'$. Then $nm = p^{k + \ell}m'n'$ and so by Euclid's lemma $p \nm m'n'$ as if it did then $p \m n'$ or $p \m m'$ but it doesn't. So $v_p (mn) = v_p(m) + v_p(n)$.
\end{proof}

\begin{nthm}[Fundamental Theorem of Arithmetic]
  Let $1 < n \in \N_1$. Then,
  \begin{enumerate}
    \item (Existence) The number $n$ can be written as a product of primes.
    \item (Uniqueness) Suppose that,
    $$ n = p_1 \dots p_r = q_1 \dots q_s $$
    where each $p_i$ and $q_j$ are prime. Assume further that,
    $$ p_1 \le p_2 \le \dots \le p_r \qquad \text{ and } \qquad q_1 \le q_2 \le \dots \le q_s $$
    Then $r = s$ and $p_i = q_i$ for all $i$
  \end{enumerate}
\end{nthm}

\begin{remark}
  If $1$ is a prime, then the Uniqueness here is broken, as,
  $$ 6 = 3 \times 2 = 3 \times 2 \times 1 = \dots $$
\end{remark}

\begin{remark}
   A consequence of the FTA is that the integral domain $\Z$ is in fact a UFD.
\end{remark}

\begin{proof}
  The existence is something we have done before. The harder part is uniqueness. Let $\ell$ be any prime. Then we have,
  \begin{align*}
    v_ell (n) &= v_\ell (p_1\dots p_r)\\
    &= v_\ell (p_1) + \dots + v_\ell (p_r)
  \end{align*}
  However,
  $$ v_{\ell} (p_i) = \begin{cases}
    1 & \text{if $\ell = p_i$}\\
    0 & \text{if $\ell \ne p_i$}
  \end{cases} $$
  Therefore,
  \begin{align*}
    v_\ell (n) &= \# \text{ of $i$ for which $\ell = p_i$}\\
    &= \# \text{ of times $\ell$ appears in the factorisation $n = p_1\dots p_r$}
  \end{align*}
  Similarly,
  $$ v_\ell (n) = \# \text{ of times $\ell$ appears in the factorisation $n = q_1\dots q_s$} $$
  Thus every prime $\ell$ appears the same number of times in each factorisation, giving the desired result.
\end{proof}


\begin{remark}
  Another way of interpreting this result is to say that for $n \in \N_1$,
  $$ n = p_1^{v_{p_1}(n)}p_2^{v_{p_2}(n)}\dots p_r^{v_{p_r}(n)} $$
  where $p_1, \dots, p_r$ are the distinct prime factors of $n$. Note that we take the empty product to be $1$, which covers the case for $n = 1$.
\end{remark}

\begin{nlemma}
  Let $n = \prod_{i=1}^r p_i^{a_i}$ where each $a_i \in \N_0$ and the $p_i$'s are distinct primes. The set of positive divisors of $n$ is the set of numbers of the form $\prod_{i=1}^r p_i^{c_i}$ where $0 \le c_i \le a_i$ for $i = 1, \dots, r$.
\end{nlemma}
\begin{proof}
  Exercise
\end{proof}

\subsection{Congruences}


\begin{ndefi}[]
  Suppose $a, b \in \Z$ and $n \in \N_1$. We write $a \equiv b \mod n$, and say `$a$ is congruent to $b$ mod $n$', if and only if $n \m (a - b)$. If $n \nm (a - b)$ we say that $a$ and $b$ are incongruent mod $n$.
\end{ndefi}

\begin{remark}
   In particular, $a \equiv 0 \mod n$ if and only if $m \m a$
\end{remark}

\begin{eg} Here are some examples:
  \begin{itemize}
    \item $4 \equiv 30\mod 13$ since $13 \m (4 - 30) = -26$
    \item $17 \not\equiv -17\mod 4$ since $17- (-17) = 34$ but $4 \nm 34$.
    \item $n$ is even if and only if $n \equiv 0\mod 2$
    \item $n$ is odd if and only if $n \c 1\mod 2$
    \item $a \c b \mod 1$ for all $a, b \in \Z$
  \end{itemize}
\end{eg}

\begin{nprop}
  Let $n \in \N_1$ being congruent mod $n$ is an equivalence relation, so,
  \begin{enumerate}
    \item Reflexive: $\fa a\in\Z, a \c a \mod n$
    \item Symmetric: $\fa a, b \in \Z$, $a \c b \mod n \implies b \c a \mod n$
    \item Transitive: $\fa a, b \in \Z$, $a \c b \mod n$ and $b \c c \mod n \implies a \c c \mod n$.
  \end{enumerate}
\end{nprop}
\begin{proof}
  The proof follows from,
  \begin{enumerate}
    \item $n \m 0$.
    \item If $n \m (a - b)$ then $n \m (b - a)$
    \item If $n \m (a - b) + (b - c) = (a - c)$
  \end{enumerate}
\end{proof}

\begin{nprop}
  Congruences respect addition, subtraction and multiplication. Then let $a, b, \a, \b \in \Z$. Suppose that $a \c \a \mod n$ and $b \c \b \mod n$. Then,
  \begin{enumerate}
    \item $a + b \c \a + \b \mod n$
    \item $a - b \c \a - \b \mod n$
    \item $ab \c \a\b \mod n$
  \end{enumerate}
  Moreover, if $f(x) \in \Z[x]$ then $f(a) \c f(\a)\mod n$
\end{nprop}
\begin{proof}
  Check that $ab \c \a\b \mod n$. Since, $a \c \a \mod n$ and so, $n \m (a - \a)$ and so $a = \a + ns$ for some $s \in \Z$, Similarly $b = \b + nt$. Hence,
  $$ ab = (\a + ns)(\b + nt) = \a\b + n(s\b + t\a + nst)$$
  and so $n \m (ab - \a\b)$. Therefore, $ab \c \a\b \mod n$, as required.
\end{proof}

\begin{eg}
  Let $n \in \N_1$ and write $n$ in decimal notation,
  $$ n = \sum_{i=0}^k a_i \times 10^i \qquad 0 \le a_i \le 9$$
  Then, define $f(x)$ by,
  $$ f(x) = \sum_{i=0}^k a_i x^i $$
  Then, since $10 \c -1\mod 11$, we see that $n = f(10) \c f(-1)\mod 11$, whence,
  $$ 11 \m n \iff 11 \m f(-1) \iff 11 \m (a_0 - a_1 + a_2 - a_3 + \dots + (-1)^k a_k) $$
  This is an easy way to test for divisibility by $11$.
\end{eg}

\begin{eg}
  Does $x^2 - 3y^2 = 2$ have a solution with $x, y \in \Z$. Let $x, y \in \Z$. Note that $x^2 - 3y^2 \c x^2 \mod 3$. Now, $x \c 0, 1, 2 \mod 3$, so $x^2 \c 0, 1, 4\mod 3 \c 0, 1\mod 3$. Hence, $x^2 - 3y^2 \c x^2 \not\c 2 \mod 3$ and so $x^2 - 3y^2 \ne 2$.
\end{eg}

\begin{remark}
  Suppose we have $f \in \Z[x_1, \dots, x_m]$ if we have $a_1, \dots, a_m \in \Z$ such that $f(a_1, \dots, a_m) = 0$ then $f(a_1, \dots, a_m) \c 0 \mod n$ for every $n \in \N$. Therefore if there exist an $n \in \N_1$ such that $f(x_1, \dots, x_m) \c 0 \mod n$ has no solution, there cannot exist $a_1, \dots, a_m \in \Z$ such that $f(a_1, \dots, a_n) = 0$.
\end{remark}

We are going to prove the following theorem,
\begin{nthm}
  There are infinitely many primes $p$ with $p \c 3\mod 4$
\end{nthm}
\begin{proof}
  Suppose that $p$ is a prime. Then $p \c 0, 1, 2, 3 \mod 4$, but $p \not\c 0 \mod 4$ because $4 \nm p$. If $p \c 2 \mod 4$ then $p = 4k + 2$ for some $k \in \Z$, so $2 \m p$ so in fact $p = 2$. Therefore there are three types of primes,
  \begin{enumerate}
    \item $p = 2$
    \item $p \c 1 \mod 4$
    \item $p \c 3 \mod 4$
  \end{enumerate}
  Let $N \in \N$ it suffices to show that there exist a type $(iii)$ prime with $p > N$. Let $4(N!) - 1$ and so $M \ge 3$ and so by the existence of FTA we can write $M = p_1\dots p_k$. If $p \le N$, then $M \c -1 \mod p$ so $p \nm M$. Hence, $p_j > N$ for all $j$. Moreover $p_j \ne 2$ for all $j$ because $M$ is odd. Therefore for each $j$ we have $p_j \c 1, 3 \mod 4$. If $p_j \c 3\mod 4$ for any $j$ then we are done. If this is not the case, then $p_j \c 1 \mod 4$ for all $j$, and so, $M \c 1 \times 1 \times \dots \times 1 \mod 4 \c 1 \mod 4$; but by definition of $M$ we have $M \c -1 \c 3 \mod 4$ - contradiction!
\end{proof}

\begin{remark}
   Congruences do not respect division, $4 \c 14 \mod 10$ but $2 \not\c 7 \mod 10$
\end{remark}

\begin{nprop}
  Let $a, b, s \in \Z$ and $d, n \in \N_1$.
  \begin{enumerate}
    \item If $a \m b \mod n$ and $d \m n$ them $a \m b \mod d$
    \item Suppose $s \ne 0$. Then $a \c b \mod n$ if and only if $as \c bs \mod ns$
  \end{enumerate}
\end{nprop}
\begin{proof}
  (i) follows from transitivity of divisibility;\\
  (ii) follows from multiplication and cancellation properties.
\end{proof}

\begin{nthm}[Cancellation law for Congruences]
  Let $a, b, c \in \Z$ and $n \in \N_1$. Let $d = \gcd (c, n)$. Then $ac \m bc \mod n \iff a \c b \mod \frac{n}{d}$. In particular, if $n$ and $c$ are coprime, then $ac \c bc \mod n \iff a \c b \mod n$.
\end{nthm}
\begin{proof}
  Since, $d = \gcd(c, n)$, we may write $n = dn'$ and $c = dc'$ where $n',c' \in \Z$. Suppose $ac \c bc \mod n$. Then $n \m c(a - b)$ and so $n' \m c'(a - b)$. However, $\gcd(n', c') = 1$ and so $n' \m (a - b)$ by Euclid's Lemma. Thus, $a \c b \mod n'$.\\
  Suppose conversely $a \c b \mod n'$ and so, $n' \m (a - b)$ and so $n \m d(a - b)$. But $d \m c$ and so $d(a - b) \m c(a - b)$ and thus $n \m c(a - b)$ by the transitivity of divisibility. Thus $ac \c bc \mod n$.
\end{proof}

\begin{nprop}
  Let $a, m, n \in \Z$. If $m$ and $n$ are coprime and if $m \m a$ and $n \m a$ then $nm \m a$.
\end{nprop}

\begin{proof}
  Since $m \m a$ we can write $a = mc$ for some $c \in \Z$. Now $n \m a = mc$ and $\gcd(m, n) = 1$ and so by Euclid's Lemma, $n \m c$. Hence, $mn \m mc = a$.
\end{proof}

\begin{ncor}
  Let $m, n \in \N$ be coprime and let $a, b \in \Z$. If $a \c b \mod m$ and $a \m b \mod n$ then $a \c b \mod mn$.
\end{ncor}
\begin{proof}
  We have $n \m (a - b)$ and $m \m (a - b)$. Since $m$ and $n$ are coprime we therefore have $mn \m (a - b).$
\end{proof}
