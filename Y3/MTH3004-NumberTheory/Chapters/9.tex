% !TEX root = ../notes.tex

\section{Law of Quadratic Reciprocity}
Here is the statement,
\begin{nthm}[LQR]
  If $p$ and $q$ are distinct odd primes, then,
  \begin{align*}
    \ls p q &= \ls q p (-1)^{\frac{p-1}{2}\frac{q-1}{2}}\\
    &= \begin{cases}
      \ls p q & \text{if } p \c 1 \mod 4 \text{ or } p \c 1 \mod 4\\
      -\ls p q & \text{if } p \c q \c 3 \mod 4
  \end{cases}
  \end{align*}
\end{nthm}

\begin{remark}
   Often forget that $p$ and $q$ are distinct odd primes.
\end{remark}

We will prove this later, but it is the most important theorem in the module.

\begin{eg}
  What is $\ls{29}{53}$? In other words, can we solve $x^2 \c 29 \mod 53$? Use LQR,
  \begin{align*}
    \ls{29}{53} &= \ls{53}{29} \\
    &= \ls{24}{29}\\
    &= \ls 2 {29} ^3 \ls 3 {29}
  \end{align*}
  Now we use LQR and the formula for $\ls 2 p$ repeatedly,
  \begin{align*}
    \ls{2}{29} &= -1 && (\text{since $29 \c -3 \mod 8$})\\
    \ls{3}{29} &= \ls{29}{3} \\
    &= \ls 2 3\\
    &= -1 && (\text{since $3 \c 3 \mod 8$})
  \end{align*}
  Thus, $\ls{29}{53} = (-1)^4 = 1$ and hence $x^2 \c 29 \mod 53$ is soluble.
\end{eg}
