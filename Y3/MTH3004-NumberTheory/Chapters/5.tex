% !TEX root = ../notes.tex


\section{Exponentiation}

\begin{nprop}
 Fix $n \in \N$ and $a \in \Z$. There exists some $r \in \N$ such that $a^r \c 1 \mod n$ if and only if $\gcd(a, n) = 1$.
\end{nprop}

\begin{proof}
  Suppose there exists $r \in \N$ such that $a^r \c 1 \mod n$. Then $a^{r-1}$ is a solution to $ax \c 1 \mod n$ and so $\gcd(a, n) = 1$ by the proposition on units of $\Z/n\Z$. Suppose conversely that $\gcd(a, n) = 1$ and so there are only finitely many possible values of $a^k \mod n$ so there exists $i, j \in \N$ with $i < j$ such that $a^i \c a^j\mod n$. Since $\gcd(a, n) = 1$ we may apply the cancellation law for congruences $i$ times obtain $a^{j-i}\c 1\mod n$/ Thus take $r = j-i$.
\end{proof}

\begin{ndefi}[Order]
  Let $n \in \N$ and $a \in \Z$ and suppose $\gcd(a, n) = 1$. Then the least $d \in \N$ such that $a^d \c 1 \mod n$ is called the order of $a\mod n$ and is written $\ord_n(a)$
\end{ndefi}

\begin{nprop}
   Let $n \in \N$ and $a\in \Z$. Suppose that $\gcd(a, n) = 1$. For $r, s \in \Z$ we have $a^r \c a^s \mod n$ if and only if $r \c s \mod \ord_n(a)$
\end{nprop}
\begin{proof}
  Let $k =\ord_n(a)$. Then $a^k \c 1 \mod n$. Now assume wlog $r > s$. Suppose $r \c s \mod k$, then there exists some $t \in \N$ such that $r = s + tk$. Hence,
  $$ a^r \c a^{s + tk} \c a^s(a^k)^t \c a^s \mod n $$
  Suppose conversely that $a^r \c a^s \mod n$. Since $\gcd(a, n)= 1$ we may apply the cancellation law $s$ times to obtain $a^{r - s}\c 1 \mod n$. By the division algorithm, there exist $u, t \in \N_0$ such that $r - s = tk + u$ where $0 \le u < k$.\\
  $$ a^{r-s} \c a^{u+tk} \c a^u(a^k)^t \c a^u\mod n $$
  and so $a^u\c 1 \mod n$. However, $0 \le u  < k$ and $k$ is the least positive integer such this is true. Hence $u = 0$. Therfore, $k \m (r - s)$, ie. $r \c s \mod k$.
\end{proof}

\begin{ncor}
   Let $n \in N$ and $a \in \Z$ and suppose that $\gcd(a, n) = 1$. Then $a^k \c 1 \mod n$ if and only if $\ord_n(a) \m k$.
\end{ncor}
\begin{proof}
  Just take $r = k$ and $s = 0$ in the above proposition.
\end{proof}

\begin{ncor}
   Let $n \in \N$ and $a \in \Z$ and suppose $\gcd(a, n) = 1$. Then the numbers $\{1, a, a^2,\dots, a^{\ord_n(k) - 1}\}$ are all incongruent $\mod n$ .
\end{ncor}
\begin{proof}
  Combine the above proposition with the proposition that says if $c, d \in \Z$ with $c \c d \mod n$ and $|c - d| < n$ then $c = d$.
\end{proof}

\subsection{Reduced Residue Systems}

\begin{ndefi}[Reduced Residue System]
  Let $n \in \N$. A subset $R \sub \Z$ is said to be a reduced residue system $\mod n$ if
  \begin{itemize}
    \item $R$ contains $\varphi(n)$ elements
    \item no two elements of $R$ are congruent $\mod n$ and,
    \item $\fa r \in R, \gcd(r, n) = 1$
  \end{itemize}
\end{ndefi}
\begin{remark}
   If $R$ is a reduced residue system $\mod n$ then,
   $$ (\Z/n\Z)^\times = \{[a]_n : a \in R\} $$
\end{remark}

\begin{nprop}
   Let $n \in \N$ and $k \in \Z$. If $\{a_1, a_2, \dots, a_{\varphi(n)}\}$ is a reduced residue system $\mod n$ and $\gcd(k, n) = 1$ then $\{ka_1, ka_2, \dots, ka_{\varphi(n)}\}$ is also a reduced redidue system $\mod n$.
\end{nprop}
\begin{proof}
  If $ka_i \c ka_j\mod n$ then by the cancellation law for congruences $a_i \c a_j\mod n$ since $\gcd(k, n) =1$. Therefore, no two elements in $\{ka_1, ka_2, \dots, ka_{\varphi(n)}\}$ are congruent $\mod n$. Moreover, since $\gcd(a_i, n) = \gcd(k, n) = 1$ we have $\gcd(ka_i, n) = 1$ so each $ka_i$ is coprime to $n$
\end{proof}

\subsection{Euler- Fermat Theorem}
\begin{nthm}[Euler-Fermat]
  Let $n \in \N$, $a \in \Z$ and suppose $\gcd(a, n) = 1$. Then $a^{\phi(n)} \c 1\mod n$.
\end{nthm}
\begin{proof}
  Let $\{b_1, \dots, n_{\varphi(n)}\}$ be a reduced residue system $\mod n$. Then since $\gcd(a, n) = 1$, then $\{ab_1, ab_2, \dots, ab_{\varphi(n)}\}$ is also a reduced residue system by the proposition on reduced residue systems. Hence the product in the first is congruent to the product of the second. Therefore,
  $$ b_1b_2\dots b_{\varphi(n)} \c a^{\varphi(n)}b_1b_2\dots b_{\varphi(n)}\mod n $$
  then by the cancellation property and $\gcd(b_i, n)$ apply it repeatedly to get the required result.
\end{proof}

\begin{ncor}
   Let $n \in \N$ and $a \in \Z$ and suppose $\gcd(a, n) = 1$. Then $\ord_n(a) \m \varphi(n)$.
\end{ncor}
\begin{proof}
  Combine the Euler-Fermat Theorem and the earlier corollary that since $\gcd(a, n) = 1$, we have $a^k = 1\mod n$ if and only if $\ord_n(a) \m k$.
\end{proof}

\begin{eg}
  If we consider $\varphi(12) = 4$. So for every $a \in \Z$ with $\gcd(a, 12) = 1$ we must have $\ord_n(a) = 1, 2$ or $4$. In fact, we can notice that with the reduced residue systems $\{1, 5, 7, 11\}$ there isn't an element with order $4$, and hence no element of order $\varphi(12)$.
\end{eg}

\begin{ncor}
   Let $p$ be a prime and let $a \in \Z$  such that $p \nmid a$. Then $a^{p-1} \c 1\mod p$
\end{ncor}
\begin{proof}
  This follows immediately as $\varphi(p) = p - 1$.
\end{proof}

\begin{eg}
  We know that $\ord_{19}(3) = 18 = \varphi(19)$ and we know $\ord_{19}(8) = 6$ which is a factor of 18.
\end{eg}

\begin{nthm}[Fermat's Little Theorem]
  Let $p$ be a prime and let $a \in \Z$. Then $a^p \c a \mod p$.
\end{nthm}
\begin{proof}
  If $p \nmid a$, this follows from the earlier corollary. If $p \m a$, then $a^p$ and $a$ are congruent to $0$ $\mod p$.
\end{proof}

\begin{remark}
   Many of the results in this section can be thought of in terms of group theory once we realise that, $(\Z/n\Z)^\times$ is just a finite abelian group. For example, $\ord_n(a)$ is just the order of $[a]_n$ in $(\Z/n\Z)^\times$. Moreover, Lagranges Theorem tells us that the order of an element divides the order of the group; so $\ord_n(a) \m \varphi(n) = \#(\Z/ n\Z)^\times$ which hence gives Euler-Fermat Theorem.
\end{remark}

\subsection{Modular Exponentiation}
Let $b \in \Z$ and $e, m \in \N$. We want a way to compute $b^e \mod m$ efficiently. We can write $e$ in binary, ie. $e = \sum_{i=0}^k a_i2^i$ where $a_i \in \{0,\,1\}$ for $0 \le i \le k$. Then we observe,
$$ b^e = b^{\left( \sum_{i=0}^k a_i2^i \right)} = \prod_{i=0}^k \left( b^{2^i} \right)^{a_i} $$

Based on this we have the following algorithm,
\begin{algorithm}
  Let $b \in \Z$ and $e, m \in \N$. Set $x = 1$ ($x$ is the product). While $e > 0$ repeat,
  \begin{enumerate}
    \item If $e$ is odd, the replace $x$ by $bx$ and reduce this $\mod m$. (If $e$ is even $x$ is not altered).
    \item Replace $b$ by $b^2$ and reduce $\mod m$
    \item If $e$ is even replace $e$ by $\frac{e}{2}$, if $e$ is odd, then replace $e$ by $\frac{e - 1}{2}$. (Drop the units in the binary expansion and shift the digits one to the right)
  \end{enumerate}
  When this is completed $x \c b^e \mod m$.
\end{algorithm}
\newpage
\begin{eg}
  We want to compute $3^{499}\mod 997$. We set $b = 3$, $e = 499$, $m = 997$ and $x = 1$.
\begin{table}[!ht]
\centering
\begin{tabular}{c|c|c|c}
step & $x\mod m$ & $b\mod m$ & $e$   \\\hline
0    & 1         & 3       & 499 \\
1    & 3         & 9       & 249   \\
2   &  27        & 81       & 124 \\
3   &  27        & 579       & 62 \\
4   &  27        & 249       & 31 \\
5   &  741       & 187       & 15 \\
6   &  981        & 74       & 7 \\
7   &  810        & 491       & 3 \\
8   &  904        & 804       & 1 \\
9   &  3        & -       & 0 \\
\end{tabular}%
\end{table}
Hence we get $3^{499}\mod 997$. Note that we don't need to calculate $b$ in the last step. Moreover we get the binary expansion of $499$, which is $111110011$ (by going from bottom to top in $e$, ignoring the $0$, letting odd be $1$ and even $0$).\\
This minimises the number of multiplications, at one step we are just multiplying two integers modulo $m$, so they are small numbers.
\end{eg}

\subsection{Polynomial Congruence}
\begin{nthm}[Legranges Polynomial Congruence Theorem]
  Let
  $$ f(x) = a_0 + a_1x + \dots + a_nx^n \in \Z[x] $$
  and let $p$ be a prime such that $p\not\m a_d$. Then $f(x) \c 0\mod p$ has at most $d$ solutions $\mod p$.
\end{nthm}

\begin{remark}
   More generally, any polynomial equation of degree $d$ over a field has at most $d$ solutions (note that $\Z/p\Z = \F_p$ is a field).
\end{remark}

\begin{proof}
  The proof is by induction on $d$. When $d = 1$ we get that,
  $$ a_1x + a_2 \c 0 \mod p $$
  since $a_1 \not\c 0\mod p$, then $\gcd(a_1, p)= 1$ and so there is exactly one solution.\\

  Assume that the theorem is true for polynomials of degree $d - 1$ and suppose for a contradiction that $f(x) \c 0 \mod p$ has $d + 1$ incongruent solutions $\mod p$ say $x_0, x_1, \dots x_d$ where $f(x_k) \c 0 \mod p$. Recall we have for $r \in \N$,
  $$ x^r - y^r = (x - y)(x^{r-1} + x^{r-2}y + \dots + xy^{t-2} + y^{y-1}) $$
  Hence,
  $$ f(x) - f(x_0) = \sum_{r=1}^n a_r(x^r - x_0^r) = \sum_{r=1}^n a_r(x - x_0)g_r(x) $$
  where each $g_r \in \Z[x]$ is of degree $r - 1$ and has leading coefficient $1$. Hence, $f(x) - f(x_0) = (x- x_0)g(x)$. Thus,
  $$ f(x_k) - f(x_0) = (x_k - x_0)g(x_k) \c 0\mod p $$
  since $f(x_k) \c f(x_0) \c 0\mod p$. But $x_k - x_0 \not\c 0 \mod p$ if $k \ne 0$ so we must have $g(x_k) \c 0 \mod p$ for each $k \ne 0$ (by cancellation law for congruences). But this means $g(x) \c 0 \mod p$ has $d$ incongruent solutions $\mod p$ - contradiction! Hence desired result is proved.
\end{proof}

\begin{ncor}
   Let $a \in \Z$ and $p$ be an odd prime. If $a^2 \c 1 \mod p$, then $a \c \pm 1 \mod p$.
\end{ncor}
\begin{proof}
  Lagranges Polynomial Theorem says that $a^2 \c 1 \mod p$ has at most two solutions and these are $a\c \pm 1\mod p$ are solutions and these must be distinct because $p$ is odd. Therefore we have found all the solutions.
\end{proof}

\begin{eg}
  Let $p$ and $q$ be distinct odd primes. Consider the congruence,
  $$ x^2 \c 1 \mod pq $$
  It is clear that $x\c \pm 1 \mod pq$ are solutions, but are there any other solutions? By the CRT we have,
  \begin{align*}
    x^2 &\c 1 \mod pq\\
    &\iff x^2 \c 1 \mod p \text{ and } x^2 \c 1 \mod q\\
    &\iff x \c \pm 1\mod p \text{ and } x\c \pm 1 \mod q\\
  \end{align*}
  Thus there are four solutions $\mod pq$. Hence,
  $$ x\c 1\mod pq \iff \begin{cases}
    x \c 1 \mod p\\
    x \c 1 \mod q
  \end{cases} $$
  and
  $$ x\c -1\mod pq \iff \begin{cases}
    x \c -1 \mod p\\
    x \c -1 \mod q
  \end{cases} $$
  and so there remains two pairs of congruences,
  $$ \begin{cases}
    x \c 1 \mod p\\
    x \c -1 \mod q
  \end{cases} \qquad \text{ and } \qquad \begin{cases}
    x \c -1 \mod p\\
    x \c 1 \mod q
  \end{cases} $$
  Note that if $x$ is a solution to one of these, then $x$ is a solution of the other.
\end{eg}
