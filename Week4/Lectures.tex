\documentclass{article}


% Packages
\usepackage{fullpage}
\usepackage{amssymb}
\usepackage{multicol}
\usepackage{amsmath}
\usepackage{amsfonts}
\usepackage{bm}
\usepackage{float}
\usepackage{tikz}
\usepackage{xcolor}
\usetikzlibrary{shapes.geometric, positioning, arrows, intersections}
\tikzset{point/.style={circle,draw=black,inner sep=0pt,minimum size=3pt}}
\usepackage{amsthm}
\usepackage{tcolorbox}
\usepackage{hyperref}
\hypersetup{
    colorlinks=true, %set true if you want colored links
    linktoc=all,     %set to all if you want both sections and subsections linked
    linkcolor=black,  %choose some color if you want links to stand out
}
\usepackage{fancyhdr}


% Macros
\newcommand{\R}{\mathbb{R}}
\newcommand{\N}{\mathbb{N}}
\newcommand{\Q}{\mathbb{Q}}
\newcommand{\Z}{\mathbb{Z}}
\newcommand{\sub}{\subset}
\renewcommand{\a}{\alpha}
\renewcommand{\b}{\beta}
\newcommand{\g}{\gamma}
\renewcommand{\d}{\delta}
\newcommand{\e}{\varepsilon}
\newcommand{\ex}{\exists\,}
\newcommand{\overbar}[1]{\mkern 1.5mu\overline{\mkern-1.5mu#1\mkern-1.5mu}\mkern 1.5mu}
\newcommand{\eR}{\overbar{\R}}
\newcommand{\xb}{\bar{x}}

%\setlength{\columnsep}{20pt}

%ToC stuff
\newtheorem{example}{Example}
\newtheorem{solution}{Solution}
%\newtheorem{definition}{Definition}[subsection]
\newtheorem{corollary}{Corollary}

\tcbuselibrary{theorems}
\newtcbtheorem[number within=section]{theorem}{Theorem}%
{colback=green!5,colframe=green!35!black,fonttitle=\bfseries}{th}
\newtcbtheorem[number within=section]{lemma}{Lemma}%
{colback=orange!5,colframe=orange!35!black,fonttitle=\bfseries}{lm}
\newtcbtheorem[number within=section]{definition}{Definition}%
{colback=blue!5,colframe=blue!35!black,fonttitle=\bfseries}{def}

 % Document stuff

\title{Week 4: Sequences}
\author{James Arthur}

\begin{document}
\maketitle
\tableofcontents
\newpage

\multicols{2}
\section{Sequences}

\noindent\begin{definition}{Limit of a Sequence}{}
  A sequence, $\{s_n \}$ converges to a limit $s$, if for every $\e > 0\, \ex N\in \Z$,
  $$ |S_n - S|< \e \quad n \ge N $$
\end{definition}\vspace{10pt}

\noindent\begin{definition}{Divergence}{}
  We say that $\displaystyle{\lim_{n\to\infty}{a_n} = \infty}$ if $\forall a \in \R$, $s_n > a$ for $n > a$. Similarly for $-\infty$.
\end{definition}\vspace{10pt}

\noindent\begin{theorem}{}{}
   Let $\displaystyle{\lim_{x\to \infty}{f(x)} = L}$, where $L\in \eR$ and suppose that $s_n = f(n)$ for large $n$, then:
   $$ \lim_{n\to\infty}{s_n} = L $$
\end{theorem}\vspace{10pt}

\noindent\begin{definition}{Subsequence}{}
  A subsequence $\{ t_k \}$ if $t_k = s_{n_k}$, where $\{ n_k \}$ is an increasing subsequence of integers.
\end{definition}\vspace{10pt}

\noindent\begin{theorem}{Uniqueness of subsequence limit}{}
   If $\displaystyle{\lim_{n\to\infty}{s_n} = s}$, then $\displaystyle{\lim_{n\to\infty}{s_{n_k}} = s \quad\forall \{s_{n_k} \}}$  of $\{ s_k \}$
\end{theorem}\vspace{10pt}
\begin{proof}
  Consider the finite case, $\forall \e > 0\,\ex N$,
  $$ |S_n - S| < \e \quad k \ge K$$
  Since, $\{ n_k \}$ is increasing $\ex K$, $n_k \ge N$ if $k > K$
  $$ |S_{n_k} - S|<\e \qquad k \ge K $$
  For infinite limits, $\forall \e >0, \, \ex N$,
  $$ S_n > n \qquad n \ge N $$
  as we know $\displaystyle{\{ n_k \}}$ is increasing, then $n_k > n$ for $n \ge N$ but $S_{n_k} > n$ for some $n \ge N$ and so the limit is infinite. For limit to $-\infty$, use the sequence $-S_n$
\end{proof}

\noindent\begin{theorem}{Limit Points of Sequences}{}
  A point $\xb$ is a limit point of a set $S$, iff there is a sequence $\{ x_n \}$ of points in $S$, $x_n \neq \xb$ for $n \ge 1$, and $\displaystyle{\lim_{n\to\infty}{x_n} = \xb}$
\end{theorem}\vspace{10pt}
\begin{proof}
  Suppose such a $\{x_n\}$ exists. Then $\forall\e>0,\, \ex N$,
  $$ 0 < |x_n - \xb| < \e \qquad \forall n \ge N $$
  Therefore every $\e$-neigh. contains $\infty$ many points of $S$ hence $\xb$ is a limit point of $S$.\\

  Now let $\xb$ be a limit point of $S$. $\forall N \ge 1$, $(\xb - \frac{1}{n}, \xb + \frac{1}{n})$ has to contain some point $x_n \in S$, $x_n \neq \xb$. Since,
  $$ |x_n - \xb| \le \frac{1}{n}\qquad m \ge n $$
  and $\displaystyle{\lim_{n\to\infty}{x_n} = \xb}$
\end{proof}

\noindent\begin{theorem}{Bounded and Subsequence Theorems}{}
  \begin{enumerate}
    \item If $\{ x_n \}$ is bounded then it has a convergent subsequence
    \item If $\{ x_n \}$ is unbounded above, then it has a subsequence $\{ x_{n_k} \}$ st,
    $$ \lim_{x\to\infty}{x_{n_k}} = \infty $$
    \item If $\{ x_n \}$ is unbounded above, then $\{ x_n \}$ has a subsequence st,
    $$ \lim_{x\to\infty}{x_{n_k}} = -\infty $$
  \end{enumerate}
\end{theorem}\vspace{10pt}
\begin{proof}
  \textbf{Proof of 1:} Let $S$ be a set of distinct numbers of $\{ s_n \}$, if $s$ is finite, then $\ex,\, \xb\in s$, which occurs infinitely often. Then,
  $$ \lim_{n\to\infty}{x_{n_k}} = \xb $$
  If $s$ is infinte, then since $s$ is bounded BWT applies, now $s$ has a limit point, $\xb$. Then by previous thm, $\ex\{y_j\}\in s$ with $y_j \neq s$,
  $$ \lim_{j\to\infty}{y_j} = \xb $$
  However, $\{ y_j \}$ may not be a subsequence of $\{ x_n \}$, so $y_j = x_{n_j}$ may not be true, where $n_j$ is increasing. So now take an increasing subsequence of $n_j$, $\{ n_j \}$, then $\{ y_{j_k} \} = \{ s_{n_{j_k}} \}$ is a subsequence. So it has the same limit as; $\{ y_j \}$
  $$ \lim_{k\to\infty}{\{ s_{n_{j_k}} \}} = \xb $$
\end{proof}

\subsection{Cauchy Sequences}

\noindent\begin{definition}{Cauchy Sequences}{}
  A sequence $\{ s_n \}$ of real numbers is said to be cauchy if $\e > 0,\,\ex N\in\N$, $n\ge N$ and $m\ge N$, then:
  $$ |S_n - S_m| < \e $$
\end{definition}\vspace{10pt}

\noindent\begin{lemma}{}{}
  Let $\{ s_n \}$ be a convergent, then it's cauchy
\end{lemma}\vspace{10pt}
\begin{proof}
  Suppose that $s_n\to s$ as $n\to\infty$. Let $\e > 0,\,\ex N, \, n\ge N$
  $$ |s_n - s|< \frac{\e}{2} $$
  Now take $m, n \ge N$, then
  \begin{align*}
    |s_n - s_m| &= |s_n - s - (s_m - s)| \\
    &\le |s_n - s| + |s_m - s| \\
    &< \e
  \end{align*}
\end{proof}

\noindent\begin{lemma}{}{}
  Let $\{ s_n \}$ be cauchy, then it's convergent
\end{lemma}\vspace{10pt}
\begin{proof}
  Let $\{ s_n \}$ be cauchy, and hence it's bounded. By thm 3.14(a), there is a convergent subsequence $\{ s_{n_k} \}$ for some $s\in\R$. Now claim, $\displaystyle{s_k \in s}$ as $\displaystyle{k \to \infty}$.\\

  Let $\e > 0,\, \ex N_2$, $k \ge N_1$, then
  $$ |s_{n_k} s| < \frac{\e}{2} $$
  Then $\ex N_2, \, m, n \ge N_2$, then $|s_m - s_n| < \frac{\e}{2}$. If $K \ge \max(m, n)$,
  \begin{align*}
    |s_k - s| &= |(s_k - s_{n_k}) - (s - s_{n_k})|\\
    &\le |s_k - s_{n_k}| + |s - s_{n_k}|\\
    &< \e
  \end{align*}
\end{proof}

\section{Series}
\noindent\begin{definition}{Series}{}
   If $\displaystyle{\{ a_k \}^\infty_k = \sum_{n = k}^\infty a_k}$ is infinite and $a_n$ is the $n^th$ term. If $\displaystyle{\sum_{k=1}^\infty} = A$, then it converges. Also we say $\displaystyle{A_n = a_k + \dots + a_n \qquad n \ge k}$ is the $n^th$ partial sum of the sum. We can also say that,
   $$ \lim_{k\to\infty}{A_n} = A $$
\end{definition}\vspace{10pt}

\noindent\begin{theorem}{Cauchy Criterion for Series}{}
   A series $\sum a_n$ converges iff $\forall \e > 0,\,\ex N$,
   $$ |a_n + a_{n-1} +\dots + a_m| < \e \qquad m \ge n \ge N \quad (*)$$
\end{theorem}\vspace{10pt}

\begin{proof}
  Let $\{ A_n \}$ be the series of partial sums of our series. Then
  $${A_m - A_{n-1} = a_n + \dots + a_m}$$
  If $(*)$ holds, then
  $$|A_m - A_{n-1}| < \e \text{ if } m \ge n \ge N \quad (**)$$
  To say $\sum a_n$ is convergent, then $\{ A_n \}$ is convergent. This is equiv to $\{ A_n \}$ being cauchy, which is what $(**)$ says.
\end{proof}

\noindent\begin{corollary}{}{}
  If $\sum a_n$ converges, then $\displaystyle{\lim_{n\to\infty}{a_n} = 0}$
\end{corollary}\vspace{10pt}
\begin{proof}
  Taking $m = n$ in the previous thm, then $\forall\e > 0,\,\ex N > 0$,
  $$ |a_n| < \e \text{ if } n \ge N $$
  which is $\displaystyle{\lim_{n\to\infty}{a_n} = 0}$
\end{proof}

\noindent\begin{corollary}({Divergence Test}){}
  If $\displaystyle{\lim_{n\to\infty}{a_n}\neq 0}$, then $\sum a_n$ divergent
\end{corollary}\vspace{10pt}



\end{document}
