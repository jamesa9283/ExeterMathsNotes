\documentclass{article}

\usepackage{fullpage}
\usepackage{multicol}
\usepackage{amsmath}
\usepackage{bm}


\newcommand{\di}{\frac{dy}{dx}}
\newcommand{\dii}{\frac{d^2y}{dx^2}}
\newcommand{\din}{\frac{d^ny}{dx^n}}
\newcommand{\dt}{\frac{dx}{dt}}
\newcommand{\dtt}{\frac{d^2x}{dt^2}}
\newcommand{\dtn}{\frac{d^nx}{dt^n}}
\newcommand{\pd}[2]{\frac{\partial#1}{\partial#2}}


\newtheorem{example}{Example}
\newtheorem{solution}{Solution}

\title{First Order Linear ODEs}
\author{James Arthur}

\begin{document}
\maketitle
\tableofcontents\newpage


\multicols{2}

\section{Recap of previous modules}
A differentiable equation is an mathematical relation involving a derivative of a dependant variable w.r.t. single/many independant variables\\

\subsection{Notation:}
$$ \dt = \dot{x},\qquad \di = y',\qquad f\left( x, y, \di,\, \dots \,, \din \right) = 0$$ $$ f_{xy} = \frac{\partial^f }{\partial x \partial y} $$

\subsection{Prerequisites: } Integration, ODEs and PDEs from Advanced Calculus.

\section{Basic Defintions and Concepts}

\subsection{Classifications}
\begin{enumerate}
  \item Use of full or partial derivative
  \item Coeeficients are functions of independant variables only / contant, otherwise non Linear
  \item The highest derivative is the order of the description DE
  \item Degree is the highest derivative in rationalised form
  \item An explicit solution is; $f = F(x, y, z, t) $ and implicit solution; $F(f, x, y, z, t) = 0$
  \item Initial Value Problem (time) or boundary value problem (space).
\end{enumerate}

\subsection{Review of integration methods}
\begin{enumerate}
  \item List of commonly used integrals (link)
  \item Polynomials, logarithms, trigonometric, inverse, hyperbolic and inverse hyperbolic trig.
\end{enumerate}

\subsection{Concepts}
\begin{enumerate}
  \item Given a DE, we want a solution.
  \item A solution is a derived relation between the dependant and independant variables without any derivative term and defined in the iterval / domain / region.
  \item replacing the solution within the domain satisfies the description DE
  \item Needs integration on one or two variables
  \item Not always analytical and closed forms possible. (could use numerical integration, iterative solution schemes.)
\end{enumerate}

\begin{enumerate}
  \item Linear higher order ODEs (Laplace transforms)
  \begin{enumerate}
    \item transforms linear ODEs in algebraic forms
    \item needs table of laplace, inverse laplace formula
  \end{enumerate}
  \item the Geometric meaning is the slope of $y(x)$: $y'(x) = f(x_0, y_0)$ implies at a point $ (x_0, y_0)$ is the slope at $\di$
  \item There is something called a direction field that we can use to visualise the DE without solving it.
  \item Curves of equal inclination $f(x, y) = c$ along which derivative is constant. Lots of paralell lines.
  \item Limitation of direction field give an overall idea about the solution but have limited accuracy
  \item Orthoganal trajectories are a family of courves that intersect another family of curves at right angles.
  For a curve $G(x, y, c) $ firstly final out $\di = f(x, y)$. General solution of the orthoganal trajector $\di = \frac{-1}{f(x, y)}$.
  \item existence is under what condition there is at least one Solution
  \item uniqueness is what condition it has at most one Solution
  item the general solution contains the constants of integration
  \item particular solution are when you ise the initial / boundary conditions
\end{enumerate}

\subsection{Famous models}
\begin{itemize}
  \item Van der Pol oscillator
  $$\begin{aligned}
    \dot x &= y \\
    \dot{y} &= \mu (1 - x^2)y - x
  \end{aligned}$$
  \item Lorentz Attractor
  $$\begin{aligned}
    \dot x &= \sigma (y - x)\\
    \dot y &= x (\rho - z) - y \\
    \dot z &= xy - \beta z \\
  \end{aligned}$$
\end{itemize}

Have to be careful between phase portrait and direction fields. Phase portraits are almost a guess and direction field as you have a solution at every point with a direction field.

\section{Analytical Solutions}

\subsection{Seperation of variables}

\begin{align*}
  g(y)y' &= f(x)\\
  \int g(y)dy = \int f(x)dx + c
\end{align*}

\subsection{Reduction to seperable form}

\begin{align*}
  y' = f(\frac{y}{x})\\
  v + x \frac{dv}{dx} &= f(v) && \text{by letting $ y = vx$}\\
  \int \frac{dv}{f(v) - v} &= \int \frac{dx}{x} + c\\
  &= \ln |x| + c\\
\end{align*}

\subsection{Exact ODEs and integrating factors}

Let us have an ODE: $M(x, y) + N(x, y)\di = 0$ which can be written as $M(x, y)dx + N(x, y)dy$. We can then write a total differential as partial derivatives:
\begin{equation}  \label{eq:1}
  du = \pd{u}{x} dx + \pd{u}{y} = 0 \implies u(x, y) = c
\end{equation}
Which then we can compare the two and we get:
$$ M = \pd{u}{x}, \qquad N = \pd{u}{y} $$ $$\implies \pd{M}{y} = \frac{\partial^2u}{\partial y \partial x}, \pd{N}{x} = \frac{\partial^2u}{\partial x \partial y} $$

So then we can say for an ODE to be exact:
$$ \pd{M}{y} = \pd{N}{x} $$

To solve the DE, take (\ref{eq:1}) and integrate with respect to u:
\begin{align*}
  du &= \pd{u}{x} dx + \pd{u}{y} = 0 && \text{(1)}\\
  u = \int M dx + K(y)= c \\
\end{align*}
















\end{document}
