\documentclass{article}

\usepackage{fullpage}
\usepackage{multicol}
\usepackage{amsmath}
\usepackage{bm}
\usepackage{amssymb}


\newcommand{\di}{\frac{dy}{dx}}
\newcommand{\dii}{\frac{d^2y}{dx^2}}
\newcommand{\din}{\frac{d^ny}{dx^n}}
\newcommand{\dt}{\frac{dx}{dt}}
\newcommand{\dtt}{\frac{d^2x}{dt^2}}
\newcommand{\dtn}{\frac{d^nx}{dt^n}}
\newcommand{\pd}[2]{\frac{\partial#1}{\partial#2}}
\newcommand{\fd}[2]{\frac{d #1}{d #2}}

\newtheorem{example}{Example}
\newtheorem{solution}{Solution}
\newtheorem{definition}{Defintion}

\title{First Order Linear ODEs}
\author{James Arthur}

\begin{document}
\maketitle
\tableofcontents\newpage


\multicols{2}

\section{Recap of previous modules}
A differentiable equation is an mathematical relation involving a derivative of a dependant variable w.r.t. single/many independant variables\\

\subsection{Notation:}
$$ \dt = \dot{x},\qquad \di = y',\qquad f\left( x, y, \di,\, \dots \,, \din \right) = 0$$ $$ f_{xy} = \frac{\partial^f }{\partial x \partial y} $$

\subsection{Prerequisites: } Integration, ODEs and PDEs from Advanced Calculus.

\section{Basic Defintions and Concepts}

\subsection{Classifications}
\begin{enumerate}
  \item Use of full or partial derivative
  \item Coeeficients are functions of independant variables only / contant, otherwise non Linear
  \item The highest derivative is the order of the description DE
  \item Degree is the highest derivative in rationalised form
  \item An explicit solution is; $f = F(x, y, z, t) $ and implicit solution; $F(f, x, y, z, t) = 0$
  \item Initial Value Problem (time) or boundary value problem (space).
\end{enumerate}

\subsection{Review of integration methods}
\begin{enumerate}
  \item List of commonly used integrals (link)
  \item Polynomials, logarithms, trigonometric, inverse, hyperbolic and inverse hyperbolic trig.
\end{enumerate}

\subsection{Concepts}
\begin{enumerate}
  \item Given a DE, we want a solution.
  \item A solution is a derived relation between the dependant and independant variables without any derivative term and defined in the iterval / domain / region.
  \item replacing the solution within the domain satisfies the description DE
  \item Needs integration on one or two variables
  \item Not always analytical and closed forms possible. (could use numerical integration, iterative solution schemes.)
\end{enumerate}

\begin{enumerate}
  \item Linear higher order ODEs (Laplace transforms)
  \begin{enumerate}
    \item transforms linear ODEs in algebraic forms
    \item needs table of laplace, inverse laplace formula
  \end{enumerate}
  \item the Geometric meaning is the slope of $y(x)$: $y'(x) = f(x_0, y_0)$ implies at a point $ (x_0, y_0)$ is the slope at $\di$
  \item There is something called a direction field that we can use to visualise the DE without solving it.
  \item Curves of equal inclination $f(x, y) = c$ along which derivative is constant. Lots of paralell lines.
  \item Limitation of direction field give an overall idea about the solution but have limited accuracy
  \item Orthoganal trajectories are a family of courves that intersect another family of curves at right angles.
  For a curve $G(x, y, c) $ firstly final out $\di = f(x, y)$. General solution of the orthoganal trajector $\di = \frac{-1}{f(x, y)}$.
  \item existence is under what condition there is at least one Solution
  \item uniqueness is what condition it has at most one Solution
  item the general solution contains the constants of integration
  \item particular solution are when you ise the initial / boundary conditions
\end{enumerate}

\subsection{Famous models}
\begin{itemize}
  \item Van der Pol oscillator
  $$\begin{aligned}
    \dot x &= y \\
    \dot{y} &= \mu (1 - x^2)y - x
  \end{aligned}$$
  \item Lorentz Attractor
  $$\begin{aligned}
    \dot x &= \sigma (y - x)\\
    \dot y &= x (\rho - z) - y \\
    \dot z &= xy - \beta z \\
  \end{aligned}$$
\end{itemize}

Have to be careful between phase portrait and direction fields. Phase portraits are almost a guess and direction field as you have a solution at every point with a direction field.

\section{Analytical Solutions}

\subsection{Seperation of variables}

\begin{align*}
  g(y)y' &= f(x)\\
  \int g(y)dy = \int f(x)dx + c
\end{align*}

\subsection{Reduction to seperable form}

\begin{align*}
  y' = f(\frac{y}{x})\\
  v + x \frac{dv}{dx} &= f(v) && \text{by letting $ y = vx$}\\
  \int \frac{dv}{f(v) - v} &= \int \frac{dx}{x} + c\\
  &= \ln |x| + c\\
\end{align*}

\subsection{Exact ODEs and integrating factors}

Let us have an ODE: $M(x, y) + N(x, y)\di = 0$ which can be written as $M(x, y)dx + N(x, y)dy$. We can then write a total differential as partial derivatives:
\begin{equation}  \label{eq:1}
  du = \pd{u}{x} dx + \pd{u}{y} = 0 \implies u(x, y) = c
\end{equation}
Which then we can compare the two and we get:
$$ M = \pd{u}{x}, \qquad N = \pd{u}{y} $$ $$\implies \pd{M}{y} = \frac{\partial^2u}{\partial y \partial x}, \pd{N}{x} = \frac{\partial^2u}{\partial x \partial y} $$

So then we can say for an ODE to be exact:
$$ \pd{M}{y} = \pd{N}{x} $$

To solve the DE, take (\ref{eq:1}) and integrate with respect to u:
\begin{align*}
  du &= \pd{u}{x} dx + \pd{u}{y} = 0 && \text{(1)}\\
  u &= \int M dx + K(y)= c \\
  \pd{u}{y} &= \pd{}{y}\left[\int M dx\right] + \frac{dK(y)}{dy} = N \\
  \implies \frac{dK}{dy} &= N - \pd{}{y}\left[ \int Mdx \right] \\
  \implies k &= \int \left[ N - \pd{}{y}\left[ \int M dx \right]\right]dy \\
\end{align*}

From this we can substitute $k(y)$ back in and get the general solution of an exact ODE:

$$ u(x, y) = \int M dx + \int{\left[ N + \pd{}{y}\left[ \int M dx \right] \right]}dy $$

\subsection{Reduction to Integrating Factors}

We have a $P(x, y)dx + Q(x, y)dy = 0 $ can be moved into $F\cdot P dx + F\cdot Q dx = 0$ and is exact.
So we hope that:
\begin{align*}
  \pd{}{y}(FP) &= \pd{}{y}(FQ)\\
  \implies \pd{F}{y}P + \pd{P}{y} &= \pd{F}{x}Q + F\pd{Q}{x}\\
\end{align*}
Now let, $F = F(x)$ only, then we can say that $\displaystyle{\pd{F}{y} = 0}$ and $\displaystyle{\pd{F}{x} = \fd{F}{x}}$
$$ \therefore \fd{F}{x} = \frac{F}{Q}\left[ \pd{P}{y} - \pd{Q}{x} \right] = F \cdot R$$
where $\displaystyle{R = \frac{1}{Q}\left[ \pd{P}{y} - \pd{Q}{x} \right]} $ and we can solve nicely as:
$$ F = Ke^{\int R dx} $$

R must be of $x$ only, not $x$ and $y$ or both. We can do the reverse with $y$, the maths is the same, but the $R$ this time, which denote as $R^*$:
$$ R^* = \frac{1}{P}\left[\pd{Q}{x} - \pd{P}{y} \right] $$

\section{Linear ODEs}

First order linear ODEs:
$$ \di + p(x)y = r(x) $$
Linear Homogenous ODEs if $r(x) = 0 $:
$$ \di + p(x)y = 0 $$
which has a solution:
$$ y(x) = ce^{-\int p(x)dx} $$



\fbox{\parbox{0.425\textwidth}{\begin{example}{
 Solve $x^3dx + y^3dy$
}\end{example}\begin{solution}{
  Firstly let us test for exactness. Our $M = x^3$ and $N=y^3$. The test says that the DE is exact if we know that $\pd{M}{y} = \pd{N}{x}$. We know that from a differential:
  $$ \pd{M}{y} = 0 = \pd{N}{x} = 0 $$
  This means that the equation is exact, so we can now plug this into the general solution to an exact equation:
  $$ u(x, y) = \int M dx + \int{\left[ N + \pd{}{y}\left[ \int M dx \right] \right]}dy $$
  and we get the solution:
  $$ u(x, y) = \frac{1}{4}x^4 + \frac{1}{4}y^4  + C$$
}\end{solution}}}\vspace{10pt}


\subsection{Nonhomogeneous Linear ODEs}

When $\displaystyle{r(x) \neq 0}$ in $\displaystyle{\dii + p(x)y = r(x)}$. We can then do the following:
\begin{align*}
  \implies dy + pydx &= rdx \\
  dy + (py - r)dx &= 0 \\
  (py - r)dx = dy = 0 \\
\end{align*}

Now we can look at the integrating factor, which we know is of form:
$$ R = \frac{1}{N}\left[\pd{M}{y} - \pd{N}{x}\right] $$
which we can sub in $\displaystyle{M = py - r}$ and $\displaystyle{N = 1}$ and we get:
$$ R = 1 \cdot (p - 0) = p(x) $$
So for a non-homogenous linear ODE, the integrating factor is $p(x)$.

\section{Non-linear ODE}
We can take a bernoulli equation:
$$ \dii + p(x)y = g(x)\cdot y^n $$
which we can write it as:
$$ \dii = gy^n - py $$

The equation is linear for $n = 0,1$, so we shall consider it without this condition. Firstly we are going to apply a substitution, $u(x) = y(x)^{1-n}$. Working it through we get:
\begin{align*}
  \fd{u}{x} &= (1-n)\left[ y(x) \right]^{-n}\dii \\
  &= (1-n)y^{-n}(gy^n - py) \\
  &= (1 - n)(g - py^{1-n}) \\
  &= (1 - n)(g - pu) \\
\end{align*}

Which is linear and solvable.

\section{Existence and Uniqueness}
An initial value problem $\displaystyle{\dii = f(x, y), \quad y(x_0) = y_0}$ has:
\begin{itemize}
  \item no Solution
  \item precisely one solution
  \item many solutions
\end{itemize}

\fbox{\parbox{0.475\textwidth}{\begin{definition}{
 Existence: Under what condition the IVP has at least one solution.
}\end{definition}}}\vspace{10pt}

More formally: A function $\displaystyle{f(x, y)}$ is continuous in some bounded rectangle $R$, $\displaystyle{R : |x - x_0| < a,\, |y - y_0|< b}$ there exists some $K$, such that $\displaystyle{|f(x, y)|\le K\quad \forall (x, y) \in R}$ then the IVP exists.\\

\fbox{\parbox{0.475\textwidth}{\begin{definition}{
 Uniqueness: Under what conditions the IVP has at most one solution.
}\end{definition}}}\vspace{10pt}

More formally: If the function and its partial derivatives $f$, $f_y$ are continuous in $R$: $\displaystyle{|f(x, y)| \le K|f_y(x, y)| \le M, \quad \forall (x, y) \in R}$












\end{document}
