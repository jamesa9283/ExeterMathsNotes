\documentclass{article}

% Packages
\usepackage{fullpage}
\usepackage{amssymb}
\usepackage{multicol}
\usepackage{amsmath}
\usepackage{amsfonts}
\usepackage{bm}
\usepackage{float}
\usepackage{tikz}
\usepackage{xcolor}
\usetikzlibrary{shapes.geometric, positioning, arrows, intersections}
\usepackage{amsthm}
\usepackage{tcolorbox}
\usepackage{hyperref}
\hypersetup{
    colorlinks=true, %set true if you want colored links
    linktoc=all,     %set to all if you want both sections and subsections linked
    linkcolor=black,  %choose some color if you want links to stand out
}

% Macros
\newcommand{\R}{\mathbb{R}}
\newcommand{\N}{\mathbb{N}}
\newcommand{\Q}{\mathbb{Q}}
\newcommand{\di}{\frac{dy}{dx}}
\newcommand{\dii}{\frac{d^2y}{dx^2}}
\newcommand{\din}{\frac{d^ny}{dx^n}}
\newcommand{\dt}{\frac{dx}{dt}}
\newcommand{\dtt}{\frac{d^2x}{dt^2}}
\newcommand{\dtn}{\frac{d^nx}{dt^n}}
\newcommand{\pd}[2]{\frac{\partial#1}{\partial#2}}
\newcommand{\fd}[2]{\frac{d #1}{d #2}}
\renewcommand{\l}{\lambda}
\newcommand{\g}{\gamma}
\renewcommand{\o}{\omega}
\newcommand{\el}{e^{\l x}}


%ToC stuff
\newtheorem{example}{Example}
\newtheorem{solution}{Solution}
%\newtheorem{definition}{Definition}[subsection]
\newtheorem{corollary}{Corollary}

\tcbuselibrary{theorems}
\newtcbtheorem[number within=section]{theorem}{Theorem}%
{colback=green!5,colframe=green!35!black,fonttitle=\bfseries}{th}
\newtcbtheorem[number within=section]{definition}{Definition}%
{colback=blue!5,colframe=blue!35!black,fonttitle=\bfseries}{def}


\title{Differential Equations Week 2 - Linear Higher Order ODEs}
\author{James Arthur}

\begin{document}
\maketitle
\tableofcontents\newpage

\multicols{2}

\section{Overview}
A higher order ODE and its $n$th derivatives, form an ODE.

We know that the solution is $n$th times differentiable\\

\noindent\begin{theorem}{Fundemental Theorem for homogenous linear ODEs}{}
   The sum and constant multiples of solutions are again solutions of the ODE
\end{theorem}\vspace{10pt}

The general solution is $\displaystyle{y(x) = c_1y_1 + \dots + c_ny_n}$, where $\{c_1,\dots, c_n \}$ are arbitrary constants and $\{y_1, \dots, y_n \}$ is a basis or fundemental system of solution.

The $n$ solutions are linearly independant when $k_1y_1+\dots+k_ny_n = 0$ implies that $\{k_1, \dots, k_n \}$ are zero.

\subsection{Initial Value Problem}
An IVP consists of the ODE, $$\displaystyle{y^{(n)} + p_{n-1}y^{(n-1)}+\dots +p_1y' + p_0y = 0}$$ and $n$ initial conditions with a given $x_0$ and several constants.\\

\noindent\begin{theorem}{Uniqueness of HODEs}{}
  If the coefficients are continuous on some open interval, $I$, and $x_0 \in I$, then the IVP has a unique solutions $y(x)$ on $I$.
\end{theorem}\vspace{10pt}

The Wronskian of $n$ solutions:
$$ W = \left|\begin{matrix}
  y_1 & y_2 & \dots & y_n \\
  y_1' & y_2' & \dots & y_n' \\
  \vdots & \vdots & \ddots & \vdots \\
  y_1^{(n-1)} & y_2^{(n-1)} & \dots & y_n^{(n-1)}\\
\end{matrix}\right| $$
has to be non-zero for linear independence.

\noindent\begin{theorem}{Existence}{}
   If the coefficients are continuous on some open interval, $I$, then the ODE has a general solution on $I$
\end{theorem}\vspace{10pt}

\noindent\begin{theorem}{General Solution}{}
   If the ODE has continuous coefficients on some open interval $I$, then every solution $y = Y(x)$  is of the form $\displaystyle{Y(x) = C_1y_1 + \dots + C_ny_n}$, where the `$y$'s are the basis and `$C$'s are the contants.
\end{theorem}\vspace{10pt}

\section{Homogenous ODEs with constant coefficients}
The ODE can be solved rather easily by subbing in $y = e^{\l x}$ and getting a characteristic equation. Hence solve and follow usual procedure.

\subsection{Real Roots}
\textit{Case 1: Distinct Roots}, then the solution is; $y_1 = e^{\l_1 x}, \dots, y_n = e^{\l_n x}$. Then linearly combine and get general solution. Now plug into the Wronskian and pull out the exponential terms:
\begin{align*}
  W &= \left|\begin{matrix}
    e^{\l_1 x} & e^{\l_2 x} & \dots & e^{\l_n x} \\
    \l_1e^{\l_1 x} & \l_2e^{\l_2 x} & \dots & \l_ne^{\l_n x} \\
    \vdots & \vdots & \ddots & \vdots \\
    \l_1^{(n-1)}e^{\l_1 x} & \l_2^{(n-1)}e^{\l_2 x} & \dots & \l_n^{(n-1)}e^{\l_n x}\\
  \end{matrix}\right| \\
  &= E\left|\begin{matrix}
    1 & 1 & \dots & 1 \\
    \l_1 & \l_2 & \dots & \l_n \\
    \vdots & \vdots & \ddots & \vdots \\
    \l_1^{(n-1)} & \l_2^{(n-1)} & \dots & \l_n^{(n-1)}\\
  \end{matrix}\right| \\
\end{align*}
where $E = e^{\l_1 + \dots + \l_n}$ and hence $W = 0$, iff the determinant on the right is zero. This is known as the vandermonde or cauchy determinant.\\

\noindent\begin{theorem}{Linearly Independence Theorem}{}
   Any number of solutions of the form $y = e^{\l x}$ are linearly independent on an open interval, $I$, iff the corresponding roots, $\l$, are all different.
\end{theorem}\vspace{10pt}

\textit{Case III: Multiple Real Roots}, For $\l_1 = \l_2$ the solutions are $y_1, y_2 = xy_1$ which are linearly independant. More generally, for real roots of order $m$, the linearly independant solutions are:
$$ e^{\l x}, xe^{\l x}, x^2e^{\l x}, \dots, x^{m-1}e^{\l x} $$

\subsection{Complex Roots}

\textit{Case 2: Simple Complex Roots}
For complex roots, they must be conjugate pairs, since the coefficients of the ODE are real. So for a $\l = \g \pm i\o$, then the two linearly independant solutions are:
$$ y_1 = e^{\g x}\cos{\o x} \qquad y_2 = e^{\g x}\sin{\o x} $$

\textit{Case 4: Multiple Complex Roots}, let us have complex double roots, $\displaystyle{\l = \g \pm i\o}$. The coresponding linearly independant solutions are: $\displaystyle{e^{\g x}\sin{\o x}, e^{\g x}\cos{\o x}, xe^{\g x}\sin{\o x}, xe^{\g x}\cos{\o x}}$. Therefore, we gain the general solution:
$$ y = e^{\g x}\big[ (A_1 + A_2x)\cos{\o x} + (B_1 + B_2x)\sin{\o x}\big] $$
For triple roots, just add $x^2$ terms.

\section{Non-homogenous Linear ODEs}
This is very similar to the second order, the proess is basically the same. We can solve via `Method of Undetermined Coefficients' or `Method of variation of paramaters'.

\subsection{Method of Undetermined Coefficients}
The basic rule is to know off, $r(x)$ and then solve. Next take a general particular integral and then follow usual solution. Then you can multiply by $x^k$ where $k$ is the number of $y_p$'s in the solution plus one.
Then if you have a summation of many particular integrals, just break it up.

\subsection{Variation of Parameters}
Particular solutions of the nonhomogenous equations:
\begin{align*}
  y_p(x) &= \sum_{k=1}^{n}y_k(x)\int\frac{W_k}{W(x)}r(x)dx\\
  &= y_1(x)\int\frac{W_1}{W(x)}r(x)dx + \dots\\
  &\quad + y_n(x)\int\frac{W_n}{W(x)}r(x)dx
\end{align*}
The wronskians $W_j(j = 1,\dots, n)$ are obtained from $W$ by replacing the $j^{th}$ column by $\begin{bmatrix}
  0 & \dots & 0 & 1 \\
\end{bmatrix}^T$, where $\{y_1, \dots, y_n \}$ are a basis.

Therefore, Wronskians $W_j(j=1, \dots, n)$ for each component are:
$$   W_1 = \left|\begin{matrix}
    0 & e^{\l_2 x} & \dots & e^{\l_n x} \\
    0 & \l_2e^{\l_2 x} & \dots & \l_ne^{\l_n x} \\
    \vdots & \vdots & \ddots & \vdots \\
    1 & \l_2^{(n-1)}e^{\l_2 x} & \dots & \l_n^{(n-1)}e^{\l_n x}\\
  \end{matrix}\right|$$ all the way to $$W_n = \left|\begin{matrix}
      e^{\l_1 x} & e^{\l_2 x} & \dots & 0 \\
      \l_1e^{\l_1 x} & \l_2e^{\l_2 x} & \dots & 0 \\
      \vdots & \vdots & \ddots & \vdots \\
      \l_1^{(n-1)}e^{\l_1 x} & \l_2^{(n-1)}e^{\l_2 x} & \dots & 1\\
    \end{matrix}\right| $$
This method should be used for ODEs greater than 2.




























\end{document}
