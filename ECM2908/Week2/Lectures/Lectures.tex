\documentclass{article}

% Packages
\usepackage{fullpage}
\usepackage{multicol}
\usepackage{amsmath}
\usepackage{amssymb}
\usepackage{mathtools}
\usepackage{bm}
\usepackage{tikz}
\usetikzlibrary{shapes.geometric, positioning}

% Macros
\newcommand{\R}{\mathbb{R}}
\newcommand{\N}{\mathbb{N}}
\newcommand{\Q}{\mathbb{Q}}
\renewcommand{\vec}[1]{\underline{\textbf{#1}}}
\newcommand{\veci}{\bm{\hat{\imath}}}
\newcommand{\vecj}{\bm{\hat{\jmath}}}
\newcommand{\veck}{\bm{\hat{k}}}
\newcommand{\e}{\varepsilon}
\newcommand{\de}{\delta}
\newcommand{\kd}{\delta_{i, j}}
\newcommand{\at}{\e_{i, j, k}}
\newcommand{\nab}{\underline{\nabla}}
\newcommand{\grad}{{\nab}\, f}
\newcommand{\pd}[2]{\frac{\partial #1}{\partial #2}}
\renewcommand{\div}{\nab \cdot}
\newcommand{\curl}{\nab \times}

\newtheorem{example}{Example}
\newtheorem{solution}{Solution}
\newtheorem{definition}{Definitions}

\title{Vector Calculus Week 2 - More Suffix Notation}
\author{James Arthur}

\begin{document}
\maketitle
\tableofcontents\newpage


\multicols{2}

\section{Gradient, Divergence and Curl}
\subsection{Gradient}

Assume we have a $f = f(x, y, z)$ or $f = f(x_1,x_2,x_3)$, so a scalar calued function. Then we define grad f as:
$$ \grad = \left(\pd{}{x}\veci + \pd{}{y}\vecj + \pd{}{z}\veck \right)\, f$$
We say grad of $f$ is a differential operator. So:
$$ \grad = \left(\pd{f}{x}\veci + \pd{f}{y}\vecj + \pd{f}{z}\veck \right) $$
and we can write it in suffix notation aswell:
$$ \left[ \grad \right]_i = \pd{}{x_i} \qquad i = 1, 2, 3$$

\subsection{Divergence}

Assume we have a vector field, $\vec{u} = \vec{u} (x, y, z, t)$. We define the divergence of this vector field as;
$$ \nab \cdot \vec{u} = \left(\pd{u_1}{x_1} + \pd{u_2}{x_2} + \pd{u_3}{x_3}\right)$$
Placing this in suffix notation, we get that:
$$ [\nab \cdot \vec{u}]_j = \pd{u_j}{x_j} $$

\subsection{Curl}

the curl of a vector field can be written as:
$$ \nab \times \vec{u} $$
To write this in suffix notation, we can just use the cross produce formula:
$$ [\nab \times \vec{u}]_i = \e_{ijk} \nab_j u_k $$
which then can be manipulated into:
$$ [\nab \times \vec{u}]_i = \e_{ijk} \pd{u_k}{x_j} \qquad j,k = 1, 2, 3 $$
where $i$ is a free index and $j, k$ are dummy suffixes, so $j, k = 1, 2, 3$

\section{Combinations of gradient, divergence and curl}%  $\grad$, $\div(\quad)$ and $\curl(\quad)$}

\subsection{Divergence of Gradient}% $\underline{\nabla} \cdot \underline{\nabla} f$}

If we take $\nab \cdot \grad$ where $f = (x_1, x_2, x_3, t)$. We can write the div of grad as:
\begin{align*}
  \div \grad &= \left(\pd{}{x}\veci + \pd{}{y}\vecj + \pd{}{z}\veck \right) \cdot \left(\pd{f}{x}\veci + \pd{f}{y}\vecj + \pd{f}{z}\veck \right) \\
  &= \pd{}{x_1}\pd{f}{x_1} + \pd{}{x_2}\pd{f}{x_2} + \pd{}{x_3}\pd{f}{x_3}\\
  &= \pd{^2 f}{x_1^2} + \pd{^2 f}{x_2^2} + \pd{^2 f}{x_3^2}\\
  &= \Delta f
\end{align*}

Where the $\Delta = \nab ^2$ is the laplacian. So how do we write this in suffix notation?
\begin{align*}
  \div \grad &= \nab_j[\grad]_j\\
  &= \pd{}{x_j} \pd{f}{x_j}\\
  &= \pd{^2 f}{x_j}\\
\end{align*}

\subsection{Curl of Gradient}

We can write the curl of gradient as:
\begin{align*}
  \left[\curl \grad\right]_i &= \e_{ijk}\nab_j\grad_k\\
  &= \e_{ijk}\pd{}{x_j}\pd{f}{x_k}\\
  &= \e_{ikj}\pd{}{x_k}\pd{f}{x_j}\\
  &= - \e_{ijk}\pd{}{x_k}\pd{f}{x_j}\\
  &= - \e_{ijk}\pd{}{x_j}\pd{f}{x_k} && \text{if $f\in c^2$}\\
  &\implies \curl\grad = 0\\
\end{align*}

\subsection{Gradient of Divergence}
Assume we have a $\vec u$, vector field, and we want $\grad \div$.

\begin{align*}
  [\grad \div]_i &= \nab_i \pd{u_j}{x_j} \\
  &= \pd{}{x_i}\pd{u_j}{x_j} \\
  &= \pd{^2 u_j}{x_i\partial x_j}
\end{align*}

\subsection{Divergence of Curl}
We can write divergence of curl as:
\begin{align*}
  [\div\curl\vec u]_i &= \pd{}{x_i}[\curl\vec u]_i \\
  &= \pd{}{x_i}\e_{ijk}\pd{u_k}{x_j} \\
   \text{$i, j, k = 1, 2, 3$, so $i \leftrightarrow j$}\\
  &= \pd{}{x_j}\e_{jik}\pd{u_k}{x_i} \\
  &= -\e_{ijk}\pd{}{x_j}\pd{u_k}{x_i} \\
  &= -\e_{ijk}\pd{}{x_i}\pd{u_k}{x_j} && \text{as $\vec u \in c^2$} \\
\end{align*}
As $\div(\curl\vec u) = -\div(\curl\vec u)$, then we know that $\div(\curl\vec u) = 0$

\subsection{Curl of Curl}
We can write curl of curl, $ \curl(\curl\vec u)$, as:
\begin{align*}
  [\curl(\curl\vec u)]_i &= \e_{ijk}\pd{}{x_j}(\curl\vec u)_k \\
  &= \e_{ijk}\pd{}{x_j}\e_{klm}\pd{u_m}{x_l} \\
  &= (\delta_{il}\delta_{jm} - \delta_{im}\delta_{jl})\pd{^2 u_m}{x_j\partial x_l} \\
  &= \delta_{il}\delta_{jm}\pd{^2 u_m}{x_j\partial x_l} - \delta_{im}\delta_{jl}\pd{^2 u_m}{x_j\partial x_l} \\
  &= \pd{^2 u_j}{x_j \partial x_i} - \pd{^2 u_i}{x_j \partial x_j}\\
  &= \pd{}{x_i}\pd{u_j}{x_j} - \pd{^2 u_i}{x_j^2} \\
  &= [\nab(\div\vec u)]_i - [\Delta\vec u]_i\\
  &= [\nab(\div\vec u) - \nab^2\vec u]_i\\
\end{align*}

\section{Scalar Field / Vector Fields Defintions}

A scalar or vector quantity is said to be a {\color{blue} field }if it is a function of position. Examples
\begin{enumerate}
  \item {\color{orange}Temperature} is a scalar field, $T = T(x, y, z) = T(\vec r)$
  \item {\color{orange}Pressure and Density} are also scalr fields $P = P(\vec r)$ and $\rho = \rho(\vec r)$
  \item if a physical quantity is a scalar we speak of a scalar field or function of position.
\end{enumerate}

\noindent
If a physical quantity is a vector, such as force $\vec F = \vec F(x, y, z)$. We speak of a {\color{blue}vector field} or {\color{blue}vector function}. \\

\noindent
A {\color{blue} vector-valued function }is an $f: A\subset \R^n \mapsto \R^m$. So, for each $\vec x = (x_1,\,\dots\,, x_n) \in A$, $f$ assigns a value $f(\vec x)$, an m-tuple, in $\R^m$. These functions, $f$, are called vector-valued functions if $m>1$ and scalar if $m=1$.\\

\noindent\fbox{\parbox{0.475\textwidth}{\begin{example}{
  Take the function, $f: (x, y, z) \mapsto (x^2 + y^2 + z^2)^\frac{3}{2}$
}\end{example}\begin{solution}{
  It's a scalar function from $\R^3$ to $\R$.
}\end{solution}}}\vspace{10pt}

\noindent\fbox{\parbox{0.475\textwidth}{\begin{example}{
   Take the function $g: (x_1, x_2, x_3) \mapsto (x_1x_2x_3, \sqrt{x_1x_3}) $
}\end{example}\begin{solution}{
  This is a vector valued function from $\R^3$ to $\R^2$
}\end{solution}}}\vspace{10pt}

To specify a temperature $T$ in a region $A$ of space requires a function $T$, $T: A\subset \R^m \mapsto \R$. $T = T(x, y, z)$.\\

\noindent
To specify the velocity of a fluid moving in space requires a map, $\vec v: \R^4 \mapsto \R^3$ where $\vec v (x, y, z, t)$ is the velocity of the fluid at $(x, y, z)$ at time $t$.\\

\noindent
When $f: U\subset\R^n \mapsto \R$, we say that $f$ is a real valued function of $n$-variables with domain $U$.\\

\noindent
Let $f: U:\R^n \mapsto \R$, then graph $f = \{ (x_1, x_2,\, \dots\,, x_n) \in \R^{n + 1} : (x_1,\, \dots\,, x^n)\}$
If $n= 1$, then we can conclude that graph $f$ is curve in $\R^2$ and if $n=2$, then graph $f$ is a surface in $\R^3$.

\subsection{Level Sets, Curves and Surfaces}
A level set is a subset of $\R^3$ on which $f$ is constant. For example, for $f(x, y, z) = x^2 + y^2 + z^2$, the set where $x^2 + y^2 + z^2 = 1$ is alevel set. A level set is a set of $(x, y, z): f(x, y, z) = c$ where $c \in \R$.\\

For functions $f(x, y)$, we speak of level curves or contours. example, $f: \R^2 \mapsto \R$, $f(x, y) = x + y + 2$, has as it's graph the inclined plane $z = x + y + 2$. The plane intersects the $xy$ plan where $z = 0$ in the line $y = -x - 2$ and the $z$-axis at $(0, 0, 2)$. For any $c \in \R$, the level curve of $c$ is the straight line: $y = -x + (c-2): L_c \{(x, y): y = - x + c - 2 \}\subset \R^2$








\end{document}
