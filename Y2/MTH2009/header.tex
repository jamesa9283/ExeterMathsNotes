\makeatletter
\ifx \nauthor\undefined
  \def\nauthor{James Arthur}
\else
\fi

\author{Based on lectures by \nlecturer \\\small Notes taken by \nauthor}
\date{\nterm\ \nyear}

\usepackage[utf8x]{inputenc}
\usepackage{alltt}
\usepackage{amsfonts}
\usepackage{amsmath}
\usepackage{amssymb}
\usepackage{amsthm}
\usepackage{booktabs}
\usepackage{caption}
\usepackage{color}
\usepackage{enumitem}
\usepackage{fancyhdr}
\usepackage{fullpage}
\usepackage{graphicx}
\usepackage{mathdots}
\usepackage{mathtools}
\usepackage{microtype}
\usepackage{multirow}
\usepackage{listings}
\usepackage{pdflscape}
\usepackage{pgfplots}
\usepackage{siunitx}
\usepackage{slashed}
\usepackage{tabularx}
\usepackage{tikz}
\usepackage{tkz-euclide}
\usepackage[normalem]{ulem}
\usepackage[all]{xy}
\usepackage{imakeidx}
\usepackage{wrapfig}

\setlength{\headheight}{20pt}
\setlength{\headsep}{10pt}

% lstLean

\definecolor{keywordcolor}{rgb}{0.7, 0.1, 0.1}   % red
\definecolor{commentcolor}{rgb}{0.4, 0.4, 0.4}   % grey
\definecolor{symbolcolor}{rgb}{0.0, 0.1, 0.6}    % blue
\definecolor{sortcolor}{rgb}{0.1, 0.5, 0.1}      % green

\def\lstlanguagefiles{lstlean.tex}
\lstset{language=lean}


\makeindex[intoc, title=Index]
\indexsetup{othercode={\lhead{\emph{Index}}}}

\ifx \nextra \undefined
  \usepackage[pdftex,
    hidelinks,
    pdfauthor={James Arthur},
    pdfsubject={Exeter Maths Notes: Year \npart\ - \ncourse},
    pdftitle={Year \npart\ - \ncourse},
  pdfkeywords={Exeter Mathematics Maths Math \npart\ \nterm\ \nyear\ \ncourse}]{hyperref}
  \title{Year \npart\ --- \ncourse}
\else
  \usepackage[pdftex,
    hidelinks,
    pdfauthor={James Arthur},
    pdfsubject={Exeter Maths Notes: Year \npart\ - \ncourse\ (\nextra)},
    pdftitle={Year \npart\ - \ncourse\ (\nextra)},
  pdfkeywords={Exeter Mathematics Maths Math \npart\ \nterm\ \nyear\ \ncourse\ \nextra}]{hyperref}

  \title{Year \npart\ --- \ncourse \\ {\Large \nextra}}
  \renewcommand\printindex{}
\fi

\pgfplotsset{compat=1.12}

\pagestyle{fancyplain}
\ifx \ncoursehead \undefined
\def\ncoursehead{\ncourse}
\fi

\lhead{\emph{\nouppercase{\leftmark}}}
\ifx \nextra \undefined
  \rhead{
    \ifnum\thepage=1
    \else
      \npart\ \ncoursehead
    \fi}
\else
  \rhead{
    \ifnum\thepage=1
    \else
      \npart\ \ncoursehead \ (\nextra)
    \fi}
\fi
\usetikzlibrary{arrows.meta}
\usetikzlibrary{decorations.markings}
\usetikzlibrary{decorations.pathmorphing}
\usetikzlibrary{positioning}
\usetikzlibrary{fadings}
\usetikzlibrary{intersections}
\usetikzlibrary{cd}
\usetikzlibrary{shapes}


\newcommand*{\Cdot}{{\raisebox{-0.25ex}{\scalebox{1.5}{$\cdot$}}}}
\newcommand {\pd}[2][ ]{
  \ifx #1 { }
    \frac{\partial}{\partial #2}
  \else
    \frac{\partial^{#1}}{\partial #2^{#1}}
  \fi
}
\ifx \nhtml \undefined
\else
  \renewcommand\printindex{}
  \DisableLigatures[f]{family = *}
  \let\Contentsline\contentsline
  \renewcommand\contentsline[3]{\Contentsline{#1}{#2}{}}
  \renewcommand{\@dotsep}{10000}
  \newlength\currentparindent
  \setlength\currentparindent\parindent

  \newcommand\@minipagerestore{\setlength{\parindent}{\currentparindent}}
  \usepackage[active,tightpage,pdftex]{preview}
  \renewcommand{\PreviewBorder}{0.1cm}

  \newenvironment{stretchpage}%
  {\begin{preview}\begin{minipage}{\hsize}}%
    {\end{minipage}\end{preview}}
  \AtBeginDocument{\begin{stretchpage}}
  \AtEndDocument{\end{stretchpage}}

  \newcommand{\@@newpage}{\end{stretchpage}\begin{stretchpage}}

  \let\@real@section\section
  \renewcommand{\section}{\@@newpage\@real@section}
  \let\@real@subsection\subsection
  \renewcommand{\subsection}{\@ifstar{\@real@subsection*}{\@@newpage\@real@subsection}}
\fi
\ifx \ntrim \undefined
\else
  \usepackage{geometry}
  \geometry{
    papersize={379pt, 699pt},
    textwidth=345pt,
    textheight=596pt,
    left=17pt,
    top=54pt,
    right=17pt
  }
\fi

\usepackage{hyperref}
\hypersetup{
    colorlinks,
    citecolor=black,
    filecolor=black,
    linkcolor=black,
    urlcolor=black
}

\ifx \nisofficial \undefined
\let\@real@maketitle\maketitle
\renewcommand{\maketitle}{\@real@maketitle\begin{center}\begin{minipage}[c]{0.9\textwidth}\centering\footnotesize These notes are not endorsed by the lecturers, and I have modified them (often significantly) after lectures. They are nowhere near accurate representations of what was actually lectured, and in particular, all errors are almost surely mine.\\ \tableofcontents\end{minipage}\end{center}}
\else
\fi

\let\endtitlepage\relax

% Theorems
\theoremstyle{definition}
\newtheorem*{aim}{Aim}
\newtheorem*{axiom}{Axiom}
\newtheorem*{claim}{Claim}
\newtheorem*{cor}{Corollary}
\newtheorem*{conjecture}{Conjecture}
\newtheorem*{defi}{Definition}
\newtheorem*{eg}{Example}
\newtheorem*{ex}{Exercise}
\newtheorem*{fact}{Fact}
\newtheorem*{law}{Law}
\newtheorem*{lemma}{Lemma}
\newtheorem*{notation}{Notation}
\newtheorem*{prop}{Proposition}
\newtheorem*{question}{Question}
\newtheorem*{rrule}{Rule}
\newtheorem*{thm}{Theorem}
\newtheorem*{assumption}{Assumption}

\newtheorem*{remark}{Remark}
\newtheorem*{warning}{Warning}
\newtheorem*{exercise}{Exercise}

\newtheorem{nthm}{Theorem}[section]
\newtheorem{nlemma}[nthm]{Lemma}
\newtheorem{nprop}[nthm]{Proposition}
\newtheorem{ncor}[nthm]{Corollary}
\newtheorem{ndefi}[nthm]{Definition}


\renewcommand{\labelitemi}{--}
\renewcommand{\labelitemii}{$\circ$}
\renewcommand{\labelenumi}{(\roman{*})}

\let\stdsection\section
\renewcommand\section{\newpage\stdsection}

% :tada: emoji, converted using svg2tikz
\definecolor{cDD2E44}{RGB}{221,46,68}
\definecolor{cEA596E}{RGB}{234,89,110}
\definecolor{cA0041E}{RGB}{160,4,30}
\definecolor{cAA8DD8}{RGB}{170,141,216}
\definecolor{c77B255}{RGB}{119,178,85}
\definecolor{c5C913B}{RGB}{92,145,59}
\definecolor{c9266CC}{RGB}{146,102,204}
\definecolor{cFFCC4D}{RGB}{255,204,77}
\def\tada{\tikz[y=0.80pt,x=0.80pt,yscale=-0.5,xscale=0.5,inner sep=0pt,outer sep=0pt]{
\path[fill=cDD2E44](11.626,7.488)..controls(11.514,7.6)and(11.429,7.735)..(11.358,7.883)--(11.35,7.875)--(0.134,33.141)--(0.145,33.152)..controls(-0.0630,33.555)and(0.285,34.375)..(0.998,35.0890)..controls(1.711,35.8020)and(2.531,36.15)..(2.934,35.942)--(2.944,35.952)--(28.21,24.735)--(28.2020,24.726)..controls(28.349,24.656)and(28.484,24.571)..(28.597,24.457)..controls(30.159,22.895)and(27.626,17.83)..(22.941,13.144)..controls(18.254,8.458)and(13.189,5.926)..(11.626,7.488)--cycle;
\path[fill=cEA596E](13,12)--(0.416,32.5060)--(0.134,33.141)--(0.145,33.152)..controls(-0.0630,33.555)and(0.285,34.375)..(0.998,35.0890)..controls(1.23,35.321)and(1.471,35.497)..(1.7070,35.646)--(17,17)--(13,12)--cycle;
\path[fill=cA0041E](23.0120,13.0660)..controls(27.682,17.738)and(30.275,22.718)..(28.8010,24.19)..controls(27.328,25.664)and(22.348,23.0720)..(17.675,18.4020)..controls(13.0040,13.73)and(10.412,8.748)..(11.885,7.275)..controls(13.359,5.8020)and(18.339,8.394)..(23.0120,13.0660)--cycle;
\path[fill=cAA8DD8](18.59,13.6090)..controls(18.391,13.77)and(18.131,13.854)..(17.856,13.824)..controls(16.988,13.73)and(16.258,13.428)..(15.747,12.951)..controls(15.2060,12.446)and(14.939,11.768)..(15.0120,11.0890)..controls(15.14,9.897)and(16.336,8.8030)..(18.375,9.0230)..controls(19.168,9.1080)and(19.522,8.853)..(19.534,8.731)..controls(19.548,8.61)and(19.257,8.285)..(18.464,8.199)..controls(17.596,8.1050)and(16.866,7.8030)..(16.354,7.326)..controls(15.813,6.821)and(15.545,6.143)..(15.619,5.464)..controls(15.749,4.272)and(16.944,3.178)..(18.981,3.399)..controls(19.559,3.461)and(19.864,3.342)..(19.993,3.265)..controls(20.0960,3.2020)and(20.137,3.142)..(20.141,3.1070)..controls(20.153,2.986)and(19.866,2.661)..(19.0710,2.575)..controls(18.522,2.515)and(18.124,2.0230)..(18.185,1.473)..controls(18.244,0.924)and(18.735,0.527)..(19.286,0.587)..controls(21.323,0.8060)and(22.259,2.129)..(22.13,3.322)..controls(22,4.516)and(20.8050,5.6080)..(18.766,5.389)..controls(18.188,5.326)and(17.886,5.446)..(17.756,5.523)..controls(17.653,5.585)and(17.611,5.646)..(17.6070,5.68)..controls(17.594,5.8020)and(17.883,6.126)..(18.678,6.212)..controls(20.715,6.432)and(21.651,7.754)..(21.522,8.947)..controls(21.393,10.139)and(20.198,11.233)..(18.16,11.0120)..controls(17.582,10.95)and(17.278,11.0700)..(17.148,11.146)..controls(17.0440,11.21)and(17.0040,11.27)..(17,11.3040)..controls(16.987,11.425)and(17.276,11.75)..(18.0700,11.836)..controls(18.618,11.896)and(19.0170,12.389)..(18.956,12.938)..controls(18.928,13.212)and(18.789,13.449)..(18.59,13.6090)--cycle(23.0010,20.16)..controls(22.7070,20.16)and(22.417,20.0310)..(22.219,19.785)..controls(21.874,19.353)and(21.945,18.724)..(22.375,18.379)..controls(22.593,18.2040)and(27.793,14.12)..(35.142,15.171)..controls(35.689,15.249)and(36.0690,15.755)..(35.991,16.3020)..controls(35.913,16.848)and(35.411,17.232)..(34.859,17.15)..controls(28.366,16.228)and(23.672,19.9040)..(23.626,19.941)..controls(23.44,20.0890)and(23.22,20.16)..(23.0010,20.16)--cycle;
\path[fill=c77B255](30.661,22.857)..controls(32.634,22.3)and(33.995,23.18)..(34.319,24.335)..controls(34.643,25.489)and(33.941,26.95)..(31.969,27.5050)..controls(31.199,27.721)and(30.968,28.0890)..(30.999,28.2060)..controls(31.0330,28.324)and(31.424,28.518)..(32.192,28.3010)..controls(34.164,27.746)and(35.525,28.626)..(35.849,29.78)..controls(36.175,30.935)and(35.471,32.394)..(33.498,32.95)..controls(32.729,33.166)and(32.497,33.535)..(32.531,33.652)..controls(32.564,33.769)and(32.954,33.963)..(33.723,33.747)..controls(34.253,33.598)and(34.8070,33.9070)..(34.956,34.438)..controls(35.1040,34.97)and(34.795,35.522)..(34.263,35.672)..controls(32.292,36.227)and(30.93,35.349)..(30.6040,34.193)..controls(30.28,33.0390)and(30.983,31.58)..(32.957,31.0240)..controls(33.727,30.8070)and(33.958,30.44)..(33.924,30.322)..controls(33.892,30.2050)and(33.5020,30.0100)..(32.734,30.226)..controls(30.76,30.782)and(29.4,29.9040)..(29.0750,28.747)..controls(28.75,27.593)and(29.453,26.134)..(31.426,25.577)..controls(32.194,25.362)and(32.425,24.992)..(32.393,24.876)..controls(32.359,24.758)and(31.97,24.564)..(31.2010,24.78)..controls(30.669,24.93)and(30.118,24.62)..(29.968,24.0890)..controls(29.819,23.559)and(30.129,23.0070)..(30.661,22.857)--cycle(5.754,16)..controls(5.659,16)and(5.562,15.986)..(5.466,15.958)..controls(4.937,15.799)and(4.637,15.242)..(4.796,14.713)..controls(5.929,10.94)and(6.956,4.919)..(5.694,3.349)..controls(5.553,3.171)and(5.34,2.996)..(4.852,3.0330)..controls(3.914,3.1050)and(4.0030,5.0840)..(4.0040,5.1040)..controls(4.0460,5.655)and(3.632,6.135)..(3.0820,6.176)..controls(2.523,6.21)and(2.0510,5.8040)..(2.0100,5.253)..controls(1.9070,3.874)and(2.336,1.218)..(4.7020,1.0390)..controls(5.758,0.959)and(6.635,1.326)..(7.254,2.0960)..controls(9.625,5.0470)and(7.218,13.6020)..(6.712,15.288)..controls(6.582,15.721)and(6.184,16)..(5.754,16)--cycle;
\path[fill=c9266CC](2,18)circle(0.0564cm);
\path[fill=c5C913B](25.5,9.5)circle(0.0423cm)(32.5,19.5)circle(0.0423cm)(23.5,31.5)circle(0.0423cm);
\path[fill=cFFCC4D](28,4)circle(0.0564cm)(32.5,8.5)circle(0.0423cm)(29.5,12.5)circle(0.0423cm)(7.5,23.5)circle(0.0423cm);
}}
% end of :tada:
\renewcommand\qedsymbol{\tada}

\everymath{\displaystyle}

\newcommand\qedsym{\hfill\ensuremath{\square}}
% Strike through
\def\st{\bgroup \ULdepth=-.55ex \ULset}

\tikzset{every picture/.style={remember picture}}
\usepackage{accents}
\newcommand\myubar[1]{%
\underaccent{\bar}{#1}}


%%%%%%%%%%%%%%%%%%%%%%%%%
%%%%% Maths Symbols %%%%%
%%%%%%%%%%%%%%%%%%%%%%%%%

% Matrix groups
\newcommand{\GL}{\mathrm{GL}}
\newcommand{\Or}{\mathrm{O}}
\newcommand{\PGL}{\mathrm{PGL}}
\newcommand{\PSL}{\mathrm{PSL}}
\newcommand{\PSO}{\mathrm{PSO}}
\newcommand{\PSU}{\mathrm{PSU}}
\newcommand{\SL}{\mathrm{SL}}
\newcommand{\SO}{\mathrm{SO}}
\newcommand{\Spin}{\mathrm{Spin}}
\newcommand{\Sp}{\mathrm{Sp}}
\newcommand{\SU}{\mathrm{SU}}
\newcommand{\U}{\mathrm{U}}
\newcommand{\Mat}{\mathrm{Mat}}

% Matrix algebras
\newcommand{\gl}{\mathfrak{gl}}
\newcommand{\ort}{\mathfrak{o}}
\newcommand{\so}{\mathfrak{so}}
\newcommand{\su}{\mathfrak{su}}
\newcommand{\uu}{\mathfrak{u}}
\renewcommand{\sl}{\mathfrak{sl}}

% Special sets
\newcommand{\C}{\mathbb{C}}
\newcommand{\CP}{\mathbb{CP}}
\newcommand{\GG}{\mathbb{G}}
\newcommand{\N}{\mathbb{N}}
\newcommand{\Q}{\mathbb{Q}}
\newcommand{\R}{\mathbb{R}}
\newcommand{\RP}{\mathbb{RP}}
\newcommand{\T}{\mathbb{T}}
\newcommand{\Z}{\mathbb{Z}}
\renewcommand{\H}{\mathbb{H}}

% Brackets
\newcommand{\abs}[1]{\left\lvert #1\right\rvert}
\newcommand{\bket}[1]{\left\lvert #1\right\rangle}
\newcommand{\brak}[1]{\left\langle #1 \right\rvert}
\newcommand{\braket}[2]{\left\langle #1\middle\vert #2 \right\rangle}
\newcommand{\bra}{\langle}
\newcommand{\ket}{\rangle}
\newcommand{\norm}[1]{\left\lVert #1\right\rVert}
\newcommand{\normalorder}[1]{\mathop{:}\nolimits\!#1\!\mathop{:}\nolimits}
\newcommand{\tv}[1]{|#1|}
\renewcommand{\vec}[1]{\boldsymbol{\mathbf{#1}}}

% not-math
\newcommand{\bolds}[1]{{\bfseries #1}}
\newcommand{\cat}[1]{\mathsf{#1}}
\newcommand{\ph}{\,\cdot\,}
\newcommand{\term}[1]{\emph{#1}\index{#1}}
\newcommand{\phantomeq}{\hphantom{{}={}}}
% Probability
\DeclareMathOperator{\Bernoulli}{Bernoulli}
\DeclareMathOperator{\betaD}{beta}
\DeclareMathOperator{\bias}{bias}
\DeclareMathOperator{\binomial}{binomial}
\DeclareMathOperator{\corr}{corr}
\DeclareMathOperator{\cov}{cov}
\DeclareMathOperator{\gammaD}{gamma}
\DeclareMathOperator{\mse}{mse}
\DeclareMathOperator{\multinomial}{multinomial}
\DeclareMathOperator{\Poisson}{Poisson}
\DeclareMathOperator{\var}{var}
\newcommand{\E}{\mathbb{E}}
\newcommand{\Prob}{\mathbb{P}}

% Algebra
\DeclareMathOperator{\adj}{adj}
\DeclareMathOperator{\Ann}{Ann}
\DeclareMathOperator{\Aut}{Aut}
\DeclareMathOperator{\Char}{char}
\DeclareMathOperator{\disc}{disc}
\DeclareMathOperator{\dom}{dom}
\DeclareMathOperator{\fix}{fix}
\DeclareMathOperator{\Hom}{Hom}
\DeclareMathOperator{\id}{id}
\DeclareMathOperator{\image}{image}
\DeclareMathOperator{\im}{im}
\DeclareMathOperator{\tr}{tr}
\DeclareMathOperator{\Tr}{Tr}
\newcommand{\Bilin}{\mathrm{Bilin}}
\newcommand{\Frob}{\mathrm{Frob}}

% Others
\newcommand\ad{\mathrm{ad}}
\newcommand\Art{\mathrm{Art}}
\newcommand{\B}{\mathcal{B}}
\newcommand{\cU}{\mathcal{U}}
\newcommand{\Der}{\mathrm{Der}}
\newcommand{\D}{\mathrm{D}}
\newcommand{\dR}{\mathrm{dR}}
\newcommand{\exterior}{\mathchoice{{\textstyle\bigwedge}}{{\bigwedge}}{{\textstyle\wedge}}{{\scriptstyle\wedge}}}
\newcommand{\F}{\mathbb{F}}
\newcommand{\G}{\mathcal{G}}
\newcommand{\Gr}{\mathrm{Gr}}
\newcommand{\haut}{\mathrm{ht}}
\newcommand{\Hol}{\mathrm{Hol}}
\newcommand{\hol}{\mathfrak{hol}}
\newcommand{\Id}{\mathrm{Id}}
\newcommand{\lie}[1]{\mathfrak{#1}}
\newcommand{\op}{\mathrm{op}}
\newcommand{\Oc}{\mathcal{O}}
\newcommand{\pr}{\mathrm{pr}}
\newcommand{\Ps}{\mathcal{P}}
\newcommand{\pt}{\mathrm{pt}}
\newcommand{\qeq}{\mathrel{``{=}"}}
\newcommand{\Rs}{\mathcal{R}}
\newcommand{\Vect}{\mathrm{Vect}}
\newcommand{\wsto}{\stackrel{\mathrm{w}^*}{\to}}
\newcommand{\wt}{\mathrm{wt}}
\newcommand{\wto}{\stackrel{\mathrm{w}}{\to}}
\renewcommand{\d}{\mathrm{d}}
\renewcommand{\P}{\mathbb{P}}
%\renewcommand{\F}{\mathcal{F}}


% CA

\newcommand{\conj}[1]{\overline{#1}}
\renewcommand{\ex}{\exists\,}
\newcommand{\fa}{\forall\,}
\newcommand{\es}{\varnothing}
\newcommand{\sub}{\subset}
\newcommand{\bus}{\supseteq}

% Greek Letters
\renewcommand{\a}{\alpha}
\renewcommand{\b}{\beta}
\renewcommand{\d}{\delta}
\newcommand{\e}{\varepsilon}
\newcommand{\g}{\gamma}
\renewcommand{\th}{\theta}
\renewcommand{\ph}{\varphi}

\renewcommand{\l}{\ell\,}
\newcommand{\di}[2]{D(#1, #2)}
\newcommand{\cdi}[2]{\overline D(#1, #2)}
\newcommand{\pdi}[2]{D'(#1, #2)}
\newcommand{\is}{\sum_{n=0}^\infty}
\newcommand{\ls}{\sum_{n = -\infty}^{\infty}}
\newcommand{\lsp}{\sum_{n = -\infty}^{-1}}



\let\Im\relax
\let\Re\relax

\DeclareMathOperator{\area}{area}
\DeclareMathOperator{\card}{card}
\DeclareMathOperator{\ccl}{ccl}
\DeclareMathOperator{\ch}{ch}
\DeclareMathOperator{\cl}{cl}
\DeclareMathOperator{\cls}{\overline{\mathrm{span}}}
\DeclareMathOperator{\coker}{coker}
\DeclareMathOperator{\conv}{conv}
\DeclareMathOperator{\cosec}{cosec}
\DeclareMathOperator{\cosech}{cosech}
\DeclareMathOperator{\covol}{covol}
\DeclareMathOperator{\diag}{diag}
\DeclareMathOperator{\diam}{diam}
\DeclareMathOperator{\Diff}{Diff}
\DeclareMathOperator{\End}{End}
\DeclareMathOperator{\energy}{energy}
\DeclareMathOperator{\erfc}{erfc}
\DeclareMathOperator{\erf}{erf}
\DeclareMathOperator*{\esssup}{ess\,sup}
\DeclareMathOperator{\ev}{ev}
\DeclareMathOperator{\Ext}{Ext}
\DeclareMathOperator{\fst}{fst}
\DeclareMathOperator{\Fit}{Fit}
\DeclareMathOperator{\Frac}{Frac}
\DeclareMathOperator{\Gal}{Gal}
\DeclareMathOperator{\gr}{gr}
\DeclareMathOperator{\hcf}{hcf}
\DeclareMathOperator{\Im}{Im}
\DeclareMathOperator{\Ind}{Ind}
\DeclareMathOperator{\Int}{Int}
\DeclareMathOperator{\Isom}{Isom}
\DeclareMathOperator{\lcm}{lcm}
\DeclareMathOperator{\length}{length}
\DeclareMathOperator{\Lie}{Lie}
\DeclareMathOperator{\like}{like}
\DeclareMathOperator{\Lk}{Lk}
\DeclareMathOperator{\Maps}{Maps}
\DeclareMathOperator{\orb}{orb}
\DeclareMathOperator{\ord}{ord}
\DeclareMathOperator{\otp}{otp}
\DeclareMathOperator{\poly}{poly}
\DeclareMathOperator{\rank}{rank}
\DeclareMathOperator{\rel}{rel}
\DeclareMathOperator{\Rad}{Rad}
\DeclareMathOperator{\Re}{Re}
\DeclareMathOperator*{\res}{res}
\DeclareMathOperator{\Res}{Res}
\DeclareMathOperator{\Ric}{Ric}
\DeclareMathOperator{\rk}{rk}
\DeclareMathOperator{\Rees}{Rees}
\DeclareMathOperator{\Root}{Root}
\DeclareMathOperator{\sech}{sech}
\DeclareMathOperator{\sgn}{sgn}
\DeclareMathOperator{\snd}{snd}
\DeclareMathOperator{\Spec}{Spec}
\DeclareMathOperator{\spn}{span}
\DeclareMathOperator{\stab}{stab}
\DeclareMathOperator{\St}{St}
\DeclareMathOperator{\supp}{supp}
\DeclareMathOperator{\Syl}{Syl}
\DeclareMathOperator{\Sym}{Sym}
\DeclareMathOperator{\vol}{vol}

\pgfarrowsdeclarecombine{twolatex'}{twolatex'}{latex'}{latex'}{latex'}{latex'}
\tikzset{->/.style = {decoration={markings,
                                  mark=at position 1 with {\arrow[scale=2]{latex'}}},
                      postaction={decorate}}}
\tikzset{<-/.style = {decoration={markings,
                                  mark=at position 0 with {\arrowreversed[scale=2]{latex'}}},
                      postaction={decorate}}}
\tikzset{<->/.style = {decoration={markings,
                                   mark=at position 0 with {\arrowreversed[scale=2]{latex'}},
                                   mark=at position 1 with {\arrow[scale=2]{latex'}}},
                       postaction={decorate}}}
\tikzset{->-/.style = {decoration={markings,
                                   mark=at position #1 with {\arrow[scale=2]{latex'}}},
                       postaction={decorate}}}
\tikzset{-<-/.style = {decoration={markings,
                                   mark=at position #1 with {\arrowreversed[scale=2]{latex'}}},
                       postaction={decorate}}}
\tikzset{->>/.style = {decoration={markings,
                                  mark=at position 1 with {\arrow[scale=2]{latex'}}},
                      postaction={decorate}}}
\tikzset{<<-/.style = {decoration={markings,
                                  mark=at position 0 with {\arrowreversed[scale=2]{twolatex'}}},
                      postaction={decorate}}}
\tikzset{<<->>/.style = {decoration={markings,
                                   mark=at position 0 with {\arrowreversed[scale=2]{twolatex'}},
                                   mark=at position 1 with {\arrow[scale=2]{twolatex'}}},
                       postaction={decorate}}}
\tikzset{->>-/.style = {decoration={markings,
                                   mark=at position #1 with {\arrow[scale=2]{twolatex'}}},
                       postaction={decorate}}}
\tikzset{-<<-/.style = {decoration={markings,
                                   mark=at position #1 with {\arrowreversed[scale=2]{twolatex'}}},
                       postaction={decorate}}}

\tikzset{circ/.style = {fill, circle, inner sep = 0, minimum size = 3}}
\tikzset{scirc/.style = {fill, circle, inner sep = 0, minimum size = 1.5}}
\tikzset{mstate/.style={circle, draw, blue, text=black, minimum width=0.7cm}}

\tikzset{eqpic/.style={baseline={([yshift=-.5ex]current bounding box.center)}}}
\tikzset{commutative diagrams/.cd,cdmap/.style={/tikz/column 1/.append style={anchor=base east},/tikz/column 2/.append style={anchor=base west},row sep=tiny}}

\definecolor{mblue}{rgb}{0.2, 0.3, 0.8}
\definecolor{morange}{rgb}{1, 0.5, 0}
\definecolor{mgreen}{rgb}{0.1, 0.4, 0.2}
\definecolor{mred}{rgb}{0.5, 0, 0}

\def\drawcirculararc(#1,#2)(#3,#4)(#5,#6){%
    \pgfmathsetmacro\cA{(#1*#1+#2*#2-#3*#3-#4*#4)/2}%
    \pgfmathsetmacro\cB{(#1*#1+#2*#2-#5*#5-#6*#6)/2}%
    \pgfmathsetmacro\cy{(\cB*(#1-#3)-\cA*(#1-#5))/%
                        ((#2-#6)*(#1-#3)-(#2-#4)*(#1-#5))}%
    \pgfmathsetmacro\cx{(\cA-\cy*(#2-#4))/(#1-#3)}%
    \pgfmathsetmacro\cr{sqrt((#1-\cx)*(#1-\cx)+(#2-\cy)*(#2-\cy))}%
    \pgfmathsetmacro\cA{atan2(#2-\cy,#1-\cx)}%
    \pgfmathsetmacro\cB{atan2(#6-\cy,#5-\cx)}%
    \pgfmathparse{\cB<\cA}%
    \ifnum\pgfmathresult=1
        \pgfmathsetmacro\cB{\cB+360}%
    \fi
    \draw (#1,#2) arc (\cA:\cB:\cr);%
}
\newcommand\getCoord[3]{\newdimen{#1}\newdimen{#2}\pgfextractx{#1}{\pgfpointanchor{#3}{center}}\pgfextracty{#2}{\pgfpointanchor{#3}{center}}}

\newcommand\qedshift{\vspace{-17pt}}
\newcommand\fakeqed{\pushQED{\qed}\qedhere}

\def\Xint#1{\mathchoice
   {\XXint\displaystyle\textstyle{#1}}%
   {\XXint\textstyle\scriptstyle{#1}}%
   {\XXint\scriptstyle\scriptscriptstyle{#1}}%
   {\XXint\scriptscriptstyle\scriptscriptstyle{#1}}%
   \!\int}
\def\XXint#1#2#3{{\setbox0=\hbox{$#1{#2#3}{\int}$}
     \vcenter{\hbox{$#2#3$}}\kern-.5\wd0}}
\def\ddashint{\Xint=}
\def\dashint{\Xint-}

\newcommand\separator{{\centering\rule{2cm}{0.2pt}\vspace{2pt}\par}}

\newenvironment{own}{\color{gray!70!black}}{}

\newcommand\makecenter[1]{\raisebox{-0.5\height}{#1}}

\mathchardef\mdash="2D

\newenvironment{significant}{\begin{center}\begin{minipage}{0.9\textwidth}\centering\em}{\end{minipage}\end{center}}
\DeclareRobustCommand{\rvdots}{%
  \vbox{
    \baselineskip4\p@\lineskiplimit\z@
    \kern-\p@
    \hbox{.}\hbox{.}\hbox{.}
  }}
\DeclareRobustCommand\tph[3]{{\texorpdfstring{#1}{#2}}}
\makeatother
